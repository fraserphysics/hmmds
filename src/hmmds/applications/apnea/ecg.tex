% Notes on modeling ECGs with HMMs directly

\documentclass[12pt]{article}
\usepackage{graphicx,color}
\usepackage{amsmath, amsfonts}
\usepackage{placeins}

\newcommand{\field}[1]{\mathbb{#1}}
\newcommand{\INTEGER}{\field{Z}}
\newcommand{\REAL}{\field{R}}
\newcommand{\COMPLEX}{\field{C}}
\newcommand{\id}{\mathbb{I}}
\newcommand{\variance}[1]{\field{V}\left[ #1 \right]}
\newcommand{\normal}[2]{{\cal N}\left(#1,#2 \right)}

\title{HMMs for ECGs}
\author{Andrew M.\ Fraser}

\begin{document}
\maketitle

\section{Introduction}
\label{sec:introduction}

I wrote explore.py to see how processing ECGs for apnea detection was
working.  Because I was disappointed by the performance of standard qrs
detectors that I got from
https://github.com/berndporr/py-ecg-detectors, I wrote my own by
building HMMs for ECG time series.

Recipes for making the models and figures described here appear in
Rules.mk.

\section{Outliers}
\label{sec:outliers}

While most of the ECG data consists of oscillations with amplitudes
less than $\pm$ 2mV, some intervals have spurious signals with $\pm$
10mV.  I call such signals \emph{lead noise}\footnote{The data in
  record \emph{C01} is less than .1mV till minute 5.1 when it goes to
  10 mV.  The useable data only starts at minute 10.24.}.  A model,
$\theta$, fit to the typical data can assign $P(y|\theta) = 0$ to lead
noise.  The observation models use inverse gamma priors for the
variance, and I use a single state, which I call the \emph{bad} state,
with prior that enforces a large variance to accomodate lead noise.
In particular the parameters are $\alpha=10^{8}$ and $\beta=10^{20}$
which in the reestimation formula
\begin{equation*}
  \sigma^2 = \frac{2 \beta + \text{data}}{2 \alpha +2 + \text{weight}}
\end{equation*}
gives $\sigma^2 \approx 10^{12}$ if there is not much data.  I arrived
at those values by trial and error.

Since lead noise can start abruptly at any time, I allow transitions
between the bad state and every other state.  It's important to ensure
that those transitions are infrequent.  The high variance makes the
likelihood for the bad state small.

\section{A Monotonic Model with 21 States}
\label{sec:mono20}

I've built HMMs with 20 normal states and 1 bad state.  Each normal
state has transitions to and from the bad state.  Otherwise the only
transitions are from a normal state to itself and to its successor.
The successor of state 19 is state 0, so the normal states progress
monotonically around a loop.

My first output model was Gaussian with the mean for state $s$ given
in terms of the mean and rms deviation of the previosu thousand
observations, the previous observation and a fixed offset, with
\begin{align*}
  s &= \sqrt{\frac{\sum_{\tau=1,000}^{\tau=1} y[t-\tau]^2}{1,000}} \\
  m &=  \frac{\sum_{\tau=1,000}^{\tau=1} y[t-\tau]}{1,000} \\
  \mu[t,i](y[:t]) &= a[i,0] \cdot s + a[i,1] \cdot m + a[i,2] \cdot
  y[t-1] + a[i,3]
\end{align*}

Veterbi decoding with such a model after training for 20 iterations on
record \emph{a01} yields Figure~\ref{fig:modelAR1k20}

\begin{figure}
  \centering
    \resizebox{0.7\textwidth}{!}{\includegraphics{modelAR1k20.pdf}}
  \caption{Performance of an HMM with 20 normal states with an output
    model that uses the previous 1,000 observations to characterize
    the mean and variance of the signal.  An ECG trace appears in the
    upper plot, while the decoded state sequence appears in the lower
    plot.  The likelihood of the data after 30 iterations of the Baum
    Welch algorithm is $5.686 \times 10^{6}$.}
  \label{fig:modelAR1k20}
\end{figure}

A simple linear auto-regressive model with
\begin{equation*}
  \mu[t,i] = a[i,0] \cdot y[t-1] + a[i,1] \cdot y[t-2] + a[i,2] \cdot
  y[t-2]  + a[i,3]
\end{equation*}
similarly yields Figure~\ref{fig:modelAR3_20}.

\begin{figure}
  \centering
    \resizebox{1.0\textwidth}{!}{\includegraphics{modelAR3_20.pdf}}
    \caption{Seems similar to \ref{fig:modelAR1k20}.  The likelihood
      of the data after 30 iterations of the Baum Welch algorithm is
      $6.046 \times 10^{6}$.}
  \label{fig:modelAR3_20}
\end{figure}

\begin{figure}
  \centering
    \resizebox{1.0\textwidth}{!}{\includegraphics{modelAR3_300.pdf}}
    \caption{Some Pig}
  \label{fig:modelAR3_300}
\end{figure}

\section{To Do}
\label{sec:todo}

\begin{itemize}
\item Build document, data, and figures using make.
\item For each base model name, eg, 20AR1k, make initializes a model
  and trains it for 30 iterations saving the training characteristic
  and the model at 5 step intervals.
\item Train on \emph{a01}
\item Plots show data from \emph{a01} and \emph{c01}
\item Train on subsegments of records to enable parallel processing.
\end{itemize}
\end{document}
