
% Notes on modeling heart rate

\documentclass[12pt]{article}
\usepackage{graphicx,color}
\usepackage{amsmath, amsfonts}
\usepackage{placeins}

\newcommand{\field}[1]{\mathbb{#1}}
\newcommand{\INTEGER}{\field{Z}}
\newcommand{\REAL}{\field{R}}
\newcommand{\COMPLEX}{\field{C}}
\newcommand{\id}{\mathbb{I}}
\newcommand{\variance}[1]{\field{V}\left[ #1 \right]}
\newcommand{\normal}[2]{{\cal N}\left(#1,#2 \right)}
\newcommand{\argmax}{\operatorname*{argmax}}

\title{HMMs for Heart Rate}
\author{Andrew M.\ Fraser}

\begin{document}
\maketitle

\section{Introduction}
\label{sec:introduction}

Here I document ideas for redoing the last chapter of my book
\emph{Hidden Markov Models and Dynamical Systems} in which I address
the CINC-2000 challenge.  I want to map the products of the ECG
analysis described in my document \emph{HMMs for ECGs} to the two
classification challenges.  The inputs are time series of heart rate
estimates and likelihoods sampled at 2 Hz and the outputs are two
stages of classification.  First: For each record determine whether or
not the patient experienced apnea; and Second: for each minute of each
record determine whether or not the patient was experiencing apnea.

\section{A Model for Record a03}
\label{sec:a03}

\textbf{To Do:}  Describe structure of the model and its performance.

\subsection{C records}
\label{sec:c_records}

Because the C records have only a few or zero minutes marked as apnea,
trying to create and train models with states for the apnea class
fails.  I build models of C records that have a single state and that
state is in the normal class.

\section{Pass1 Challenges}
\label{sec:pass1}

I built a model called \emph{two\_ar3\_masked6.1} with the following
features:
\begin{itemize}
\item Models detected peaks and forces them to be on chains
\item There are \emph{two} chains, one for normal and one for apnea
\item The lengths of the \emph{intervals} between peaks are modeled
\item The heart rate signal is fit to AR3 models
\end{itemize}
Here I will refer to that model as simply the \emph{HMM}.

The HMM does pass2 classification for the \emph{a} records reasonably
well except for \emph{a09} and \emph{a18}.  But those records are
almost all apnea, and the HMM has very high likelihood for them.  I'd
hoped that I could rely on high likelihood to identify problematic
\emph{x} records and mark them as all apnea, but that didn't work.

\begin{figure}
  \centering
    \resizebox{0.8\textwidth}{!}{\includegraphics{statistics.pdf}}
    \caption{Possible pass1 statistics.  The upper and middle plots
      are normalized power spectral density estimates.  The sum of
      channels from 1.0 to 3.6 cpm is $F(PSD)$ in the lower plot.
      Note that records a11 and c10 are out of order for separating
      \emph{a} records from \emph{c} records.  The y-axis in the lower
      plot is the log likelihood of a particular HMM for data from
      each record.  The vertical dotted line only differs from the
      diagonal line in its classification of x06, x17 and x18.  The
      HMM classifies most minutes of those records as normal, and they
      look like c records visually.}
  \label{fig:statistics}
\end{figure}

Figure~\ref{fig:statistics}\footnote{The PSDs are normalized to have
  the same power for frequencies above 0.3 cpm)} illustrates pass1.
  The following \emph{x} records fall close to the boundary:
\begin{description}
\item[x06] Pass1 says A which seems \textbf{wrong}.  But HMM says this
  is mostly normal, which seems correct.
\item[x09] Pass1 says A which seems correct, but HMM says this is
  mostly normal.
\item[x10] Pass1 says A which seems correct.  HMM misses some of the
  apnea
\item[x13] Pass 1 says A which seems correct.  HMM misses some of the
  apnea.
\item[x15] Pass1 says A.  Maybe it's B.  HMM results seem plausible.
\item[x17] Pass1 says A which seems \textbf{wrong}, but hmm says this
  is mostly normal.
\item[x23] Pass1 says A which seems correct.  HMM has trouble
  detecting apnea.
\item[x24] Pass1 says N which seems correct.  HMM says almost all
  normal.
\item[x33 \& x34] Pass1 says N.  I don't know if that's correct.
  Visually I guess there might be a few episodes of apnea.  HMM
  agrees.  I think \emph{x33} and \emph{x34} are different electrodes
  measuring the same record.
\end{description}

Here is my guess at classification of the records:
\begin{description}
\item[C]
  x03 (classified as A),
  x04,
  x06,
  x11,
  x17,
  x18 (classified as A),
  x22,
  x24,
  x29,
  x35
\item[B]
  x13 (A),
  x15 (A),
  x26 (N),
  x33 (N),
  x34 (N)
\item[A]
  x01, x02, x05, x07, x08, x09, x10, x12, x14, x16, x19, x20,
  x21, x23, x25, x27, x28, x30, x31, x32
\end{description}

x29 suffers alternans not apnea.

Setting the threshold for statistic 1 between 0.34 and 0.36 classifies
all of the \emph{a} and \emph{c} records records correctly except c10.  These are the records for which statistic 1 is between 0.28
and 0.42:
\begin{description}
\item[x24]  Need to trim first 15 minutes.  Looks normal to me.
\item[c09] Like c02.  Hour to hour heart rate changes.
\item[c07] Low hour to hour heart rate variation.
\item[c02] Hour to hour heart rate changes.
\item[x33 x34] These are different measurements of the same thing.
  I think it's a B record.
\item[b04] HMM correctly says mostly normal.
\item[c08] HMM says some apnea, but expert says none.
\item[x06] First 5.2 minutes need to be trimmed.  I guess it's a B
  record.
\item[x17] Last 5 minutes need to be trimmed.  Apnea 126-150.
\item[x18] Need to trim first 14.9 minutes.  206-230 looks like apnea
  oscillations, but respiration look sort of OK.
\item[x15] Apnea 120 - 140, 180 - 210, 350-390, 490-515.  More hour to
  hour variation than c07.
\item[a11] PMD for HMM is .64
\item[a06] Amplitude of apnea related heart rate oscillations is
  small.  The hour to hour changes may be helpful.  Classifying the
  minutes looks hard.
\item[x10] Apnea 220 - 250, 360-390
\item[x13] I guess it's B.
\item[x07] I see apnea oscillations.
\item[a09] Respiration signal is very weak.  Heart rate changes hour
  to hour but not as much as a10.  The amplitude of the oscillations
  in the heart rate during apnea is small.  It seems difficult to me.
\item[c10] Heart rate has apnea like oscillations but returns to same
  baseline.  During those oscillations the respiration signal also
  oscillates, but perhaps not as deeply as a10.
\item[a10] Heart rate changes hour to hour
\item[x09] HMM misses clear apnea episodes
\item[x23] Apnea 120-185 but respiration looks sort of OK
\end{description}

\textbf{Conclude:} Statistic 1 alone is adequate for pass1.

\section{Work on Pass2}
\label{sec:pass2}

\begin{table*}
  \centering
  \input{two.tex}
  \caption[Performance]{Performance of model
    \emph{two\_ar3\_masked6.1}.  Records a09 and a18 are marked all
    apnea based on their high likelihood.}
  \label{tab:pass2performance}
\end{table*}

\section{To Do}
\label{sec:todo}

\begin{description}
\item[Support different observation models:] Identify the code for
  reading data to each file with a pickled model
\item[Multiple Chains:] Try HMM with more chains for apnea peaks.
\end{description}

\section{Unlucky Thoughts}
\label{sec:unlucky}

Here are some ideas that didn't work out\footnote{He's not stupid.
  He's just unlucky when he thinks.}:
\begin{description}
\item[Cumulative distribution of respiration] I looked at the CDF of
  the respiration signal.  I hoped to use it to resolve records in
  pass-1 that the low frequency PSD left on the border.
\item[Use ECG Likelihood] My idea was to use the likelihood of the ECG
  model to modify the variances of the heart rate models.  I didn't
  see any structure to exploit when I looked at plots of the log
  likelihood, $\log\left(P(y[t]|y[:t]) \right)$, of the heart rate
  model against the log likelihood of the ECG model.
\item[Multiple Chains] In model\_init.py I made chains with lengths
  numpy.arange(20, 40, 2) for both normal and apnea.  Initially the
  models had AR order 1.  After writing code to change the AR order, I
  found the log likelihood per time step varied radically between
  records.  Here is the last line from training an AR-4 model for 20
  iterations:
\begin{verbatim}
19 L0 -5.875e+03 L1  6.323e+03 L2  1.646e+03 L3  5.682e+03 L4  4.964e+03
L5 -1.093e+04 L6  3.797e+03 L7  1.924e+03 L8 -4.514e+03 L9  4.667e+03
L10 -1.755e+03 L11  5.678e+03 L prior -9.081e+06 U/n -4.1274e+01
\end{verbatim}
  I found that the weights for states in the chains were low and the
  variances were high, from (3.32607 0.99734) to (731.01504 0.79096).
  On the other hand, for the apnea with a transition to itself, the
  weight and variance was (130965.95661 0.04808).  The classification
  performance was bad.

  I think the sampling rate is too high.  Since the width of the
  low-pass filter for the heart rate data in utilities.py is
  \emph{low\_pass\_width=2 * numpy.pi / (15 * PINT('seconds'))},
  samples that are less than 15/2 seconds apart are redundant.  Forty
  samples per minute is 1.5 seconds per sample or 5 times 15/2.

  I will try changing the sample rate to 8 samples per minute and the
  chain lengths to (4, 8, 1).
\item[Disparate Likelihoods:] The command and last iteration of
  training ar3 are
\begin{verbatim}
python apnea_train.py --AR_order 3 --iterations 20 --records a01 a02 a03 a04 a05 a07 a12 a13 a14 a15 a17 a19 --type masked --trim_start 25 ../../../../build/derived_data/apnea/models/a_ar1_masked ../../../../build/derived_data/apnea/models/a_ar3_masked

19 L0 -6.009e+03 L1  5.868e+03 L2  7.069e+02 L3  4.772e+03 L4  4.285e+03 L5 -1.360e+04 L6  1.951e+03 L7 -2.085e+02 L8 -4.664e+03 L9  3.691e+03 L10 -2.632e+03 L11  7.025e+03 L prior -1.084e+06 U/n -4.9251e+00
\end{verbatim}
  which implies the following likelihoods for individual records:
  
  \begin{tabular*}{1.0\linewidth}{rr} 
    a01 & -6.009e+03 \\
    a02 & 5.868e+03 \\
    a03 & 7.069e+02 \\
    a04 & 4.772e+03 \\
    a05 & 4.285e+03 \\
    a07 & -1.360e+04 \\
    a12 & 1.951e+03 \\
    a13 & -2.085e+02 \\
    a14 & -4.664e+03 \\
    a15 & 3.691e+03 \\
    a17 & -2.632e+03 \\
    a19 & 7.025e+03
  \end{tabular*}
  Also from that training here is a summary of the probability of
  occupying key states averaged over all of the training data:
  
  \begin{tabular*}{1.0\linewidth}[c]{rrllll}
    index & name & weight &  variance & a/b &    alpha \\ \hline
  0 & N\_2\_0     &    2.74e+03  &  0.0449 &  0.2& 5.00e+01 \\
  2 & N\_3\_0     &    1.38e+03  &  0.0847 &  0.2& 5.00e+01 \\
  5 & N\_4\_0     &    1.55e+03  &   0.076 &  0.2& 5.00e+01 \\
  9 & N\_5\_0     &    481       &   0.572 &  0.2& 5.00e+01 \\
 14 & N\_6\_0     &    568       &   0.158 &  0.2& 5.00e+01 \\
 20 & N\_7\_0     &    967       &  0.0665 &  0.2& 5.00e+01 \\
 27 & N\_8\_0     &    3.91e+03  &  0.0272 &  0.2& 5.00e+01 \\
 35 & N\_noise   &    1.31e-18  &     100 &100.0& 9.60e+04 \\
 36 & N\_switch  &    1.17e+04  &  0.0635 &  0.2& 5.00e+01 \\
 37 & A\_2\_0     &    3.92e+03  &  0.0775 &  0.2& 5.00e+01 \\
 39 & A\_3\_0     &    4.11e+03  &  0.0319 &  0.2& 5.00e+01 \\
 42 & A\_4\_0     &    4.4e+03   &  0.0205 &  0.2& 5.00e+01 \\
 46 & A\_5\_0     &    2.71e+03  &  0.0605 &  0.2& 5.00e+01 \\
 51 & A\_6\_0     &    7.86e+03  &  0.0196 &  0.2& 5.00e+01 \\
 57 & A\_7\_0     &    1.9e+03   &  0.0973 &  0.2& 5.00e+01 \\
 64 & A\_8\_0     &    1.28e+03  &   0.585 &  0.2& 5.00e+01 \\
 72 & A\_noise   &    2.28e-17  &     100 &100.0& 9.60e+04 \\
 73 & A\_switch  &    2.63e+04  &   0.059 &  0.2& 5.00e+01
  \end{tabular*}
  I want code to make a similar table for isolated records.

\end{description}

\section{Miscellaneous}
\label{sec:misc}
\begin{verbatim}

develop-shell$ time make -j 8 ../../../../build/derived_data/apnea/models/unmasked_all_ac  
real	174m16.579s
user	2786m13.984s
sys	1m26.283s

develop-shell$ time make ../../../../build/derived_data/apnea/models/unmasked_all_ac
real	3m15.912s
user	20m57.698s
sys	0m13.803s

\end{verbatim}

\end{document}
http://www.ms.k.u-tokyo.ac.jp/sugi/2010/RIMS2010.pdf Density Ratio
Estimation: A Comprehensive Review

https://neuropsychology.github.io/NeuroKit/examples/ecg_delineate/ecg_delineate.html
Locate P, Q, S and T waves in ECG

https://docs.scipy.org/doc/scipy/reference/generated/scipy.datasets.electrocardiogram.html#scipy.datasets.electrocardiogram

http://compbio.fmph.uniba.sk/papers/04wabi.pdf The Most Probable
Labeling Problem in HMMs and Its Application to Bioinformatics

https://probml.github.io/pml-book/book2.html Probabilistic Machine
Learning: Advanced Topics by Kevin Patrick Murphy.  MIT Press, 2023.

https://archive.siam.org/books/textbooks/fr18_book.pdf Iterative
Methods for Optimization C.T. Kelly  (Saved to cathcart)
