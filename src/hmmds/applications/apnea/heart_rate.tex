
% Notes on modeling heart rate

\documentclass[12pt]{article}
\usepackage{graphicx,color}
\usepackage{amsmath, amsfonts}
\usepackage{placeins}

\newcommand{\field}[1]{\mathbb{#1}}
\newcommand{\INTEGER}{\field{Z}}
\newcommand{\REAL}{\field{R}}
\newcommand{\COMPLEX}{\field{C}}
\newcommand{\id}{\mathbb{I}}
\newcommand{\variance}[1]{\field{V}\left[ #1 \right]}
\newcommand{\normal}[2]{{\cal N}\left(#1,#2 \right)}
\newcommand{\argmax}{\operatorname*{argmax}}

\title{HMMs for Heart Rate}
\author{Andrew M.\ Fraser}

\begin{document}
\maketitle

\section{Introduction}
\label{sec:introduction}

Here I document ideas for redoing the last chapter of my book
\emph{Hidden Markov Models and Dynamical Systems} in which I address
the CINC-2000 challenge.  I want to map the products of the ECG
analysis described in my document \emph{HMMs for ECGs} to to the two
classification challenges.  The inputs are time series of heart rate
estimates and likelihoods sampled at 2 Hz and the outputs are
two stages of classification.  First: For each record determine
whether or not the patient experienced apnea; and Second: for each
minute of each record determine whether or not the patient was
experiencing apnea.

\section{A Model for Record a03}
\label{sec:a03}

\textbf{To Do:}  Describe structure of the model and its performance.

\subsection{Fit a Separate Models to Each Record}
\label{sec:selves}

\textbf{To Do:} Imitating the approach in \emph{HMMs for ECGs}, use
the same structure to build an HMM for each of the labeled records.

Then strip out the class portions of those models and see if it's
possible to use the cross entropies to do pass-1 classification.

\subsection{C records}
\label{sec:c_records}

Because the C records have only a few or zero minutes marked as apnea,
trying to create and train models with states for the apnea class
fails.  I build models of C records that have a single state and that
state is in the normal class.


\section{Exploiting Likelihood from ECG Models}
\label{sec:exploit_likelihood}

The ECG models provide both estimates of the heart rate and at time
series of the conditional probabilities $p(y[t]|y[:t])$.  I believe
that when those probabilities are small the estimated heart rate is
unreliable.  I intend to incorporate that belief into my models of
heart rate.

\textbf{To Do:} Using s

\section{Pass1 Challenges}
\label{sec:pass1}

\begin{description}
\item[Statistic 1] Spectral power in the range of low frequency apnea
        oscillations.  Channels 22 to 62.
\end{description}

Setting the threshold for statistic 1 between 0.34 and 0.355 classifies
all of the \emph{a} and \emph{c} records records correctly except a06
and c10.  These are the records for which statistic 1 is between 0.3
and 0.375:
\begin{description}
\item[a06] Amplitude of apnea related heart rate oscillations is
  small.  The hour to hour changes may be helpful.  Classifying the
  minutes looks hard.  With a threshold of 0.34 in statistic 1, the
  only records that get misclassified are a06 and c10.
\item[c10] Heart rate has apnea like oscillations but returns to same
  baseline.  During those oscillations the respiration signal also
  oscillates, but perhaps not as deeply as a10.
\item[a10] Heart rate changes hour to hour
\item[a09] Respiration signal is very weak.  Heart rate changes hour
  to hour but not as much as a10.  The amplitude of the oscillations
  in the heart rate during apnea is small.  It seems difficult to me.
\item[c02] Hour to hour heart rate changes.  But I bet that the
  troubling value of statistic 1 is due to spikes in the heart rate
  that could be masked by looking at negative spikes in the log
  likelihood.
\item[c09] Like c02.  Hour to hour heart rate changes.  But I bet that
  the troubling value of statistic 1 is due to spikes in the heart
  rate that could be masked by looking at negative spikes in the log
  likelihood.
\item[c07] Low hour to hour heart rate variation.
\item[x24]  Need to trim first 15 minutes.  Looks normal to me.
  Filter on log likelihood would help with statistic 1.
\item[x33 x34] These are different measurements of the same thing.
  It looks normal.  Perhaps filtering based on log likelihood would
  help a little with statistic 1.
\item[x13] I guess it's normal.  Filter on likelihood would help.
\item[x15] Apnea 120 - 140, 180 - 210, 350-390, 490-515.  More hour to
  hour variation than c07.
\item[x07] I see apnea oscillations.  Filtering on likelihood might
  lead to miss-classification.
\item[x17] Last 5 minutes need to be trimmed.  Apnea 126-150.
  Filtering by likelihood would not remove many heart rate spikes.
  ???
\item[x06] First 5.2 minutes need to be trimmed.  I guess normal.
  Filtering on likelihood would help.
\item[x10] Apnea 220 - 250, 360-390
\item[x18] Need to trim first 14.9 minutes.  206-230 looks like apnea
  oscillations, but respiration look sort of OK
\item[x23] Apnea 120-185 but respiration looks sort of OK
\end{description}

Two ideas come from the review of those records: 1. Use the power in
very low frequencies as an additional statistic; 2. Use negative
spikes in the log likelihood of the ECG signal to reduce the weight
given to heart rate estimates.  I've been thinking of using the log
likelihood of the ECG signal to influence the variance of the output
models for the HMMs I use for minute by minute apnea classification.
After I figure that out, I may be able to the technique implement item
2 above.


\section{To Do}
\label{sec:todo}

\begin{description}
\item[Pass-1] Use statistic\_plots.py.  Continue using three classes
  like the book.  Look at \textbf{b} and \emph{x} files when drawing
  the boundaries.  Explore changes in PSD estimation.  I think using
  longer FFTs improves classification.  Perhaps using more overlap
  would help too.
\item[move] apnea/models to heart\_rate/models in build directory
\item[c records have bad starts] 1,2,4,9 and 10 have several minutes
  with disconnected leads.
\item[Tied Observations] Multiple states with the same observation
  model
\item[Old Code] Consider removing or rewriting pass1.py, pass2.py, and
  score.py 
\end{description}

\section{Unlucky Thoughts}
\label{sec:unlucky}

Here are some ideas that didn't work out\footnote{He's not stupid.
  He's just unlucky when he thinks.}:
\begin{description}
\item[Cumulative distribution of respiration] I looked at the CDF of
  the respiration signal.  I hoped to use it to resolve records in
  pass-1 that the low frequency PSD left on the border.
\item[Use ECG Likelihood] My idea was to use the likelihood of the ECG
  model to modify the variances of the heart rate models.  I didn't
  see any structure to exploit when I looked at plots of the log
  likelihood, $\log\left(P(y[t]|y[:t]) \right)$, of the heart rate
  model against the log likelihood of the ECG model.
\item[Multiple Chains] In model\_init.py I made chains with lengths
  numpy.arange(20, 40, 2) for both normal and apnea.  Initially the
  models had AR order 1.  After writing code to change the AR order, I
  found the log likelihood per time step varied radically between
  records.  Here is the last line from training an AR-4 model for 20
  iterations:
\begin{verbatim}
19 L0 -5.875e+03 L1  6.323e+03 L2  1.646e+03 L3  5.682e+03 L4  4.964e+03
L5 -1.093e+04 L6  3.797e+03 L7  1.924e+03 L8 -4.514e+03 L9  4.667e+03
L10 -1.755e+03 L11  5.678e+03 L prior -9.081e+06 U/n -4.1274e+01
\end{verbatim}
  I found that the weights for states in the chains were low and the
  variances were high, from (3.32607 0.99734) to (731.01504 0.79096).
  On the other hand, for the apnea with a transition to itself, the
  weight and variance was (130965.95661 0.04808).  The classification
  performance was bad.

  I think the sampling rate is too high.  Since the width of the
  low-pass filter for the heart rate data in utilities.py is
  \emph{low\_pass\_width=2 * numpy.pi / (15 * PINT('seconds'))},
  samples that are less than 15/2 seconds apart are redundant.  Forty
  samples per minute is 1.5 seconds per sample or 5 times 15/2.

  I will try changing the sample rate to 8 samples per minute and the
  chain lengths to (4, 8, 1).

\end{description}

\section{Miscellaneous}
\label{sec:misc}
\begin{verbatim}

develop-shell$ time make -j 8 ../../../../build/derived_data/apnea/models/unmasked_all_ac  
real	174m16.579s
user	2786m13.984s
sys	1m26.283s

develop-shell$ time make ../../../../build/derived_data/apnea/models/unmasked_all_ac
real	3m15.912s
user	20m57.698s
sys	0m13.803s

\end{verbatim}

\begin{table}
  \centering
  \input{like_table.tex} 
  \caption[Log Likelihoods]{Log Likelihoods of models trained on one
    data set and evaluated on another}
  \label{tab:log_likelihoods}
\end{table}

\begin{figure}
  \centering
    \resizebox{1.0\textwidth}{!}{\includegraphics{statistics.pdf}}
    \caption{Possible pass1 statistics.}
  \label{fig:statistics}
\end{figure}

\end{document}
