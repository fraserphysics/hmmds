
% Notes on modeling heart rate

\documentclass[12pt]{article}
\usepackage{graphicx,color}
\usepackage{amsmath, amsfonts}
\usepackage{placeins}

\newcommand{\field}[1]{\mathbb{#1}}
\newcommand{\INTEGER}{\field{Z}}
\newcommand{\REAL}{\field{R}}
\newcommand{\COMPLEX}{\field{C}}
\newcommand{\id}{\mathbb{I}}
\newcommand{\variance}[1]{\field{V}\left[ #1 \right]}
\newcommand{\normal}[2]{{\cal N}\left(#1,#2 \right)}
\newcommand{\argmax}{\operatorname*{argmax}}

\title{HMMs for Heart Rate}
\author{Andrew M.\ Fraser}

\begin{document}
\maketitle

\section{Introduction}
\label{sec:introduction}

Here I document ideas for redoing the last chapter of my book
\emph{Hidden Markov Models and Dynamical Systems} in which I address
the CINC-2000 challenge.  I want to map the products of the ECG
analysis described in my document \emph{HMMs for ECGs} to to the two
classification challenges.  The inputs are time series of heart rate
estimates and likelihoods sampled at 2 Hz and the outputs are
two stages of classification.  First: For each record determine
whether or not the patient experienced apnea; and Second: for each
minute of each record determine whether or not the patient was
experiencing apnea.

\section{A Model for Record a03}
\label{sec:a03}

\textbf{To Do:}  Describe structure of the model and its performance.

\subsection{Fit a Separate Models to Each Record}
\label{sec:selves}

\textbf{To Do:} Imitating the approach in \emph{HMMs for ECGs}, use
the same structure to build an HMM for each of the labeled records.

Then strip out the class portions of those models and see if it's
possible to use the cross entropies to do pass-1 classification.

\subsection{C records}
\label{sec:c_records}

Because the C records have only a few or zero minutes marked as apnea,
trying to create and train models with states for the apnea class
fails.  I build models of C records that have a single state and that
state is in the normal class.


\section{Exploiting Likelihood from ECG Models}
\label{sec:exploit_likelihood}

The ECG models provide both estimates of the heart rate and at time
series of the conditional probabilities $p(y[t]|y[:t])$.  I believe
that when those probabilities are small the estimated heart rate is
unreliable.  I intend to incorporate that belief into my models of
heart rate.

\textbf{To Do:} Using simple AR-HMM models of the heart rate time
series, plot likelihood of heart rate data against likelihood of the
ecg data.

\textbf{To Do:}  Make an AR model of the joint (heart
rate, ecg likelihood) observations in which the variance of the heart
rate values is a function of the likelihood values.

\section{To Do}
\label{sec:todo}

\begin{description}
\item[move] apnea/models to heart\_rate/models in build dir
\item[Drop 25 minutes for pass 1] Do it consistently in the training
  of the models and the pass 1 classification.
\item[c records have bad starts] 1,2,4,9 and 10 have several minutes
  with disconnected leads.
\end{description}

\section{Miscellaneous}
\label{sec:misc}
\begin{verbatim}

develop-shell$ time make -j 8 ../../../../build/derived_data/apnea/models/unmasked_all_ac  
real	174m16.579s
user	2786m13.984s
sys	1m26.283s

develop-shell$ time make ../../../../build/derived_data/apnea/models/unmasked_all_ac
real	3m15.912s
user	20m57.698s
sys	0m13.803s

\end{verbatim}

\begin{table}
  \centering
  \input{like_table.tex} 
  \caption[Log Likelihoods]{Log Likelihoods of models trained on one
    data set and evaluated on another}
  \label{tab:log_likelihoods}
\end{table}

\begin{figure}
  \centering
    \resizebox{1.0\textwidth}{!}{\includegraphics{statistics.pdf}}
    \caption{Possible pass1 statistics.}
  \label{fig:statistics}
\end{figure}

\end{document}
