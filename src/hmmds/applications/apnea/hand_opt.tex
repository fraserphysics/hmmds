
% Notes on modeling heart rate

\documentclass[12pt]{article}
\usepackage{graphicx,color}
\usepackage{amsmath, amsfonts}
\usepackage{placeins}
\usepackage{verbatim}
\usepackage{showlabels}

\newcommand{\field}[1]{\mathbb{#1}}
\newcommand{\INTEGER}{\field{Z}}
\newcommand{\REAL}{\field{R}}
\newcommand{\COMPLEX}{\field{C}}
\newcommand{\id}{\mathbb{I}}
\newcommand{\variance}[1]{\field{V}\left[ #1 \right]}
\newcommand{\normal}[2]{{\cal N}\left(#1,#2 \right)}
\newcommand{\argmax}{\operatorname*{argmax}}
\newcommand{\BestModel}{\emph{two\_ar5\_masked}}

\title{Hand Optimization}
\author{Andrew M.\ Fraser}

\begin{document}
\maketitle

\newcommand{\StudyCaption}[2]{
  \caption{Performance vs #1 on the training records (\emph{a} \emph{b}
    and \emph{c} records excluding defective records).  The models
    have #2.}
}

\newcommand{\StudyFigs}[2]{

\begin{figure}
  \centering
  \resizebox{0.8\textwidth}{!}{\includegraphics{#1_threshold.pdf}}
  \StudyCaption{detection threshold}{#2}
  \label{fig:#1_threshold}
\end{figure}

\begin{figure}
  \centering
    \resizebox{0.8\textwidth}{!}{\includegraphics{#1_errors_vs_ar.pdf}}
    \StudyCaption{autoregressive order}{#2}
  \label{fig:#1_ar_order_study}
\end{figure}

\begin{figure}
  \centering
    \resizebox{0.8\textwidth}{!}{\includegraphics{#1_errors_vs_fs.pdf}}
    \StudyCaption{heart rate sample frequency}{#2}
  \label{fig:#1_sample_frequency_study}
\end{figure}

\begin{figure}
  \centering
    \resizebox{0.8\textwidth}{!}{\includegraphics{#1_errors_vs_lpp.pdf}}
    \StudyCaption{cutoff period in seconds of low pass filter for heart rate}{#2}
  \label{fig:#1_lpp_study}
\end{figure}

\begin{figure}
  \centering
    \resizebox{0.8\textwidth}{!}{\includegraphics{#1_errors_vs_rc.pdf}}
    \StudyCaption{center frequency in cpm for extracting respiration signal}{#2}
  \label{fig:#1_rc_study}
\end{figure}

\begin{figure}
  \centering
    \resizebox{0.8\textwidth}{!}{\includegraphics{#1_errors_vs_rw.pdf}}
    \StudyCaption{width in cpm of filter for extracting respiration signal}{#2}
  \label{fig:#1_rw_study}
\end{figure}

\begin{figure}
  \centering
    \resizebox{0.8\textwidth}{!}{\includegraphics{#1_errors_vs_rs.pdf}}
    \StudyCaption{cutoff frequency in cpm for smoothing the
      respiration signal}{#2}
  \label{fig:#1_rs_study}
\end{figure}

\begin{table*}
  \centering
  \input{#1_score.tex}
  \caption[Performance]{Performance of a model that has #2.}
  \label{tab:#1_score}
\end{table*}}

\section{Combining Respiration and Heart Rate}
\label{sec:combination}

Here I combine a vector heart rate and respiration signal with a
multiple state state structure.  The parameters to choose include:
\begin{description}
\item[Parameters:] for the (Vector Autoregressive Gaussian) model
  of the 2-d vectors with components respiration and low-pass filtered
  heart rate.  The parameters include:
  \begin{description}
  \item[AR order] 8 % MULTI_AR
  \item[sample rate] 4 % MULTI_FS
  \item[low pass period] 65 seconds % MULTI_LPP
  \item[respiration pass center] 12 per minute % MULTI_RC
  \item[respiration pass width] 3.6 per minute % MULTI_RW
  \item[envelope smooth] .47 per minute % MULTI_RS
  \item[$\psi$ and $\nu$] Priors for each state
  \end{description}
\item[Detector threshold] 0.7 % MULTI_THRESHOLD
\end{description}

I chose the parameter values in foregoing list to get the best
performance\footnote{AR order 10 reduces the error count from 2,270 to
  2,260.  I chose AR order 8 to reduce over fitting.} on the training
data.

In the following figures I've plotted the number of classification
errors on the training data against each of those parameters.  The
training records are the \emph{a} records and the \emph{b} and
\emph{c} except \emph{b05, c04,} and \emph{c06} which are
defective\footnote{The ECG for b05 is not usable.  c04 has an
  arrhythmia in which the interval between beats is very short as
  frequently as every third beat.  And c06 is a shifted version of
  c05.}.

\StudyFigs{multi}{multiple states and the observation models are
  vector autoregressive models for heart rate and respiration}

\begin{table*}
  \centering
  \input{test_score.tex}
  \caption[Performance]{Performance on the test data.}
  \label{tab:test_score}
\end{table*}

\end{document}

#1_threshold.pdf
#1_errors_vs_ar.pdf
#1_errors_vs_lpp.pdf
#1_errors_vs_rc.pdf
#1_errors_vs_rw.pdf
#1_errors_vs_rs.pdf
#1_score.tex

In the book, I got to error rate of .1176 on the training data (1952
errors) and .1305 (2253 errors) on the test data.  Cheating by peaking
at the expert data to set thresholds I got 1410 errors with the v4s
models.

Performance of multi models with cheating:

       A->N   N->A  N_error
       3A 2N  787   705   1492
       3A 3N  651   638   1289  (changed psi & nu)

Performance without cheating

2024-03-17: 3 normal states, 3 apnea states, threshold based on Pass1,
log_likelihoods of 6 state model with classes trained on APLUSNAMES
and c_model.  2094 errors on all training data.

2024-03-18: Improve on yesterday by using models low_classless and
high_classless instead of c_model and \$(MULTI_BEST) for threshold
calculations.  1940 errors on all training data.

2024-03-20: Finished attempt to use one model for each training
record.  I'd hoped to use likelihood's to define nearest neighbors and
then use the neighbor's model to classify a record's data.  That would
have generalized to the x-files.  It didn't work well at all.

2024-03-21: Focus on parameter of models \$(MODELS)/high_classless and
\$(MODELS)/low_classless that give statistics for the threshold
function.  1805 errors on all training data.  N_minutes = 6457 + 10155
= 16612.  Error fraction is 1805/16612 = 0.109.

2024-03-23: Found deadline for submission to SciPy-24 was Feb 27 and
that meeting will be in Tacoma.  I won't go.  I'm ready to move beyond
tinkering with models for the apnea data.  I wrote to PhysioNet asking
how to score classification of testing data.

2024-04-29 Changed training set from: APLUSNAMES =
$(ANAMES) b01 b02 b03 b04 c08 c10 to: APLUSNAMES = $(ANAMES) b01 b02
c02 c09 c10 because the derived heart rates for c08, b03, and b04 have
unsatisfactory spikes.  After modifying the training set performance
in Table 4 (multi state with cheat threshold) dropped to 1424, 0.09,
and the values in Table 5 (uses the best m and b values) were 2337
0.14.  The optimum parameter values stayed the same

Plan:

1. Review x-files and consider modifying QRS code to address failures.

2. View x-files and estimate best threshold for each.

3. Create submission file based on thresholds from code.

4. Create submission file based on thresholds from viewing data.

5. If possible score test data using cheat best threshold for each
file.

6. Combine heart_rate.tex and hand_opt.tex into document that reports
on interesting or valuable lessons from my work on the apnea data.

7. Rewrite last chapter of the book.

8. Revise software including build of book.

9. Start asking for help.

Here are notes on the apnea data:

x-files with untrimmed beginnings: 01, 04, 06, 18, 22, 24, 26, 28, 29,
35

Guesses about which x-files are ``Normal'' (like c-files) based on:

Pass1   Visual 64   128 1024 2048 4096
                    x03            x03
x04     x04    x04  x04  x04  x04  x04
x06     x06
x11     x11              x11  x11  x11
                                   x15
x17     x17              x17
                    x18  x18  x18
x22     x22    x22  x22  x22  x22  x22
x24     x24              x24  x24  x24
x29     x29    x29       x29  x29  x29
x33     x33    x33  x33  x33  x33  x33
x34     x34    x34  x34  x34  x34  x34
x35     x35    x35  x35  x35  x35  x35

x11 has atypical beats, but my QRS code tracks and provides OK heart
rate

x29 has alternans.  I think it's a ``Normal'' record

x26 Has atypical beats and my QRS code occasionally also fails for
normal beats.  I think that the atypical beats queers the training and
leads to the failures on normal beats.

Here are the thresholds to use for the x-files (first column from
python printfit.py default threshold_statistics.pkl ):

Record  Fit    Visual  QRS OK?
x01     0.42    -1.7   yes
x02     2.19     1.0   yes
x03     1.15     1.1   yes
x04     2.81     3.4   extra beats
x05     0.55     0.0   yes
x06     0.94     2.8   yes
x07    -1.64    -1.0   yes
x08     0.92    -0.4   yes
x09    -2.00     0.3   yes
x10    -0.26     0.0   yes    
x11    -1.39           Arrhythmia ?PVC distorts resp
x12     1.12     0.4   yes
x13    -1.66    -0.3   Arrhythmia, negative ecg spikes
x14    -1.92    -0.7   yes
x15     0.49    -0.6   yes
x16    -0.06     0.2   yes
x17     1.48           yes  c?
x18     0.95    -2.3   yes
x19     3.73     0.4   yes Some arrhythmia, but ok
x20     0.19    -0.7   yes
x21     0.10           yes
x22     0.22           yes
x23    -1.48           yes
x24     3.75           yes
x25     0.04           yes
x26    -1.46     0.0   arrhythmia and extra beats
x27    -0.68           yes
x28    -0.86           yes
x29    10.66           yes But serious arrhythima
x30    -0.58     0.8   yes
x31     5.00           yes
x32     2.08           yes
x33     0.19           yes
x34     0.22           yes
x35     0.08           yes

X33 and x34 are shifted measurements of the same event, as are c05 and
c06.

x04 OK after about t=88

x11 QRS-HMM is OK.  Easy to see PVC peaks in ECG.  At coarse time
scale, I can't separate PVC peaks from low frequency excursions.
Solution: Mask positive peaks 3\%, low likelihood 1\%, and long
inter-beat periods (hr=55bpm)

x13 QRS-HMM is OK.  Easy to see PVC valleys in ECG.  Solution: Mask on
negative peaks 5\% and low likelihood 1\%.

x19 QRS-HMM is OK, but Arrhythmia may cause high fit threshold that
throws off classification.  Solution: Mask on low likelihood 10\% and
set threshold to 1.2

x26 QRS-HMM inserts extra beats.  Arrhythmia missing R.  Solution:
Custom alpha-beta variance prior.  Mask on negative peaks 3\%, low
likelihood 3\%, and long inter-beat periods (48 bpm).

x29 QRS-HMM is OK.  Episodes of arrhythmia with frequent premature
beats.  Sometimes every 5th beat is premature.  Solution: Mark this
record as normal and don't fix the derived signals.


python fit2threshold.py ../../../../build/derived_data/apnea/models/multi_ar12fs4lpp65rc13rw3.0rs.455_masked threshold_statistics.pkl --records a01 a02 a03 a04 a05 a06 a07 a08 a09 a10 a11 a12 a13 a14 a15 a16 a17 a18 a19 a20 b01 b02 b03 b04 c08 c10 --mb 0.927 0.0305 --fig_path fit2threshold.pdf
errors=1841
a01 -0.00  0.82 -0.82
a02 -1.80 -1.05 -0.75
a03  2.40  2.73 -0.34
a04 -0.60 -0.67  0.06
a05 -0.60 -1.09  0.49
a06 -1.20 -0.63 -0.57
a07  1.20  0.57  0.63
a08  1.20  0.03  1.17
a09 -2.40 -1.58 -0.82
a10 -0.00 -0.75  0.75
a11 -2.40 -0.61 -1.79
a12 -1.20 -1.23  0.03
a13 -0.00 -0.02  0.02
a14 -0.00  1.72 -1.72
a15 -0.60 -0.73  0.13
a16  0.60  0.26  0.34
a17 -0.00 -0.83  0.83
a18 -2.40 -0.61 -1.79
a19 -0.00  0.35 -0.35
a20 -1.20 -0.79 -0.41
b01  2.40  0.18  2.22
b02  1.80  0.08  1.72
b03  1.20  0.81  0.39
b04  2.40  3.26 -0.86
c08  2.40  3.17 -0.77
c10  2.40  0.34  2.06
     best  fit   difference

2024-06-15 Start on HMM called ``threshold'' for estimating thresholds
delta threshold = .12 n_states = 42

errors=1294
a01 -1.00 -0.31 -0.69
a02 -2.33 -2.72  0.39 [...]

2024-06-21 But with 42 degrees of freedom, one can fit 26 thresholds
exactly.

Changes: (1) Limit rank of SVD to 8; (2) VARG covariance the same for
all states except noise;  (3) Use 100 steps between -3 and +3 for
calculating the best thresholds.

Makefile says --power_dict hr_respiration .25 threshold 4

Rank = 8, delta mu = .2, variance = .49 n_states = 26
errors=1926
    best  fit   difference  in qxfit2threshold
a01 -0.03 -0.62  0.59
a02 -2.21 -1.45 -0.76 [...]

Why is minimum of fit thresholds -1.45?

delta mu = .2, variance = 0.04, n_states = 26
***Training failed***

delta mu = .2, variance = 0.09, n_states = 26
errors=2120
    best  fit   difference  in qxfit2threshold
a01 -0.03  0.19 -0.22
a02 -2.21  0.11 -2.32

delta mu = .1, variance = 0.09, n_states = 50
***Training failed***

delta mu = .1, variance = 0.25, n_states = 50, rank = 8
errors=2038
    best  fit   difference  in qxfit2threshold
a01 -0.03 -0.83  0.80
a02 -2.21 -0.33 -1.89

delta mu = .1, variance = 1.0, n_states = 50, rank = 8
errors=2136
    best  fit   difference  in qxfit2threshold
a01 -0.03 -0.69  0.66
a02 -2.21 -1.08 -1.13
minimum of fit is -1.08 occurs for 8 records

delta mu = .05, variance = 1.0, n_states = 98, rank = 9
errors=2047
    best  fit   difference  in qxfit2threshold
a01 -0.03 -0.03 -0.00
a02 -2.21 -1.45 -0.76
minimum of fit is -1.45 occurs for 7 records

2024-07-23 Errors on test data: 3704 21\%.  Not satisfactory given the
complexity of the effort.  Next, I will try to get close to 80\% with
a simple approach.
