
% Notes on modeling heart rate

\documentclass[12pt]{article}
\usepackage{graphicx,color}
\usepackage{amsmath, amsfonts}
\usepackage{placeins}
\usepackage{verbatim}
\usepackage{showlabels}

\newcommand{\field}[1]{\mathbb{#1}}
\newcommand{\INTEGER}{\field{Z}}
\newcommand{\REAL}{\field{R}}
\newcommand{\COMPLEX}{\field{C}}
\newcommand{\id}{\mathbb{I}}
\newcommand{\variance}[1]{\field{V}\left[ #1 \right]}
\newcommand{\normal}[2]{{\cal N}\left(#1,#2 \right)}
\newcommand{\argmax}{\operatorname*{argmax}}
\newcommand{\BestModel}{\emph{two\_ar5\_masked}}

\title{Hand Optimization}
\author{Andrew M.\ Fraser}

\begin{document}
\maketitle

\newcommand{\StudyCaption}[2]{
  \caption{Performance vs #1 on the \emph{a} records combined with and
    \emph{b} and \emph{c} records that are close to the pass-1
    boundary appear in the upper plot. Thresholds that get the right
    number of apnea minutes appear in the lower plot.  The models have
    #2.}
  }
  \newcommand{\StudyFigs}[2]{

\begin{figure}
  \centering
  \resizebox{0.8\textwidth}{!}{\includegraphics{#1_threshold.pdf}}
  \StudyCaption{detection threshold}{#2}
  \label{fig:#1_threshold}
\end{figure}

\begin{figure}
  \centering
    \resizebox{0.8\textwidth}{!}{\includegraphics{#1_errors_vs_pt.pdf}}
    \StudyCaption{prominence threshold}{#2}
  \label{fig:#1_prominence_study}
\end{figure}

\begin{figure}
  \centering
    \resizebox{0.8\textwidth}{!}{\includegraphics{#1_errors_vs_ip.pdf}}
    \StudyCaption{exponential weight of interval component}{#2}
  \label{fig:#1_interval_weight_study}
\end{figure}

\begin{figure}
  \centering
    \resizebox{0.8\textwidth}{!}{\includegraphics{#1_errors_vs_ar.pdf}}
    \StudyCaption{autoregressive order}{#2}
  \label{fig:#1_ar_order_study}
\end{figure}

\begin{figure}
  \centering
    \resizebox{0.8\textwidth}{!}{\includegraphics{#1_errors_vs_fs.pdf}}
    \StudyCaption{heart rate sample frequency}{#2}
  \label{fig:#1_sample_frequency_study}
\end{figure}

\begin{figure}
  \centering
    \resizebox{0.8\textwidth}{!}{\includegraphics{#1_errors_vs_lpp.pdf}}
    \StudyCaption{cutoff period in seconds of low pass filter for heart rate}{#2}
  \label{fig:#1_lpp_study}
\end{figure}

\begin{figure}
  \centering
    \resizebox{0.8\textwidth}{!}{\includegraphics{#1_errors_vs_rc.pdf}}
    \StudyCaption{center frequency in cpm for extracting respiration signal}{#2}
  \label{fig:#1_rc_study}
\end{figure}

\begin{figure}
  \centering
    \resizebox{0.8\textwidth}{!}{\includegraphics{#1_errors_vs_rw.pdf}}
    \StudyCaption{width in cpm of filter for extracting respiration signal}{#2}
  \label{fig:#1_rw_study}
\end{figure}

\begin{figure}
  \centering
    \resizebox{0.8\textwidth}{!}{\includegraphics{#1_errors_vs_rs.pdf}}
    \StudyCaption{cutoff frequency in cpm for smoothing the
      respiration signal}{#2}
  \label{fig:#1_rs_study}
\end{figure}

\begin{table*}
  \centering
  \input{#1_score.tex}
  \caption[Performance]{Performance of a
    model that has #2.}
  \label{tab:#1_score}
\end{table*}}

\section{Combining Intervals, Respiration and Heart Rate}
\label{sec:combination}

Next, I combine a vector heart rate and respiration signal with a four
state state structure and interval modeling.  The parameters to choose
include:
\begin{description}
\item[VARG parameters:] for the (Vector Autoregressive Gaussian) model
  of the 2-d vectors with components respiration and low-pass filtered
  heart rate.  The parameters include:
  \begin{description}
  \item[AR order] % ar
  \item[sample rate] % fs
  \item[low pass width] 7.5 per minute % lpw
  \item[respiration pass center] 15 per minute % rc
  \item[respiration pass width] 3 per minute % rw
  \item[envelope smooth] 1.5 per minute % es
  \item[normalization] 
  \item[$\psi$ and $\nu$] Priors for each state
  \end{description}
\item[Peak parameters] For finding peaks.  Note: The signal used is
  from the low pass filter for the VARG signal.
  \begin{description}
  \item[minimum prominence] 
  \item[distance] 0.417 minutes
  \item[window length] 1.42 minutes
  \end{description}
\item[Detector threshold] 
\item[Weighting power] 
\end{description}

\StudyFigs{v4s}{four states and the observation models combine
  the likelihood of the interval between peaks with a vector
  autoregressive model for heart rate and respiration}

\clearpage % flush floats
\section{Varying the Detector Threshold}
\label{sec:threshold}

Notice that in table~\ref{tab:v4s_score} only three records have 30\%
of the classification errors.  Figures \ref{fig:v4s_threshold_a01}
to\ref{fig:v4s_threshold_b02} show the dependence of the number of
errors on the detector threshold, and the best threshold varies from
about 0.13 to 3,500.  In the next section, \ref{sec:psd_threshold}, I
describe a technique for customizing the detector threshold for each record.

\newcommand{\ThresholdStudy}[2]{
  \begin{figure}
    \centering
    \resizebox{0.8\textwidth}{!}{\includegraphics{v4s_threshold_#1.pdf}}
    \caption{The number of classification errors as a function
      detector threshold for record #1 appears in the upper plot.  The
      best threshold for record #1 is about #2.}
    \label{fig:v4s_threshold_#1}
  \end{figure}
}

\ThresholdStudy{a01}{.16}
\ThresholdStudy{a03}{100}
\ThresholdStudy{a07}{3.0}
\ThresholdStudy{a10}{2,000}
\ThresholdStudy{a11}{1,300}
\ThresholdStudy{a12}{0.13}
\ThresholdStudy{a19}{2,000}
\ThresholdStudy{b02}{3,500}

\clearpage
\subsection{Custom Thresholds}
\label{sec:psd_threshold}

I found Fourier Power Spectral Density estimates (PSDs) useful for
classifying whole records for Pass1.  I've used PSDs again for
customizing detector thresholds.  My PSD estimates are vectors with
2,149 components.  I use the following 1-d statistic to set
thresholds:
\begin{equation}
  \label{eq:z_sum}
  z_{\text{sum}}(\text{record}) \equiv \sum_{i=0}^{2048}
  \frac{\text{PSD}_i(\text{record}) - \mu_i}{\sigma_i}
\end{equation}
Figure~\ref{fig:shift_threshold} illustrates a four parameter function
that maps PSDs to thresholds, and classification results using that
function appear in Table~\ref{tab:abcd_score}.

\begin{figure}
  \centering
  \resizebox{0.5\textwidth}{!}{\includegraphics{shift_threshold.pdf}}
  \caption[A function of PSDs for varying detector thresholds.]{The
    axes of the upper plot are the function of a PSD defined in
    \eqref{eq:z_sum} and the detector threshold.  The markers for each
    record are placed at the threshold that minimizes the number of
    classification errors for that record.  The piecewise affine line
    is a four parameter function that maps
    $z_{\text{sum}}: \mapsto \text{threshold}$.  The parameters have
    been optimized to minimize the classification errors over all of a
    set of records.  The four lower plots illustrate the dependence of
    the total number of errors on each of the four parameters.  }
  \label{fig:shift_threshold}
\end{figure}

\begin{table*}
  \centering
  \input{abcd_score.tex}
  \caption[Performance]{Classification performance when thresholds are
  functions of PSDs}
  \label{tab:abcd_score}
\end{table*}

\clearpage

\subsection{Further Optimization}
\label{sec:further}

I am pleased with the improvement from 18\%
(Table~\ref{tab_v4s_score}) to \%14 (Table~\ref{tab:shift_threshold}).
It's close to the error rate of 11.8\% that the models I describe in
the first edition of the book obtained.  I expected that I could get
some more improvement by revisiting parameter choices using threshold
function \ref{fig:shift_threshold}, but when I used the threshold
function, the optimal parameter values stayed the same.  See
Fig.~\ref{fig:abcd}

\begin{figure}
  \centering
  \resizebox{0.4\textwidth}{!}{\includegraphics{abcd_ip.pdf}}
  \resizebox{0.4\textwidth}{!}{\includegraphics{abcd_pt.pdf}}\\
  \resizebox{0.4\textwidth}{!}{\includegraphics{abcd_ar.pdf}}
  \resizebox{0.4\textwidth}{!}{\includegraphics{abcd_fs.pdf}}\\
  \resizebox{0.4\textwidth}{!}{\includegraphics{abcd_lpp.pdf}}
  \resizebox{0.4\textwidth}{!}{\includegraphics{abcd_rc.pdf}}\\
  \resizebox{0.4\textwidth}{!}{\includegraphics{abcd_rw.pdf}}
  \resizebox{0.4\textwidth}{!}{\includegraphics{abcd_rs.pdf}}
  \caption[Parameter sensitivity with variable thresholds.]{Survey of
    8 parameters.}
  \label{fig:abcd}
\end{figure}


\end{document}
