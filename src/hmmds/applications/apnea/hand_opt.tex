
% Notes on modeling heart rate

\documentclass[12pt]{article}
\usepackage{graphicx,color}
\usepackage{amsmath, amsfonts}
\usepackage{placeins}
\usepackage{verbatim}
\usepackage{showlabels}

\newcommand{\field}[1]{\mathbb{#1}}
\newcommand{\INTEGER}{\field{Z}}
\newcommand{\REAL}{\field{R}}
\newcommand{\COMPLEX}{\field{C}}
\newcommand{\id}{\mathbb{I}}
\newcommand{\variance}[1]{\field{V}\left[ #1 \right]}
\newcommand{\normal}[2]{{\cal N}\left(#1,#2 \right)}
\newcommand{\argmax}{\operatorname*{argmax}}
\newcommand{\BestModel}{\emph{two\_ar5\_masked}}

\title{Hand Optimization}
\author{Andrew M.\ Fraser}

\begin{document}
\maketitle

\newcommand{\StudyCaption}[2]{
  \caption{Performance vs #1 on the \emph{a} records combined with and \emph{b} and
    \emph{c} records that are close to the pass-1 boundary. The models
    have #2.}
  }
  \newcommand{\StudyFigs}[2]{

\begin{figure}
  \centering
  \resizebox{0.8\textwidth}{!}{\includegraphics{#1_threshold.pdf}}
  \StudyCaption{detection threshold}{#2}
  \label{fig:#1_threshold}
\end{figure}

\begin{figure}
  \centering
    \resizebox{0.8\textwidth}{!}{\includegraphics{#1_errors_vs_pt.pdf}}
    \StudyCaption{prominence threshold}{#2}
  \label{fig:#1_prominence_study}
\end{figure}

\begin{figure}
  \centering
    \resizebox{0.8\textwidth}{!}{\includegraphics{#1_errors_vs_ip.pdf}}
    \StudyCaption{exponential weight of interval component}{#2}
  \label{fig:#1_interval_weight_study}
\end{figure}

\begin{figure}
  \centering
    \resizebox{0.8\textwidth}{!}{\includegraphics{#1_errors_vs_ar.pdf}}
    \StudyCaption{autoregressive order}{#2}
  \label{fig:#1_ar_order_study}
\end{figure}

\begin{figure}
  \centering
    \resizebox{0.8\textwidth}{!}{\includegraphics{#1_errors_vs_fs.pdf}}
    \StudyCaption{heart rate sample frequency}{#2}
  \label{fig:#1_sample_frequency_study}
\end{figure}

\begin{figure}
  \centering
    \resizebox{0.8\textwidth}{!}{\includegraphics{#1_errors_vs_lpp.pdf}}
    \StudyCaption{cutoff period in seconds of low pass filter for heart rate}{#2}
  \label{fig:#1_lpp_study}
\end{figure}

\begin{figure}
  \centering
    \resizebox{0.8\textwidth}{!}{\includegraphics{#1_errors_vs_rc.pdf}}
    \StudyCaption{center frequency in cpm for extracting respiration signal}{#2}
  \label{fig:#1_rc_study}
\end{figure}

\begin{figure}
  \centering
    \resizebox{0.8\textwidth}{!}{\includegraphics{#1_errors_vs_rw.pdf}}
    \StudyCaption{width in cpm of filter for extracting respiration signal}{#2}
  \label{fig:#1_rw_study}
\end{figure}

\begin{figure}
  \centering
    \resizebox{0.8\textwidth}{!}{\includegraphics{#1_errors_vs_rs.pdf}}
    \StudyCaption{cutoff frequency in cpm for smoothing the
      respiration signal}{#2}
  \label{fig:#1_rs_study}
\end{figure}

\begin{table*}
  \centering
  \input{#1_score.tex}
  \caption[Performance]{Performance of a
    model that has #2.}
  \label{tab:#1_score}
\end{table*}}

\subsection{Combining Intervals, Respiration and Heart Rate}
\label{sec:combination}

Next, I combine a vector heart rate and respiration signal with a four
state state structure and interval modeling.  The parameters to choose
include:
\begin{description}
\item[VARG parameters:] for the (Vector AutoRegressive Gaussian) model
  of the 2-d vectors with components respiration and low-pass filtered
  heart rate.  The parameters include:
  \begin{description}
  \item[AR order] % ar
  \item[sample rate] % fs
  \item[low pass width] 7.5 per minute % lpw
  \item[respiration pass center] 15 per minute % rc
  \item[respiration pass width] 3 per minute % rw
  \item[envelope smooth] 1.5 per minute % es
  \item[normalization] 
  \item[$\psi$ and $\nu$] Priors for each state
  \end{description}
\item[Peak parameters] For finding peaks.  Note: The signal used is
  from the low pass filter for the VARG signal.
  \begin{description}
  \item[minimum prominence] 
  \item[distance] 0.417 minutes
  \item[window length] 1.42 minutes
  \end{description}
\item[Detector threshold] 
\item[Weighting power] 
\end{description}

\StudyFigs{v4s}{four states and the observation models combine
  the likelihood of the interval between peaks with a vector
  autoregressive model for heart rate and respiration}

\end{document}
