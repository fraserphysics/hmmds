% The presentation for 2025-05-11 at the SIAM DS meeting
\documentclass[dvipsnames]{beamer}
\setbeamertemplate{navigation symbols}{} %no nav symbols
\usepackage{graphicx,xcolor}
\usepackage{amsmath, amsfonts}% ToDo: compatible w/siammathtime.sty?
\usepackage{amsthm}% Note: Conflicts with newsiambook
\usepackage{xspace}
\usepackage{bm} % Bold Math
\usepackage{rotating} % for sidewaysfigure
\usepackage{afterpage}
\usepackage{booktabs}       % for nicer looking tables
\usepackage{dcolumn}        % decimal point aligned columns
\renewcommand{\th}{^{\text th}}
\newcommand{\field}[1]{\mathbb{#1}}
\newcommand{\INTEGER}{\field{Z}}
\newcommand{\REAL}{\field{R}}
\newcommand{\COMPLEX}{\field{C}}
\newcommand{\EV}{\field{E}}
\newcommand{\y}{\mathbf{y}}
\newcommand{\x}{\mathbf{x}}
\newcommand{\s}{{\bf s}}
\newcommand{\bS}{{\bf S}}
\newcommand{\Y}{{\bf Y}}
\newcommand{\Tsamp}{\tau_s }
\newcommand{\ColorComment}[3]{}
\newcommand{\argmin}{\operatorname*{argmin}}
\newcommand{\argmax}{\operatorname*{argmax}}
\newcommand{\Normal}{{\mathcal{N}}}
\newcommand{\NormalE}[3]{{\mathcal{N}}\left.\left(#1,#2\right)\right|_{#3}}
\newcommand{\transpose}{^\top}
\newcommand{\ceil}[1]{\lceil#1\rceil}
\newcommand{\bceil}[1]{\left\lceil#1\right\rceil}
\newcommand{\floor}[1]{\lfloor#1\rfloor}
\newcommand{\bfloor}[1]{\left\lfloor#1\right\rfloor}
\newcommand{\states}{{\cal S}}
\newcommand{\outputs}{{\cal Y}}
\newcommand{\State}{S}
\newcommand{\Output}{Y}
\newcommand{\parameters}{\theta}
\newcommand{\parametersPrime}{\theta'}% for EM1.gpt

\newcommand{\ti}[2]{{#1}{(#2)}}                  % Time Index
%%% \newcommand{\ts}[3]{{#1}{(t\=#2,\ldots,#3)}}         % Time Sequence
\newcommand{\ts}[3]{#1_{#2}^{#3}}                    % Time Sequence
%%% \newcommand{\ts}[3]{\left\{ #1(l) \right\}_{l=#2}^{#3}}  % Time Sequence
\newcommand{\id}{{\bf I}}
\newcommand{\ie}{i.e.\xspace}
\newcommand{\eg}{e.g.\xspace}
\newcommand{\etal}{et al.\xspace}
\newcommand{\iid}{i.~i.~d.\xspace}
\newcommand{\apost}{\emph{a posteriori}\xspace}
\newcommand{\apri}{\emph{a priori}\xspace}
\newcommand{\plotsize}{\small}
\newcommand{\mlabel}[1]{\label{#1}}
\newcommand{\EMmap}{{\mathcal T}} %

\newcommand{\Green}[1]{{\color{ForestGreen}#1}}
\newcommand{\Red}[1]{{\color{red}#1}}
\newcommand{\Blue}[1]{{\color{blue}#1}}
\newcommand{\given}{~|~}


\title{Estimating Dynamics and Data Assimilation to Detect Obstructive
Sleep Apnea}

\author{Andrew M.\ Fraser}
\date{2025-5-11}
\institute{SIAM DS25}

\usetheme{default}
\usefonttheme[]{serif}
\begin{document}
\frame{\titlepage}

%\frame{\frametitle{Outline}\tableofcontents}

\section{Computers IN Cardiology 2000}%
\label{sec:apnea}

\frame{ \frametitle{Goal of CINC 2000: Use ECG to Detect Apnea}
  Computers in Cardiology 2000 Challenge: Classify EKG
  \begin{columns}[c]
    \column{0.45\textwidth}%
    \begin{center}
    Normal\\
    \resizebox{\textwidth}{!}{\includegraphics{a03erN.pdf}}
    \end{center}
    \column{0.45\textwidth}%
    \begin{center}
      Apnea\\
      \resizebox{\textwidth}{!}{\includegraphics{a03erA.pdf}}
    \end{center}
  \end{columns}
  Intermediate Objectives: \\Detect QRS Pattern $\rightarrow$ Estimate
  Heart Rate
}

\frame{\frametitle{See Apnea in Heart Rate}
  \resizebox{0.9\textwidth}{!}{\includegraphics{a03HR.pdf}}
}

\frame{\frametitle{Respiration in Spectrogram}
  \resizebox{0.9\textwidth}{!}{\includegraphics{sgram.jpg}}
  
  James McNames saw respiration between 10 and 20 cpm.
}

\frame{\frametitle{Different Strokes}
  \resizebox{0.6\textwidth}{!}{\includegraphics{a03a10b03c02.pdf}}
  
  The waveforms differ between records.
}

\frame{\frametitle{Unvarying PQRST Duration}
  \resizebox{\textwidth}{!}{\includegraphics{constant_a03.pdf}}

  At different heart rates the shape and duration of the PQRST pattern
  doesn't change.  Only the delay between the sequences changes.
}

\frame{\frametitle{HMM State Structure}
  \begin{center}
    \resizebox{0.5\textwidth}{!}{\input{ecg_hmm.pdf_t}}
  \end{center}
  Code fits deterministic chain of 49 states to unvarying PQRST data.
  Variable residence time in states $s_0, s_1$ and $s_2$ accommodates
  delay between PQRST sequences.
}

\frame{\frametitle{Heart Rate from Viterbi Algorithm}
  \begin{center}
    \resizebox{0.5\textwidth}{!}{\includegraphics{ecg2hr.pdf}}    
  \end{center}
}

\frame{\frametitle{2-d Vector Observations}
  \begin{center}
    \resizebox{0.8\textwidth}{!}{\includegraphics{explore.pdf}}
  \end{center}
  Observation at $t$ is
  \begin{equation*}
    \bm{y}[t] \equiv \left(
    \text{Low Pass}[t], \text{Respiration}[t]
    \right)
  \end{equation*}
}

\frame{\frametitle{Observation Model Given $s[t]= s$}
  \begin{center}
    \resizebox{0.8\textwidth}{!}{\input{apnea_observation.pdf_t}}
  \end{center}\medskip
  \begin{equation*}
    \left.
    \begin{bmatrix}
      \text{hr}\\
      \text{resp}
    \end{bmatrix}[t]\right|s \sim \Normal\left( \bm{\mu}_{s,t}, \Sigma_s
  \right) \text{ with } \bm{\mu}_{s,t} = L_s(\text{past 5 observations})
  \end{equation*}
}

\frame{\frametitle{Schematic of HMM from Graphviz}
  \begin{center}
    \resizebox{2.0\textwidth}{!}{\includegraphics{viz.pdf}}    
  \end{center}
  7 apnea states and 4 normal states.
}

\frame{\frametitle{Training and Classification}
  Choose maximum likelihood parameters given the training data,
  $y_{\text{training}}$
  \begin{equation*}
    \bm{\hat \theta} = \argmax_\theta P(y_{\text{training}} | \theta).
  \end{equation*}
  
  For each testing record consider all of the data, $y[0:T]$, and for
  each state, $s_i$, and each time, $t$, calculate the probability of
  being in that state at that time,
  \begin{equation*}
    w_{t,i} \equiv P\left(s[t]=s_i | y[0:T], \bm{\hat \theta} \right).
  \end{equation*}

Finally classify each time using the ratio
\begin{equation*}
  R[t] \equiv \frac{\sum_{i\in\text{Apnea}}w_{t,i}}{\sum_{i\in\text{Normal}}w_{t,i}}.
\end{equation*}
}

\frame{\frametitle{Classification Performance}
  \begin{center}
    \resizebox{0.7\textwidth}{!}{\includegraphics{hr_resp_class.pdf}}
  \end{center}
  Error rates:
  \begin{description}
  \item[14\%] On training data
  \item[15\%] On testing data
  \item[7.5\%] Classification by eye won in 2000
  \item[3.6\%] Zhou and Kang 2024 Wavelets + NNs + \ldots
  \end{description}
}

\frame{\frametitle{Multi-Level Modeling?}
  %
  Here, I have integrated separate simplistic dynamical models of
  these three aspects of physiology:
  \begin{itemize}
  \item Dynamics of single heart beat
  \item Dynamics of respiration
  \item Dynamics of apnea
  \end{itemize}
  Separating those aspects seems appropriate and useful in a way that
  I would like to quantify.
}

\frame{\frametitle{The End}
  Questions?
}

\end{document}

arXiv:1105.1476v2 [stat.CO] 7 Sep 2012
EM algorithm and variants: an informal tutorial
Alexis Roche∗
https://arxiv.org/pdf/1105.1476

Notes from Mac's Claude generated document:

[1] **** 2013 Prevalence is increasing 17\% among 50-70 year old men
Free https://pubmed.ncbi.nlm.nih.gov/23589584/

P. E. Peppard et al., “Increased prevalence of sleep-disordered
breathing in adults,” American Journal of Epidemiology, vol. 177,
no. 9, pp. 1006–1014, 2013.

[2] 2017 These limitations have driven research into simpler, more
accessible diagnosis methods that can reach the millions of people
with undiagnosed sleep apnea

V. K. Kapur et al., “Clinical practice guideline for diagnostic
testing for adult obstructive sleep apnea,” Journal of Clinical Sleep
Medicine, vol. 13, no. 3, pp. 479–504, 2017.

[3] 2002 from CINC2000.  These heart pattern changes show up in the
ECG and can help identify when apnea is happening

T. Penzel, J. McNames, P. de Chazal, B. Raymond, A. Murray, and
G. Moody, “Systematic comparison of different algorithms for apnoea
detection based on electrocardiogram recordings,” Medical and
Biological Engineering and Computing, vol. 40, no. 4, pp. 402–407,
2002.

[4] 2015 Paywall. Ignore.  From the abstract: ``It uses two novel
features derived from the ECG, and two well-known features in heart
rate variability analysis, namely the standard deviation and the
serial correlation coefficients of the RR interval time series.''

C. Varon, A. Caicedo, D. Testelmans, B. Buyse, and S. Van Huffel, “A
novel algorithm for the automatic detection of sleep apnea from
single-lead ECG,” IEEE Transactions on Biomedical Engineering,
vol. 62, no. 9, pp. 2269–2278, 2015.

[5] 2009 Ignore By analyzing these subtle changes, researchers can estimate
breathing patterns without direct respiratory measurements

W. Karlen, C. Mattiussi, and D. Floreano, “Sleep and wake
classification with ECG and respiratory effort signals,” IEEE
Transactions on Biomedical Circuits and Systems, vol. 3, no. 2,
pp. 71–78, 2009.

From the abstract: ``We propose a single-sensor photoplethysmographic (PPG)-based automated multi-stage sleep classification. This experimental study recorded the PPG during the entire night's sleep of 10 patients. Data analysis was performed to obtain 79 features from the recordings, which were then classified according to sleep stages.''

[11] Ignore 2011 applied HMMs to model the time-varying nature of ECG signals
during sleep and reported 85% accuracy in detecting apnea events.

M. Migliorini, J. M. Kortelainen, J. Pärkkä, K. M. Tenhunen,
H. M. Himanen, and M. J. Bianchi, “Monitoring nocturnal heart rate
with bed sensor,” Methods of Information in Medicine, vol. 50, no. 4,
pp. 356–360, 2011.

From https://pubmed.ncbi.nlm.nih.gov/24889150/:
``The aim of this study is to assess the reliability of the estimated Nocturnal Heart Rate (HR), recorded through a bed sensor, compared with the one obtained from standard electrocardiography (ECG)''

[14] Paywall E. Urtnasan, J. U. Park, E. Y. Joo, and K. J. Lee, “Automated
detection of obstructive sleep apnea events from a single-lead
electrocardiogram using a convolutional neural network,” Journal of
Medical Systems, vol. 42, no. 6, p. 104, 2018.

[15] Ignore.  Claude lies about citation. used long short-term memory
(LSTM) networks, a type of RNN, to capture temporal patterns in ECG
signals. Their method achieved 97.3% accuracy on the Physionet Apnea-
ECG dataset.

L. Chen, X. Zhang, and C. Song, “An automatic screening approach for
obstructive sleep apnea diagnosis based on single-lead
electrocardiogram,” IEEE Transactions on Automation Science and
Engineering, vol. 17, no. 2, pp. 1056–1067, 2020.

[20] 2011 Traditional 24–48 hour ECG recorders have been adapted with sleep
apnea detection algorithms for overnight screening

N. A. Collop et al., “Clinical guidelines for the use of unattended
portable monitors in the diagnosis of obstructive sleep apnea in adult
patients,” Journal of Clinical Sleep Medicine, vol. 7, no. 5,
pp. 531–548, 2011.  I think Claude lies about citation.

https://jcsm.aasm.org/doi/10.5664/jcsm.27032  2007
From Abstract: ``At a minimum, PM must record airflow, respiratory effort, and blood oxygenation. The airflow, effort, and oximetric biosensors conventionally used for in-laboratory PSG should be used in PM.''

[23] ignore can't find. 2018 Adhesive patches such as the VitalPatch or ePatch can record ECG
for multiple days and are being used for sleep apnea screening

M. Sharma et al., “Ambulatory sleep monitoring via a patch-based
device,” Sleep Medicine, vol. 43, pp. 64–70, 2018.

[24] 2003 ignore too old.  Studies by Flemons et al. [24] and Masa et
al. [25] have shown that home-based monitoring can effectively
identify most cases of moderate to severe sleep apnea.

W. W. Flemons et al., “Home diagnosis of sleep apnea: a systematic
review of the literature,” Chest, vol. 124, no. 4, pp. 1543–1579,
2003.

[25] ignore 2011 is ``The effect of supplemental oxygen in obesity
hypoventilation syndrome,”

[29] ignore 2016 Penzel et al. [29] noted that these comorbidities can
confound detection algorithms by creating similar patterns to those
caused by apnea.

T. Penzel et al., “Digital analysis and technical specifications,”
Journal of Clinical Sleep Medicine, vol. 12, no. 3, pp. 429–449, 2016.

I find it to be 2007.  ``The evidence review suggested that computer
scoring and quantitative analysis of sleep is still in the formative
stage of development.''

--------------------------------------------------------------------
Jiao, M., Song, C., Xian, X., Yang, S., & Liu, F. (2024). Deep
Attention Networks With Multi-Temporal Information Fusion for Sleep
Apnea Detection. IEEE open journal of engineering in medicine and
biology, 5, 792–802. https://doi.org/10.1109/OJEMB.2024.3405666

https://pmc.ncbi.nlm.nih.gov/articles/PMC11505982/
Uses CINC2000 data 0.9106 accuracy with neural nets.

-----
Zhou Y, Kang K. Multi-Feature Automatic Extraction for Detecting
Obstructive Sleep Apnea Based on Single-Lead Electrocardiography
Signals. Sensors (Basel). 2024;24(4):1159. Published 2024 Feb
9. doi:10.3390/s24041159

https://pubmed.ncbi.nlm.nih.gov/38400317/
Uses CINC2000 data and does well with neural nets.
