% Abstract for SIAM Dynamical Systems meeting May 2025 in Denver

\documentclass{article}
\usepackage{amsmath, amsfonts}
\newcommand{\parameters}{\theta}
\newcommand{\parametersPrime}{\theta'}
\newcommand{\Normal}{{\mathcal{N}}}
\newcommand{\qmle}{Q_{\mathtt{MLE}}(\parametersPrime, \parameters)}
\newcommand{\qmap}{Q_{\mathtt{MAP}}(\parametersPrime, \parameters)}

\newcommand{\field}[1]{\mathbb{#1}}
\newcommand{\INTEGER}{\field{Z}}
\newcommand{\REAL}{\field{R}}
\newcommand{\COMPLEX}{\field{C}}
\newcommand{\EV}{\field{E}}

\begin{document}

\subsection*{Submission Format}
\label{SIAM}
From
https://www.siam.org/conferences-events/siam-conferences/ds25/submissions/:

It is recommended the first speaker provide an overview of the topic
area. Each minisymposium speaker should submit an abstract no longer
than 1,500 characters, including spaces. The organizing committee will
referee minisymposium proposals. The number of minisymposia may be
limited to retain an acceptable level of parallelism in sessions.

\subsection*{Thoughts}
\label{thoughts}

Title of Minisymposium: Dynamical Systems Models of Health and
Sickness

Talks:

Using Bayesian Data Augmentation for State Space Models to Make the
"dark Matter” in Epidemics Visible Hengartner,

Modeling Spatial Infectious Disease Dynamics: Why Ancestral Genomic
Admixture Is Important Schiff,

Estimating Dynamics and Data Assimilation to Detect Obstructive Sleep
Apnea Fraser,

Statistical Mechanics of Chemical Kinetics and the Chemical Master
Equation Pearson,


I will talk about scales from nanometers to continents.

I'm going to talk about ECG -> Apnea detection.  My models work on
three levels:

1. The PQRST sequence in an individual ECG has a fairly constant form
because it arises from wave propagation in cardiac tissue.  I ignore
details of wave propagation.

2. The effect of respiration on the timing of the P waves

3. The effect of CO2 concentration on the timing of P waves

I first spoke about ECG -> Apnea detection at SIAM DS 2001 in a
minisymposium that Kevin Vixie organized.  I presented flawed ideas.
Linda Theil suggested we write a book for SIAM.

\subsection*{Overall Abstract}
\label{overall}

We argue that health and sickness are multi-scale phenomena ranging
from the sub micron sub millisecond spatio-temporal chemical dynamics
inside individual cells to the spatio-temporal dynamics of epidemics
with scales of years and and continents.  Building complete models is
hopeless and such models would be useless.  A key to usefulness is
choosing what to ignore.


\subsection*{Estimating Dynamics and Data Assimilation to Detect
  Obstructive Sleep Apnea}
\label{sec:estimating}

Revisiting the challenge of using ECGs to detect obstructive sleep
apnea which was posed by the Computers in Cardiology Meeting in 2000,
I model dynamics at two levels.  First I fit the PQRST sequences in
the ECG of each patient with a model that has rigid timing and use
the delay between T waves and subsequent P waves to estimate heart
rate.  The analysis relies on the idea that once a P wave is launched
in the SA node, the rest of the pattern follows from wave propagation
through the roughly fixed physiology of the patient's heart.  The
model ignores the mechanisms of wave propagation in the heart and how
that produces the ECG signal.

The second model, of heart beat timing, accounts both for how CO2
levels varying over time scales of tens of seconds affects heart rate
and for how inhalation and exhalation advance and retard heart beat
timing respectively.  The model ignores details of the PQRST pattern
and any physiology.

At the 2001 Snowbird meeting I presented an algorithm and nice results
addressing the challenge based on the idea of finding a maximum
probability sequence of classifications.  In any remaining time, I
will talk about the exponential complexity of finding maximum
probability sequences of classifications and how luck gave me nice
results in 2001.

\end{document}
A Comparative Study of ECG-derived Respiration in Ambulatory
Monitoring using the Single-lead ECG:
https://www.nature.com/articles/s41598-020-62624-5
