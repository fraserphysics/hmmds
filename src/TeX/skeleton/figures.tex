% software.tex, part of dshmm, is the LaTeX file for explaining and
% displaying figures that appear in my book "Hidden Markov Models and
% Dynamical Systems"

% Copyright 2021 Andrew M. Fraser.

% Dshmm is free software: you can redistribute it and/or modify it
% under the terms of the GNU General Public License as published by the
% Free Software Foundation, either version 3 of the License, or (at your
% option) any later version.

% Dshmm is distributed in the hope that it will be useful, but WITHOUT
% ANY WARRANTY; without even the implied warranty of MERCHANTABILITY or
% FITNESS FOR A PARTICULAR PURPOSE.  See the GNU General Public License
% for more details.

% See the file gpl.txt in the root directory of the dshmm distribution
% or see <http://www.gnu.org/licenses/>.

% To build from command line, first type:
%export TEXINPUTS=../figs:.:/usr/share/texlive/texmf-dist/tex///

%%% Note: The following settings will cause many overfull hbox false alarms:
\documentclass[prelim,showlabels]{book}

\usepackage[letter,center]{crop}
\crop
\usepackage{graphicx,color}
\usepackage{amsmath, amsfonts}% ToDo: compatible w/siammathtime.sty?
\usepackage{amsthm}% Note: Conflicts with newsiambook
\usepackage{xspace}
\usepackage{bm} % Bold Math
\usepackage{rotating} % for sidewaysfigure
\usepackage{afterpage}
\usepackage{booktabs}       % for nicer looking tables
\usepackage{dcolumn}        % decimal point aligned columns
\usepackage{url}
\usepackage{showlabels}
\newtheorem{theorem}{Theorem}
\renewcommand{\th}{^{\text th}}
\newcommand{\field}[1]{\mathbb{#1}}
\newcommand{\INTEGER}{\field{Z}}
\newcommand{\REAL}{\field{R}}
\newcommand{\COMPLEX}{\field{C}}
\newcommand{\EV}{\field{E}}
\newcommand{\y}{\mathbf{y}}
\newcommand{\x}{\mathbf{x}}
\newcommand{\s}{{\bf s}}
\newcommand{\bS}{{\bf S}}
\newcommand{\Y}{{\bf Y}}
\newcommand{\Tsamp}{\tau_s }
%\newcommand{\ColorComment}[3]{{\color{#1}\textbf{#2:} #3}}
\newcommand{\ColorComment}[3]{}
\newcommand{\argmin}{\operatorname*{argmin}}
\newcommand{\argmax}{\operatorname*{argmax}}
\newcommand{\Normal}{{\mathcal{N}}}
\newcommand{\NormalE}[3]{{\mathcal{N}}\left.\left(#1,#2\right)\right|_{#3}}
\newcommand{\transpose}{^\top}
\newcommand{\ceil}[1]{\lceil#1\rceil}
\newcommand{\bceil}[1]{\left\lceil#1\right\rceil}
\newcommand{\floor}[1]{\lfloor#1\rfloor}
\newcommand{\bfloor}[1]{\left\lfloor#1\right\rfloor}
\newcommand{\states}{{\cal S}}
\newcommand{\outputs}{{\cal Y}}
\newcommand{\State}{S}
\newcommand{\Output}{Y}
\newcommand{\parameters}{\theta}
\newcommand{\parametersPrime}{\theta'}% for EM1.gpt

\newcommand{\ti}[2]{{#1}{(#2)}}                  % Time Index
%%% \newcommand{\ts}[3]{{#1}{(t\=#2,\ldots,#3)}}         % Time Sequence
\newcommand{\ts}[3]{#1_{#2}^{#3}}                    % Time Sequence
%%% \newcommand{\ts}[3]{\left\{ #1(l) \right\}_{l=#2}^{#3}}  % Time Sequence
\newcommand{\id}{{\bf I}}
\newcommand{\ie}{i.e.\xspace}
\newcommand{\eg}{e.g.\xspace}
\newcommand{\etal}{et al.\xspace}
\newcommand{\iid}{i.~i.~d.\xspace}
\newcommand{\apost}{\emph{a posteriori}\xspace}
\newcommand{\apri}{\emph{a priori}\xspace}
\newcommand{\plotsize}{\small}
\newcommand{\mlabel}[1]{\label{#1}}
\newcommand{\biburl}[1]{\\\url{#1}} % URL in a "note" gets a line by itself
\newcommand{\EMmap}{{\mathcal T}} %
%\pagestyle{empty}

\author{Andrew M. Fraser}
\title{Software for Hidden Markov Models and Dynamical Systems}

\begin{document}
\frontmatter
\maketitle

\chapter{Preface}
\label{chap:preface}

This document describes \emph{dshmm}, a revision of the software I
used to produce \emph{Hidden Markov Models and Dynamical Systems},
hereinafter \emph{the book}.  When I started working on the book, I
decided to structure it as a software project using version control to
help with collaboration and keep track of changes.  I used gnu
\emph{make} to control building the figures and formatting the text.
I hoped that using the structure of a software project would produce a
single package that achieved the following goals:
\begin{itemize}
\item Keep a record of all techniques
\item Allow readers to read, run, modify, and redistribute the code
\item Allow readers to verify results
\item Make formatting techniques available to other authors
\end{itemize}

\section*{Building}
\label{sec:building}

To build this document, from a command line, go to the root directory
(just above \emph{TeX}) and enter the following
\begin{center}
  \texttt{make software.pdf}
\end{center}
Some of the derived files (those that take days to build and those for
which the present code does not work) are included in the
\emph{derived\_data} directory of distribution.  If you want to
rebuild those files, simply remove them and rebuild this document
using the above procedure.

\subsection*{showlabels}
\label{sec:showlabels}

The text \texttt{\{sec:showlabels\}} that appears in the margin is
produced by the package \emph{showlabels}.  I find it helpful as I
write \LaTeX documents to be able to see the symbolic names of labeled
items.  You can eliminate the labels by removing the line
\texttt{\textbackslash usepackage\{showlabels\}} from the file
\emph{software.tex} and rebuilding this document, \emph{software.pdf}.

\section*{Changes}
\label{sec:changes}

A preliminary version of dshmm produced the figures in this draft
document.  Ultimately, each figure in the book will map to a figure in
this document.  Below, I list the major differences between dshmm and
the software that I used to produce the first edition of the book:
\begin{description}
\item[Error described and addressed:] The algorithm for decoding
  \emph{class} sequences described in Chapter 6 of the book does not
  work in general.  In designing the flawed algorithm, I imitated the
  Viterbi algorithm which finds the best \emph{state} sequence with
  complexity that is \emph{linear} in the number of time steps.  In
  the book, I claimed that my algorithm was also linear in the number
  time steps and that it finds the best \emph{classification}
  sequence.  While the complexity is linear, the flawed algorithm may
  fail to find the best sequence, and in fact it may fail to find a
  \emph{possible} sequence.  For details, see
  Section~\ref{sec:V4Class} of this document.
\item[Test Suite:] I have written tests using python's unit test
  framework.  Those tests are part of the new distribution.  That
  system revealed to me the error of the previous item.
\item[Favor Pretty Code Over Pretty Pictures:] Karl Hegbloom and I
  wrote complex code to make the book pretty.  In this version, I
  sacrifice appearance of the plots if that lets me write simpler
  code.
\item[Python3:] The new code runs under python3
  exclusively.
\item[Numpy, SciPy, Cython, \ldots] The new and improved tools that
  have been created since I started the first version combined with
  things I've learned about writing code has produced much clearer
  code.
\item[Sphinx:] I use sphinx to format documentation embedded in the
  source.
\item[Matplotlib and XFig:] I make all of the figures with Matplotlib
  and XFig rather than the mix of tools including gnuplot that Karl
  and I used for the book.  My code that uses Matplotlib is ugly.
  While I take some responsibility for that ugliness, I don't
  recommend Matplotlib with any enthusiasm over gnuplot.
\item[SCons:] I've switched from gnu-make to SCons for building.  I
  don't recommend SCons over gnu make with any enthusiasm either.
\end{description}

\subsection*{Differences}
The data for most of the figures and tables in this document are not
identical to data for the figures and tables in the book.  My code is
different and the libraries on which my code builds are different.
Many details of the results depend on features like random number
generators.  I have focused on making the new code easier to read
rather than getting it to reproduce the results of the old code
exactly.  While I've found the old code embarrassingly difficult to
read, I am pleased to have found that it mostly runs and gives correct
results.  As of 2021-09-01 the only substantial error that I've found
in the old is the bogus algorithm described in
Chapter~\ref{chap:apnea} of the book for decoding class sequences.  If
I find other major flaws, I will note them in revisions to this
document.

\section*{Copyright and License}

SIAM owns the copyright to the \LaTeX source files for the book,
namely: \emph{algorithms.tex, introduction.tex, appendix.tex,
  main.tex, toys.tex, continuous.tex, real.tex,} and
\emph{variants.tex}.  You may obtain those files from the SIAM website
for the book, but you can only use them to see how the book
was made.  The \LaTeX source files are subject to the following
restrictions:
\begin{itemize}
\item You may not use them to print a copy of the book
\item You may not further distribute them
\item You may not distribute modifications of the files
\end{itemize}

I (Andrew M. Fraser) own the copyright to all the other files, and I
make those files available under the terms of version 3 of the GNU
General Public License as published by the Free Software Foundation,
(http://www.gnu.org/licenses/).

\section*{Structure of the Document}
Chapters~\ref{chap:introduction} through~\ref{chap:apnea} mirror
chapters in the book.  The figures in those chapters map bijectively
to the figures in the corresponding chapters of the book and the
numbering is the same.  While the figures in the book illustrate
concepts in the text, in this document the text explains the software
that makes the figures.  I will place other material in this preface
and in appendices.  The first appendix addresses the erroneous analysis
of class decoding that appears in Chapter~\ref{chap:apnea} of the
book.

\section*{To Do}
\begin{description}
\item[fig:LikeLor] Perhaps try for larger $n_\text{states}$
\item[Fix fig:mm:]
\item[Class-Decode:] Study the performance of ad hoc algorithms for
  decoding class.  Write an appendix describing the study.  Apply such
  an algorithm to the apnea problem of Chapter 6.
\item[Clean up EKF and put it in code/hmm]  Get SCons to build all
  files in derived\_data/laser and remove them from the repository
\item[Write new Hview code] Data for Figs.~\ref{fig:ToyTS1}
  and~\ref{fig:ToyStretch}

\end{description}

\mainmatter
\chapter{Introduction}
\label{chap:introduction}

\section{Laser Example}

The red trace in Fig.~\ref{fig:LaserLP5} plots the first 250 points in
Tang's data file \emph{LP5.DAT}.  The blue trace is the result of an
optimization that searches for parameters of the model described in
the book that minimizes the sum of the squared differences between the
simulation and the data.  The program \emph{Laser\_data.py} calls the
scipy.optimize.fmin\_powell routine to do the optimization and makes
the file \emph{data/LaserLP5}.  The optimization takes about 6
minutes.  The script \emph{Laser\_plots.py} makes
\emph{figs/LaserLP5.pdf} from \emph{data/LaserLP5}.
\begin{figure}[htbp]
  \centering{\resizebox{\textwidth}{!}{\includegraphics{LaserLP5.pdf}}
  }
  \caption[Laser intensity measurements.]%
  {Laser intensity measurements.  The trace labeled \emph{Laser} is a
    plot of laser intensity measurements provided by Tang \etal.  The
    trace labeled \emph{Simulation} plots a numerical simulation of
    the Lorenz system (1.1) with parameters $r=21.16$,
    $s=1.792$, $b=0.3670$, and measurement parameters $\Tsamp=0.1435$,
    $S_g = 7.071$, and $O_g =15.16$.  I used the optimization
    procedure described in the text to select these parameters.  The
    simulated intensities were derived from the state by $\ti{y}{t} =
    S_g \cdot (\ti{x_1}{t})^2 + O_g$.  I specified an absolute error
    tolerance of $10^{-7}$ per time step for the numerical
    integrator.}
  \label{fig:LaserLP5}
\end{figure}

\begin{figure}[htbp]
  \centering{\resizebox{\textwidth}{!}{\includegraphics{LaserLogLike.pdf}}
  }
  \caption{Log likelihood as function of $s$ and $b$.  Other parameters were
    taken from the vector $\hat \theta$ that maximizes the likelihood
    $P(\ts{y}{1}{250}|\theta)$ (see Eqn.~1.3).}
  \label{fig:LaserLogLike}
\end{figure}

After finishing the optimization for Fig.~\ref{fig:LaserLogLike}, the
data for Figs.~\ref{fig:LaserStates},~\ref{fig:LaserForecast},
and~\ref{fig:LaserHist} are made in less than a minute.
\begin{figure}[htbp]
  \centering{\resizebox{\textwidth}{!}{\includegraphics{LaserStates.pdf}}
  }
  \caption[State trajectory $\ts{\hat
    x}{1}{250}$.]%
  {State trajectory $\ts{\hat x}{1}{250}$ estimated from observation
    sequence $\ts{y}{1}{250}$. (see Eqn.~1.4.)
    Components $x_1$ and $x_3$ of the Lorenz system (see
    Eqn.~1.1) are plotted.  Recovering the familiar
    Lorenz figure suggests both that the laser data is \emph{Lorenz
      like} and that the algorithm for estimating states from
    observations is reasonable.}
  \label{fig:LaserStates}
\end{figure}

Figure~\ref{fig:LaserForecast} is not as nice as the figure in the book.
I hope to rework the optimization code yet again to find a stable
period five orbit that matches the data.
\begin{figure}[htbp]
  \centering{\resizebox{\textwidth}{!}{\includegraphics{LaserForecast.pdf}}
  }
  \caption[Forecast observation sequence.]%
  {Forecast observation sequence.  I set the noise terms $\eta$ and
    $\epsilon$ to zero and iterated Eqn.~1.2 400 times to
    generate the forecast $\ts{\hat y}{251}{650}$.  I started with
    the initial condition $\hat x$ defined by Eqn.~1.5.
    The forecast begins to fail noticeably after $t=500$.  The failure
    suggests that the period five cycle in the forecast is unstable.
    The laser cycle must have been stable to appear in the data.  Thus
    an essential characteristic of the model is wrong.}
  \label{fig:LaserForecast}
\end{figure}

%\section{State Space Models}
\addtocounter{section}{1}
\section{Discrete HMMs}
\label{sec:intro_hmm}
\index{discrete hidden Markov model}%

Karl Hegbloom and I drew Fig.~\ref{fig:mm} using the XFig utility.
Karl set up a method that let us use the same drawing for
Fig.~\ref{fig:dhmm} too.  In his words:
\begin{quote}
  Both \emph{Markov\_mm.pdf\_t} and \emph{Markov\_dhmm.pdf\_t} are
  generated from the same .fig file.  The output arcs and their labels
  are at depth 40 and the rest is at depth 50.  When \emph{mm.pdf\_t}
  is created, the space taken up by the output arcs is blank, so we
  need to trim it off using the \emph{-K} switch to \emph{fig2dev}.
\end{quote}

\begin{figure}[htbp]
  \centering{\resizebox{\textwidth}{!}{\input Markov_mm.pdf_t }
  }
  \caption[A Markov model.]{A Markov model}
  \label{fig:mm}
\end{figure}

\begin{figure}[htbp]
  \centering{\input{Markov_dhmm.pdf_t}
  }
  \caption[A hidden Markov model.]{A hidden Markov model}
  \label{fig:dhmm}
\end{figure}

\subsection{Example: Quantized Lorenz Time Series}
\label{sec:QuantizedLorenz}

I used \emph{scipy.odeint} or the fourth order Runge Kutta formula
implemented in Cython, to integrate the Lorenz system and make the
data for Figs.~\ref{fig:TSintro}-\ref{fig:Statesintro}.  I have not
found Matplotlib to be much easier to use than gnuplot.
\begin{figure}[htbp]
  \centering{\resizebox{\textwidth}{!}{\includegraphics{TSintro.pdf}}
  }
  \caption[Generating the observations $\ts{y}{1}{40}$.]%
  {Generating the observations $\ts{y}{1}{40}$.  The curve in the
    upper plot depicts the first component $\ti{x_1}{\tau}$ of an
    orbit of the Lorenz system (1.1), and the points
    marked {\scriptsize\raise0.5ex\hbox{$\bm{\diamond}\!$}} indicate the
    values sampled with an interval $\Tsamp = 0.15$.  The points
    in the lower plot are the quantized values $\ti{y}{t} \equiv
    \bceil{\frac{\ti{x_1}{t \cdot \Tsamp}}{10} + 2}$, where
    $\ceil{u}$ is the least integer greater than or equal to $u$.  }
  \label{fig:TSintro}
\end{figure}

 \begin{figure}[htbp]
   \centering{\resizebox{\textwidth}{!}{\includegraphics{STSintro.pdf}}
   }
   \caption{A plot of the state sequence found by Viterbi decoding a quantized
     time series from the Lorenz system.  Here the number of the
     decoded state $\ti{s}{t}$ is plotted against time $t$.  Although
     it is hard to see any structure in the plot because the numbers
     assigned to the states are not significant,
     Fig.~1.9 illustrates that the decoded states
     are closely related to positions in the generating state space.}
   \label{fig:STSintro}
 \end{figure}

 Figure~\ref{fig:Statesintro} is the picture on the cover of the book.
%%% This is a large color figure on a page by itself. (butterfly)
%%%
 \begin{figure}[htb]
  \centering{\resizebox{\textwidth}{!}{\includegraphics{Statesintro.pdf}}
  }
   \caption[Relationship between states of HMM and Lorenz system.]%
   {The relationship between the hidden states of an HMM and the
     original coordinates of the Lorenz system.}
   \label{fig:Statesintro}
 \end{figure}
%%%\afterpage{\clearpage}%% Print this right here or let it float to end of chapter?

\subsection{Example: Hidden States as Parts of Speech}
\label{sec:POSpeech}

I've copied Table~\ref{tab:POS} from the source for the book.  Since
the actual table of words is the result of the new code
(\emph{po\_speech.py}), the explanations do not match the word groups.
Using different seeds for the random number generator produces
different groupings.  For now, each run takes 52 minutes.  In the
future, I will write the core routines in Cython which will reduce the
time by an order of magnitude.
\begin{table}[htb]
  \caption[Words most frequently associated with each state.]%
  {Words most frequently associated with each state.  While I have no
    interpretation for three of the states, some of the following
    interpretations of the other states are strikingly successful.}
  \begin{center}{\plotsize%
      \fbox{%
      \begin{tabular}[t]{r@{\hspace{0.28em}}p{19em}}
        1  -- & \rule{0pt}{2.5ex}Adjectives \\
        2  -- & \rule{0pt}{2.5ex}Punctuation and other tokens that appear at the end of phrases \\
        6  -- & \rule{0pt}{2.5ex}Capitalized articles and other tokens that appear at the beginning of phrases \\
        7  -- & \rule{0pt}{2.5ex}Objective pronouns and other words with similar functions \\
        8  -- & \rule{0pt}{2.5ex}Nouns
      \end{tabular}\qquad%
      \begin{tabular}[t]{r@{\hspace{0.28em}}p{10em}}
        9  -- & \rule{0pt}{2.5ex}Nouns \\
        10 -- & \rule{0pt}{2.5ex}Helping verbs \\
        11 -- & \rule{0pt}{2.5ex}Nominative pronouns \\
        12 -- & \rule{0pt}{2.5ex}Articles \\
        13 -- & \rule{0pt}{2.5ex}Conjunctions \\
        14 -- & \rule{0pt}{2.5ex}Prepositions \\
        15 -- & \rule{0pt}{2.5ex}Relative pronouns
      \end{tabular}}\\[2.0ex]
      \begin{tabular}{|@{\hspace{0.10em}}r@{\hspace{0.40em}}|*{10}{@{\hspace{0.28em}}l@{\hspace{0.28em}}}|}
%        %\hline \input{po_speech} [0.5ex]
%        \hline
      \end{tabular}
    }\end{center}
  \label{tab:POS}
\end{table}

\subsection{Remarks}
\label{sec:DHMMRemarks}

Figures~\ref{fig:dhmm_net} and~\ref{fig:nonmm} are XFig drawings.
\begin{figure}[htbp]
  \centering{\input{Markov_dhmm_net.pdf_t}
  }
  \caption[Bayes net schematic for a hidden Markov model.]%
  {Bayes net schematic for a hidden Markov model.  The drawn edges
    indicate the dependence and independence relations: Given
    $\ti{S}{t}$, $\ti{Y}{t}$ is conditionally independent of
    everything else, and given $\ti{S}{t-1}$, $\ti{S}{t+1}$, and
    $\ti{Y}{t}$, $\ti{S}{t}$ is conditionally independent of
    everything else.}
  \label{fig:dhmm_net}
\end{figure}

\begin{figure}[htbp]
  \centering{\input{nonmm.pdf_t}
  }
  \caption[An HMM that cannot be represented by a
  Markov model.]%
  {An HMM that cannot be represented by a Markov model of any order.
    Consider the string of observations ``$b,a,a,\ldots,a,a,a$''.  For
    each ``$a$'' in the string, the previous non-``$a$'' observation
    was ``$b$''.  Since the model will not produce another ``$b$'' before
    it produces a ``$c$'', the next observation can be either a
    ``$c$'' or another ``$a$'', but not a ``$b$''.  Because there is
    no limit on the number of consecutive ``$a$'s'' that can appear,
    there is no limit on how far back in the observation sequence you
    might have to look to know the probabilities of the next
    observation.}
  \label{fig:nonmm}
\end{figure}

\chapter{Basic Algorithms}
\label{chap:algorithms}

\section{The Forward Algorithm}
\label{sec:forward}
\index{forward algorithm|textbf}

Figure~\ref{fig:forward} is an XFig drawing that invokes many \LaTeX
commands defined in the file \emph{software.tex}.
\begin{sidewaysfigure}[htbp]
  %% 0.75\textheight is approx 5.7 in.  The .eps is exactly that size.
  %% The widths for the minipages below are taken by measuring the
  %% ovals inside of XFig.
  \centering{\plotsize%
    %% Column a
    \def\colaa{$\alpha(s_1,t-1)$}%
    \def\colab{$\alpha(s_2,t-1)$}%
    \def\colac{$\alpha(s_3,t-1)$}%
    %% Column b
    \def\colba{$P \left(s_1|\ts{y}{1}{t-1} \right)$}%
    \def\colbb{$P \left(s_2|\ts{y}{1}{t-1} \right)$}%
    \def\colbc{$P \left(s_3|\ts{y}{1}{t-1} \right)$}%
    \def\sumeqforwardAthree{%
      \begin{minipage}[t]{1.8in}
        \raggedright%
        Weighted\\%
        sum of prior\\%
        $\alpha$'s\\%
        Eqn.~2.4c
      \end{minipage}}%
    %% Column c
    \def\colca{$P \left(s_1,\ti{y}{t}|\ts{y}{1}{t-1} \right)$}%
    \def\colcb{$P \left(s_2,\ti{y}{t}|\ts{y}{1}{t-1} \right)$}%
    \def\colcc{$P \left(s_3,\ti{y}{t}|\ts{y}{1}{t-1} \right)$}%
    \def\prdeqforwardBtwo{%
      \begin{minipage}[t]{2.3in}
        \raggedright%
        Multiply\\%
        observation\\%
        probability\\%
        Eqn.~2.5b
      \end{minipage}}%
    %% Column d
    \def\coldb{$P(\ti{y}{t}|\ts{y}{1}{t-1})$}%
    \def\prdeqforwardC{%
      \begin{minipage}[t]{1.8in}
        \raggedright%
        Add, to get\\%
        $\gamma(t)$, $P \left(\ti{y}{t}|\ts{y}{1}{t-1} \right)$\\%
        Eqn.~2.6
      \end{minipage}}%
    %% Column e
    \def\colea{$\alpha(s_1,t)$}%
    \def\coleb{$\alpha(s_2,t)$}%
    \def\colec{$\alpha(s_3,t)$}%
    \def\quoteqforwardD{%
      \begin{minipage}[t]{1.8in}
        \raggedright%
        Normalize\\%
        new $\alpha$'s\\%
        Eqn.~2.7
      \end{minipage}}%
    \input{forward.pdf_t}
  }
  \vspace{5 em}
  \caption[Dependency relations in the forward
  algorithm.]%
  {Dependency relations in the forward algorithm (See
    Eqns.~2.4-2.7 in the text).  The
    figure indicates the calculations the algorithm executes to
    incorporate the observation at time $t$ for a three state model.}
  \label{fig:forward}
\end{sidewaysfigure}
\afterpage{\clearpage}%% Print this right here please.

%\section{The Backward Algorithm}
\addtocounter{section}{1}

\section{The Viterbi Algorithm}
\label{sec:viterbi}

Figure~\ref{fig:viterbi} is straight \LaTeX source that is part of the
file \emph{software.tex}.
%%%
%%% fig:viterbi
%%%
\begin{figure}[htbp]
  \begin{center}
    \def\tnext{_{\text{next}}}%
    \def\told{_{\text{old}}}%
    \def\tbest{_{\text{best}}}%
    \fbox{
      \begin{minipage}{0.90\textwidth}
        \begin{tabbing}
          XX\=XX\=XX\=XX\=XX\=XX\=XX\=XX\= \kill
          Initialize: \> \+ \\
          for each $s$\\ \> \+
          $\nu\tnext (s) = \log \left( P_{\ti{Y}{1},\ti{S}{1}}
            \left(\ti{y}{1},s \right)\right)$ \\ \\ \< \- \< \-
          Iterate: \> \+ \\
          for $t$ from 2 to $T-1$\\ \> \+
          Swap $\nu\tnext \leftrightarrow \nu\told$\\
          for each $s\tnext$\\ \> \+
             \\ for each $s\told$\\ \> \+
                $\omega(s\told,s\tnext) = \nu(s\told,t) + \log\left(
            P(s\tnext|s\told) \right)$  \\ \> \+
             $ + \log\left( P(\ti{y}{t+1}|s\tnext) \right)$\\
          \< \- \< \- %This stuff is for tabs \< \-
          \\ \# Find best predecessor\\
          $B(s\tnext,t+1) = \argmax_{s\told} \omega(s\told,s\tnext) $ \\
          \\ \# Update $\nu$\\
          $\nu\tnext(s\tnext) =\,$ \= $\omega(B(s\tnext,t+1),s\tnext)$ \\
          \< \- \< \- \< \- %This stuff is for tabs \< \-
          Backtrack: \> \+ \\
          $\bar s = \argmax_s \nu\tnext(s)$ \\
          $\ti{\hat s}{T}  = \bar s$ \\
          for $t$ from $T-1$ to $1$  \\ \> \+
          $ \bar s = B(\bar s,t+1)$  \\
          $\ti{\hat s}{t} = \bar s$
        \end{tabbing}
      \end{minipage}
    }
    \caption[Pseudocode for the Viterbi Algorithm]%
    {Pseudocode for the Viterbi Algorithm}
    \label{fig:viterbi}
  \end{center}
\end{figure}
Figure~\ref{fig:viterbiB} is an XFig drawing that invokes commands
defined in \emph{software.tex}.
%%%
%%% fig:viterbiB
%%%
\begin{sidewaysfigure}[htbp]
  %% 0.75\textheight is approx 5.7 in.  The .eps is exactly that size.
  %% The widths for the minipages below are taken by measuring the
  %% ovals inside of XFig.
  \centering{\plotsize%
    %% Column a
    \def\colaa{$\nu(s_1,t-1)$}%
    \def\colab{$\nu(s_2,t-1)$}%
    \def\colac{$\nu(s_3,t-1)$}%
    %% Column b
    \def\colba{$ \begin{matrix} B(s_1,t) = \argmax_{{\tilde s}}\\%
        \log\left(P(s_1|{\tilde s}) \right)\\+ \nu({\tilde s},t-1)
      \end{matrix}$}%
    \def\colbb{$ \begin{matrix} B(s_2,t) = \argmax_{{\tilde s}}\\%
        \log\left(P(s_2|{\tilde s}) \right)\\+ \nu({\tilde s},t-1)
      \end{matrix}$}%
    \def\colbc{$ \begin{matrix} B(s_3,t) = \argmax_{{\tilde s}}\\%
        \log\left(P(s_3|{\tilde s}) \right)\\+ \nu({\tilde s},t-1)
      \end{matrix}$}%
    \def\bestpred{%
      \begin{minipage}[t]{3.9in}
        \raggedright%
        For each state $s$ find the best\\%
        predecessor $\tilde s$, \ie, the\\%
        one that maximizes\\%
        $\log\left(P(s|\tilde s) \right) + \nu(\tilde s,t-1)$.\\%
        The bolder lines indicate best\\%
        predecessors.
      \end{minipage}}%
    %% Column c
    \def\colca{$\nu(s_1,t)$}%
    \def\colcb{$\nu(s_2,t)$}%
    \def\colcc{$\nu(s_3,t)$}%
    \def\newnu{%
      \begin{minipage}[t]{2.7in}
        \raggedright%
        For each state $s$\\%
        calculate $\nu(s,t)$\\%
        by including the\\%
        conditional probability\\%
        of the observation $\ti{y}{t}$,\\%
        \ie, $ \nu(s,t) = \log\left(P(\ti{y}{t}|s) \right) $\\
        \qquad$+ \log\left(P(s|B(s,t)) \right)$\\
        \qquad$+ \nu(B(s,t),t-1)$.
      \end{minipage}}
    %%
    \input{viterbiB.pdf_t}
  }
  \vspace{7em}
  \caption[Dependency relations in the Viterbi algorithm.]%
  {Dependency relations in the Viterbi algorithm.}
  \label{fig:viterbiB}
\end{sidewaysfigure}

\subsection{General Decoding}
\label{sec:GenDecode}

\subsection{MAP Sequence of States or Sequence of MAP States?}
\label{sec:sequenceMAP}

Figure~\ref{fig:sequenceMAP} is an XFig drawing.
\begin{figure}[htbp]
  \centering{\plotsize%
    \input{sequenceMAP.pdf_t}
  }
  \caption{HMM used to illustrate that the maximum \apost\ sequence of states is
    not the same as the sequence of maximum \apost\ states.}
\label{fig:sequenceMAP}
\end{figure}

\section{The Baum-Welch Algorithm}
\label{sec:baum_welch}

%\subsection{Weights and Reestimation}
\addtocounter{subsection}{1}

\subsubsection{Reestimation}

Table~\ref{tab:reestimation} is somewhat obscure straight \LaTeX.
\begin{table}[htbp]
  \caption[Summary of reestimation formulas.]%
  {Summary of reestimation formulas.\index{reestimation formulas}}
  \centering{\plotsize%
    \begin{minipage}{.7\textwidth}
      Note that formulas for $w(s,t)$ and ${\tilde w}({\tilde s},s,t)$
      appear in Eqns.~2.20 and 2.22 respectively.
    \end{minipage}\\[1ex]
%    \begin{tabular*}{0.98\textwidth}[H]{|l|r|l|}
    \begin{tabular}[H]{|p{5.5em}|c|c|}
      \hline
      \rule{0pt}{2.5ex}Description & Expression & New Value \\
      \hline
      \rule{0pt}{2.5ex}Initial State Probability
      & $P_{\ti{S}{1}|\ti{\parameters}{n+1}} \left(s|\ti{\parameters}{n+1} \right)$
      & {\normalsize $ w(s,1) $} \\[1.5ex]
      \hline State Transition Probability
      & $P_{\ti{S}{t+1}|\ti{S}{t},\ti{\parameters}{n+1}} \left({\tilde
          s}|s, \ti{\parameters}{n+1} \right)$
      & \raisebox{-2.5ex}[3ex][6ex]{\Large $ \frac {\sum_{t=1}^{T-1} {\tilde
            w}({\tilde s},s,t)} {\sum_{s'\in\states} \sum_{t=1}^{T-1}
          {\tilde w}(s',s,t)}$} \\[2.5ex]
      \hline Conditional Observation Probability
      & $P_{\ti{Y}{t}|\ti{S}{t},\ti{\parameters}{n+1}} \left(y|s,
        \ti{\parameters}{n+1} \right)$
      & \raisebox{-3.3ex}[2.5ex][6ex]{\Large $ \frac {\sum_{t:\ti{y}{t}=y} w(s,t)}
        {\sum_{t} w(s,t)}$} \\[2.0ex]
      \hline
%    \end{tabular*}}
    \end{tabular}}
  \label{tab:reestimation}
\end{table}

Fig.~\ref{fig:train} is also obscure straight \LaTeX.
%%%
%%% fig:train
%%%
\begin{figure}[htbp]
  \begin{center}
    %\small%
    \def\assign{\leftarrow}%
    \def\oldmodel{\ti{\parameters}{n}}%
    \def\newmodel{\ti{\parameters}{n+1}}%
% Tabbing commands:
% \=    Set a stop
% \>    Skip to the next stop
% \<    Go back a stop
% \\    New line
% \+    Move left margin right one stop
% \-    Move the left margin left one stop
\fbox{
  \begin{minipage}{0.90\textwidth}
    \begin{tabbing}
      XX\=XX\=XX\=XX\=XX\=XX\=XX\=XX\= \kill
      Notation: \> \+ \\ \\
      \begin{minipage}[b]{1.0\textwidth}
        $\oldmodel$ is the model, or equivalently the set of
        parameters, after $n$ iterations of the Baum-Welch algorithm.
      \end{minipage}\\ \\%
      %%
      \begin{minipage}[b]{1.0\textwidth}
        $\bm{\alpha}_n$ is the set of conditional state probabilities
        calculated on the basis of the $n^{\text{th}}$ model and the
        data $\ts{y}{1}{T}$.  See Eqns.~2.2 and 2.7.
        \begin{equation*}
          \bm{\alpha}_n \equiv \left\{ P_{ S(t)| \ts{Y}{1}{t},\oldmodel} \left( s |
             \ts{y}{1}{t},\oldmodel\right) : \forall s \in \states\, \& \, 1
           \leq t \leq T \right\}
        \end{equation*}
      \end{minipage}\\ \\%
      %%
      \begin{minipage}[b]{1.0\textwidth}
         $\bm{\beta}_n$ is a set of values calculated on the basis of the
         $n^{\text{th}}$ model $\oldmodel$ and the data $\ts{y}{1}{T}$.  See
         Eqns.~2.11 and 2.12.
         \begin{equation*}
         \bm{\beta}_n \equiv \left\{ \frac{P_{\ts{Y}{t+1}{T}|\ti{S}{t}}
         \left(\ts{y}{t+1}{T} | s \right)} {P
         \left(\ts{y}{t+1}{T}|\ts{y}{1}{t} \right)} : \forall s \in
         \states\, \& \, 1 \leq t < T \right\}
         \end{equation*}
      \end{minipage}\\ \\%
      %%
      \begin{minipage}[b]{1.0\textwidth}
         $\bm{\gamma}_n$ is the set of conditional observation probabilities
         calculated on the basis of the $n^{\text{th}}$ model $\oldmodel$ and
         the data $\ts{y}{1}{T}$.  See Eqns.~2.3 and 2.6.
         \begin{equation*}
         \bm{\gamma}_n \equiv \left\{P \left(\ti{y}{t} | \ts{y}{1}{t-1},
         \oldmodel\right) : \, 2 \leq t \leq T \right\}
         \end{equation*}
      \end{minipage}\\ \\%
      %%
      \<\-
      Initialize: \> \+ \\
      Set $n=1$ and choose $\ti{\parameters}{1}$\\ \\ \< \-
      Iterate: \> \+ \\
      $\newmodel \assign \text{reestimate} \left( \ts{y}{1}{T},
      \bm{\alpha}_n, \bm{\beta}_n, \bm{\gamma}_n, \oldmodel\right)$ XX\= \kill
      $\left(\bm{\alpha}_n,\bm{\gamma}_n\right) \assign
      \text{forward}(\ts{y}{1}{T},\oldmodel)$ \> See Section 2.1 page 20\\
      $\bm{\beta}_n \assign \text{backward}(\bm{\gamma}_n, \ts{y}{1}{T},
      \oldmodel)$ \> See Section 2.2 page 25\\
      $\newmodel \assign \text{reestimate} \left( \ts{y}{1}{T},
      \bm{\alpha}_n, \bm{\beta}_n, \bm{\gamma}_n, \oldmodel\right)$ \> See Table
      2.2 page 33 \\
      $n \assign n+1$ \\
      Test for completion
    \end{tabbing}
  \end{minipage}
}
\caption[Baum-Welch model parameter optimization.]%
{Summary and pseudo-code for optimizing model parameters by iterating
  the Baum-Welch algorithm. \index{Baum-Welch algorithm}}
    \label{fig:train}
  \end{center}
\end{figure}

\subsection{Remarks}
\label{sec:AlgApp}

\subsubsection{Multiple Maxima}
\label{sec:MultiMax}

\begin{figure}
  \centering{\resizebox{\textwidth}{!}{\includegraphics{TrainChar.pdf}}
  }
  \caption[Convergence of the Baum-Welch algorithm.]{%
    Convergence of the Baum-Welch algorithm.  Here I have plotted
    $\frac{\log \left( P \left(\ts{y}{1}{T}|\ti{\parameters}{n}
        \right)\right)}{T}$ (the log likelihood per step) as a
    function of the number of iterations $n$ of the Baum-Welch
    algorithm for five different initial models $\ti{\parameters}{1}$.
    I used the same sequence of observations $\ts{y}{1}{T}$ that I
    used for Fig.~1.9, and I used different seeds for a random number
    generator to make the five initial models.  Note the following
    characteristics: The five different initial models all converge to
    different models with different likelihoods; the curves intersect
    each other as some models improve more with training than others;
    convergence is difficult to determine because some curves seem to
    have converged for many iterations and later rise significantly.
    Although it appears that three of the initial models all converge
    to -0.5, close examination of the data suggests that they are
    converging to different models with different log likelihoods per
    step.}
  \label{fig:TrainChar}
\end{figure}

\section{The EM algorithm}
\label{sec:EM}
\index{estimate maximize (EM) algorithm|textbf}%

\begin{figure}[htbp]
  \centering{\resizebox{\textwidth}{!}{\includegraphics{GaussMix.pdf}}
  }
  \caption[Two iterations of the EM
  algorithm.]%
  {Two iterations of the EM algorithm.  I use the algorithm to search
    for the parameters of Eqn.~2.49 that maximize the
    likelihood of the ten simulated observations that appear above in
    the row labeled $\ti{y}{t}$.  The triple of rows labeled
    $\ti{\parameters}{1}$ report the weighting calculated using
    Eqn.~2.50 used in Eqns.~2.51 to recalculate
    the conditional means, $\mu_1$ and $\mu_2$, for
    $\ti{\parameters}{2}$ in the M step, and the next triple of rows,
    labeled $\ti{\parameters}{2}$, report the same quantities for the
    calculation of $\ti{\parameters}{3}$.  The parameters of the first
    three models appear in the row just below the boxed table.
    $\ti{\parameters}{3}$ is the triple of parameters produced by two
    iterations of the EM algorithm, $\lambda = 0.603, ~ \mu_1 =
    -2.028,~ \mu_2 = 1.885$.  On the axes of the upper plot, I
    illustrate $P(x|\parameters)$ as defined in
    Eqn.~2.49 for two sets of model parameters: The
    dashed line depicts $\parameters = (0.5, -2, 2)$, the distribution
    used to simulate the data, and the solid line depicts
    $\ti{\parameters}{1} = (0.5, -1, 1)$, the starting distribution I
    chose for the EM algorithm.  On the bottom axes I plot the
    simulated observations as marks on the abscissa and
    $P(y|\ti{\parameters}{3})$ as a solid line.}
  \label{fig:GaussMix}
\end{figure}

\addtocounter{subsection}{1}
%\subsection{Monotonicity}

\afterpage{\clearpage} % Flush float
\newpage
\subsection{Convergence}

\subsubsection{A contrived example}
\label{sec:contrived}

\begin{figure}[htbp]
  \centering{\resizebox{\textwidth}{!}{\includegraphics{EM.pdf}}
  }
  \caption[An illustration of the EM algorithm.]%
  {An illustration of the EM algorithm for an experiment in which a
    coin is thrown four times, first a head is observed
    $(\ti{y}{1}=1)$, then a tail is observed $(\ti{y}{2}=0)$, and
    finally two results are unobserved with $s_h$ and $s_t$ being the
    number of unobserved heads and tails respectively.  The goal is to
    find the maximum likelihood value of $\parameters$, the
    probability of heads.  The log likelihood function for the
    complete data is $L_{\parameters} = (s_h + 1)\log(\parameters) +
    (s_t +1)\log(1-\parameters)$.  The auxiliary function
    $Q(\parameters',\parameters) = (1+2\parameters)\log(\parameters')
    + (1 + 2(1-\parameters))\log(1-\parameters')$ appears on the left,
    and the map $\EMmap(\parameters)$ appears on the right.  Note that
    $\parameters^* = \frac{1}{2}$ is the fixed point of $\EMmap$
    (where the plot intersects the slope 1 reference line) and it is stable
    because the slope of $\EMmap$ is less than one.}
  \label{fig:EM}
\end{figure}

\chapter{Variants and Generalizations}
\label{chap:variants}

\emph{Laser\_plots.py} uses pylab to make Fig.~\ref{fig:LaserHist} directly
from Tang's file \emph{data/LP5.DAT}.
\begin{figure}[htbp]
  \centering{\resizebox{\textwidth}{!}{\includegraphics{LaserHist.pdf}}
  }
  \caption[Histogram of Tang's laser measurements.]%
  {Histogram of Tang's laser measurements.  Even though neither $y=5$
    nor $y=93$ occurs in $\ts{y}{1}{600}$, it is more plausible that
    $y=93$ would occur in future measurements because of what happens
    in the neighborhood.  Discarding the numerical significance of the
    bin labels would preclude such an observation. }
  \label{fig:LaserHist}
\end{figure}

\section{Gaussian Observations}
\label{sec:gaussian}
\index{Gaussian observation}

\subsection{Independent Scalar Observations}
\label{sec:ScalarGaussian}

%%%
%%% fig:ScalarGaussian
%%%
\begin{figure}[htbp]
  \centering{\plotsize%
    \setlength{\unitlength}{1in}%
    \begin{tabular}[H]{cc}
      \begin{picture}(0,0)
        \put(-0.1,0.8){\makebox{\normalsize\textbf{(a)}}}
      \end{picture}%
      {\def\prba{$0.93$}%
        \def\prbb{$0.13$}%
        \def\prbc{$0.07$}%
        \def\prbd{$0.87$}%
        \def\lbla{\parbox[t]{1.8in}{$\mu=-1$\\$\sigma^2=1$}}%
        \def\lblb{\parbox[t]{1.8in}{$\mu=1$\\$\sigma^2=1$}}%
        \input{ScalarGaussian.pdf_t}
      }%
      %\smallskip%
      &
      \begin{picture}(0,0)
        \put(0.0,0.8){\makebox{\normalsize\textbf{(b)}}}
      \end{picture}%
      \hspace{2em}
      \resizebox{0.37\textwidth}{!}{\includegraphics{SGO_b.pdf}}
      %
      %\smallskip%
      \\
      \begin{picture}(0,0)
        \put(0.0,0.8){\makebox{\normalsize\textbf{(c)}}}
      \end{picture}%
      \hspace{2em}
      \resizebox{0.37\textwidth}{!}{\includegraphics{SGO_c.pdf}}%
      \smallskip%
      &
      \begin{picture}(0,0)
        \put(0.0,0.8){\makebox{\normalsize\textbf{(d)}}}
      \end{picture}%
      \hspace{2em}
      \resizebox{0.37\textwidth}{!}{\includegraphics{SGO_d.pdf}}%
      %\smallskip%
      \\
      \begin{picture}(0,0)
        \put(-0.1,0.8){\makebox{\normalsize\textbf{(e)}}}
      \end{picture}%
      {\def\prba{$0.5$}%
        \def\prbb{$0.5$}%
        \def\prbc{$0.5$}%
        \def\prbd{$0.5$}%
        \def\lbla{\parbox[t]{1.8in}{$\mu=-2$\\$\sigma^2=2$}}%
        \def\lblb{\parbox[t]{1.8in}{$\mu=2$\\$\sigma^2=2$}}%
        \input{ScalarGaussian.pdf_t}
      }&
      \begin{picture}(0,0)
        \put(-0.1,0.8){\makebox{\normalsize\textbf{(f)}}}
      \end{picture}%
      {\def\prba{$0.92$}%
        \def\prbb{$0.12$}%
        \def\prbc{$0.08$}%
        \def\prbd{$0.88$}%
        \def\lbla{\parbox[t]{1.8in}{$\mu=-0.74$\\$\sigma^2=1.09$}}%
        \def\lblb{\parbox[t]{1.8in}{$\mu=1.17$\\$\sigma^2=1.27$}}%
        \input{ScalarGaussian.pdf_t}
      }
      \end{tabular}}
    \caption[An HMM with scalar Gaussian
    observations.]%
    {An HMM with scalar Gaussian observations.  A state diagram
      appears in \emph{(a)}.  The half-life of the first state is
      about ten and the half life of the second state is about five,
      \ie, $0.93^{10} \approx 0.87^5 \approx 0.5$.  A simulated state
      sequence and observation sequence appear in \emph{(b)} and
      \emph{(c)} respectively.  Using the model parameters from
      \emph{(a)} and the observation sequence from \emph{(c)}, the
      Viterbi algorithm estimates the state sequence that appears in
      \emph{(d)} which is satisfyingly similar to the state sequence
      in \emph{(b)}.  Finally, starting from the initial model
      depicted in \emph{(e)} and using the observation sequence
      depicted in \emph{(c)}, 50 iterations of the Baum-Welch
      algorithm produces the model depicted in \emph{(f)} which is
      satisfyingly similar to \emph{(a)}.}
  \label{fig:ScalarGaussian}
\end{figure}

\subsection{Singularities of the likelihood function and regularization}
\label{sec:regularization}

Figure~\ref{fig:MLEfail} is an XFig drawing.
\begin{figure}[htbp]
  \centering{\plotsize%
    \setlength{\unitlength}{1in}%
    \begin{tabular}[H]{cc}
      \begin{picture}(0,0)
        \put(-0.1,0.8){\makebox{\normalsize\textbf{(a)}}}
      \end{picture}%
      {\def\prba{$0.5$}%
      \def\prbb{$0.5$}%
      \def\prbc{$0.5$}%
      \def\prbd{$0.5$}%
      \def\lbla{\parbox[t]{1.8in}{$\mu=0.0$\\$\sigma^2=16$}}%
      \def\lblb{\parbox[t]{1.8in}{$\mu=3.6$\\$\sigma^2=0.016$}}%
      \input{ScalarGaussian.pdf_t}
      }&
         \begin{picture}(0,0)
           \put(-0.1,0.8){\makebox{\normalsize\textbf{(b)}}}
         \end{picture}%
         {\def\prba{$0.99$}%
         \def\prbb{$1.0$}%
         \def\prbc{$0.01$}%
         \def\prbd{$0.0$}%
         \def\lbla{\parbox[t]{1.8in}{$\mu=-0.09$\\$\sigma^2=1.90$}}%
      \def\lblb{\parbox[t]{1.8in}{$\mu=3.62$\\$\sigma^2=1.7\times 10^{-6}$}}%
      \input{ScalarGaussian.pdf_t}
      }%
    \end{tabular}}%
  \caption[An illustration of trouble with
  maximum likelihood.]%
  {An illustration of trouble with maximum likelihood.  Here I have
    used the same implementation of the Baum-Welch algorithm that I
    used to produce Fig.~3.2\emph{(f)}, but
    rather than starting with the model in
    Fig.~3.2\emph{~(c)}, I started the algorithm with
    the initial model depicted in (a) above.  Six iterations of the
    algorithm produced the suspicious model depicted in \emph{(b)} above.}
  \label{fig:MLEfail}% numbers from ScalarGaussian.py
\end{figure} \marginpar{FixMe: Two rather than six.}

\section{Related Models}
\label{sec:related}
It takes 11 minutes for \emph{VStatePic.py} to make the state data for
Fig.~\ref{fig:VARGstates} from \emph{data/lorenz.xyz}.
\begin{figure}[p]
  \centering{\plotsize%
    \includegraphics[width=1.0\textwidth]{VARGstates.pdf}
  }
  \caption[Vector autoregressive observation models.]%
  {Plots of decoded states using an HMM with vector autoregressive
    observations.  Here the observations are a trajectory of three
    dimensional state vectors from the Lorenz system.  In each state
    the observation $\ti{y}{t}$ is modeled as a Gaussian with a mean
    that is an affine (linear plus fixed offset) function of the
    observation $\ti{y}{t-1}$.  The empty boxes correspond to states
    that do not appear in the decoded sequence of states.  In
    comparing with Fig.~1.9 which used a model with
    coarsely quantized observations, notice that large regions near
    the fixed points at centers of the spirals are represented by a
    single state.  These large regions occur because the dynamics are
    approximately linear over their extents.}
  \label{fig:VARGstates}
\end{figure}

\chapter[Continuous States and Observations]{Continuous States and
  Observations and Kalman Filtering}
 \label{chap:continuous}
 % \addtocounter{chapter}{1} % Skip chapter without figures
 
\section{Backwards Information Filter}
\label{sec:BackwardsInformation}

I had trouble doing smoothing on simulated Lorenz data.  The dynamics
are nonlinear both forward and backwards, and they are extremely
unstable going backwards.  I was able to do it by keeping track of
inverse state covariances and state means.  I didn't find that exact
procedure anywhere else.  I use the following notation for inverse
covariances (called \emph{information matrices}):
\begin{align}
  \Phi_b &\equiv \Sigma_b^{-1} && \text{Information at time $t$
                                  updated with $y[t]$}\\
  \Phi_\beta &\equiv \Sigma_\beta^{-1} && \text{Information at time
                                          $t$ backcast from $t+1$
                                          before update}                             
\end{align}
The $\beta$ parameters characterize the \emph{backcast} distribution,
$x[t]|y[t+1:] \sim \Normal\left(\mu_\beta[t], \Phi_\beta^{-1}[t]\right)$. 
And the $b$ parameters characterize the \emph{updated} distribution,
$x[t]|y[t:] \sim \Normal\left(\mu_b[t], \Phi_b^{-1}[t]\right)$. 
The iteration starts at the final time $T$ with the initial information
set to zero.  And I set the initial mean to the estimate
from the forward pass through the data:
\begin{align}
  \label{betaT_cheat}
  \mu_\beta[T] &= \mu_\alpha[T] \\
  \Phi_\beta[T] &= 0
\end{align}
While Eqn.\eqref{betaT_cheat} seems to illegitimately use information
from the forward pass in the backward pass, the value of
$\Phi_\beta[T]$ makes the initial prior completely uninformative.
Because the Lorenz system is extremely unstable backwards in time, if
the backwards pass starts or ever gets very far off of the attractor
the integration diverges.

I will first derive the recursion formulas for the stationary linear
case where the notation is easier.
\begin{align}
  \Phi_b &= \Phi_\beta + G^T \Sigma_\epsilon G && \text{Equation
                                                  4.32a} \\
  \mu_b &= \mu_\beta + \Sigma_b G^T \Sigma_\epsilon^{-1} \left[y -
          G\mu_\beta \right] && \text{Equation 4.32b}\\
  \label{eq:linear_mu_b}
  &= \mu_\beta + \Phi_b^{-1} G^T \Sigma_\epsilon^{-1} \left[ y - G
    \mu_\beta \right] \\
  &\equiv \mu_\beta + z && \text{Where $z$ is the least square
                           solution of} \\
  \Phi_bz &= G^T \Sigma_\epsilon^{-1} \left[ y - G \mu_\beta \right]
\end{align}

Letting $F$ denote the system map backwards in time, ie, $x[t] = F
x[t+1]$, I derive the backcast formulas I use as follows:
\begin{align}
  \label{eq:linear_mu_beta}
  \mu_\beta &= F \mu_b && \text{Equation 4.33a}\\
  \label{sumInside}
  \Phi_\beta^{-1} &= F \left( \Sigma_\eta + \phi_b^{-1} \right) F^T &&
                                                                  \text{Equation
                                                                  4.33b}
  \\
  \Phi_\beta &= {F^T}^{-1}  \left( \Sigma_\eta + \phi_b^{-1}
               \right)^{-1} F^{-1} \\
  \label{singularPhib}
  \left(\Sigma_\eta + \phi_b^{-1} \right)^{-1} &= \left( \id + \Phi_b
                                                 \Sigma_\eta
                                                 \right)^{-1} \Phi_b
                       && \text{Verify by multiplying} \\
  \Phi_\beta &= {F^T}^{-1} \left( \id + \Phi_b \Sigma_\eta
               \right)^{-1} \Phi_b F^{-1}
\end{align}
Notes: (1) The advantage of the right hand side of \eqref{singularPhib}
over the left is that $\Phi_b$ may be singular while $\left( \id +
  \Phi_b \Sigma_\eta \right)$ will be invertible. (2) In the
forecast, the state noise covariance, $\Sigma_\eta$, is added outside of
the multiplication by $F$, while it is inside in the backcast, \eqref{sumInside},
because in the model assumes state noise is added \emph{after} the transition
from $t$ to $t+1$.

\subsection{Nonlinearities}
\label{sec:backNonlinear}

Suppose that the state map and observation maps are not linear and
have affine approximations:
\begin{align}
  f(x+\delta) &= f(x) + F\cdot \delta \\
  g(x+\delta) &= g(x) + G \cdot \delta
\end{align}
Then the formulas for the informations $\Phi_b$ and $\Phi_\beta$ are
not changed, but the equations for $\mu$ become:
\begin{align}
  \mu_\beta &= f(\mu_b) \\
  \mu_b &= \mu_\beta + \Phi_b^{-1} G^T \Sigma_\epsilon^{-1} \left[ y -
          g(\mu_\beta) \right]
\end{align}


\chapter[Performance Bounds]{Performance Bounds and a Toy Problem}
\label{chap:toys}
\index{toy problems}

\subsection{Lorenz System}
\label{sec:lorenz_system}

Here is the Lorenz system:
\begin{align*}
  \dot x &\equiv f(x) && \text{with components} \\
  \dot x_0 &= s x_1 - s x_0 \\
  \dot x_1 &= -x_0 x_2 + r x_0 - x_1 \\
  \dot x_2 &= x_0 x_1 - b x_2
\end{align*}
Solving for fixed points
\begin{align*}
  x_1 &= x_0 \\
  x_0(r-1-x_2) &=0 \\
  x_0^2 &= b(r-1) &&\text{with solutions}\\
  x_0, x_1, x_2 &= 0,0,0 &&\text{or}\\
  x_0 &= x_1 = \pm \sqrt{b(r-1)}, \quad x_2 = r-1.
\end{align*}
The derivative of $f$ is
\begin{equation*}
  \frac{d}{dx} f(x) \equiv F(x) = 
  \begin{bmatrix}
    -s    & s   & 0 \\
    r-x_2 & -1  & -x_0 \\
    x_1   & x_0 & -b
  \end{bmatrix}
\end{equation*}
I use the derivative to integrate the tangent equation
\begin{equation*}
  \frac{d}{dt} \frac{d x[t]}{d x[0]} = F(x[t]) x[t].
\end{equation*}
I also evaluate the derivative at the fixed point with $x_0>0$ finding
\begin{equation*}
  F(x_{\text{fixed}}) =
  \begin{bmatrix}
    -s            &   s           & 0 \\
    1             &  -1           & -\sqrt{b(r-1)} \\
    \sqrt{b(r-1)} & \sqrt{b(r-1)} & -b
  \end{bmatrix}
\end{equation*}
A spectral decomposition of that derivative gives the tangent to the
unstable manifold at the fixed point.

Wikipedia says that the foci are unstable for
\begin{equation*}
  r > s \frac{s+b+3}{s-b-1}.
\end{equation*}
With the values $s=10$ and $b=8/3$ the threshold is
\begin{equation*}
  r_{\text{bifurcation}} = 10\frac{47}{19} \approx 24.7368
\end{equation*}

My numerical experiments show that for
$17 < r < r_{\text{bifurcation}}$ there an unstable periodic orbit at
finite amplitude.

\subsubsection{Lorenz Example}

I make Figs.~\ref{fig:ToyTS1} and~\ref{fig:ToyStretch} from the files
\emph{Save\_Hview\_T\_100}, \emph{Save\_Hview\_T\_118}, and
\emph{Save\_Hview\_T\_119} which I made using the old version of GUI
program \emph{Hview.py}.  You can use the new version to make similar
files and figures, but I like the way the old ones look.  The script
\emph{ToyA.py} makes the figures from the data.
\begin{figure}[htbp]
  \centering{\resizebox{\textwidth}{!}{\includegraphics{ToyTS1.pdf}}
  }
  \caption[Extended Kalman filter for one step forecasting.]%
    {Extended Kalman filter for one step forecasting with simulation
    parameters:\\
    \begin{tabular}[c]{ll}
       $\Tsamp=0.25$ & Sample interval \\
       $\sigma_\eta = 10^{-6}$ & Standard deviation of state noise \\
       $\sigma_\epsilon = 0.01$ & Standard deviation of measurement
       noise \\
       $\Delta = 10^{-4}$ & Measurement quantization \\
     \end{tabular}\\
    A time series of observations appears in the upper plot.  The
    middle plot characterizes the one-step forecast distributions
    $P_{\gamma} \left(\ti{y}{t} \right) \equiv P
    \left(\ti{y}{t}|\ts{y}{1}{t-1},\theta \right) =
    \NormalE{\ti{\mu_\gamma}{t}}{\ti{\sigma^2_\gamma}{t}}{\ti{y}{t}}$;
    the first trace is the standard deviations of the forecasts and
    the second trace is the difference between the actual observation
    and the mean of the forecast.  The logs of the likelihoods of the
    forecasts, $\log(P_{\gamma} \left(\ti{y}{t} \right))$, appear in
    the bottom plot.}
  \label{fig:ToyTS1}
\end{figure}
%%%
\begin{figure}[htbp]
  \centering{\resizebox{\textwidth}{!}{\includegraphics{ToyStretch.pdf}}
  }
  \caption[Dynamical stretching increases state variance.]%
  {These plots illustrate dynamical stretching increasing the variance
    of the conditional distribution in state space corresponding to
    time steps 118 and 119 in Fig.~5.1.  In each plot, the larger
    ellipse represents the \emph{forecast} state distribution $P_{a}
    \left(\ti{x}{t} \right) \equiv P
    \left(\ti{x}{t}|\ts{y}{1}{t-1},\theta \right) =
    \NormalE{\mu_a}{\Sigma_a}{\ti{x}{t}}$ and the smaller ellipse
    represents the \emph{updated} state distribution $P_{\alpha}
    \left(\ti{x}{t} \right) \equiv P \left(\ti{x}{t}|\ts{y}{1}{t},
      \theta \right) =
    \NormalE{\mu_\alpha}{\Sigma_\alpha}{\ti{x}{t}}$.  For each
    distribution, an ellipse depicts the level set $(x-\mu)\transpose
    \Sigma^{-1} (x-\mu) =1$ in the $x_1\times x_3$ plane.  Since the
    observations provide information about the value of $x_1$, the
    updated distributions vary less in the $x_1$ direction than the
    corresponding forecasts.  To aid comparisons, the $x_1$ range is
    $0.2$ and the $x_3$ range is $0.01$ in each plot.  In the $x_1$
    direction, the standard deviation of the updated distribution
    $P_{\alpha}(\ti{x}{t})$ at $t=118$ (the smaller of the two
    ellipses on the left) is 0.007.  The dynamics map that
    distribution to the forecast distribution $P_{a}(\ti{x}{t})$ at
    $t=119$ (the larger of the two ellipses on the right) for which
    the standard deviation in the $x_1$ direction is more than ten
    times larger.}
  \label{fig:ToyStretch}
\end{figure}

\begin{figure}[htbp]
  \centering{\resizebox{\textwidth}{!}{\includegraphics[trim = 45mm
      20mm 25mm 10mm ]{ToyH.pdf}} }
  \caption[Average log likelihood of one step forecasts.]%
  {Average log likelihood of one step forecasts as a function of time
    step $\Tsamp$ and filter parameter $\tilde \sigma_\epsilon$.  To
    simulate measurements for this figure, I used the parameters:\\
    \begin{tabular}[c]{ll}
     $\sigma_\eta = 10^{-6}$ & Standard deviation of state noise \\
     $\sigma_\epsilon = 10^{-10}$ & Standard deviation of measurement noise \\
     $\Delta = 10^{-4}$ & Measurement quantization \\
     $T=5,000$ & Number of samples
   \end{tabular}\\
   For both plots, the vertical axis is the average log likelihood of
   the one-step forecast $-\hat h \equiv \frac{1}{T} \sum_{t=1}^T
   \log \left( P \left(\ti{y}{t}|\ts{y}{1}{t-1}, \theta
     \right)\right)$.  On the left I plot $-\hat h$ as a function of
   both $\Tsamp$, the time step, and $\tilde \sigma_\epsilon$, the
   standard deviation of the measurement noise model used by the
   Kalman filter.  On the right ``$\circ$''%% Top row of dots
   indicates the
   performance of filters that use measurement noise models that depend on the
   sampling time through the formula $\tilde \sigma_\epsilon(\Tsamp) =
   10^{0.4 \Tsamp -4.85}$, which closely follows the ridge top in the plot
   on the left, ``$\diamond$''%% Bottom row of dots
   indicates the performance of
   filters that use $\tilde \sigma_\epsilon = 10^{-4}$, \ie the
   measurement quantization level, and the solid line traces
   Eqn.~5.2 in the text.}
   \label{fig:ToyH}
   % Build Hsurvey, HtauS and ToyH.pdf on watcher: time scons figs/ToyH.pdf
   % real    150m1.812s
   % user    150m1.399s
\end{figure}
%\afterpage{\clearpage}%% Print this right here please.

% \section{Fidelity Criteria and Entropy}
% \label{sec:fidelity}

% \section{Stretching and Entropy}

% \section{Lyapunov Exponents and Pesin's Formula}
% \label{sec:PesinFormula}
\addtocounter{section}{3}
\section{Benettin's Procedure for Calculating Lyapunov Exponents Numerically}
\label{sec:Benettin}

Figure~\ref{fig:QR} is a boring collection of straight \LaTeX ~ and
three XFig drawings.
\begin{figure}[htbp]
  \centering{\plotsize%
    \def\Mone{$ \begin{bmatrix} \begin{bmatrix} e_1
        \end{bmatrix} & \begin{bmatrix} e_2 \end{bmatrix} \end{bmatrix} $}%
    \def\Mtwo{$ R \begin{bmatrix} \begin{bmatrix} e_1
        \end{bmatrix} & \begin{bmatrix} e_2 \end{bmatrix} \end{bmatrix} $}%
    \def\Mthree{$ QR \begin{bmatrix} \begin{bmatrix} e_1
        \end{bmatrix} & \begin{bmatrix} e_2 \end{bmatrix} \end{bmatrix} $}%
    \def\Mfour{$ \begin{bmatrix} 1 & 0 \\ 0 & 1 \end{bmatrix} $}%
    \def\Mfive{$ \begin{bmatrix} 5 & 0.15 \\ 0 & 0.2 \end{bmatrix} $}%
    \def\Msix{$ \begin{bmatrix}3 & 0.25 \\ -4 & 0 \end{bmatrix} $}%
     \input{QR.pdf_t}
  }
  \caption[The action of the $Q$ $R$ factors of a matrix on a unit square.]%
  {The action of the $Q$ $R$ factors of a matrix on a unit square.
    Here $A=
    \begin{bmatrix} 3 & 0.25 \\ -4 & 0 \end{bmatrix}$, $Q=
    \begin{bmatrix} 0.6 & 0.8 \\ -0.8 & 0.6 \end{bmatrix}$, and $R=
    \begin{bmatrix} 5 & 0.15 \\ 0 & 0.2 \end{bmatrix}$.  $R$ stretches
    the $x$ component by a factor of five and shears $y$ components in
    the $x$ direction and shrinks them by a factor of five with a net
    effect of preserving areas.  $Q$ simply rotates the stretched
    figure.  Each parallelepiped in the bottom row is constructed from
    the columns of the corresponding matrix in the middle row.  The
    algebraic formulas for those vectors appear in the top row.  Note
    that $R$ determines the changes in length and area, and that $Q$
    does not effect either.}
  \label{fig:QR}
\end{figure}

\begin{figure}[htb]
  \centering{\includegraphics[width=1.0\textwidth]{benettin.pdf}
  }
  \caption[Lyapunov exponent calculation for
  the Lorenz system.]%
  {Lyapunov exponent calculation for the Lorenz system.  In the upper
    part, the three lighter traces are plots of $\frac{1}{T}
    \sum_{t=1}^T \log\left( \left| \ti{r_{1,1}}{t} \right| \right)$
    and the heavier traces the 5\% and 95\% limits on 1,000 separate
    runs.  The lower part is the same except that $\left|
      \ti{r_{1,1}}{t} \right|$ is augmented by a noise term with
    amplitude $\frac{\sigma_\eta}{\Delta} = 0.01$ (See
    Eqn.~5.49.  The shapes of the traces are almost
    unchanged except for uniform shift up of about 0.03.  I conclude
    that a test model that is the same as the generating model except
    that the state noise is $\frac{\sigma_\eta}{\Delta} = 0.01$ would
    have a cross entropy of about 0.936 nats, while the largest Lyapunov
    exponent is $\hat \lambda_1 \approx 0.906$ nats.}
  \label{fig:benettin}
\end{figure}

% \section{A Practical Performance Bound}
% \label{sec:PracticalBound}
\addtocounter{section}{1}
\section{Approaching the Bound}
\label{sec:approach}

Fig.~\ref{fig:LikeLor} from the data.  Note: To make this figure you
must have at least two gigabytes of RAM.
\begin{figure}[htbp]
  \centering{\resizebox{\textwidth}{!}{\includegraphics{LikeLor.pdf}}
  }
  \caption[Entropy gap, $\hat
  \delta_{\mu||\theta}$ vs number of states in HMMs]%
  {Entropy gap, $\hat \delta_{\mu||\theta}$ vs number of states in
    HMMs.  The upper trace plots estimates of cross entropy $\hat
    h(\mathcal{B},F,\mu||\theta)$ for a sequence of HMMs vs the number
    of discrete state in the models.  I built the models using actual
    Lorenz state space trajectories as described in the text.  The
    lower trace is an estimate of the entropy rate, $\hat h(F,\mu)) =
    \hat \lambda_1$, of the true process based on Lyapunov exponents
    estimated by the Benettin procedure.  The distance between the
    curves is the \emph{entropy gap} $\hat \delta_{\mu||\theta}$. The
    gap seems to be going to zero, suggesting that an HMM with enough
    states might perform at least as well as any other model based on
    any other technology.  Each model was built using the same sample
    trajectory of 8,000,000 points in the original state space, and
    the cross entropy estimates are based on a test sequence of 10,000
    observations.}
  \label{fig:LikeLor}
\end{figure}

\chapter{Obstructive Sleep Apnea}
\label{chap:apnea}

\section{The  Challenge and the Data}
\label{sec:challenge}

\subsection{The Data}
\label{sec:data}
\begin{figure}
  \centering{\resizebox{\textwidth}{!}{\includegraphics{a03erA.pdf}}
  }
  \caption[A segment of record a03]%
  {A segment of record a03.  Two cycles of a large apnea
    induced oscillation in $SpO_2$ are drawn in the lower plot.  The
    middle plot is the oronasal airflow signal, and the upper plot is
    the ECG (units of both ONR and ECG are unknown).  The time
    axis is marked in \emph{hours:minutes}.  Notice the increased
    heart rate just after 0:58 and just before 0:59.}
  \label{fig:a03erA}
\end{figure}

\begin{figure}
  \centering{\resizebox{\textwidth}{!}{\includegraphics{a03erN.pdf}}
  }
  \caption[A segment of record a03]%
  {A segment of record a03 taken during a period of normal
    respiration.  Signals the same as in Fig.~6.1.}
  \label{fig:a03erN}
\end{figure}

\begin{figure}
  \centering{\resizebox{\textwidth}{!}{\includegraphics{a03HR.pdf}}
  }
  \caption[A segment of record 03 at the end of an episode of apnea]%
  {A segment of record 03 at the end of an episode of apnea with
    indications in both the $SpO_2$ signal and the heart rate
    \emph{(HR)} signal.  The expert marked the time before 1:00 as
    apnea and the time afterwards as normal.}
  \label{fig:a03erHR}
\end{figure}

\section{First Classification Algorithms and Two Useful Features}
\label{sec:NVG}

% \subsection{Using Information from Experts to Train}
\addtocounter{subsection}{1}
\subsection{Nonlinear Dynamics}
\label{sec:NLD}

\begin{figure}
  \centering{\resizebox{\textwidth}{!}{\includegraphics{ApneaNLD.pdf}}
  }
  \caption[Nonlinear effects]%
  {Nonlinear effects: The upper plot seems to be a period two
    oscillation.  The lower plot is approximately sawtooth.}
  \label{fig:ApneaNLD}
\end{figure}

\subsection{The Excellent Eye of Dr.\ McNames}
\label{sec:mcnames}

\begin{figure}
  \centering{\resizebox{\textwidth}{!}{\includegraphics{sgram.pdf}}
  }
  \caption[Information about respiration in high
  frequency phase variations]%
  {Information about respiration in high frequency phase variations.
    This is the $a11$ record roughly between minutes 40 and 225.  The
    upper plot is heart rate (bandpass filtered 0.09-3.66 cpm), the
    middle plot is a spectrogram of the phase jitter in the heart
    rate, and the lower plot is the expert classification.  A single
    band of spectral power between about 10 and 20 cpm without much
    power below the band in the spectrogram indicates normal
    respiration.}
  \label{fig:sgram}
\end{figure}
\afterpage{\clearpage} % Flush float
\newpage
\section{Decoding Sequences of Classifications}
\label{sec:V4Class}

In the book I presented the algorithm described in
Fig.~\ref{fig:viterbiC} for finding the \emph{best} classification
sequence,
\begin{equation*}
  \hat c_1^T \equiv \argmax_{c_1^T} P(y_1^T,c_1^T),
\end{equation*}
given an observation sequence $y_1^T$.  I claimed that the number of
computations required is a linear function of $T$.  In 2013, as I was
developing a test suite for a new version of the software, I
discovered a configuration for which the algorithm failed to find a
\emph{possible} classification sequence even though it was using the
model that generated the test data.

Examining the failure, I discovered that for each of the ``best''
class histories ending in each of the classes at a particular time
$t$, the conditional probability of a particular state $s$ given that
class history was 0.  However, at time $t+1$, the only state that
could produce $y(t+1)$ required that $S(t)=s$.  Thus the algorithm
calculated $P(y_1^{t}, \hat c_1^{t}) = 0$ and died.

I had designed the algorithm thinking that the class sequence was
Markov and that future evidence could not change what was the best
history leading to any class.  The following drawing makes my error
obvious.
\begin{center}
  \resizebox{0.75\columnwidth}{!}{\input{class_net.pdf_t}}
\end{center}
Blocking out $s_{t+1}$ separates the past from the future, but
blocking out $c_{t+1}$ doesn't.

Given measurements up to time step $t$, for any class history $c_1^t$,
one can calculate a \emph{score} or value of
\begin{equation*}
  \psi(s,c_1^t) \equiv P(y_1^t, s(t) = s, c_1^t)
\end{equation*}
for every possible value $s$ of the ending state $s(t)$.  I define
\emph{complete domination} of one class history $^a c_1^t$ by a
second $^b c_1^t$ as
\begin{equation*}
  \psi(s, ^b \!c_1^t) \geq \psi(s, ^a\! c_1^t),~ \forall s,
\end{equation*}
which I denote $^b \! c_1^t \preceq \,^a\!c_1^t$.  Because a completely
dominated class history will never be required as a part of a best
class sequence, one can safely discard them as one makes a forward
pass through a data set.  However, to ensure optimality one must
retain and propagate every partial history that is not completely
dominated by another retained partial history.

One might hope that for a class with a small number of states, that a
small number of histories would completely dominate all others.  For a
class with only two states, the figure below illustrates both that
hope and the fact that it is only a \emph{hope} and not a necessity.
To ensure optimality one may need to keep a number of class histories
that is exponential in sequence length $t$.  That is too many to
retain for most realistic applications.

\begin{center}
%  \resizebox{\textwidth}{!}{\input mono.pdf_t }\medskip\\
  \parbox{0.9\textwidth}{ In the drawings above, each dot represents a
    different class history and for that class history the coordinates
    of the dot are $\psi(s_1, c_1^t)$ and $\psi(s_2, c_1^t)$.  The
    drawing on the left illustrates \emph{complete dominance}.  The
    class history corresponding to the upper right dot completely
    dominates all the other class histories and it is the only one
    that must be retained and propagated.  In the drawing on the
    right, no class history is completely dominated, and to ensure
    finding an optimal class sequence each history must be retained
    and propagated.  For even two states in a class there is no limit
    on the number of class histories that one may have to retain to
    ensure finding an optimal class sequence.}
\end{center}

\subsection{Finding a \emph{Pretty Good Class} Sequence}
\label{sec:prettygood}

Rather than finding the \emph{best} class sequence given a sequence of
observations $y_1^T$, I now seek a \emph{good} sequence.  In a forward pass
through the data sequence, at each $t$ I retain a class history for
each ending class and a class history for each state.  Specifically, I
retain the set:
\begin{equation*}
  \left\{ \argmax_{c_1^t:c_1^{t-1}\text{retained}}
    \psi(s,c_1^t), ~ \forall s \right\} \cup
  \left\{ \argmax_{c_1^t:c(t)=c\& c_1^{t-1}\text{retained}}
    \psi(s,c_1^t), ~ \forall c \right\}.
\end{equation*}
Future versions of this document will report on the performance of
this algorithm and perhaps on other approaches.


%
\begin{figure}[htbp]
  \begin{center}
    \newcommand{\tnext}{_{\text{next}}}
    \newcommand{\told}{_{\text{old}}}
    \newcommand{\tbest}{_{\text{best}}}
    \newcommand{\ick}{\log \left(\sum_{s} g(s,c\tbest) f(t+1,s,c\tbest) \right)}
    \fbox{
      \begin{minipage}{\columnwidth}
        \begin{tabbing}
          XX\=XX\=XX\=XX\=XX\=XX\=XX\=XX\= \kill
          Initialize: \> \+ \\
          for each $c$\\ \> \+
          $\nu\tnext (c) = \log \left( \sum_{s} g(s,C) P_{\ti{Y}{1},\ti{S}{1}}
            \left(\ti{y}{1},s \right)\right)$ \\ \\ \< \- \< \-
          Iterate: \> \+ \\
          for $t$ from 1 to $T$\\ \> \+
          Swap $\nu\tnext \leftrightarrow \nu\told$\\
          for each $c\tnext$\\ \> \+
          \\ \# Find best predecessor\\
          $c\tbest = \argmax_{c\told}\left( \nu\told(c\told) + \ick \right)$ \\
          \\ \# Update $\nu$\\
          $\nu\tnext(c\tnext) =\,$ \= $\nu\told(c\tbest) + \ick$ \\ \> \-
          \\ \# Update predecessor array\\
          Predecessor[$c\tnext,t$] = $c\tbest$\\
          \\ \# Update $\phi$\\
          for $s$ in $c\tnext$\\ \> \+
          Assign $\phi\tnext(s,c\tnext)$ using Eqn.~6.7\\ \\
          \< \- \< \- \< \- %This stuff is for tabs \< \-
          Backtrack: \> \+ \\
          $\ts{c}{1}{t} = \ts{\hat c}{1}{t}(\bar c)$ , where $\bar c =
          \argmax_c \nu\tnext(c)$ at $t=T$
        \end{tabbing}
      \end{minipage}
    }
    \caption[\textbf{This algorithm does not work!}  In the book, this
    figure was entitled \emph{Pseudocode for the Viterbi Algorithm}.]%
    {\textbf{This algorithm does not work!}  In the book, this figure
      was entitled \emph{Pseudocode for the Viterbi algorithm for
        class sequences}.}
    \label{fig:viterbiC}
  \end{center}
\end{figure}

\section{Assembling the Pieces}
\label{sec:Pieces}

\subsection{Extracting a Low Pass Filtered Heart Rate}
\label{sec:LPHR}

\subsection{Extracting Respiration Information}
\label{sec:RESP}

\begin{figure}
  \centering{
    \resizebox{0.65\textwidth}{!}{\includegraphics{LDA1.pdf}}
    \resizebox{0.3\textwidth}{!}{\includegraphics{LDA2.pdf}}
  }
  \caption[Linear discriminant analysis of phase
  jitter periodograms]%
  {Linear discriminant analysis of phase jitter periodograms.  The
    plot in the upper left, shows the following mean periodograms:
    $\mu_C$, the mean for the \emph{c} records; $\mu_{N}$, the mean of
    the minutes that the expert classified as normal in the $a$
    records; and $\mu_{A}$, the mean of the minutes that the expert
    classified as apnea in the $a$ records.  The lower left plot shows
    the basis vectors that result from the linear discriminant
    analysis.  Scatter plots of the three classes projected on the
    basis $(v_1,v_2)$ appear on the right.}
  \label{fig:LDA}
\end{figure}


\subsection{Classifying Records}
\label{sec:ClassRec}

\begin{figure}[tbhp]
  \centering{\resizebox{\textwidth}{!}{\includegraphics{pass1.pdf}}
  }
  \caption[The first pass classifier]%
  {The first pass classifier.  I've plotted the location of each
    record using the log likelihood ratio $llr$ and the ratio
    statistic $R$.  Records to the left of the line $2.39-\frac{llr}{2}$
    are in the $L$ group.  Records to the right of the line
    $2.55-\frac{llr}{2}$ are in the $H$ group.  And those in between are
    in the $M$ group.}
  \label{fig:pass1}
\end{figure}

\subsection{Model Topology and Training Data}
\label{sec:topology}

Figure~\ref{fig:structure} is a simple XFig drawing.
\begin{figure}
  \centering{\plotsize%
    \def\Azero{$A_H$}%
    \def\Aone{$A_{I1}$}%
    \def\Atwo{$A_{I2}$}%
    \def\Athree{$A_{P11}$}%
    \def\Afour{$A_{P12}$}%
    \def\Afive{$A_{P21}$}%
    \def\Asix{$A_{P22}$}%
    \def\Nzero{$N_H$}%
    \def\None{$N_{I1}$}%
    \def\Ntwo{$N_{I2}$}%
    \def\Nthree{$N_{I3}$}%
    \def\Nfour{$N_{I4}$}%
    \def\Nfive{$N_{P11}$}%
    \def\Nsix{$N_{P12}$}%
    \def\azero{$A_H$}%
    \def\aone{$A_{I1}$}%
    \def\atwo{$A_{P11}$}%
    \def\athree{$A_{P12}$}%
    \def\nzero{$N_H$}%
    \def\none{$N_{I1}$}%
    \def\ntwo{$N_{I2}$}%
    \def\nthree{$N_{I3}$}%
    \def\nfour{$N_{P11}$}%
    \def\nfive{$N_{P12}$}%
    \def\ModH{\large$Mod_H$}%
    \def\ModL{\large$Mod_M$ and $Mod_L$}%
%    \input{structure.pdf_t}
  }
  \caption[Structure of HMMs for minute by minute classification]%
  {Structure of HMMs for minute by minute classification in the second
    pass of my procedure.  I used the structure on the left for those
    records classified as $H$ on the first pass and the structure on
    the right for those records classified as $M$ or $L$ on the first
    pass.}
  \label{fig:structure}
\end{figure}

\subsection{Tunable Parameters}
\label{sec:tune}
%
It takes a long time make the data for Fig.~\ref{fig:PFsurvey}.  To
let readers avoid that delay, I've included the data in the source
package.  If you want to build the data, simply remove
\emph{data/PFsurveyH}, make \emph{data/PFsurveyH} and copy it to
\emph{data/PFsurvey}.
\begin{figure}
  \centering{
    \resizebox{1.0\textwidth}{!}{
%      \includegraphics[trim = 40mm 10mm 10mm 10mm ]{pf_H.pdf}
    }
  }
  \caption[The response of classification performance to changes in
  \emph{Pow} and \emph{Fudge}]%
  {The response of classification performance to changes in \emph{Pow}
    and \emph{Fudge}.  I've plotted the performance of $Mod_H$ trained
    and evaluated on the \emph{H} group of records.  As described in
    the text, \emph{Pow} governs the relative weighting of the low
    pass heart rate signal to the respiration characteristics and
    \emph{Fudge} is a bias for choosing the \emph{normal}
    classification.  The $Z$ axis is the fraction of minutes
    classified correctly.  It takes about 3.75 days to make the data
    for this plot. }
  \label{fig:PFsurvey}
\end{figure}

\subsection{Results}
\label{sec:results}

The results in Table~\ref{tab:result1} are part of the typed in \LaTeX
source.  I should write code that creates a file of results and then
import the results into the text.
\begin{table}
  \caption[Performance with tuned values of \emph{Fudge} and \emph{Pow} on training]%
  {Performance with tuned values of \emph{Fudge} and
    \emph{Pow} on training records.  I've sorted the list in order of
    how well the code classified each of the minutes in each record.  For
    each record, the number in the column labeled $N\rightarrow A$ is
    the number of minutes labeled as \emph{normal} by the expert that
    the code labeled as \emph{apnea}.  The interpretations of the
    other columns are similar.}
  \centering{\plotsize%
    \begin{tabular}{|llllll|}
      \hline
      Record & $N\rightarrow N$ &  $N\rightarrow A$ &  $A\rightarrow N$ &  $A\rightarrow A$ & \% Right \\
      \hline
      \rule{0pt}{2.0ex}%
      a11 &  198 &   46 &  138 &   84 & 0.6052 \\
      b02 &  255 &  169 &   14 &   79 & 0.6460 \\
      a06 &  276 &   27 &  140 &   66 & 0.6719 \\
      a08 &  197 &  114 &   24 &  165 & 0.7240 \\
      b01 &  362 &  105 &    2 &   17 & 0.7798 \\
      a07 &   96 &   93 &   12 &  309 & 0.7941 \\
      a18 &   37 &   14 &   82 &  356 & 0.8037 \\
      b03 &  296 &   71 &   13 &   60 & 0.8091 \\
      a03 &  175 &   98 &    0 &  246 & 0.8112 \\
      a20 &  184 &   10 &   78 &  237 & 0.8271 \\
      a15 &   91 &   50 &   36 &  332 & 0.8310 \\
      a05 &  147 &   30 &   44 &  232 & 0.8366 \\
      a16 &  140 &   21 &   51 &  269 & 0.8503 \\
      a13 &  213 &   38 &   28 &  215 & 0.8664 \\
      a09 &   90 &   24 &   39 &  342 & 0.8727 \\
      a10 &  404 &   13 &   49 &   50 & 0.8798 \\
      a14 &   69 &   57 &    2 &  381 & 0.8841 \\
      a17 &  302 &   24 &   32 &  126 & 0.8843 \\
      a02 &   72 &   36 &   19 &  401 & 0.8958 \\
      a19 &  289 &    8 &   30 &  174 & 0.9242 \\
      a12 &   14 &   29 &    3 &  530 & 0.9444 \\
      b04 &  418 &    0 &   10 &    0 & 0.9766 \\
      a01 &   11 &    8 &    0 &  470 & 0.9836 \\
      a04 &   35 &    4 &    2 &  451 & 0.9878 \\
      c07 &  424 &    0 &    4 &    0 & 0.9907 \\
      c05 &  462 &    0 &    3 &    0 & 0.9935 \\
      c09 &  465 &    0 &    2 &    0 & 0.9957 \\
      c10 &  429 &    0 &    1 &    0 & 0.9977 \\
      c03 &  452 &    1 &    0 &    0 & 0.9978 \\
      c06 &  466 &    0 &    1 &    0 & 0.9979 \\
      c02 &  500 &    0 &    1 &    0 & 0.9980 \\
      c01 &  483 &    0 &    0 &    0 & 1.0000 \\
      c04 &  481 &    0 &    0 &    0 & 1.0000 \\
      c08 &  513 &    0 &    0 &    0 & 1.0000 \\
      \hline
      \rule{0pt}{2.0ex}%
      sum & 9046 & 1090 &  860 & 5592 & 0.8824\\
      \hline
    \end{tabular}}
  \label{tab:result1}
\end{table}

Table~\ref{tab:cinc2000} is a copy of the table in the book.  It does
not reflect the performance of the new code.
\begin{table}
  \caption[The scores described in this chapter]%
  {Here are the scores described in this chapter interspersed
    with the top scores from the CINC2000 website
    (\url{http://www.physionet.org/challenge/2000/top-scores.shtml}).}
  \centering{\plotsize%
    \begin{tabular}{|l|p{25em}|l|}
      \hline
      Score & Entrant & Entries \\
      \hline
      \rule{0pt}{2.25ex}%
      92.62 & J McNames, A Fraser, and A Rechtsteiner
      Portland State University, Portland, OR, USA & 4 \\
      92.30 & B Raymond, R Cayton, R Bates, and M Chappell
      Birmingham Heartlands Hospital, Birmingham, UK & 8 \\
      89.36 & P de Chazal, C Henehan, E Sheridan, R Reilly, P Nolan,
      and M O'Malley
      University College - Dublin, Ireland  & 15 \\
      87.56 & M Schrader, C Zywietz, V von Einem, B Widiger, G Joseph
      Medical School Hannover, Hannover, Germany &	9 \\
      87.30 & MR Jarvis and PP Mitra
      Caltech, Pasadena, CA, USA &	3 \\
      \hline
      \rule{0pt}{2.25ex}%
      86.95 & Second entry in this chapter.  Adjust \emph{Fudge} to get
      fraction of apnea minutes in the test records to match the fraction of
      apnea minutes in the training records & \\
      \hline
      \rule{0pt}{2.25ex}%
      86.24 & First entry in this chapter.  Models and parameters tuned to
      the training records. & \\
      \hline
      \rule{0pt}{2.25ex}%
      85.63 & Z Shinar, A Baharav, and S Akselrod
      Tel-Aviv University, Ramat-Aviv, Israel &	1 \\
      85.54 & C Maier, M Bauch, and H Dickhaus
      University of Heidelberg, Heilbronn, Germany & 5 \\
      84.49 & JE Mietus, C-K Peng, and AL Goldberger
      Beth Israel Deaconess Medical Center, Boston, MA, USA (unofficial
      entry) & \\
      \hline
    \end{tabular}}
  \label{tab:cinc2000}
\end{table}
%

\end{document}

Unused: figs/ScalarGaussian.pdf_t figs/sgauss1.pdf_t
figs/sgauss2.pdf_t figs/ToyCross.pdf_t figs/ToyPractical.pdf_t

ToDo:

DDS is derived_data_synthetic
RD is raw_data

https://physionet.org/content/apnea-ecg/1.0.0/

a03erA
a03erN
a03erHR
ApneaNLD
sgram
LDA1
LDA2
pass1
pf_H

https://github.com/rlabbe/Kalman-and-Bayesian-Filters-in-Python/

%%% Local Variables:
%%% eval: (TeX-PDF-mode)
%%% End:
