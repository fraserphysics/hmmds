% Derived from learning10poster.tex, ds13.tex and scipy2016.tex.
% Documetnation for poster mode at
% http://www.tex.ac.uk/ctan/macros/latex/contrib/textpos/textpos.pdf

\newcommand{\plotsize}{\tiny} % Size of font in labels
\documentclass[a0,showboxes]{a0poster}
%\documentclass[a0]{a0poster}
\usepackage{amsmath}
\usepackage{amsfonts}
%\usepackage{graphix}
\pagestyle{empty}
\setcounter{secnumdepth}{0}

% The textpos package is necessary to position textblocks at arbitary 
% places on the page.
\usepackage[absolute]{textpos}

% Graphics to include graphics. Times is nice on posters, but you
% might want to switch it off and go for CMR fonts.
%\usepackage{graphics,wrapfig,times}
\usepackage[pdftex]{rotating} %AMF

% These colours are tried and tested for titles and headers. Don't
% over use color!
\usepackage{color}
%\definecolor{DarkBlue}{rgb}{0.1,0.1,0.5}
\definecolor{DarkBlue}{rgb}{0.1,0.1,0.8}
\definecolor{Red}{rgb}{0.9,0.0,0.1}

% see documentation for a0poster class for the size options here
\let\Textsize\normalsize
\def\Head#1{\noindent\hbox to \hsize{\hfil{\LARGE\color{DarkBlue} #1}}\bigskip}
\def\LHead#1{\noindent{\LARGE\color{DarkBlue} #1}\bigskip}
\def\Subhead#1{\noindent{\large\color{DarkBlue} #1}\bigskip}
\def\Title#1{\noindent{\VeryHuge\color{Red} #1}}

\TPGrid[50mm,50mm]{23}{16}      % 4 cols of width 5, plus 3 gaps

\parindent=0pt
\parskip=0.5\baselineskip

\begin{document}

% Understanding textblocks is the key to being able to do a poster in
% LaTeX. In
%
%    \begin{textblock}{wid}(x,y)
%    ...
%    \end{textblock}
%
% the first argument gives the block width in units of the grid
% cells specified above in \TPGrid; the second gives the (x,y)
% position on the grid, with the y axis pointing down.

\begin{textblock}{12}(0,0) \Title{Entropy, Cross Entropy and Data Assimilation}
\end{textblock}

\begin{textblock}{4}(14,0.05)
{\Huge\color{DarkBlue} Andrew M.~Fraser}
\end{textblock}

\begin{textblock}{5}(0,1.3)
  \LHead{ODE $\rightarrow$ Observations}
\end{textblock}

\begin{textblock}{5}(0,2.0)
  \resizebox{0.9\textwidth}{!}{\includegraphics{particles_a.pdf}}\\
  \resizebox{0.9\textwidth}{!}{\includegraphics{TSintro.pdf}}
\end{textblock}

\begin{textblock}{5}(0,8)
  \LHead{Observations $\rightarrow$ States}
\end{textblock}

\begin{textblock}{3}(0,10.0)
  \resizebox{0.99\textwidth}{!}{\includegraphics[width=3cm]{LaserLP5.pdf}}\\
  \resizebox{0.7\textwidth}{!}{\includegraphics{LaserStates.pdf}}
\end{textblock}

\begin{textblock}{5}(6,1.3)
  \LHead{Data Assimilation}
\end{textblock}

\begin{textblock}{5}(6,2)
  \resizebox{0.8\columnwidth}{!}  {\input{Markov_dhmm.pdf_t}}\bigskip
  
  \resizebox{0.7\textwidth}{!}{\includegraphics{Statesintro.pdf}}
\end{textblock}

\begin{textblock}{5}(6,8)
  \LHead{Variations on a Theme}
  Put integrals here
  \begin{description}
  \item[Kalman Filter] 
  \item[Extended Kalman Filter] 
  \item[Hidden Markov Model] 
  \item[Particle Filter] 
  \end{description}
\end{textblock}

\begin{textblock}{5}(12.0,1.3)
  \LHead{Entropy}
\end{textblock}

\begin{textblock}{5}(12.0,9)
    \resizebox{1.0\textwidth}{!}{\includegraphics[]{ToyStretch.pdf}}
\end{textblock}

\begin{textblock}{5}(12,12)
  \resizebox{1.0\textwidth}{!}{\includegraphics{LikeLor.pdf}}
\end{textblock}


\begin{textblock}{5}(18.0,1.3)
  \LHead{Particle Filter}
\end{textblock}

\begin{textblock}{5}(18,2.0)
  \resizebox{0.9\textwidth}{!}{\includegraphics{particles_b.pdf}}\\
  \resizebox{0.9\textwidth}{!}{\includegraphics{entropy_particle.pdf}}
\end{textblock}

\end{document}

%%%---------------
%%% Local Variables:
%%% eval: (TeX-PDF-mode)
%%% End:
