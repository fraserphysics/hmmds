% Derived from learning10poster.tex, ds13.tex and scipy2016.tex.
% Documetnation for poster mode at
% http://www.tex.ac.uk/ctan/macros/latex/contrib/textpos/textpos.pdf

\newcommand{\field}[1]{\mathbb{#1}}
\newcommand\REAL{\field{R}}
\newcommand\INTEGER{\field{Z}}
\newcommand{\EV}[2]{\field{E}_{#1}\left[#2\right]}
\newcommand{\argmax}{\operatorname*{argmax}}
\newcommand{\plotsize}{\tiny} % Size of font in labels
\documentclass[a0,showboxes]{a0poster}
%\documentclass[a0]{a0poster}
\usepackage{amsmath}
\usepackage{amsfonts}
%\usepackage{graphix}
\pagestyle{empty}
\setcounter{secnumdepth}{0}

% The textpos package is necessary to position textblocks at arbitary 
% places on the page.
\usepackage[absolute]{textpos}

% Graphics to include graphics. Times is nice on posters, but you
% might want to switch it off and go for CMR fonts.
%\usepackage{graphics,wrapfig,times}
\usepackage[pdftex]{rotating} %AMF

% These colours are tried and tested for titles and headers. Don't
% over use color!
\usepackage{color}
%\definecolor{DarkBlue}{rgb}{0.1,0.1,0.5}
\definecolor{DarkBlue}{rgb}{0.1,0.1,0.8}
\definecolor{Red}{rgb}{0.9,0.0,0.1}

% see documentation for a0poster class for the size options here
\let\Textsize\normalsize
\def\Head#1{\noindent\hbox to \hsize{\hfil{\LARGE\color{DarkBlue} #1}}\bigskip}
\def\LHead#1{\noindent{\LARGE\color{DarkBlue} #1}\bigskip}
\def\Subhead#1{\noindent{\large\color{DarkBlue} #1}\bigskip}
\def\Title#1{\noindent{\VeryHuge\color{Red} #1}}

\TPGrid[50mm,50mm]{23}{16}      % 4 cols of width 5, plus 3 gaps

\parindent=0pt
\parskip=0.5\baselineskip

\begin{document}

% Understanding textblocks is the key to being able to do a poster in
% LaTeX. In
%
%    \begin{textblock}{wid}(x,y)
%    ...
%    \end{textblock}
%
% the first argument gives the block width in units of the grid
% cells specified above in \TPGrid; the second gives the (x,y)
% position on the grid, with the y axis pointing down.

\begin{textblock}{13}(0,0) \Title{Entropy, Cross Entropy and Data Assimilation}
\end{textblock}

\begin{textblock}{4}(14,0.05)
{\Huge\color{DarkBlue} Andrew M.~Fraser}
\end{textblock}

\begin{textblock}{5}(0,1.3)
  \LHead{ODE $\rightarrow$ Observations}
  \begin{align*}
    \dot x &= F(x) \\
    x[0: 10000] &= \text{integrate}(F, 10000) \\
    \text{bins} &= [-10,0,10] \\
    y &= G(x) \\
    y[0:10000] &= \text{digitize}(x[0:10000,0], \text{bins}])
  \end{align*}
  \resizebox{0.9\textwidth}{!}{\includegraphics{particles_a.pdf}}\\
  \resizebox{0.9\textwidth}{!}{\includegraphics{TSintro.pdf}}
\end{textblock}

\begin{textblock}{5}(0,9.5)
  \LHead{Data Assimilation:\\Observations $\rightarrow$ States $\in \REAL^3$}\\
  Laser data from Tang and Weiss\\
  Extended Kalman smoothing for state space trajectory estimate
  \begin{align*}
    \hat x &= \argmax_x P(x|y) \\
  \end{align*}
  \resizebox{0.8\textwidth}{!}{\includegraphics[width=3cm]{LaserLP5.pdf}}\\
  \resizebox{0.8\textwidth}{!}{\includegraphics{LaserStates.pdf}}
\end{textblock}

\begin{textblock}{5.5}(6,1.3) %
  \LHead{Data Assimilation:\\ Observations
    $\rightarrow$ States $\in
    \INTEGER$}\\
  Parameters, $\theta$, of Hidden Markov Model (HMM) with states
  $s\in\INTEGER$ and observations $y\in\INTEGER$:
  \begin{description}
  \item[$P_{S \leftarrow S}$:] State to state transition probabilities
  \item[$P_{Y \leftarrow S}$:]  Conditional observation probabilities
  \end{description}
  Estimation algorithms:
  \begin{description}
  \item[Forward Filter:] Conditional probability of states
    $P(x[t]|y[0:t+1], \theta]$
  \item[MLE Parameters:] (Forward-Backward, or Baum Welch)
    \begin{equation*}
      \hat \theta = \argmax_\theta P(y[0:1000]|\theta)
    \end{equation*}
  \item[MAP States:] (Viterbi)
    \begin{equation*}
      \hat s[0:1000] = \argmax_{s[0:1000]}P(s[0:1000]|y[1000], \theta)
    \end{equation*}
  \end{description}
  \begin{center}
    \resizebox{0.6\columnwidth}{!}
  {\input{Markov_dhmm.pdf_t}}
  \resizebox{0.7\textwidth}{!}{\includegraphics{Statesintro.pdf}}
  \end{center}
\end{textblock}

\begin{textblock}{5.5}(6,12)
  \LHead{Variations on a Theme}\\
  \textbf{Forward data assimilation} alternates between \textbf{update} $\&$
  \textbf{forecast}.\vspace{-1cm}
  \begin{description}
  \item[Update:] \parbox{.8\textwidth}{
    \begin{align*}
      \alpha(x,t) &\equiv P(X[t]=x|y[0:t+1] \\
      \alpha(x,t) &\propto a(x,t) P_{Y \leftarrow X}(y[t]|x)
    \end{align*}
    One must evaluate $\int \alpha(x,t) dx$ to normalize the update.}
  \item[Forecast:] \parbox{.8\textwidth}{
    \begin{align*}
      a(x,t) &\equiv P(X[t]=x|y[0:t]) \\
      (x,t) &= \int \alpha(\chi,t-1) P_{X \leftarrow X}(x|\chi) d \chi
    \end{align*}} \vspace{-1cm}
  \end{description}
  \begin{description}
  \item[Kalman Filter] $P_{X \leftarrow X}$ and $P_{Y \leftarrow X}$
    are linear with Gaussian residuals.
  \item[Extended Kalman Filter] Kalman filter for nonlinear functions
    with local linear approximations.
  \item[Hidden Markov Model] States and observations are finite sets.
  \item[Particle Filter] Monte-Carlo for integrals.  Probabilities
    represented by clouds of points.
  \end{description}
\end{textblock}

\begin{textblock}{5}(12.0,1.3)
  \LHead{Entropy and Lyapunov Exponents}\\
  \textbf{Entropy} for true model $P_\mu$
  \begin{equation*}
    h(\mu) \equiv \lim_{n \rightarrow \infty} -\frac{1}{n}
      \EV{\mu}{\log(P_\mu (y[0:n]))}
    \end{equation*}
    For $\forall y[0:n] \in A_\epsilon^{(n)}$, the typical or plausible set
    \begin{align*}
      \frac{-\log(P_\mu (y[0:n]))}{n} &= h \pm \epsilon & 
      \text{definition} h \text{ is the rate that prob} \rightarrow 0. \\
      \text{Pr} \left\{  A_\epsilon^{(n)} \right\} &> 1-\epsilon \\
      \left|  A_\epsilon^{(n)} \right| & \leq e^{n(h+\epsilon)} & 
          h \text{is the rate that } A_\epsilon^{(n)} \text{grows}.
    \end{align*}
    \textbf{Cross Entropy} of other model $\theta$ wrt true $\mu$ 
    \begin{align*}
    h(\mu||\theta) & \equiv \lim_{n \rightarrow \infty} -\frac{1}{n}
      \EV{\mu}{\log(P_\theta (y[0:n]))} \\
      h(\mu||\theta) - h(\mu) & \geq 0 \quad \text{equality } \rightarrow
                                         \mu = \theta \text{ almost everywhere}
    \end{align*}
    \textbf{Lyapunov exponents}, $\lambda_i$ characterize the
    exponential rates that trajectories converge or diverge.
    Estimate them numerically with Benettin's procedure that requires
    integrating tangent equation.   Work of Ruelle, Pesin,
    Ledrappier, Young says that for the Lorenz system the largest
    exponent is equal to the entropy, ie,
    \begin{equation*}
      h = \lambda_0 \approx 0.906.
    \end{equation*}
    So \textbf{0.906} is a lower \textbf{bound} for the cross entropy
    of an model of time series from the Lorenz system.
    \begin{center}
      \resizebox{1.0\textwidth}{!}{\includegraphics{LikeLor.pdf}}
      Use HMMs with many states to approach the bound.
    \end{center}
\end{textblock}

\begin{textblock}{5}(12.0,13)
    \resizebox{1.0\textwidth}{!}{\includegraphics[]{ToyStretch.pdf}}
\end{textblock}


\begin{textblock}{5}(18.0,1.3)
  \LHead{Particle Filter}
\end{textblock}

\begin{textblock}{5}(18,2.0)
  \resizebox{0.9\textwidth}{!}{\includegraphics{particles_b.pdf}}\\
  \resizebox{0.9\textwidth}{!}{\includegraphics{entropy_particle.pdf}}
\end{textblock}

\end{document}

%%%---------------
%%% Local Variables:
%%% eval: (TeX-PDF-mode)
%%% End:
