% Notes on modeling ECGs with HMMs directly

\documentclass[12pt]{article}
\usepackage{graphicx,color}
\usepackage{amsmath, amsfonts}
\usepackage{placeins}

\newcommand{\field}[1]{\mathbb{#1}}
\newcommand{\INTEGER}{\field{Z}}
\newcommand{\REAL}{\field{R}}
\newcommand{\COMPLEX}{\field{C}}
\newcommand{\id}{\mathbb{I}}
\newcommand{\variance}[1]{\field{V}\left[ #1 \right]}
\newcommand{\normal}[2]{{\cal N}\left(#1,#2 \right)}
\newcommand{\argmax}{\operatorname*{argmax}}
\newcommand{\SixPlots}[1]{
  \begin{figure}
    \centering
    \resizebox{1.0\textwidth}{!}{\includegraphics{ecgs_#1.pdf}}
    \caption{Subsets of data from six CINC-2000 records.}
    \label{fig:ecgs_#1}
  \end{figure}
}

\title{All Records}
\author{Andrew M.\ Fraser}

\begin{document}
\maketitle

\section{Introduction}
\label{sec:introduction}

The data for the \emph{Computers in Cardiology Challenge} in 2000,
CINC2000, consists of ECG recordings of 69 patients in sleep
studies\footnote{The description of the challenge and the data are
available at https://archive.physionet.org/challenge/2000/.  PhysioNet
has records for 70 patients, but the record named b05 is not
usable.}.  The challenge was to ``demonstrate the efficacy of
ECG-based methods for apnea detection using a large,
well-characterized, and representative set of data.''

I am reworking an approach to the challenge, and I don't understand
why the ECG waveforms in some of the records are so different from the
others.  I suspect that lead placement caused much of the variation.
However, I also suspect that a few of the records show pathologies in
addition to apnea.  I would like to talk to an expert about the
variations.

In Section~\ref{sec:strange_records} I have a few notes on
peculiarities I see in some of the records, and in
Section~\ref{sec:all_records} I have plots of a few seconds of each
record.

\section{Strange Records}
\label{sec:strange_records}

\begin{description}
\item[a02] Small R wave.  See Figure~\ref{fig:ecgs_a01}.
  % There are changes at minutes 129.2, 175.8, 188.1, 206.8, etc.
  % Maybe the patient is moving?
\item[a10] Inverted signal?  See Figure~\ref{fig:ecgs_a07}.
\item[a12] Small inverted signal?  See Figure~\ref{fig:ecgs_a07}.
\item[a20] Double R wave.  The same double pattern appears throughout
  the record.  I think it's a characteristic of the patient.  See
  Figure~\ref{fig:ecgs_a19}.
\item[c02] Negative R wave?  See Figure~\ref{fig:ecgs_c01}.
\item[c05] I think this is the same as c06.
  %Peak at minute 9.235 in c05 is the same as peak at 10.569 in c06.
\item[x05] S bigger than R.  See Figure~\ref{fig:ecgs_x03}.
\item[x06] S bigger than R.  See Figure~\ref{fig:ecgs_x03}.
\item[x11] Atypical beats.  See Figure~\ref{fig:x11_332.83}.
\item[x13] Occasional big negative spikes.  See
  Figure~\ref{fig:x13_165.5}.
\item[x17] Strange waveform.  I think T is the biggest feature.  Bad
  lead placement?  See Figure~\ref{fig:ecgs_x15}.
\item[x24] S bigger than R?  See Figure~\ref{fig:ecgs_x21}.
\item[x26] What's happening at times 1:05:47.5,
  1:05:55 and 1:05:57 in Figure~\ref{fig:x26_65.78}?
  % minutes 65.8 and 65.95
\item[x29] Frequent short inter-beat intervals.  See
  Figure~\ref{fig:x29_328.95}.
  % Except minutes 296 to 317 are normal.
\item[x33] Small inverted signal.  See Figure~\ref{fig:ecgs_x33}.
\item[x34] Same as x33.
  % -2mV peak at 7.55234 in record x33 appears in record x34 at 8.7190
\end{description}

% x11_332.83 x13_165.5 x26_65.78 x29_328.95
\newcommand{\StrangePlot}[2]{
  \begin{figure}
    \centering
    \resizebox{1,0\textwidth}{!}{\includegraphics{#1.pdf}}
    \caption{#2}
    \label{fig:#1}
  \end{figure}
}

\StrangePlot{x11_332.83}{In record x11 the beat at 5:32:51 is strange.}
\StrangePlot{x13_165.5}{In record x13 there are occasional negative
  waves instead of R waves.}
\StrangePlot{x26_65.78}{In record x26 there's something
  different at 1:05:47.5, 1:05:55 and 1:05:57.}
\StrangePlot{x29_328.95}{In record x29 there are frequent short
  inter-beat intervals.  Here at 5:25:00.8 and 5:25:09.8}.

\clearpage

\section{All Records}
\label{sec:all_records}

Here I plot a sample of each of the 69 records provided by PhysioNet
for CINC2000.

\SixPlots{a01}
\SixPlots{a07}
\SixPlots{a13}
\SixPlots{a19}
\SixPlots{c01}
\SixPlots{c07}
\SixPlots{x03}
\SixPlots{x09}
\SixPlots{x15}
\SixPlots{x21}
\SixPlots{x27}
\SixPlots{x33}

\end{document}
