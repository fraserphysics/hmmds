\def\toysLambdaOne{0.906}
\input{toy_values.tex}
% A typical def in toy_values.tex is \def\toyToyhnt{200}.  The initial
% ``toy'' indicated it is for this chaper, toys.tex.  The following
% ``Toyh'' indicates that the value is defined in the script toy_y.py.
% Then final ``nt'' indicates that the value comes from args.n_t in that
% script.

\chapter{Toy Problems and Performance Bounds}
\label{chap:toys}

Having developed algorithms for fitting model parameters, one might
reasonably ask how well the models so produced perform.  In this
chapter we argue that the exercise of fitting models to data from
chaotic dynamical systems is interesting because \emph{Lyapunov
  exponent} calculations give a quantitative benchmark against which
to compare model performance.  The idea is that the stretching or
local instability of dynamics, which Lyapunov exponents characterize,
limits the predictability \index{predictability} of sequences of
observations.  We will start by examining a toy example derived from
the Lorenz system that informally introduces the main ideas.  From
there we will review definitions of entropy and Lyapunov exponents and
results from information theory and ergodic theory that connect the
ideas.  Finally we explain a simple calculation that can determine
that a proposed model is fundamentally suboptimal.

We suppose that a \emph{true} stochastic process that assigns
probabilities to sequences of observations exists.  Many of the terms
that we define are \emph{expected values} with respect to those
probabilities, and we find that sample sequences converge to those
expected values.  We use $P_{*|\mu}$ to denote these \emph{true}
probabilities\footnote{Following Kolmogorov, modern probability theory
  is cast a subfield of measure theory.  The measure theory literature
  uses the Greek letter $\mu$ for a function or \emph{measure} that
  maps sets to $\REAL^+$.  In earlier chapters, we have used
  $P_{*|\theta}$ to denote parametric families of distributions.  We
  introduce the mongrel notation $P_{*|\mu}$ here to make our notation
  for comparisons between a true distribution and a parametric model
  natural.  The meaning of the Greek letter $\mu$ here is not related
  to its use to denote the mean of a distribution.}, and we use them
to define expected values without delving in to theoretical questions
about their existence.

\subsubsection{Lorenz Example}

As an example, we have simulated a version of the %
\index*{Lorenz system} (Eqn.~\eqref{eq:Lorenz}) modified to fit the
form of Eqn.~\eqref{eq:contnoise},
\begin{align*}
  \ti{x}{t+1} &= F(\ti{x}{t},t) + \ti{\eta}{t}\\
  \ti{y}{t}   &= G(\ti{x}{t},t) + \ti{\epsilon}{t}).
\end{align*}
We have used the \index*{extended Kalman filter} described in
chapter~\ref{chap:continuous} to obtain parametric probability
functions $P\left(\ti{y}{t}|\ts{y}{1}{t-1} , \theta \right)$ that
approximate $P\left(\ti{y}{t}|\ts{y}{1}{t-1}, \mu \right)$, \ie, the
conditional distribution of the measurement at time $t$ given all
previous measurements.  Our code for generating sequences of
measurements has the following characteristics:
  \begin{description}
  \item[State] 
    \begin{equation*}
      x \equiv 
      \begin{bmatrix}
        x_1\\x_2\\x_3
      \end{bmatrix}
    \end{equation*}
  \item[Time step] We obtain the map $F$ by numerically integrating
    Eqn.~\eqref{eq:Lorenz} for time intervals of length $t_s$ with an
    absolute error tolerance of $\toyHviewatol$.
  \item[iid state noise] 
    \begin{equation*}
      \ti{\eta}{t} \sim \Normal(0, \begin{bmatrix} 1 & 0 & 0\\0 & 1 &
        0\\0&0&1 \end{bmatrix} \sigma_\eta^2)
    \end{equation*}
  \item[Measurement function] A simple projection
    \begin{equation*}
      G(x) = x_1 = \begin{bmatrix} 1,0,0 \end{bmatrix} \cdot x
    \end{equation*}
  \item[iid measurement noise] 
    \begin{equation*}
      \ti{\epsilon}{t} \sim \Normal(0, \sigma_\epsilon^2)
    \end{equation*}
  \item[Quantization] The observations are quantized with a resolution
    $\Delta=\toyHviewystep$.  We analyze quantized measurements rather than
    continuous measurements because they provide a \emph{finite} rather
    than \emph{infinite} amount of information and they are
    characterized by \emph{coordinate invariant} probability mass
    functions rather than \emph{coordinate dependent} probability
    density functions.
\end{description}
Recall that for the extended Kalman filter $\ti{\mu_\gamma}{t}$ and
$\ti{\sigma_\gamma}{t}$ completely characterize
$ P(\ti{y}{t}|\ts{y}{1}{t-1},\theta)$ with
\begin{equation*}
  P(\ti{y}{t}|\ts{y}{1}{t-1},\theta) =
  \NormalE{\ti{\mu_\gamma}{t}}{\ti{\sigma^2_\gamma}{t}}{\ti{y}{t}}.
\end{equation*}
(We use a lower case sigma here because the observations are scalars.)
We obtain affine maps for the approximation $F(x+\delta,t) \approx
\left[ DF(x)\right]\delta + F(x)$ %
\nomenclature[rDF]{$\left[ DF(x)\right]\delta$}{The \emph{derivative}
  of the function $F$ at $x$ applied to the vector $\delta$.  This
  notation emphasizes that $DF(x)$ is a linear map.}
%
by numerically integrating both the Lorenz system and the tangent
equations.  We use those approximations with Eqns.~\eqref{eq:KForeMu}
and \eqref{eq:KUpdate} on page \pageref{eq:KUpdate} to implement the
recursive calculation of $\ti{\mu_\gamma}{t}$ and
$\ti{\sigma_\gamma}{t}$ described by Eqns.~\eqref{eq:t274} to
\eqref{eq:IUpdate} on page \pageref{eq:IUpdate}.

Figures~\ref{fig:ToyTS1} and \ref{fig:ToyStretch} depict a simulation
in which dynamical stretching, \ie, the linear instability of
$\left[ DF(x)\right]$, occasionally limits predictability.  We chose
the parameters, specified in the caption of Fig.~\ref{fig:ToyTS1}, so
that dynamical noise and measurement quantization are negligible
compared to the effects of measurement noise and dynamical stretching.
In the middle plot of Fig.~\ref{fig:ToyTS1} notice that while for most
times the forecast deviation of the prediction
$\ti{\sigma_\gamma}{t}$ is very close to the size of the measurement
noise $\sigma_\eta = \toyHviewSigmaEpsilon$, occasionally the forecast deviations are
many times larger.  The log likelihood per time step which appears in
bottom plot of the figure is low when either the forecast deviations
are large or when the difference between the mean of the forecast and
the actual observation is much larger than the predicted deviation,
\ie,
$\ti{\sigma^2_\gamma}{t} <<
\left(\ti{y}{t}-\ti{\mu_\gamma}{t}\right)^2$.

The largest excursion of $\ti{\sigma_\gamma}{t}$ in
Fig.~\ref{fig:ToyTS1} occurs at $t=\toyHviewTmaxSigma$.
Figure~\ref{fig:ToyStretch} illustrates the stretching action of the
map $[DF]$ that occurs then.

\begin{figure}[htbp]
  \centering{
    \includegraphics[width=1.0\textwidth]{ToyTS1.pdf}}
  \caption[\comment{fig:ToyTS1 }Extended Kalman filter for one step forecasting.]%
    {Extended Kalman filter for one step forecasting with simulation
    parameters:\\
    \begin{tabular}[c]{ll}
       $t_s=\toyHviewdt$ & Sample interval \\
       $\sigma_\eta = \toyHviewSigmaEta$ & Standard deviation of state noise \\
       $\sigma_\epsilon = \toyHviewSigmaEpsilon$ & Standard deviation of measurement noise \\
       $\Delta = \toyHviewystep$ & Measurement quantization \\
    \end{tabular}\\
    A time series of observations appears in the upper plot.  The
    middle plot characterizes the one-step forecast distributions
    $P_{\gamma} \left(\ti{y}{t} \right) \equiv P
    \left(\ti{y}{t}|\ts{y}{1}{t-1},\theta \right) =
    \NormalE{\ti{\mu_\gamma}{t}}{\ti{\sigma^2_\gamma}{t}}{\ti{y}{t}}$.
    The standard deviations of the forecasts appear in the first
    trace, and the differences between the actual observations
    and the means of the forecasts appear in the second trace.  The
    logs of the likelihoods of the forecasts, $\log(P_{\gamma}
    \left(\ti{y}{t} \right))$, appear in the bottom plot.}
  \label{fig:ToyTS1}
\end{figure}
%%%
\marginpar{FixMe: $x_0$ or $x_1$?}
\begin{figure}[htbp]
  \centering{\resizebox{\textwidth}{!}{\includegraphics{ToyStretch.pdf}}
  }
  \caption[\comment{fig:ToyStretch }Dynamical stretching increases
  state variance.]%
  {These plots illustrate dynamical stretching increasing the variance
    of the conditional distribution in state space between time steps
    $\toyHviewTmaxSigma$ and $\toyHviewTmaxSigmaPlusOne$ in
    Fig.~\ref{fig:ToyTS1}.  In each plot, the \emph{forecast} state
    distribution ellipse represents
    $P_{a} \left(\ti{x}{t} \right) \equiv P
    \left(\ti{x}{t}|\ts{y}{1}{t-1},\theta \right) =
    \NormalE{\mu_a}{\Sigma_a}{\ti{x}{t}}$ and the \emph{update} state
    distribution ellipse represents
    $P_{\alpha} \left(\ti{x}{t} \right) \equiv P
    \left(\ti{x}{t}|\ts{y}{1}{t}, \theta \right) =
    \NormalE{\mu_\alpha}{\Sigma_\alpha}{\ti{x}{t}}$.  For each
    distribution, an ellipse depicts the level set
    $(x-\mu)\transpose \Sigma^{-1} (x-\mu) =1$ in the $x_1\times x_3$
    plane.  To support visual comparisons, the $x_1$ range the $x_3$
    range are the same in each of the plots.  In mapping the updated
    distribution at $t=\toyHviewTmaxSigma$ (on the left) to the
    forecast distribution at time $t=\toyHviewTmaxSigmaPlusOne$ (on the
    right), the function $F$ that implements state dynamics stretches
    the ellipse by about a factor of 10 in both directions.}
  \label{fig:ToyStretch}
\end{figure}

Figure~\ref{fig:ToyH} illustrates the behavior of $-\hat h$, the sample
average of the log likelihood of forecasts, for a group of simulations
with parameters that are quite different from those in
Figs.~\ref{fig:ToyTS1} and \ref{fig:ToyStretch}.  Given model parameters
$\theta$ and a sample sequence $\ts{y}{1}{T}$ of length $T$, we define
\begin{align*}
  -\hat h &\equiv \frac{1}{T} \sum_{t=1}^T
    \log \left( P \left(\ti{y}{t}|\ts{y}{1}{t-1}, \theta
      \right)\right)\\
    &=  \frac{1}{T} \log \left( P \left(\ts{y}{1}{T}| \theta
      \right)\right).
\end{align*}
The negative of this sample log likelihood is an estimate of the
\emph{cross entropy rate} \index{entropy!cross rate}
\begin{equation*}
  h(\mu||\theta) \equiv  \lim_{T \rightarrow \infty} - \frac{1}{T}
  \EV_{\mu} \log \left( P \left(\ts{Y}{1}{T}|\theta \right) \right)
\end{equation*}
which in turn is bounded from below by the \emph{entropy rate}.  We
discuss both entropy rate and cross entropy rate in
Section~\ref{sec:pesin}, and in Section~\ref{sec:PesinFormula} we
review the \emph{Pesin Formula}.  That formula says that the largest
\emph{Lyapunov exponent} $\lambda_1$, a characterization of the
stretching action of the dynamics, is equal to the entropy rate, a
characterization of the predictability.  For good models, we expect
the log likelihood of forecasts to fall off with a slope of
$-\lambda_1$ as the sample time $t_s$ increases, and for the best
possible model we expect
\begin{equation}
  \label{eq:bound1}
  h(t_s) = \lambda_1 t_s.
\end{equation}

We have chosen the parameters\footnote{ In making Fig.~\ref{fig:ToyH},
  we wanted simulations close to the bound of Eqn.~\eqref{eq:bound1}.
  We found that at larger values of $t_s$ and $\Delta$, extended
  Kalman filters performed better if given models with larger state
  noise than the noise actually used to generate the data, \ie $\tilde
  \sigma_\eta > \sigma_\eta$.  We believe that the effect is the
  result of the larger errors that occur as the affine approximation
  $F(x+\delta) \approx [DF(x)] \delta + F(x)$ fails to track the
  nonlinearities of the Lorenz system over larger intervals in state
  space.  By making $\tilde \sigma_\eta$ larger, the errors are
  accommodated as state noise.  We chose the state noise of the
  generating process to be an order of magnitude larger than the
  absolute integration tolerance of $\toyHviewatol$.  We then chose the
  quantization level and sample times to be as large as possible, but
  still small enough that we could have $\tilde \sigma_\eta =
  \sigma_\eta$ without losing performance.  That led to the values
  $\tilde \sigma_\eta = \sigma_\eta = \toyToyhdevstategenerate$, $\Delta = \toyToyhystep$,
  and $0 < t_s \leq 0.5$.} specified in the caption of
Fig.~\ref{fig:ToyH} with a large measurement quantization size so that
the log likelihood is limited primarily by dynamical stretching and
the Gaussian assumption for $ P \left(\ti{y}{t}|\ts{y}{1}{t-1}, \theta
\right)$.  We are pleased to observe that the overall slope of the
plot on the left in Fig.~\ref{fig:ToyH} is consistent with the
estimate $\hat \lambda_1 = \toysLambdaOne$ (base $e$) that we obtain using the
Benettin procedure described in Section~\ref{sec:Benettin}.
%%% code/python/lorenzLyap.py

\begin{figure}[htbp]
  \centering{
    \includegraphics[width=1.0\textwidth]{ToyH.pdf}}
  \caption[\comment{fig:ToyH }Average log likelihood of one step forecasts.]%
  {Average log likelihood of one step forecasts as a function of time
    step $t_s$ and filter parameter $\tilde \sigma_\epsilon$.  To
    simulate measurements for this figure, we used the parameters:\\
    \begin{tabular}[c]{ll}
     $\sigma_\eta = \toyToyhdevstategenerate$ & Standard deviation of state noise \\
     $\sigma_\epsilon = \toyToyhdevmeasurement$ & Standard deviation of measurement noise \\
     $\Delta = \toyToyhystep$ & Measurement quantization \\
     $T=\toyToyhnt$ & Number of samples
    \end{tabular}\\
    For both plots, the vertical axis is the average log likelihood of
    the one-step forecast $-\hat h \equiv \frac{1}{T} \sum_{t=1}^T
    \log \left( P \left(\ti{y}{t}|\ts{y}{1}{t-1}, \theta
      \right)\right)$.  On the left we plot $-\hat h$ as a function of
    both $t_s$, the time step, and $\tilde \sigma_\epsilon$, the
    standard deviation of the measurement noise model used by the
    extended Kalman filter.  On the right ``$\circ$''%% Top row of dots
    indicates the
    performance of filters that use measurement noise models that depend on the
    sampling time through the formula $\tilde \sigma_\epsilon(t_s) =
    10^{0.4 t_s -4.85}$, which closely follows the ridge top in the plot
    on the left, ``$\diamond$''%% Bottom row of dots
    indicates the performance of
    filters that use $\tilde \sigma_\epsilon = \toyToyhystep$, \ie the
    measurement quantization level, and the solid line traces
    Eqn.~\eqref{eq:ToyH} in the text.}
   \label{fig:ToyH}
 \end{figure}
 \marginpar{FixMe: formula $10^{0.4 t_s -4.85}$?} 
%\afterpage{\clearpage}%% Print this right here please.

In the plot on the right in Fig.~\ref{fig:ToyH} we see that for a
class of filters in which the standard deviation of the model
measurement noise $\tilde \sigma_\epsilon$ is set to the quantization
size $\Delta$, the log likelihood closely follows the approximation
\begin{equation}
  \label{eq:ToyH}
  \hat h(t_s) = \log\left(\text{erf}\left(\frac{1}{2\sqrt{2}}\right)\right)
  + \lambda_1 t_s,
\end{equation}
where \emph{erf} is the error function\footnote{The error function is
  defined by $\operatorname{erf}(x) = \frac{2}{\sqrt{\pi}}\int_0^x
  e^{-t^2} dt$.}.  We explain the nonzero intercept in
Eqn.~\eqref{eq:ToyH} by observing that in the limit of small sampling
interval ($t_s \rightarrow 0$) and zero noise ($\sigma_\eta
\rightarrow 0$ and $\sigma_\epsilon \rightarrow 0$), only one discrete
observation $\ti{y}{t}=\bar y$ is possible given a history
$\ts{y}{1}{t-1}$.  For data drawn from that limiting case, a Kalman
filter with parameters $\tilde \sigma_\epsilon = \Delta$ and
$\sigma_\eta \rightarrow 0$ would make a forecast with a density
$P(\ti{y}{t}|\ts{y}{1}{t-1}) = \NormalE{\bar
  y}{(\Delta)^2}{\ti{y}{t}}$.  Integrating that density over the
quantization interval yields
 \begin{align*}
   P_{\ti{y}{t}|\theta}(\bar y) &= \int_{\bar y -
     \frac{\Delta}{2}}^{\bar y + \frac{\Delta}{2}} \frac{1}{\sqrt{2\pi
       (\Delta)^2}}e^{-\frac{(y-\bar
       y)^2}{2(\Delta)^2}} dy\\
   &= \int_{-\frac{1}{2}}^{\frac{1}{2}} \frac{1}{\sqrt{2\pi
     }}e^{-\frac{1}{2}s^2} ds\\
   &= \text{erf} \left( \frac{1}{2\sqrt{2}}\right) \\
   &\approx 0.3829 \\
   \log \left( P_{\ti{y}{t}|\theta}(\bar y) \right) &\approx -0.9599.
 \end{align*}
\marginpar{FixMe: 0.3829 and -0.9599?}
Given the simplicity of the analysis, Eqn.~\eqref{eq:ToyH} fits the
simulations in Fig.~\ref{fig:ToyH} remarkably well.


\section{Fidelity Criteria and Entropy}
\label{sec:fidelity}

Stochastic models are fit to an enormous variety of measured phenomena
and the most appropriate measure of the fidelity of a model to
measurements depends on the application.  Such phenomena include long
and short term weather, financial markets, computer data, electric
power demand, and signals and noise in communication or
instrumentation.  In many cases one makes decisions based on a model
and those decisions change the \emph{cost} of future events.  The
expected cost of basing decisions on a model $P_{*|\theta}$ depends in
a complicated fashion on many factors including how the cost of acting
on a decision depends on lead time and which aspects of the modeled
phenomenon are important.  For the Lorenz example at the beginning of
this chapter we implicitly assumed \emph{stationarity} and
\emph{ergodicity} and characterized model quality in terms of the
average of the log of the likelihood.  For a stationary ergodic
system, the log-likelihood is tied to $D(\mu||\theta)$, the
\emph{relative entropy} of the model $P_{*|\theta}$ given $P_{*|\mu}$
(see Eqn.~\eqref{eq:RelativeEntropy}).  Although relative entropy is
not an appropriate performance measure for every application, it is a
common tool for problems in information theory and statistics.  See
Cover and Thomas\cite{Cover91} for many of these including the
application of relative entropy to a theory of gambling.  Relative
entropy is exactly the right performance measure for data compression.
The arithmetic coding algorithm (see the review by Witten, Neal, and
Cleary\cite{Witten87}) for compressing a sequence of symbols uses a
model $P_{*|\theta}$ to make decisions that affect the cost in a
manner that depends on the symbol values that actually occur.  The
relative entropy $D(\mu||\theta)$ is the expected value of the number
of bits wasted by the algorithm if it uses a model $P_{*|\theta}$ for
decisions when in fact $P_{*|\mu}$ is true.  More accurate models lead
to better compression.

\subsection{Definitions}
\label{sec:hDef}

Now, to solidify the discussion, we make some formal definitions.

\subsubsection{Stochastic process}
\index{stochastic process}
We are interested in sequences of states $\ts{X}{1}{T}$ and
measurements  $\ts{Y}{1}{T}$ each of which can be thought of as a
\emph{random function} on the domain $\left\{ 1,2,\ldots,T \right\}$,
\ie, a \emph{stochastic process}.

\subsubsection{Entropy of a discrete random variable}
\index{entropy}
If a discrete random variable $U$ takes on the values
$u_1,u_2,\ldots,u_n$ with probabilities $P(u_1),P(u_2),\ldots,P(u_n)$,
then the \emph{entropy} of $U$ is
\nomenclature[rHU]{$H(U)$}{The entropy of a discrete random variable $U$.}
\begin{equation}
  \label{eq:HDef}
  H(U) \equiv - \EV \log \left( P(U) \right) = -\sum_{k=1}^n P(u_k)
  \log \left( P(u_k) \right).
\end{equation}
Entropy quantifies the the uncertainty in $U$ before its value is
known and the information or degree of surprise in discovering its
value.  If the base of the logarithm in Eqn.~\eqref{eq:HDef} is 2,
then the units of $H(U)$ are called \emph{bits}.  We will use natural
logarithms with Euler constant $e$ as the base.  For natural
logarithms the units of $H(U)$ are called \emph{nats}.

\subsubsection{Differential entropy of a continuous random variable}
\index{entropy!differential of a continuous random variable}
If $U$ is a continuous random variable with a probability density
function $P$ then the \emph{differential entropy} of $U$ is
\newcommand{\hdiff}{\tilde H}
\begin{equation}
  \label{eq:hdiffDef}
  \hdiff(U) \equiv  - \EV \log \left( P(U) \right) = - \int P(u) \log
  \left( P(u) \right) du.
\end{equation}
\nomenclature[rhv]{$\hdiff(U)$}{Differential entropy of the continuous random
  variable $U$.}  Notice that the differential entropy depends on the
coordinates of $U$.

\subsubsection{Conditional entropy}
\index{entropy!conditional}
The \emph{conditional entropy} of $U$ given $V$ is
\nomenclature[rHUV]{$H(U"|V)$}{The \emph{conditional entropy} of $U$ given
  $V$.}
\begin{equation*}
  H(U|V) \equiv - \EV \log \left( P(U|V) \right) = -\sum_{i,j} P(u_i,v_j)
  \log \left( P(u_i|v_j) \right).
\end{equation*}
Factoring the probability of sequences
\begin{equation*}
  P_{\ts{z}{1}{T} } = P _{\ti{z}{1} }
  \prod_{t=2}^T P _{\ti{z}{t}|\ts{z}{1}{t-1} }
\end{equation*}
is equivalent analyzing entropy into the sum
\begin{equation*}
  H(\ts{Z}{1}{T}) = H(\ti{Z}{1}) + \sum_{t=2}^T H
  \left(\ti{Z}{t}|\ts{Z}{1}{t-1} \right).
\end{equation*}

\subsubsection{Relative entropy of two probability functions}
\index{entropy!relative} The \emph{relative entropy} between two
probability functions $P_{*|\mu}$ and $P_{*|\theta}$ with the same
domain ${\cal Z}$ is %
\nomenclature[rDmu]{$D(\mu"|"|\theta)$}{The \emph{relative entropy}
  between two probability functions $P_{*"|\mu}$ and $P_{*"|\theta}$.}
\begin{align}
  \label{eq:RelativeEntropy}
  D(\mu||\theta) &\equiv \EV_{\mu} \log \left(
    \frac{P(Z|\mu)}{P(Z|\theta)} \right) \\
  &= \sum_{z \in \cal{Z}} P(z|\mu) \log\left(
  \frac{P(z|\mu)}{P(z|\theta)} \right) \nonumber.
\end{align}
The relative entropy is coordinate independent.  The relative entropy
between two probability functions $P_{*|\mu}$ and $P_{*|\theta}$ is never
negative and is zero if and only if the functions are the same on all
sets with finite probability.  We use $D(\mu||\theta)$ to
characterize the fidelity of a model $P_{*|\theta}$ to a \emph{true}
distribution $P_{*|\mu}$

\subsubsection{Cross entropy of two probability functions}
\index{entropy!cross}
While some authors use the terms \emph{relative entropy} and
\emph{cross entropy} interchangeably to mean the quantity
$D(\mu||\theta)$ that we defined in Eqn.\eqref{eq:RelativeEntropy}, we
define the cross entropy to be
\begin{align}
  \label{eq:CrossEntropy}
  H(\mu||\theta) &\equiv -\EV_{\mu} \log \left(P(Z|\theta) \right) \\
  &= -\sum_{z \in \cal{Z}} P(z|\mu) \log(P(z|\theta)) \nonumber
\end{align}
and note that
\begin{equation*}
  D(\mu||\theta) = H(\mu||\theta) - H(\mu).
\end{equation*}
The cross entropy is the negative expected log-likelihood of a model.
It is greater than the entropy unless the model $P_{*|\theta}$ is the same
as $P_{*|\mu}$ for all sets with finite probability.

\subsubsection{Stationary}
\index{stationary}
A stochastic process is \emph{stationary} if probabilities are
unchanged by constant shifts in time, \ie, for any two integers
$T\geq 1$ and $\tau\geq 0$
\begin{equation*}
  P_{\ts{Z}{1}{T}} =  P_{\ts{Z}{1+\tau}{T+\tau}}.
\end{equation*}

\subsubsection{Ergodic}
Roughly, in an \emph{ergodic} \index{ergodic|textbf} process you can
get anywhere from anywhere else.  Let ${\cal X}$ be the set of states
for a stationary stochastic process with probabilities $P_{*|\mu}$.
The process is ergodic if for any two subsets of ${\cal X}$, $A$ and
$B$ with $P(A|\mu) > 0$ and $P(B|\mu) > 0$ there is a time $T$ such
that the probability of going from set $A$ to set $B$ in time $T$ is
greater than zero.  Birkhoff's ergodic theorem says that for an
ergodic process, time averages converge to expected values with
probability one.

\subsubsection{Entropy rate}
For a discrete stochastic process $X$, the \index*{entropy!rate} is
\nomenclature[rhx]{$h(X)$}{The \emph{entropy rate} of a stochastic process $X$.}
\begin{equation}
  \label{eq:hrate1}
  h(X) \equiv \lim_{T \rightarrow \infty} \frac{1}{T} H(\ts{X}{1}{T}).
\end{equation}
If the process is stationary
\begin{equation}
  \label{eq:hrate2}
  h(X) = \lim_{T \rightarrow \infty} H(\ti{X}{T}|\ts{X}{1}{T-1}).
\end{equation}
If the process is stationary and Markov
\begin{equation}
  \label{eq:hrate3}
  h(X) =  H(\ti{X}{T+1}|\ti{X}{T})~\forall T.
\end{equation}
And if the process is ergodic
\begin{equation}
  \label{eq:hrate4}
  h(X) = \lim_{T \rightarrow \infty} - \frac{1}{T} \log \left( P \left(\ts{x}{1}{T}|\mu \right) \right)
\end{equation}
with probability one.

We similarly define the relative entropy rate and the cross entropy
rate.  For an ergodic process $X$ with true probabilities $P_{*|\mu}$ and
model probabilities $P_{*|\theta}$
\begin{align}
  \label{eq:hrate5}
  h(\mu||\theta) &\equiv  \lim_{T \rightarrow \infty} - \frac{1}{T}
  \EV_\mu \log \left( P \left(\ts{X}{1}{T}|\theta \right) \right) \\
  &=  \lim_{T \rightarrow \infty} - \frac{1}{T} \log \left( P
  \left(\ts{x}{1}{T}| \theta \right) \right) \nonumber
\end{align}
with probability one.

\subsubsection{Entropy rate of a partition $\mathcal{B}$}
\index{partition} Let ${\cal X}$, the set of states for a
stationary stochastic process with probabilities $P_{*|\mu}$, be a
continuum, and let $\mathcal{B} = \left\{ \beta_1,\beta_2,\ldots
  \beta_n \right\}$ be a partition of ${\cal X}$ into a finite number
of non-overlapping subsets.  By setting $\ti{b}{t}$ to the index of
the element of $\mathcal{B}$ into which $\ti{x}{t}$ falls, we can map
any sequence $\ts{x}{1}{T}$ into a sequence $\ts{b}{1}{T}$ thus
defining a discrete stationary stochastic process $B$.  Applying the
definition of entropy rate to the process $B$ yields the definition of
the entropy rate as a function of partition $\mathcal{B}$.  Suppose in
particular that on some set ${\cal X}$ the map $F:{\cal X}\mapsto{\cal
  X}$ and the probability $P_{*|\mu}$ define an ergodic process, that
$\mathcal{B}$ is a partition of ${\cal X}$, and that the model
probability function $P_{*|\theta}$ assigns probabilities to sequences
of partition indices $\ts{b}{1}{T}$.  We define the entropy rate $
h(\mathcal{B},F,\mu)$ and the \index*{cross entropy rate} %
$h(\mathcal{B},F,\mu||\theta)$ as follows
\begin{align}
  \label{eq:hrate6}
  h(\mathcal{B},F,\mu) &\equiv \lim_{T \rightarrow \infty} \frac{1}{T}
  H(\ts{B}{1}{T}) \\
  &= \lim_{T \rightarrow \infty} \frac{1}{T} \EV_\mu \log\left(P
  \left(\ts{B}{1}{T}| \mu \right) \right) \nonumber\\
  \label{eq:hrate7}
  h(\mathcal{B},F,\mu||\theta) &\equiv \lim_{T \rightarrow \infty}
  \frac{1}{T} \EV_\mu \log\left(P \left(\ts{B}{1}{T} | \theta \right)
  \right). \nonumber  
\end{align}
\subsubsection{Kolmogorov Sinai entropy $h_{KS}$}
\index{Kolmogorov Sinai entropy|textbf}%
\index{entropy!Kolmogorov Sinai|see{Kolmogorov Sinai entropy}}%
\nomenclature[rhw]{$h_{KS}$}{Kolmogorov Sinai entropy.}%
As before, suppose that on some set ${\cal X}$, the map $F:{\cal
  X}\mapsto{\cal X}$ and the probability $P_{*|\mu}$ define an ergodic
process.  The least upper bound over all partitions $\mathcal{B}$ on
the entropy rate $h(\mathcal{B},F,\mu)$ is the called the
\emph{Kolmogorov Sinai entropy}
\begin{equation}
  \label{eq:hKS}
  h_{KS}(F,\mu) \equiv \sup_{\mathcal{B}} h(\mathcal{B},F,\mu).
\end{equation}

\section{Stretching and Entropy}
\label{sec:pesin}

Here we will outline theory that connects ideas from dynamics to ideas
from probability.  The main results say that average dynamical
stretching (Lyapunov exponents) is proportional to average uncertainty
(entropy) in measurement sequences.  First we will give some examples,
then we will quote definitions and theorems without proof.

\subsection{Maps of the unit circle}
\label{sec:PesinExamples}
\index{unit circle!map of}
\index{map of unit circle}

\subsubsection{Two $x$ mod one}
\label{sec:TwoX}

The map of the unit interval $[0,1)$ into itself defined by
\begin{align}
  \ti{x}{n+1} &= F_2(\ti{x}{n})\\
  \label{eq:twox}
  &\equiv 2x \mod 1
\end{align}
is continuous if we identify the points $0$ and $1$.  Notice that if
we use the partition
\begin{equation}
  \label{eq:partition2}
  \mathcal{B}_2 = \left\{ \beta_0 = [0,\frac{1}{2}),\beta_1 =
  [\frac{1}{2}, 1 )\right\},  
\end{equation}
a symbol sequence $\ts{b}{0}{\infty}$ provides the coefficients of a
base two power series that identifies the starting point, \ie
\begin{equation*}
  \ti{x}{0} = \sum_{t=0}^\infty  \ti{b}{t} \left(\frac{1}{2}\right)^{t+1}.
\end{equation*}
Further, if we assign a uniform probability measure $\mu$ to the interval
\begin{equation*}
  h_{KS}(F_2,\mu) =h(\mathcal{B}_2,F,\mu) = \log(2).
\end{equation*}

\subsubsection{Three $x$ mod one}
\label{sec:ThreeX}

Analogously, by defining the map
\begin{align}
  \ti{x}{n+1} &= F_3(\ti{x}{n})\\
  \label{eq:threex}
  &\equiv 3x \mod 1,
\end{align}
and using the partition
\begin{equation}
  \label{eq:partition3}
  \mathcal{B}_3 = \left\{ \beta_0 = [0,\frac{1}{3}),\beta_1 =
    [\frac{1}{3}, \frac{2}{3} ), \beta_2 = [\frac{2}{3}, 1 )\right\},
\end{equation}
and uniform probability measure $\mu$, we find
\begin{equation*}
  h_{KS}(F_3,\mu) =h(\mathcal{B}_3,F_3,\mu) = \log(3).
\end{equation*}

This is the result that motivated Kolmogorov and Sinai to define the
entropy $h_{KS}$.  They were addressing the isomorphism problem, \eg, ``Is
there a relabeling of points in the unit interval that makes $F_2$ the
same as $F_3$?''.  Since the characteristic $h_{KS}$ is independent of
the coordinates or labeling used, the fact that $h_{KS}(F_3,\mu) \neq
h_{KS}(F_2,\mu)$ provided a negative answer to the question.

Notice that the Kolmogorov entropy is equal to the average of the log
of the slope of the map.  Specifically, the slope of $F_2$ is 2 and
$h_{KS}(F_2,\mu) = \log(2)$ while the slope of $F_3$ is 3 and
$h_{KS}(F_3,\mu) = \log(3)$.  The rule that entropy is proportional to
the log of the average slope is not true in general.  The next example
provides a counter example and suggests a correction factor.

\subsubsection{Dynamics on a \index*{Cantor set}}
\label{sec:Cantor}

While every point in the entire unit interval can be represented as a
base three power series, \ie
\begin{equation*}
  \forall x \in [0,1),~ \exists d_0^\infty :~x  = \sum_t^\infty
  \ti{d}{t} \left(\frac{1}{3}\right)^{t+1} \text{ with } \ti{d}{t} \in
  \left\{ 0,1,2 \right\} ~ \forall t,
\end{equation*}
the middle third Cantor set consists of the points in the unit
interval that can be represented as base three power series that
exclude the digit ``1''.  The symbol sequences produced by applying
the map $F_3$ and partition $\mathcal{B}_3$ to the middle third Cantor
set are the sequences of coefficients in the base three expansions of
the starting points, \ie, they consist exclusively of $0's$ and $2's$.
Given any finite sequence $d_0^n$, we define the set of infinite
coefficient sequences $\left\{d_0^n,\ldots \right\}$ as those that
begin with the sequence $d_0^n$.  Now we define a probability measure
$\mu_c$ in terms of such sets of infinite sequences,
\begin{equation}
  \label{eq:muC}
  \mu_c\left( \left\{ d_0^n,\ldots \right\} \right) \equiv 
  \begin{cases}
    2^{-(n+1)} & \text{if ``1'' does not appear in } d_0^n\\
    0 & \text{if ``1'' does appear in } d_0^n
  \end{cases}
\end{equation}
With this measure we find
\begin{equation*}
  h_{KS}(F_3,\mu_c) = \log(2).
\end{equation*}

The following isomorphism or relabeling of the unit interval connects
$(F_2,\mu)$ to $(F_3,\mu_c)$:
\begin{enumerate}
\item Find the binary expansion $\ts{b}{0}{\infty}$ of the original
  point $x$
\item Create a new sequence $\ts{d}{0}{\infty}$ by replacing every
  occurrence of ``1'' in  $\ts{b}{0}{\infty}$ with ``2''
\item Map $x$ to $y$ where $y$ is described by the base three
  expansion $\ts{d}{0}{\infty}$
\end{enumerate}

The \index*{Hausdorff dimension}\footnote{Sets and characteristics of
  sets with non-integer dimensions are sometimes called
  \emph{fractal}.} of the middle third Cantor set is
$\delta = \frac{\log(2)}{\log(3)}$, and that is the factor that is
missing in the formula connecting entropy and stretching.
\begin{align}
  h_{KS}(F_3,\mu_c) &= \log(2) \nonumber \\
  &= \frac{\log(2)}{\log(3)} \log(3) \nonumber \\
  \label{eq:CantorCorrect}
  &= \delta \log(\text{stretching factor})
\end{align}
Now we turn to the definitions and theorems that express the above
idea precisely.

\section{Lyapunov Exponents and Pesin's Formula}
\label{sec:PesinFormula}

Vixie\cite{vixie02} has reviewed the work of
Ruelle\cite{ruelle-1978-1}, Pesin\cite{pesin77},
Young\cite{young95-1}, and others who established the relationship
between smooth dynamics and entropy.  Here we reiterate a few of those
results using the following notation:
\begin{description}
\item[$X$] An $n$-dimensional manifold
\item[$F:X\mapsto X$] An invertible $C^2$ (continuous with continuous
  first and second derivatives) map of the manifold into itself
\item[$\mu$] A probability measure on $X$ that is invariant under $F$
\item[$x$] A point on the manifold
\item[$TX(x)$] The tangent space of $X$ at $x$
\item[$v$] An element of $TX(x)$
\end{description}

We define the asymptotic growth rate of the direction $v$ at $x$ as
\begin{equation}
  \label{eq:growthVX}
  \lambda(F,x,v) \equiv \lim_{t\rightarrow \infty} \frac{1}{t} \log
  \left( \left\| [DF^t(x)] v \right\| \right).
\end{equation}
\index*{Oseledec's theorem} ~\cite{young95-1,katok95, mane87} says
that at almost every $x$ the limit exists for every $v$, and that
although the value of the limit depends on $v$, that as $v$ varies, it
only takes on $r \leq n$ discrete values called the \emph{spectrum of
  \index*{Lyapunov exponents}}
\begin{equation}
  \label{eq:LyapunovSpectrum}
  \lambda_1(F,x) > \lambda_2(F,x) > ... > \lambda_r(F,x).
\end{equation}
The tangent space $TX(x)$ is the direct sum of subspaces $E_i \subset
TX(x)$ associated with each exponent, with
\begin{equation*}
  \lambda(F,x,v) = \lambda_i(F,x) ~ \forall v \in E_i
\end{equation*}
and
\begin{equation*}
  TX(x) = \bigoplus_{i=1}^r E_i. %\bigoplus?
\end{equation*}
The dimension of $E_i$ is called the \emph{multiplicity} $m_i$ of the
exponent $\lambda_i$. If $\mu$ is ergodic with respect to $F$, then
the spectrum $\left\{ \lambda_i \right\}$ is the same almost
everywhere.

We want to use \index*{Pesin's formula}\cite{pesin77} which implies
that if $\mu$ is smooth and ergodic, then the entropy is equal to the
sum of the positive Lyapunov exponents, \ie
\begin{equation}
  \label{eq:pesin}
  h_{KS}(F,\mu) = \sum_{i:\lambda_i >0} m_i \lambda_i.
\end{equation}
In light of the correction for fractal dimension that we saw in
Eqn.~\eqref{eq:CantorCorrect} and the ubiquity of fractal measures in
chaotic systems, we should review Ledrappier and Young's explanation
(See \cite{young95-1} for an overview) of the effect of fractal
measures on Pesin's formula.

Ledrappier and Young's formula is given in terms of the dimensions of
the conditional measures on the nested family of \emph{unstable
  foliations} of $F$.  For a point $x\in X$ and $i$ such that
$\lambda_i>0$ we define
\begin{equation}
  \label{eq:Foliation}
  W^i(x) \equiv \left\{ y \in X \text{ such that } \lim_{n \rightarrow
  \infty} \sup \frac{1}{t} \log \left( d\left( F^{-t}(x), F^{-t}(y)
  \right) \right) <
  -\lambda_i \right\}.
\end{equation}
For an intuitive picture, consider a trajectory $x(t)$ that passes
through $x$ at time $t=0$; any trajectory $y(t)$ that has separated
from $x(t)$ at a rate of at least $\lambda_i$, passes through the
manifold $W^i(x)$ at time $t=0$.

Let $\delta_i$ be the Hausdorff dimension of the conditional measure
that $\mu$ defines on $W^i(x)$.  For an ergodic $\mu$, $\delta_i$ will
be constant almost everywhere.  Further, let $\gamma_i$ be the
incremental dimension
\begin{equation*}
  \gamma_i \equiv
  \begin{cases}
    \delta_1 & i=1\\
    \delta_i - \delta_{i-1} & i > 1
  \end{cases}
\end{equation*}
Now \index*{Ledrappier and Young's formula} is
\begin{equation}
  \label{eq:Ledrappier}
  h_{KS}(F,\mu) = \sum_{i : \lambda_i > 0} \lambda_i \gamma_i.
\end{equation}

Note that Pesin's formula holds if the measure $\mu$ is smooth in the
unstable directions.  Such measures are called SRB (Sinai Ruelle
Bowen) measures.  \index{SRB measure} Tucker has found that the Lorenz
system has an SRB measure and says that numerical simulations of
Lorenz's system are ``real''~\cite{tucker99}.

\subsection{A theoretical bound on model likelihood}
\label{sec:TheoreticalBound}

Now we have the terms that we need to discuss theoretical bounds on
the expected log likelihood of models of discrete observations of a
chaotic dynamical system.  Given $X$, $F$, and $\mu$ as described
above, if the multiplicity of each exponent is $m_i = 1$ then we know:
\begin{align}
  \label{eq:Bound1}
  h({\cal{B}},F,\mu) &\equiv - \lim_{t \rightarrow
    \infty} \EV_\mu \log \left( P \left( \ti{b}{t}|\ts{b}{1}{t-1}, \mu
    \right) \right) \\
  \label{eq:Bound2}
  h({\cal{B}},F,\mu||\theta) &\equiv - \lim_{t \rightarrow \infty}
  \EV_\mu \log \left( P \left( \ti{b}{t}|\ts{b}{1}{t-1}, \theta
    \right) \right) \\
  \label{eq:Bound3}
  h({\cal{B}},F,\mu) &\leq h({\cal{B}},F,\mu||\theta)  &&
  \text{Equality } \iff \theta = \mu \text{ a.e.}\\
  \label{eq:Bound4}
  h({\cal{B}},F,\mu) &\leq h_{KS}(F,\mu)  &&
  \text{Equality } \iff {\cal{B}} \text{ generating} \\
  \label{eq:Bound5}
  h_{KS}(F,\mu) &= \sum_{i : \lambda_i > 0} \lambda_i \gamma_i\\
  \label{eq:Bound6}
  h_{KS}(F,\mu) &\leq \sum_{i : \lambda_i > 0} \lambda_i && \mu \text{
  smooth on } W^i \Rightarrow \text{Equality}
\end{align}
with the following justifications
\begin{description}
\item[\eqref{eq:Bound1} {\mdseries and} \eqref{eq:Bound2}:] Definition
\item[\eqref{eq:Bound3}:] Gibbs inequality, \eqref{eq:GibbsIE}
\item[\eqref{eq:Bound4}:] The definition of $h_{KS}(F,\mu)$ is that it is the
  \emph{supremum} over all partitions ${\cal{B}}$
\item[\eqref{eq:Bound5}:] This is Ledrappier and Young's formula
  \eqref{eq:Ledrappier}
\item[\eqref{eq:Bound6}:] Because in \eqref{eq:Bound5} $0 \leq \gamma_i \leq
  1~ \forall i$
\end{description}
Thus we have the following two theorems:
\begin{theorem}[Lyapunov exponent bound on likelihood]
  If $\mu$ is ergodic and smooth in the unstable directions and
  $\cal{B}$ is a generating partition, then for any model $\theta$ of
  the stochastic process $B$ consisting of $F,\mu$, and $\cal{B}$
  \begin{equation}
    \label{eq:BoundTheorem}
    h({\cal{B}},F,\mu||\theta) \geq \sum_{i : \lambda_i > 0} \lambda_i
  \end{equation}
\end{theorem}
\begin{theorem}[Entropy gap] \label{GapTheorem}
  \index{entropy!gap|textbf}
  If $\mu$ is ergodic (not necessarily smooth in the unstable
  directions).  Then for an optimal model $\theta$ of the stochastic
  process $B$ consisting of $F,\mu$, and $\cal{B}$ ($\cal{B}$ not
  necessarily generating)
  \begin{equation}
    \label{eq:GapTheorem1}
    h({\cal{B}},F,\mu||\theta) = h({\cal{B}},F,\mu) \leq  \chi \equiv \sum_{i :
    \lambda_i > 0} \lambda_i
  \end{equation}
  and if for some other model $\nu$
  \begin{equation}
    \label{eq:GapTheorem2}
    h({\cal{B}},F,\mu||\nu) \geq \chi
    % \equiv \sum_{i : \lambda_i > 0} \lambda_i m_i
  \end{equation}
  then the model $\nu$ is not optimal.
\end{theorem}

In the next section, we will describe a numerical procedure for
estimating Lyapunov exponents, and in the following section we will
argue that one can reasonably use Eqn.~\eqref{eq:GapTheorem2} with
numerical simulations to quantitatively characterize the
non-optimality of a model.

\section{Benettin's Procedure for Calculating Lyapunov Exponents Numerically}
\label{sec:Benettin}

We begin reviewing \index*{Benettin's procedure}\cite{Benettin80} for
estimating Lyapunov exponents by using the Lorenz system as an
example.  The Lorenz system is
\begin{align*}
  \dot x = F(x) = 
  \begin{bmatrix}
    s(x_2-x_1)\\ x_1(r - x_3) -x_2 \\ x_1 x_2 - bx_3.
  \end{bmatrix}
\end{align*}
Note that
\begin{equation*}
  D F(x) = 
  \begin{bmatrix}
    -s & s & 0 \\ r-x_3 & -1 & -x_1 \\ x_2 & x_1 & -b
  \end{bmatrix}
\end{equation*}
where $\left(D F(x)\right)_{i,j} \equiv \frac{\partial F_i(x)}{\partial
  x_j}$.  Let $\Phi$ denote solutions to the differential equation
with
\newcommand{\DM}{{\cal{D}}} % Derivative Matrix, or tangent
\newcommand{\ct}{\tau}      % Continuous time
\begin{equation*}
  \ti{x}{\ct} \equiv \Phi(\ti{x}{0},\ct).
\end{equation*}
Lyapunov exponents are defined (recall Eqn.~\eqref{eq:growthVX}) in
terms of the long time behavior of the derivative matrix
\begin{equation*}
  \ti{\DM}{\ti{x}{0},\ct} \equiv D_{\ti{x}{0}} \Phi(\ti{x}{0},\ct).
\end{equation*}
Interchanging the order of differentiation with respect to $\ti{x}{0}$
and $\ct$ and applying the chain rule yields a \emph{linear}
differential equation for $\DM$:
\begin{align*}
  \dot \DM(\ti{x}{0},\ct) &=  \frac{d}{d\ct} D_{\ti{x}{0}}
  \Phi(\ti{x}{0},\ct)\\
  &= \left. D F(x) \right|_{x=\Phi(\ti{x}{0},\ct)}
  \DM(\ti{x}{0},\ct).
\end{align*}
Thus, given initial conditions $\ti{x}{0}$ and $\ti{\DM}{0} = \id $
%\begin{bmatrix}  1 & 0 & 0 \\ 0 & 1 & 0 \\ 0 & 0 & 1\end{bmatrix}$
one can  use an off-the-shelf routine to find $\begin{bmatrix}
  \ti{x}{\ct}\\ \ti{\DM}{\ct}\end{bmatrix}$ by integrating 
\begin{equation}
  \label{eq:TangentODE}
  \dot {\begin{bmatrix} \ti{x}{\ct}\\\ti{\DM}{\ct}\end{bmatrix}} = 
  \begin{bmatrix}
    F(x) \\  [D F(x)] \DM
  \end{bmatrix}.
\end{equation}

Given a computer with infinite precision, for a range of time
intervals $\tau$, one could:
\begin{enumerate}
\item Integrate Eqn.~\eqref{eq:TangentODE} to obtain $\ti{\DM}{\ct}$
\item \label{item:SVD} Do singular value decompositions (SVD's)
  \begin{equation}
    \label{eq:TangentSVD}
    \ti{U}{\ct} \ti{S}{\ct} \ti{V\transpose}{\ct} = \ti{\DM}{\ct},
  \end{equation}
  where $\ti{U}{\ct}$ and $\ti{V}{\ct}$ are orthogonal and $\ti{S}{\ct}$ is
  diagonal
\item Look for approximate convergence of the finite time Lyapunov
  exponent estimates:
  \begin{equation}
    \label{eq:lambdaSVD}
    \ti{\tilde \lambda_i}{\ct} \equiv \frac{1}{\ct} \log( \ti{S_{i,i}}{\ct} ) .
  \end{equation}
\end{enumerate}
On a real computer, the procedure fails because the ratio of the
largest and smallest singular values
$\frac{\ti{S_1}{\ct}}{\ti{S_d}{\ct}}$ grows exponentially with $\ct$
and becomes larger than the precision of the machine.

Rather than using an SVD decomposition for each $\tau$ in
step~\ref{item:SVD} above, one could use a QR decomposition:
\begin{equation}
  \label{eq:TangentQR}
  \ti{Q}{\ct} \ti{R}{\ct} = \ti{\DM}{\ct}.
\end{equation}
A QR decomposition factors the matrix $\ti{\DM}{\ct}$ into a product
of two matrices the first of which $\ti{Q}{\ct}$ is orthogonal and the
second of which $\ti{R}{\ct}$ is upper triangular.  One could use the
intuitive Gram Schmidt procedure, but there are algorithms that behave
better numerically (see, \eg \cite{GandL3} or \cite{Press92}).
Although the diagonal elements of $\ti{R}{\ct}$ are not equal to the
diagonal elements of $\ti{S}{\ct}$, the finite time estimates
\begin{equation}
  \label{eq:lambdaQR}
  \ti{\hat \lambda_i}{\ct} \equiv \frac{1}{\ct} \log( \left|
  \ti{R_{i,i}}{\ct} \right| )
\end{equation}
and the $\ti{\tilde \lambda_i}{\ct}$ defined in
Eqn.~\eqref{eq:lambdaSVD} converge to the same values\footnote{In the
  SVD of Eqn.~\eqref{eq:TangentSVD}, the first column of $\ti{V}{\ct}$
  specifies the direction of the initial vector in the tangent space
  with the largest stretching.  The exponential stretching rate is the
  Lyapunov exponent $\lambda_1$.  However, with probability one, a
  randomly chosen vector will have the same stretching rate.  The
  estimate $\ti{\hat \lambda_1}{\ct}$ of Eqn~\eqref{eq:lambdaQR} is
  based on the stretching rate of the first standard basis vector, \eg,
  $[1,0,0]$.
  Similar arguments using the growth rates of areas, volumes,
  hyper-volumes, etc. support using the estimates $\ti{\hat
    \lambda_i}{\ct}$ of Eqn~\eqref{eq:lambdaQR} for $i=2,3,\ldots$}.

Using Eqn.~\eqref{eq:lambdaQR} does not address the problem of finite
machine precision for long time intervals $\tau$, but Benettin et al.
recommend calculating $\log( \left| \ti{R_{i,i}}{\ct} \right| )$ by
breaking the interval into $N$ smaller steps of duration $\Delta\tau$
in a way that does address finite precision.  Letting $\ti{A}{n}$
denote the one time step derivative
\begin{equation}
  \label{eq:Adef}
  \ti{A}{n} \equiv D \Phi( \ti{x}{(n-1) \Delta\tau}, \Delta\tau )
\end{equation}
the chain rule implies
\begin{equation*}
  D \Phi( \ti{x}{0}, N \Delta\tau ) = \prod_{n=1}^N \ti{A}{n}.
\end{equation*}
If, for each $n$, one calculates\footnote{To calculate $\ti{Q}{n}$ and
  $\ti{r}{n}$ for each $n$, one can either:
  \begin{enumerate}
  \item Integrate Eqn.~\eqref{eq:TangentODE} for a time interval $\Delta\tau$
    with the initial condition $\begin{bmatrix} \ti{x}{(n-1)\Delta\tau} \\
      \ti{Q}{n-1} \end{bmatrix}$ to obtain $\begin{bmatrix}
      \ti{x}{n\Delta\tau} \\ \ti{A}{n} \ti{Q}{n-1}\end{bmatrix}$ and then
    calculate a QR factorization of $\ti{A}{n} \ti{Q}{n-1}$, the
    second component of the result.
  \item As above, but use the identity matrix instead of $\ti{Q}{n-1}$
    as the second component of the initial condition for the
    integration which yields the result $\begin{bmatrix}
    \ti{x}{n\Delta\tau} \\ \ti{A}{n}\end{bmatrix}$, then calculate a QR
    factorization of the product $\ti{A}{n}\ti{Q}{n-1}$
  \end{enumerate} }
                                %
the pair $\left( \ti{Q}{n}, \ti{r}{n}
\right)$ defined by
\begin{align*}
  \ti{Q}{0} &= \id \\
  \ti{Q}{n} \ti{r}{n} &\equiv \ti{A}{n} \ti{Q}{n-1},
\end{align*}
where $\ti{Q}{n}$ and $\ti{r}{n}$ are obtained by a QR factorization
of the product $\ti{A}{n} \ti{Q}{n-1}$, then induction yields
\begin{equation*}
  \prod_{n=1}^N \ti{A}{n} =\ti{Q}{N} \prod_{n=1}^N \ti{r}{n}.
\end{equation*}
Since $\prod_{n=1}^N \ti{r}{n}$ is upper triangular, we have the QR
factorization
\begin{align}
  \label{eq:QR1}
  D \Phi( \ti{x}{0}, N \Delta\tau ) &= \ti{Q}{N} \ti{R}{N} \\
  \label{eq:QR2}
  \ti{R}{N} &= \prod_{n=1}^N \ti{r}{n}.
\end{align}
And since $\ti{R_{i,i}}{N} = \prod_{n=1}^N \ti{r_{i,i}}{n}$,
\begin{equation}
  \label{eq:rii}
  \log( \left| \ti{R_{i,i}}{n} \right| ) = \sum_{n=1}^N \log( \left|
    \ti{r_{i,i}}{n} \right| ).
\end{equation}
Substituting this result into Eqn.~\eqref{eq:lambdaQR} constitutes the
Benettin procedure.  The action of a matrix on a unit square is
factored into components $Q$ and $R$ and sketched in
Fig.~\ref{fig:QR}.  Results of applying the procedure to the Lorenz
system appear in Fig.~\ref{fig:benettin}.
\begin{figure}[htbp]
  \centering{\plotsize%
    \def\Mone{$ \begin{bmatrix} \begin{bmatrix} e_1
        \end{bmatrix} & \begin{bmatrix} e_2 \end{bmatrix} \end{bmatrix} $}%
    \def\Mtwo{$ R \begin{bmatrix} \begin{bmatrix} e_1
        \end{bmatrix} & \begin{bmatrix} e_2 \end{bmatrix} \end{bmatrix} $}%
    \def\Mthree{$ QR \begin{bmatrix} \begin{bmatrix} e_1
        \end{bmatrix} & \begin{bmatrix} e_2 \end{bmatrix} \end{bmatrix} $}%
    \def\Mfour{$ \begin{bmatrix} 1 & 0 \\ 0 & 1 \end{bmatrix} $}%
    \def\Mfive{$ \begin{bmatrix} 5 & 0.15 \\ 0 & 0.2 \end{bmatrix} $}%
    \def\Msix{$ \begin{bmatrix}3 & 0.25 \\ -4 & 0 \end{bmatrix} $}%
    \input{QR.pdf_t}}
  \caption[\comment{fig:QR }The action of the $Q$ $R$ factors of a matrix on a unit square.]%
  {The action of the $Q$ $R$ factors of a matrix on a unit square.
    Here $A=
    \begin{bmatrix} 3 & 0.25 \\ -4 & 0 \end{bmatrix}$, $Q=
    \begin{bmatrix} 0.6 & 0.8 \\ -0.8 & 0.6 \end{bmatrix}$, and $R=
    \begin{bmatrix} 5 & 0.15 \\ 0 & 0.2 \end{bmatrix}$.  $R$ stretches the
    $x$ component by a factor of five and shears $y$ components in the
    $x$ direction and shrinks them by a factor of five with a net
    effect of preserving areas (The determinants of
    $A$ and $R$ are both 1.0).  $Q$ simply rotates the stretched figure.}
  \label{fig:QR}
\end{figure}
\begin{figure}[htb]
  \centering{\includegraphics[width=1.0\textwidth]{benettin.pdf}
  }
  \caption[\comment{fig:benettin }Lyapunov exponent calculation for
  the Lorenz system.]%
  {Lyapunov exponent calculation for the Lorenz system.  In the upper
    part, the colored traces are plots of
    $\frac{1}{T} \sum_{t=1}^T \log\left( \left| \ti{r_{1,1}}{t}
      \right| \right)$ (See Eqn.~\eqref{eq:rii}) for three
    different initial conditions and the black traces the 5\% and 95\%
    limits on 1,000 separate runs.  The lower part is the same except
    that $\left| \ti{r_{1,1}}{t} \right|$ is augmented by a noise term
    with amplitude $\frac{\sigma_\eta}{\Delta} = 0.01$ (See
    Eqn.~\eqref{eq:LE.aug}).}
  \label{fig:benettin}
\end{figure}

\section{A Practical Performance Bound}
\label{sec:PracticalBound}

Consider the following two cases:
\begin{itemize}
\item State space dynamics perturbed by noise
\item Simulated dynamics perturbed by numerical truncation
\end{itemize}
The definition of \index*{Kolmogorov Sinai entropy} in the two cases yields
extremely different answers.  If the perturbations are random noise,
then the supremum of $h(\mathcal{B},F,\mu)$ over $\mathcal{B}$ does
not exist and $h_{KS}$ is unbounded.  On the other hand, if the
perturbations are numerical truncation and the process is a digital
simulation, then all observation sequences converge to periodic cycles
and $h_{KS} = 0$.  Thus, the \emph{strict} definition of the
Kolmogorov Sinai entropy is useless as a bound on the cross entropy of
models in numerical simulations.  Here we argue however, that
numerical Lyapunov exponent estimates nonetheless provide a
\emph{practical} reference for the performance of models.

If you are working on a new model building procedure that takes
\emph{training} samples $\left\{ \ts{y}{\tau_1}{T_1},
  \ts{y}{\tau_2}{T_2}, \ldots, \ts{y}{\tau_N}{T_N}, \right\}$ and
produces a family of parameterized conditional probability functions
$P \left(\ti{y}{t}|\ts{y}{1}{t-1},\theta \right)$, we recommend
numerically estimating the \emph{entropy gap} (see
Theorem~\ref{GapTheorem}) $\delta_{\mu||\theta} =
h({\cal{B}},F,\mu||\theta) - \sum_{i : \lambda_i > 0} \lambda_i$ to
characterize the fidelity of the resulting models $\theta$ to
generating processes.  As a debugging tool, it is reasonable to choose
some parameters $\theta'$ for a model class, use that model to
generate training data, and then verify that as the size of the
training data set increases the proposed model building procedure
recovers the parameters $\theta'$.  However such a test fails to
consider how well the proposed procedure and model class work on the
realistic case of data generated by processes outside the model class.
Even though the test we propose does not provide \emph{correct} model
parameters against which to compare fitted parameters, it does provide
a reference against which to compare model performance.  Specifically,
we advocate the following numerical experiment for evaluating a model
building procedure:
\begin{enumerate}
\item \label{PPB1} Use a numerical simulation of a chaotic dynamical
  system to generate training data and testing data.  For simplicity,
  consider a system with a single positive exponent $\lambda_1$
\item \label{PPB2} Quantize the data to a few thousand levels
\item \label{PPB3} Run the Benettin procedure on the system, to
  estimate Lyapunov exponents
\item \label{PPB4} Substitute the estimated exponents into Ledrappier
  and Young's formula, Eqn.~\eqref{eq:Ledrappier} with $\gamma_1 = 1$
  to get $\hat h(F,\mu)) = \hat \lambda_1$, an estimated entropy rate.
  If the partition $\mathcal{B}$ is fine enough, $\hat
  h(\mathcal{B},F,\mu)) = \hat \lambda_1$ will be a good estimate.
\item \label{PPB5} Produce $P_{*|\theta}$ by applying the new model building
  procedure to the training data
\item \label{PPB6} Estimate the cross entropy by evaluating the
  likelihood of the model on long sequences of testing data
  \begin{equation}
    \label{eq:ToyCE}
    \hat h(\mathcal{B},F,\mu||\theta) = \frac{-1}{T} \sum_{t=1}^T \log \left(
    P \left(\ti{y}{t}|\ts{y}{1}{t-1},\theta \right)\right)
  \end{equation}
\item \label{PPB7} Calculate an entropy gap \index{entropy!gap} by
  subtracting the two estimates
  \begin{equation*}
    \hat \delta_{\mu||\theta} = \hat h(\mathcal{B},F,\mu||\theta)
    -\hat h(\mathcal{B},F,\mu)).
  \end{equation*}
  For an optimal model, expect the gap to be zero.  If the gap is much
  larger than zero, conclude that the new procedure is suboptimal.
\end{enumerate}
The test is reasonable only if, subject to some constraints, $\EV \hat
\delta_{\mu||\theta} \geq 0$ is a tight bound and the variance of
$\hat \delta_{\mu||\theta}$ is small.  Below, we argue that a model
that uses knowledge of the generating process and has smooth
probability densities in state space achieves the bound with equality
and thus the bound is tight.

In this class of models, the probability measure for the generating
process is not necessarily ergodic or even stationary; it is derived
from a uniform density over a box that covers possible starting
conditions, and it includes a little bit of noise in the dynamics so
that even in the long time limit it does not become fractal.  Because
the probability is smooth, the models cannot exploit fractal
properties that might exist in the modeled system and consequently
$\gamma$, the Ledrappier and Young correction to the Pesin formula, is
irrelevant.  More specifically we consider a model class with the
following properties: \newcommand{\Lic}{L_{\text{i.c.}}}
\begin{description}
\item[Probability density] The probability density for the initial
  state $P(\ti{x}{1}|\theta)$ is a uniform distribution on a cube in
  $X$ that has length $\Lic$ on each side.
\item[State noise] The model has noise in the state dynamics,
  \begin{equation}
    \label{eq:PracticalStateMap}
    \ti{x}{t+1} = F(\ti{x}{t}) + \ti{\eta}{t},
  \end{equation}
  where $\ti{\eta}{t}$ are \iid Gaussian with $\ti{\eta}{t} \sim
  \Normal(0,\id \sigma_\eta^2)$.  We suppose that $F$ is the same as
  the \emph{true} function of the \emph{modeled} system, but that
  noise in the modeled system is smaller or zero.
\item[Measurement function] We let the measurement function be the
  same for the model as for the true system, \ie, a discrete partition
  with resolution $\Delta$.  We have in mind a uniform quantization of
  a single component of $X$ such as we used for Fig.~\ref{fig:ToyH}.
\end{description}
The only difference between the true system and the model is that the
state noise in the model may be larger than the state noise in the
true system.

With this framework we can draw samples randomly from a true
distribution $P_{*|\mu}$ and consider model probabilities
$P_{*|\theta}$ without having to find a stationary distribution.  In
sidestepping the key issue of a stationary distribution, we have
sacrificed ergodicity which is the basis of the definition of a
Lyapunov exponent as a global property.  Empirically, however, the
convergence of the Benettin procedure is similar for any initial
condition (See Fig.~\ref{fig:benettin}).  Relying on this empirical
observation, we suppose for some time interval $\tau$ that the
stretching factor is independent of initial conditions with
\begin{equation}
  \label{eq:Stretch}
  \ti{S}{\tau} = \prod_{t=1}^\tau \ti{r_{1,1}}{t} = e^{\hat \lambda \tau}.
\end{equation}

In the limit of small noise $\sigma_\eta \rightarrow 0$, one can
calculate $P(\ts{y}{1}{T}|\theta)$ for any sequence of observations as
the probability of the set of initial conditions that are consistent
with  $\ts{y}{1}{T}$, \ie, the pre-image of $\ts{y}{1}{T}$,
\begin{equation*}
  P(\ts{y}{1}{T}|\theta) = \int_{ \left\{ x:\ts{Y}{1}{T}(x) =
      \ts{y}{1}{T} \right\} } P(x|\theta) dx.
\end{equation*}
For a $d$ dimensional system, the volume of such pre-images is typically
\begin{equation*}
  \frac{\Delta e^{-\hat \lambda T}}{O(T^d)} < \text{Vol} < \Delta e^{-\hat \lambda T},
\end{equation*}
and since the density of initial conditions is smooth, for large $T$
we find
\begin{subequations}
\begin{equation}
  \label{eq:back.image}
  \frac{1}{T} \log \left( P(\ts{y}{1}{T}|\theta) \right) \approx -\hat
  \lambda.
\end{equation}

Rather than going backwards in time to analyze the pre-image of
$\ts{y}{1}{T}$, we can think about the forward image of the volume of
initial conditions under the map $\Phi(T)$.  To first order, the
distribution is a uniform probability density over a parallelepiped
that extends a distance $\Lic \ti{s}{T}$ in the direction of the first
column of the orthogonal matrix $\ti{Q}{t}$ in Eqn.~\eqref{eq:QR1}.
The measurement partition divides the image into elements that have a
characteristic size of $\Delta$, yielding
\begin{equation}
  \label{eq:forward.image}
  \frac{1}{T} \log \left( P(\ts{y}{1}{T}|\theta) \right) \approx -\hat
  \lambda
\end{equation}
\end{subequations}
again.  Given the enormous stretching that occurs, it is clear that
the image of the volume of allowed initial conditions will resemble
steel wool more than a parallelepiped, but the exponential nature of
the stretching is all that matters, and in the small noise limit we
have
\begin{equation}
  \label{eq:zero.noise}
  h({\cal{B}},F,\mu||\theta) \approx \hat \lambda
\end{equation}

One might wonder whether finite state noise $\sigma_\eta$ invalidates
\eqref{eq:zero.noise} or if perhaps a correction of order
$\frac{\sigma_\eta}{\Delta}$ will suffice.  As long as $\sigma_\eta <<
\Lic e^{\hat \lambda T}$, the analysis of the size of the image of the
volume of initial conditions under $\Phi(T)$ that leads to
\eqref{eq:forward.image} is adequate.  The noise term $\sigma_\eta$ in
the model will however transfer probability from observation sequences
permitted by the true system to sequences that it does not allow,
thereby increasing the cross entropy.  At each time step the effect
roughly augments the stretching in each state space direction with a
term of size $\frac{\sigma_\eta}{\Delta}$.  Since the noise at each
time is independent of the noise at all other times, the effects add
in quadrature.  We can estimate an upper bound on the total effect by
replacing $ \left| \ti{r_{i,i}}{n} \right|$ with $ \left|
  \ti{r_{i,i}}{n} \right| + \frac{\sigma_\eta}{\Delta}$ for each $i$
and $n$ in Eqn.~\eqref{eq:rii} of the Benettin procedure, \ie,
\begin{equation}
  \label{eq:LE.aug}
  \hat \lambda_{\text{aug},i} \equiv \frac{1}{N}  \sum_{n=1}^N \log( \left|
    \ti{r_{i,i}}{n} \right|  + \frac{\sigma_\eta}{\Delta}).
\end{equation}
Notice that knowledge of $\hat \lambda_i$ and $\frac{\sigma_\eta}
{\Delta}$, is not sufficient to calculate $\hat
\lambda_{\text{aug},i}$.  If the stretching were uniform with $\left|
  \ti{r_{i,i}}{n} \right| = e^{\lambda} ~\forall n$, the augmented
result would be $\hat \lambda_{\text{aug},i} = \log\left( e^\lambda +
  \frac{\sigma_\eta} {\Delta} \right)$, but the result increases
without bound\footnote{If for example,
  \begin{equation*}
     \left| \ti{r_{i,i}}{n} \right| = \begin{cases}
       \delta^{N-1} e^\lambda & n=0\\
       \frac{1}{\delta} e^\lambda & \text{otherwise} \end{cases}
  \end{equation*}
  then $\hat \lambda_i = \lambda$, but $\lim_{\delta \rightarrow
    0}\hat \lambda_{\text{aug},i} = \infty$. } as $ \left| \ti{r_{i,i}}{n}
\right|$ varies more with $n$.  In Fig.~\ref{fig:benettin} we compare
$\hat \lambda_{\text{aug},1}$ with $\hat \lambda_{1}$ for the Lorenz
system.  For noise with an amplitude $\frac{\sigma_\eta}{\Delta} =
0.01$, the figure indicates an augmentation of $\hat
\lambda_{\text{aug},1} - \hat \lambda_{1} \approx 0.03$, which is
roughly ten times the augmentation that uniform stretching would
produce.  From the figure, we conclude that the Benettin procedure
produces a robust practical upper bound on model performance.


\section{Approaching the Bound}
\label{sec:approach}
\longpage%%% Maybe not required for MathTime... look for half
         %%% paragraph after the fig:LikeLor to see if it is needed.

Although the \emph{slope} of the plot in Fig.~\ref{fig:ToyH} (the log
likelihood per time step attained by extended Kalman filters) matches
the entropy bound, and we are satisfied with our explanation of the
nonzero intercept, an example of a model with a likelihood close to
the bound without any offset would be more satisfying.  To find such
model, we returned to the source of coarsely quantized Lorenz
observations that we used for Fig.~\ref{fig:Statesintro} in the
introduction.  That figure illustrates the association of each of the
twelve discrete hidden states of an HMM with particular regions in
$\REAL^3$, the state space of the Lorenz system.  Recall that we
generated the observations by integrating the Lorenz system with a
time step of $\tau_{\mathtt{sample}}=0.15$ and quantized the first
component into one of four levels.  Although the cross entropy of that
twelve state model is not very close to the bound based on our
Lyapunov exponent estimate ($\hat h(\mathcal{B},F,\mu)) = \hat
\lambda_1 = 0.136~\text{nats} = 0.196~\text{bits}$), it seems
\marginpar{FixMe: 0.15, 0.136, 0.196?}
plausible that by using HMMs with more states, we might get higher
likelihoods.  In fact it is true, but we are surprised at how many
states we needed.

Our first attempt was to train a sequence of models with ever more
hidden states.  We initialized each model randomly and ran many
iterations of the Baum-Welch algorithm on quantized observations.
Even with with many iterations, we did not build any promising models.

In our second attempt, we exploited our knowledge of the Lorenz
dynamics in $X=\REAL^3$ as follows:
\begin{enumerate}
\item Generate training data $\ts{x}{1}{T}$ and $\ts{y}{1}{T}$ by
  integrating the Lorenz system.  Here $\ti{x}{t}$ is a point in the
  original state space and $\ti{y}{t}$ is a quantized observation that
  can take one of four values.
\item Generate testing data $\ts{y}{T+1}{T+N}$ by continuing the integration.
\item Find a set of discrete states $\left\{s_1, s_2, \ldots,
    s_m\right\}$ by partitioning the original space with a uniform
  grid of resolution $\Delta_x$.  We only constructed states for those
  partition elements that were occupied by at least one member of
  $\ts{x}{1}{T}$.
\item Build an initial HMM using the training sequence.  We set the
  state transition probabilities by counting the frequency with which
  the partitioned $X$ sequence made each possible transition.
  Similarly, we set the observation model by counting the frequency
  with which each partition element was associated with each possible
  observation.
\item \label{step:estimate1} Estimate the cross entropy ($ \hat
  h(\mathcal{B},F,\mu||\theta)$ See Eqn.~\eqref{eq:ToyCE}) of the
  initial model by calculating its log likelihood per time step on the
  testing data.
\item \label{step:train} Improve the model with several Baum-Welch
  iterations.
\item \label{step:estimate2} Estimate the cross entropy of the trained
  model by calculating its log likelihood per time step on the testing
  data.
\end{enumerate}

As hoped, we found that as we reduced the resolution, the number of
states increased and the cross entropy estimates decreased.  We found
however that the computational cost of training models with large
numbers of states was prohibitive.  Although the cross entropy
estimates in step~\ref{step:estimate2} were lower than the estimates
in step~\ref{step:estimate1}, it was computationally cheaper to attain
any given cross entropy value by simply reducing the state partition
resolution enough, omitting steps~\ref{step:train}
and~\ref{step:estimate2}, and using the estimate from
step~\ref{step:estimate1}.

Since there is no training, \ie, Baum-Welch iterations, in this
abbreviated procedure, we could calculate the likelihood with a
variant of the forward algorithm that does not store or return
$\alpha$ or $\gamma$ values for the entire sequence.  In fact, given
typical observations $\ts{y}{1}{t}$ up to time $t$, only a small
fraction of the states have nonzero probability.  Code that exploits
these features is many orders of magnitude cheaper than code for the
vanilla forward algorithm.

\longpage%%% Maybe not required for MathTime... look for half
         %%% paragraph after the fig:LikeLor to see if it is needed.
Results of the abbreviated procedure appear in Fig.~\ref{fig:LikeLor}.
\marginpar{FixMe: numbers in this paragraph?}
For the rightmost point on the curve we found a model with 1,490,202
hidden states and a cross entropy of 0.153 nats or 0.221 bits.  By
connecting this model to a simple data compression routine one could
compress the test data (or presumably any other long sequence from the
source) down to 0.221 bits per sample, which is 13\% more than the
0.196 bits per sample that the Lyapunov exponent estimate suggests is
possible.  Address space limits of 32 bit Python made it difficult to
build larger models and extend the plot to the right.
\begin{figure}[htbp]
  \centering{
    \includegraphics[width=1.0\textwidth]{LikeLor.pdf}}
  \caption[\comment{fig:LikeLor }Entropy gap, $\hat \delta_{\mu||\theta}$ vs number of states in HMMs]%
  {Entropy gap, $\hat \delta_{\mu||\theta}$ vs number of
    states in HMMs.  The upper trace plots estimates of cross entropy
    $\hat h(\mathcal{B},F,\mu||\theta)$ for a sequence of HMMs vs the
    number of discrete state in the models.  We built the models using
    actual Lorenz state space trajectories as describe in the text.
    The lower trace is an estimate of the entropy rate, $\hat
    h(F,\mu)) = \hat \lambda_1$, of the true process based on Lyapunov
    exponents estimated by the Benettin procedure.  The distance
    between the curves is the \emph{entropy gap} $\hat
    \delta_{\mu||\theta}$. The gap seems to be going to zero,
    suggesting that an HMM with a few \emph{billion} states might
    perform at least as well as any other model based any other
    technology.  Each model was built using the same 8,000,000 sample
    trajectory in the original state space, and the cross entropy
    estimates are based on a test sequence of 10,000 observations.}
  \label{fig:LikeLor}
\end{figure}
\marginpar{FixMe: Numbers in fig:LikeLor?}
\afterpage{\clearpage}%% Print this right here please.
\newpage
% \ToDo{remember to cite Yariv Ephraim
% ph\_barbe at hotmail.com recommends Biscarat '94 (a paper on
% stochastic EM)}

%%%
%%% Local Variables:
%%% TeX-master: "main"
%%% eval: (load-file "hmmkeys.el")
%%% End:
