\appendix

\chapter{Formulas for Matrices and Gaussians}
\label{cha:MatrixFormulas}

Here we review some material necessary for deriving
Eqns.~\eqref{eq:KUpdate}-\eqref{eq:smoothing} on page
\pageref{eq:KUpdate}.  Similar material appears in Appendix A of
Kailath et al.\cite{KSH00}.

\subsubsection{Block Matrix Inverse}
\index{block matrix inverse}%
\index{matrix inverse|see{block matrix inverse}}%

If $G$ is an $n\times n$ invertible matrix, $K$ is an $m\times m$
invertible matrix, and $H$ and $J$ are $n\times m$ and $m\times n$
respectively, then direct matrix multiplication verifies that
\begin{subequations}
  \label{eq:MatrixInverse}
  \begin{align}
    \begin{bmatrix}
      (G-HK^{-1}J)^{-1} & -(G-HK^{-1}J)^{-1}HK^{-1} \\
      -(K-JG^{-1}H)^{-1}JG^{-1} & (K-JG^{-1}H)^{-1}
    \end{bmatrix}
    \begin{bmatrix}
      G & H \\ J & K
    \end{bmatrix}
    &=
    \begin{bmatrix}
      \id & 0 \\ 0 & \id
    \end{bmatrix}\\ \nonumber \\
    \begin{bmatrix}
      G & H \\ J & K
    \end{bmatrix}
    \begin{bmatrix}
      (G-HK^{-1}J)^{-1} & -G^{-1}H(K-JG^{-1}H)^{-1} \\
      -K^{-1}J(G-HK^{-1}J)^{-1} & (K-JG^{-1}H)^{-1}
    \end{bmatrix}
    &=
    \begin{bmatrix}
      \id & 0 \\ 0 & \id
    \end{bmatrix},
  \end{align}
\end{subequations}
assuming that $(G-HK^{-1}J)^{-1}$ and $(K-JG^{-1}H)^{-1}$ exist.
One can derive other expressions for the inverse by using the Sherman
Morrison Woodbury formula (Eqn.~\eqref{eq:SMW}) to expand terms in
Eqn.~\eqref{eq:MatrixInverse}.

By noting
\begin{multline*}
  \begin{bmatrix}
    (G-HK^{-1}J)^{-1} & -(G-HK^{-1}J)^{-1}HK^{-1} \\
    0 & (K-JG^{-1}H)^{-1}
  \end{bmatrix}
  \begin{bmatrix}
    G & H \\ J & K
  \end{bmatrix}\\
  =
  \begin{bmatrix}
    \id & 0 \\
     (K-JG^{-1}H)^{-1}J & (K-JG^{-1}H)^{-1} K
  \end{bmatrix}
\end{multline*}
and taking the determinant of both sides
\renewcommand{\det}[1]{\left| #1 \right|}
\begin{equation*}
  \det{(G-HK^{-1}J)^{-1}} \cdot \det{(K-JG^{-1}H)^{-1}} \cdot 
  \det{\begin{bmatrix}
    G & H \\ J & K
  \end{bmatrix}} = \det{(K-JG^{-1}H)^{-1}} \cdot \det {K}
\end{equation*}
one finds the following formula for determinants
\index{block matrix determinant}
\index{determinant|see{block matrix determinant}}
\begin{equation}
  \label{eq:BlockDet}
  \det{ \begin{bmatrix} G & H \\ J & K \end{bmatrix}} = \det{K}
  \det{(G-HK^{-1}J)}
\end{equation}
\subsubsection{Sherman Morrison Woodbury Formula}
\index{Sherman Morrison Woodbury formula}

If $G$ and $ K $ are invertible matrices, $H$ and $ J $ have
dimensions so that $\left( G + H K J \right)^{-1}$ makes sense and
exists, and $\left( J G^{-1}H + K ^{-1} \right)^{-1}$ exists, then
\begin{equation}
  \label{eq:SMW}
  \left( G + H K  J  \right)^{-1} = G^{-1} - G^{-1}H\left( J G^{-1}H +
   K ^{-1}\right)^{-1} J G^{-1}.
\end{equation}
Multiplying both sides by $\left( G + H K  J  \right)$ verifies the
formula.  Equation~\eqref{eq:SMW} is called the Sherman Morrison
Woodbury formula.

To invert $(A^{-1} + C\transpose B^{-1} C)$, when $A$ is an $n\times
n$ matrix and $B$ is an $m \times m$ matrix, if $n > m$ one can use
\eqref{eq:SMW} to write
\begin{equation}
  \label{eq:MatrixInversion}
  \left(A^{-1} + C\transpose  B ^{-1}C \right)^{-1} = A - AC\transpose
  \left( CAC\transpose +  B  \right)^{-1}CA.
\end{equation}
The right hand side requires inverting an $m \times m$ matrix while
the left hand side requires inverting an $n \times n$ matrix.

\subsubsection{Marginal and Conditional Distributions of a Gaussian}
\index{marginal distribution, Gaussian}
\index{conditional distribution, Gaussian}

Suppose that $W = \begin{bmatrix} U \\ V \end{bmatrix}$ is a Gaussian
random variable with an $n$ dimensional component $U$ and an $m$
dimensional component $V$.  We write its distribution
%
\nomenclature[rNormal]{$\Normal \left( \mu,\Sigma \right)$}{A
  \emph{normal} or Gaussian distribution function; $\mu$ is an $n$
  dimensional vector and $\Sigma$ is an $n\times n$ matrix.  Writing
  $X\sim \Normal \left( \mu,\Sigma \right)$ means $X$ is distributed
  normally with mean $\mu$ and covariance $\Sigma$ and the
  probability density at any particular vector $x$ is
  $\NormalE{\mu}{\Sigma}{x}$.
} % end \nomenclature
%
\nomenclature[rNormalE]{$\NormalE{\mu}{\Sigma}{x}$}{The value of a
  Gaussian probability density function evaluated at the $n$
  dimensional vector $x$, \ie
  \begin{equation*}
    \NormalE{\mu}{\Sigma}{x} \equiv \frac{1}
    {\sqrt{(2\pi)^n\left"|\Sigma\right"|}} e^{ -\frac{1}{2}
      (x-\mu)\transpose \Sigma^{-1} (x-\mu)}
  \end{equation*}
  where $\left"|\Sigma\right"|$ is the determinant of the covariance
  matrix $\Sigma$.
} % end \nomenclature
\begin{equation*}
  W \sim \Normal \left( \mu_W,\Sigma_W \right) \text{ or equivalently
  } P(w) = \NormalE{\mu_W}{\Sigma_W}{w}
\end{equation*}
with
\begin{equation*}
  \mu_W = \begin{bmatrix} \mu_U\\\mu_V \end{bmatrix} \text{ and }
  \Sigma_W =   \begin{bmatrix} \Sigma_{UU} & \Sigma_{UV}\\ \Sigma_{VU}
  & \Sigma_{VV} \end{bmatrix} \equiv \begin{bmatrix} A & C \\
  C\transpose & B \end{bmatrix},
\end{equation*}
where we have introduced $A \equiv \Sigma_{UU}$, $B \equiv
\Sigma_{VV}$, and $C \equiv \Sigma_{UV}$ to shorten the notation.
If we denote
\begin{equation*}
    \Sigma_W^{-1} =   \begin{bmatrix} D & F \\ F\transpose & E
  \end{bmatrix},
\end{equation*}
then from Eqns.~\eqref{eq:MatrixInverse} and ~\eqref{eq:SMW}
\begin{subequations}
  \label{eq:MI:ABCDEF}
  \begin{align}
    \label{eq:MI:D}
    D &= (A - CB^{-1}C\transpose)^{-1} & &= A^{-1} + A^{-1}C E
    C\transpose A^{-1}\\
    \label{eq:MI:E}
    E &= (B - C\transpose A^{-1}C)^{-1} & &= B^{-1} + B^{-1}C\transpose
    D C B^{-1} \\
    \label{eq:MI:F}
    F &= -A^{-1}C E & &= -DCB^{-1}.
  \end{align}
\end{subequations}
In this notation, the marginal distributions are
\begin{subequations}
  \label{eq:Gauss-Marginal}
  \begin{align}
    P\left(u \right) &= \int P\left(u,v \right) dv \\
    &= \NormalE{\mu_U}{A}{u} \\
    P\left(v \right) &= \int P\left(u,v \right) du \\
    &= \NormalE{\mu_V}{B}{v},
  \end{align}
\end{subequations}
and the conditional distributions are
\begin{subequations}
  \label{eq:Gauss-Conditional}
  \begin{align}
    P\left(u |v \right) &= \frac{ P\left(u,v \right)}{P\left(v \right)} \\
    &= \NormalE{\mu_U + CB^{-1}(v-\mu_v)}{D^{-1}}{u} \\
    P\left(v |u \right) &= \frac{ P\left(v,u \right)}{P\left(u \right)} \\
    &= \NormalE{\mu_V + C\transpose A^{-1} (u - \mu_U)}{E^{-1}}{v}
  \end{align}
\end{subequations}
Notice that the covariance of the \emph{marginal} distribution of $U$
is given by the $UU$ block of $\Sigma_W$, but that the inverse
covariance of the \emph{conditional} distribution of $U$ is given by
the $UU$ block of $\Sigma_W^{-1}$.

As a check of these formulas, we examine $P\left(u |v \right) P\left(v
\right)$ and find
\begin{align*}
  P\left(u |v \right) P\left(v \right) &= \frac{\sqrt{\left| D
      \right|}} {\sqrt{(2\pi)^n}} e^{ -\frac{1}{2} \left(u - \mu_U -
      CB^{-1}(v - \mu_V) \right)\transpose D \left(u - \mu_U -
      CB^{-1}(v - \mu_V) \right)
  } \\
  &\quad \times \frac{1}{\sqrt{(2\pi)^m \left| B \right|}} e^{ -\frac{1}{2}
    (v-\mu_V)\transpose B^{-1} (v-\mu_V)}\\
                                %
    &= \frac{1} {\sqrt{(2\pi)^{n+m} \left| \Sigma_W
      \right| }} \exp \bigg( -\frac{1}{2} \Big[ \\
  &\quad \left( u - \mu_U - CB^{-1}(v - \mu_V) \right)\transpose D
       \left(u - \mu_U - CB^{-1}(v - \mu_V) \right)  \\
     &\quad + (v-\mu_V)\transpose B^{-1} (v-\mu_V) \Big]\bigg)\\
                                  %
     &= \frac{1} {\sqrt{(2\pi)^{n+m} \left| \Sigma_W \right| }}\\
     &\quad\times e^{ -\frac{1}{2} \left((u-\mu_U)\transpose D (u-\mu_U) +
         2 (v-\mu_V)\transpose F\transpose (u-\mu_U) +
         (v-\mu_V)\transpose E (v-\mu_V)\right)}\\
     &= P \left(u,v \right)
\end{align*}
all is right with the world.  In the above, Eqn.~\eqref{eq:BlockDet}
implies that $\frac{\sqrt{\det{D}}}{\sqrt{\det{B}}} =
\frac{1}{\sqrt{\det{\Sigma_W}}}$.


\subsubsection{Completing the Square}
\index{completing the square}

Some of the derivations in section~\ref{sec:KDerive} rely on a
procedure called \emph{completing the square}, which we illustrate
with the following example.  Suppose that the function $f(u)$ is the
product of two $n$ dimensional Gaussians,
$\Normal\left(\mu_1,\Sigma_1\right)$ and
$\Normal\left(\mu_2,\Sigma_2\right)$, \ie
\begin{align}
  \label{eq:Qu.a}
  f(u) &= \frac{1} {\sqrt{(2\pi)^n \left| \Sigma_1 \right| }}
  e^{-\frac{1}{2} ( u - \mu_1)\transpose \Sigma_1^{-1} ( u - \mu_1)}
  \frac{1} {\sqrt{(2\pi)^n \left| \Sigma_2 \right| }}
  e^{-\frac{1}{2} ( u - \mu_2)\transpose \Sigma_2^{-1} ( u - \mu_2)}\\
  \label{eq:Qu.b}
  &= \frac{1} {\sqrt{(2\pi)^{2n} \left| \Sigma_1 \right| \left|
        \Sigma_2 \right| }} e^{-\frac{1}{2}\big[ ( u -
    \mu_1)\transpose \Sigma_1^{-1} ( u - \mu_1) + ( u -
    \mu_2)\transpose \Sigma_2^{-1} ( u - \mu_2)\big]}\\
  \label{eq:Qu.c}
  &\equiv \frac{1} {\sqrt{(2\pi)^{2n} \left| \Sigma_1 \right| \left|
        \Sigma_2 \right| }} e^{-\frac{1}{2}\big[Q(u)\big]}.
\end{align}
By expanding the function $Q(u)$ in the exponent, we find:
\begin{align}
  \label{eq:Qu.d}
  Q(u) &= u\transpose \left( \Sigma_1^{-1} + \Sigma_2^{-1} \right) u -
  2 u\transpose  \left( \Sigma_1^{-1} \mu_1 + \Sigma_2^{-1} \mu_2
  \right) + \mu_1\transpose \Sigma_1^{-1} \mu_1 + \mu_2\transpose
  \Sigma_2^{-1} \mu_2 \\
  \label{eq:Qu.e}
  &= u\transpose q u - 2 u\transpose l + s
\end{align}
where the quadratic, linear, and scalar terms are
\begin{align*}
  q &= \left( \Sigma_1^{-1} + \Sigma_2^{-1} \right) \\
  l &= \left( \Sigma_1^{-1} \mu_1 + \Sigma_2^{-1} \mu_2 \right) \\
  s &= \mu_1\transpose \Sigma_1^{-1} \mu_1 + \mu_2\transpose
  \Sigma_2^{-1} \mu_2
\end{align*}
respectively.

\emph{Completing the square} means finding values $\mu$, $\Sigma$, and
$R$ for which Eqn.~\eqref{eq:Qu.e} takes the form
\begin{equation}
  \label{eq:Qu.f}
  Q(u) = (u - \mu)\transpose \Sigma^{-1} (u - \mu) + R,
\end{equation}
where $R$ is not a function of $u$.  One can verify by substitution
that the solution is
\begin{align*}
  \Sigma^{-1} &= q\\
  \mu &= \Sigma l\\
  R &= s -  \mu\transpose \Sigma^{-1} \mu.
\end{align*}
For the product of Gaussians example \eqref{eq:Qu.a},
\begin{subequations}
  \label{eq:GaussianProduct}
  \begin{align}
  \Sigma^{-1} &= \Sigma_1^{-1} + \Sigma_2^{-1} \\
  \mu &= \Sigma \left( \Sigma_1^{-1} \mu_1 + \Sigma_2^{-1} \mu_2
  \right) \\
  &= \left( \Sigma_1^{-1} + \Sigma_2^{-1} \right)^{-1} \left(
  \Sigma_1^{-1} \mu_1 + \Sigma_2^{-1} \mu_2  \right) \\
  R &= \mu_1\transpose \Sigma_1^{-1} \mu_1 + \mu_2\transpose
  \Sigma_2^{-1} \mu_2 -  \mu\transpose \Sigma^{-1} \mu \\
  &= \mu_1\transpose \Sigma_1^{-1} \mu_1 + \mu_2\transpose
  \Sigma_2^{-1} \mu_2 - \left( \Sigma_1^{-1} \mu_1 + \Sigma_2^{-1}
  \mu_2  \right) \transpose \left( \Sigma_1^{-1} + \Sigma_2^{-1}
  \right)^{-1} \left( \Sigma_1^{-1} \mu_1 + \Sigma_2^{-1} \mu_2
  \right).
\end{align}
\end{subequations}
In words the product of two Gaussian density functions is an
unnormalized Gaussian density function in which the inverse covariance
is the sum of the inverse covariances of the factors and the mean is
the average of the factor means weighted by the inverse covariances.

\chapter{EM Convergence Rate}
\label{chap:em_appendix}

Now we calculate a linear approximation of the behavior of the EM
algorithm in the neighborhood of a fixed point.  The manipulations
here require that the probability and likelihood functions have
continuous first and second derivatives which we implicitly assume.
We let $\Phi$ denote the action of one iteration of the algorithm and
let $\hat \theta$ denote a fixed point with
\begin{align}
  \ti{\theta}{n+1} &= \Phi\left( \ti{\theta}{n} \right) \\
  \Phi(\hat \theta) &= \hat \theta && \text{with Taylor series} \\
  \Phi(\theta) &= \hat \theta + \left[ \frac{\partial
                 \Phi(\theta')}{\partial \theta'} \right]_{\hat
                 \theta} (\theta - \hat \theta) + \text{Remainder} \\
  Q(\theta, \theta') &\equiv \EV_{S|y,\theta'} \log\left(P(y,S \given
  \theta \right) \\
  \Phi(\theta') &= \argmax_\theta Q(\theta, \theta'). 
\end{align}
For a given value of $\theta'$, the derivative of $\Phi(\theta,
\theta')$ at a maximum is zero, and we write
\begin{align}
  \Psi(\theta,\theta')
  &\equiv \frac{\partial Q(\theta,
    \theta')}{\partial \theta} \\
  \Psi(\Phi(\theta'), \theta')
  &= 0 \\
  \frac{d\Psi(\Phi(\theta'), \theta')}{d \theta'}
  &= 0 \\
  \frac{d\Psi(\Phi(\theta'), \theta')}{d \theta'}
  &= \left. \frac{\partial \Psi(\theta, \theta')}{\partial \theta}
    \right|_{\Phi(\theta'), \theta'} \left.\frac{\partial
    \Phi(\theta)}{\partial \theta} \right|_{\theta'} +
    \left. \frac{\partial \Psi(\theta, \theta')}{\partial \theta'}
    \right|_{\Phi(\theta'), \theta'} \\
  &= \left. \frac{\partial^2 Q(\theta, \theta')}{\partial \theta^2}
    \right|_{\Phi(\theta'), \theta'} \left.\frac{\partial
    \Phi(\theta)}{\partial \theta} \right|_{\theta'} +
    \left. \frac{\partial^2 Q(\theta, \theta')}{\partial \theta'
    \partial \theta} \right|_{\Phi(\theta'), \theta'} =0 \\
  \label{eq:two_second_derivatives}
  \left.\frac{\partial \Phi(\theta)}{\partial \theta}
  \right|_{\theta'}
  &= - \left[ \left. \frac{\partial^2 Q(\theta,  \theta')}{\partial
    \theta^2} \right|_{\Phi(\theta'), \theta'} \right]^{-1} \left[
    \left. \frac{\partial^2 Q(\theta, \theta')}{\partial \theta'
    \partial \theta} \right|_{\Phi(\theta'), \theta'} \right].
\end{align}
Manipulating the first of the two second derivatives in
\eqref{eq:two_second_derivatives} we find
\begin{align}
  \frac{\partial^2 Q(\theta,  \theta')}{\partial \theta^2}
  &= \frac{\partial^2}{\partial \theta^2} \EV_{S \given y, \theta'}
    \log \left( P(y, S \given \theta \right) \\
  &= \frac{\partial^2}{\partial \theta^2} \EV_{S \given y, \theta'}
    \left( \log \left( P(y \given \theta \right) + \log \left( P(S
    \given y, \theta \right) \right)\\
  &=  \frac{\partial^2}{\partial \theta^2} \log \left( P(y \given
    \theta \right) + \EV_{S \given y, \theta'} \left(
    \frac{\partial^2}{\partial \theta^2} \log \left( P(S \given y,
    \theta \right) \right) \\
  \label{eq:second_partial_information}
  &= -J_y - I_{S|y},
\end{align}
where $J_y$ is called the \emph{observed information} that $y$
provides about $\theta$, and $I_{S|y}$ is the \emph{Fisher information}
of the unobserved data.  Now manipulating the second of the two second
derivatives in \eqref{eq:two_second_derivatives} we find
\begin{align}
  \frac{\partial^2 Q(\theta,  \theta')}{\partial \theta \partial
  \theta'}
  &= \frac{\partial^2}{\partial \theta \partial \theta'} \EV_{S \given y, \theta'}
    \log \left( P(y, S \given \theta \right) \\
  &= \frac{\partial^2}{\partial \theta \partial \theta'} \EV_{S \given y, \theta'}
    \left( \log \left( P(y \given \theta \right) + \log \left( P(S
    \given y, \theta \right) \right)\\
  &=  \frac{\partial^2}{\partial \theta \partial \theta'} \log \left( P(y \given
    \theta \right) + \frac{\partial}{\partial \theta'} \EV_{S \given y, \theta'} \left(
    \frac{\partial}{\partial \theta} \log \left( P(S \given y,
    \theta \right) \right) \\
  &=  \frac{\partial}{\partial \theta'} \EV_{S \given y, \theta'} \left(
    \frac{\partial}{\partial \theta} \log \left( P(S \given y,
    \theta \right) \right) \\
  &= \frac{\partial}{\partial \theta'} \sum_s P(s \given y, \theta') %
    \frac{ %
    \frac{\partial P(s \given y,\theta )}{\partial \theta}
    }{P(s \given y,\theta )}  \\
  &= \sum_s P(s \given y, \theta') %
    \frac{ %
    \frac{\partial P(s \given y,\theta' )}{\partial \theta'}
    }{P(s \given y,\theta' )}
    \frac{ %
    \frac{\partial P(s \given y,\theta )}{\partial \theta}
    }{P(s \given y,\theta )}
\end{align}
At the fixed point $\hat \theta = \theta = \theta'$ and
\begin{align}
  \left. \frac{\partial^2 Q(\theta,  \theta')}{\partial \theta
  \partial \theta'} \right|_{\hat \theta, \hat \theta}
  &= \EV_{S \given y, \hat \theta} \left( \frac{\partial}{\partial
    \theta} \log \left( P(S \given y, \theta )\right) \right)^2\\
  \label{eq:mixed_partial_information}
  &\equiv I_{S|y}.
\end{align}
Combining \eqref{eq:mixed_partial_information} and
\eqref{eq:second_partial_information} with
\eqref{eq:two_second_derivatives} we write
\begin{equation}
  \label{eq:information_em_derivative}
  \left.\frac{\partial \Phi(\theta)}{\partial \theta}
  \right|_{\theta'} = \left[J_y + I_{S|y}  \right]^{-1} I_{S|y}.
\end{equation}
Equation~\eqref{eq:information_em_derivative} appeals to our
intuition.  If the observed information is much larger than the
unobserved information, the derivative is small and the convergence is
fast and if the unobserved information dominates the derivative is
close to one and the convergence is slow.

\chapter{Notes on Software}
\label{cha:Software}

All of the software we've used (both the code we've written and the
software that code depends on) for this book is free software.  After
fetching our code, typing ``make'' in the top level directory of the
hmmds project will create a copy of the book in a file called
\emph{main.pdf} after some hours of computation.

\section{Dependencies}
\label{sec:SWdep}

Our code relies on many free software packages.  Although we developed
the code on NixOS systems, the code will run on any of the many other
platforms for which the dependencies are available.  In addition to
the standard gnu development tools, we require at least the following
and their dependencies:
  \begin{description}
  \item[Python] The interpreted, interactive programming language, and
    the following Python packages
    \begin{description}
    \item[numpy] 
    \item[scipy] 
    \item[matplotlib]
    \item[cython] 
    \item[pint] 
    \item[pathlib2] For py-ecg-detectors
    \item[pywavelets] For py-ecg-detectors
    \item[pygraphviz] For making pdf graph of HMM
    \item[pandas] For wfdb package from PhysioNet
    \item[pyqtgraph] For GUIs
    \item[pyqt5] GUIs import PyQt5
    \end{description}
  \item[bash] 
  \item[gnumake] 
  \item[xfig]
  \item[fig2dev] 
  \item[wget]
  \item[graphviz] 
  \item[texlive] TeX, LaTeX and much supporting software.  We use the
    following packages:
    \begin{description}
    \item[tcolorbox] 
    \item[environ] 
    \item[trimspaces] 
    \item[listingssutf8] 
    \item[latexmk]
    \item[type1cm] 
    \item[showlabels] 
    \item[xstring] 
    \item[framed] 
    \end{description}
  \end{description}

\section{Data}
\label{sec:SWdata}

We used the following sets of data for examples:
\begin{description}
\item[Tang's laser data] Carl Otto Weiss \index{Weiss, C.\ O.} mailed
  us a CD full of data from various experiments that he and Tang did
  in the 1980s and 1990s.  Although we analyzed many of the files, we
  finally used only a file called \emph{LP5.DAT} in the book (see
  Section~\ref{sec:laser}).  The file \emph{LP5.DAT} is included in
  \emph{hmmdsbook}.
\item[H.~L.~Mencken's \emph{A Book of Prefaces}] We used Mencken's
  book for the parts of speech example in Section~\ref{sec:POSpeech}.
  Although the code fetches the book from www.gutenberg.org as of
  April, 2007, we planned to include the parsed text in
  \emph{hmmdsbook}.
\item[CINC2000 ECG data] We used Dr. Thomas Penzel's ECG measurements
  throughout Chapter~\ref{chap:apnea}.  Although the code fetches the
  the data from\\
  www.physionet.org/physiobank/database/apnea-ecg as of April, 2007,
  we planned to include much smaller files that contain estimates of
  the timing of heart beats in \emph{hmmdsbook}.
\end{description}

\section{Clarity and Efficiency}
\label{sec:Clarity}

We wrote earlier versions of the code for this book in C to make it
run faster than Python.  In particular the C code exploited sparsity
in the probability arrays in models.  Since we wrote those early
versions, SciPy has provided support for sparse matrices making our C
code obsolete.  Now we have Python code for all of the algorithms
described in the book.

We also provide Cython code for a few of the algorithms.  While the
Cython code is faster, it is harder to read and debug.  The interfaces
to call Cython code match the interfaces to the Python code.  We
recommend developing with the Python code and after that if you need
to run faster, try the Cython versions.

Here is the heart of our simple Python implementation of the forward
algorithm described in about 4 pages in Section~\ref{sec:forward}.  It
looks pretty simple here.
\label{code:forward}
\begin{verbatim}
        # last is a conditional distribution of state probabilities.
        # What it is conditioned on changes as the calculations
        # progress.
        last = numpy.copy(self.p_state_initial.reshape(-1))
        for t, likelihood_t in enumerate(self.state_likelihood):
            last *= likelihood_t  # Element-wise multiply
            self.gamma_inv[t] = 1 / last.sum()
            last *= self.gamma_inv[t]
            self.alpha[t, :] = last
            last[:] = numpy.dot(last, self.p_state2state)
\end{verbatim}


Here is the heart of our simple implementation of the backward
algorithm in Python.  It is even simpler.
\label{code:backward}
\begin{verbatim}
        # last and beta are analogous to last and alpha in forward(),
        # but the precise interpretations are more complicated.
        last = numpy.ones(self.n_states)
        for t in range(len(self.state_likelihood) - 1, -1, -1):
            self.beta[t, :] = last
            last *= self.state_likelihood[t] * self.gamma_inv[t]
            last[:] = numpy.dot(self.p_state2state, last)
\end{verbatim}

%%% Local Variables:
%%% TeX-master: "main"
%%% eval: (load-file "hmmkeys")
%%% mode: LaTeX
%%% End:

% LocalWords:  Welch Viterbi HMM
