\appendix

\chapter{Formulas for Matrices and Gaussians}
\label{cha:MatrixFormulas}

Here we review some material necessary for deriving
Eqns.~\eqref{eq:KUpdate}-\eqref{eq:smoothing}.  Similar material
appears in Appendix A of Kailath et al.\cite{KSH00}.

\subsubsection{Block Matrix Inverse}
\index{block matrix inverse}%
\index{matrix inverse|see{block matrix inverse}}%

If $G$ is an $n\times n$ invertible matrix, $K$ is an $m\times m$
invertible matrix, and $H$ and $J$ are $n\times m$ and $m\times n$
respectively, then direct matrix multiplication verifies that
\begin{subequations}
  \label{eq:MatrixInverse}
  \begin{align}
    \begin{bmatrix}
      (G-HK^{-1}J)^{-1} & -(G-HK^{-1}J)^{-1}HK^{-1} \\
      -(K-JG^{-1}H)^{-1}JG^{-1} & (K-JG^{-1}H)^{-1}
    \end{bmatrix}
    \begin{bmatrix}
      G & H \\ J & K
    \end{bmatrix}
    &=
    \begin{bmatrix}
      \id & 0 \\ 0 & \id
    \end{bmatrix}\\ \nonumber \\
    \begin{bmatrix}
      G & H \\ J & K
    \end{bmatrix}
    \begin{bmatrix}
      (G-HK^{-1}J)^{-1} & -G^{-1}H(K-JG^{-1}H)^{-1} \\
      -K^{-1}J(G-HK^{-1}J)^{-1} & (K-JG^{-1}H)^{-1}
    \end{bmatrix}
    &=
    \begin{bmatrix}
      \id & 0 \\ 0 & \id
    \end{bmatrix},
  \end{align}
\end{subequations}
assuming that $(G-HK^{-1}J)^{-1}$ and $(K-JG^{-1}H)^{-1}$ exist.
One can derive other expressions for the inverse by using the Sherman
Morrison Woodbury formula (Eqn.~\eqref{eq:SMW}) to expand terms in
Eqn.~\eqref{eq:MatrixInverse}.

By noting
\begin{multline*}
  \begin{bmatrix}
    (G-HK^{-1}J)^{-1} & -G^{-1}H(K-JG^{-1}H)^{-1} \\
    0 & (K-JG^{-1}H)^{-1}
  \end{bmatrix}
  \begin{bmatrix}
    G & H \\ J & K
  \end{bmatrix}\\
  =
  \begin{bmatrix}
    \id & 0 \\
     (K-JG^{-1}H)^{-1}J & (K-JG^{-1}H)^{-1} K
  \end{bmatrix}
\end{multline*}
and taking the determinant of both sides
\renewcommand{\det}[1]{\left| #1 \right|}
\begin{equation*}
  \det{(K-JG^{-1}H)^{-1}} \det{K} = \det{(G-HK^{-1}J)^{-1}}
  \det{(K-JG^{-1}H)^{-1}} \det{
  \begin{bmatrix}
    G & H \\ J & K
  \end{bmatrix}}
\end{equation*}
one finds the following formula for determinants
\index{block matrix determinant}
\index{determinant|see{block matrix determinant}}
\begin{equation}
  \label{eq:BlockDet}
  \det{ \begin{bmatrix} G & H \\ J & K \end{bmatrix}} = \det{K}
  \det{(G-HK^{-1}J)}
\end{equation}
\subsubsection{Sherman Morrison Woodbury Formula}
\index{Sherman Morrison Woodbury formula}

If $G$ and $J$ are invertible matrices, $H$ and $K$ have dimensions so
that $\left( G + HJK \right)$ makes sense, and $\left(KG^{-1}H +
  J^{-1} \right)^{-1}$ exists, then
\begin{equation}
  \label{eq:SMW}
  \left( G + HJK \right)^{-1} = G^{-1} - G^{-1}H\left(KG^{-1}H +
  J^{-1}\right)^{-1}KG^{-1}.
\end{equation}
Multiplying both sides by $\left( G + HJK \right)$ verifies the
formula.  Equation~\eqref{eq:SMW} is called the Sherman Morrison
Woodbury formula.  The special case
\begin{equation}
  \label{eq:MatrixInversion}
  \left(L^{-1} + H\transpose J^{-1}H \right)^{-1} = L - LH\transpose
  \left( HLH\transpose + J \right)^{-1}HL
\end{equation}
is efficient when the dimension, $m$, of $J$ is smaller than the
dimension $n$ of $L$ because an $n\times n$ matrix inversion on the
left is replaced by an $m \times m$ matrix inversion on the right.

\subsubsection{Marginal and Conditional Distributions of a Gaussian}
\index{marginal distribution, Gaussian}
\index{conditional distribution, Gaussian}

Suppose that $W = \begin{bmatrix} U \\ V \end{bmatrix}$ is a Gaussian
random variable with an $n$ dimensional component $U$ and an $m$
dimensional component $V$.  We write its distribution
%
\nomenclature[rNormal]{$\Normal \left( \mu,\Sigma \right)$}{A
  \emph{normal} or Gaussian distribution function; $\mu$ is an $n$
  dimensional vector and $\Sigma$ is an $n\times n$ matrix.  Writing
  $X\sim \Normal \left( \mu,\Sigma \right)$ means $X$ is distributed
  normally with mean $\mu$ and covariance $\Sigma$ and the
  probability density at any particular vector $x$ is
  $\NormalE{\mu}{\Sigma}{x}$.
} % end \nomenclature
%
\nomenclature[rNormalE]{$\NormalE{\mu}{\Sigma}{x}$}{The value of a
  Gaussian probability density function evaluated at the $n$
  dimensional vector $x$, \ie
  \begin{equation*}
    \NormalE{\mu}{\Sigma}{x} \equiv \frac{1}
    {\sqrt{(2\pi)^n\left"|\Sigma\right"|}} e^{ -\frac{1}{2}
      (x-\mu)\transpose \Sigma^{-1} (x-\mu)}
  \end{equation*}
  where $\left"|\Sigma\right"|$ is the determinant of the covariance
  matrix $\Sigma$.
} % end \nomenclature
\begin{equation*}
  W \sim \Normal \left( \mu_W,\Sigma_W \right) \text{ or equivalently
  } P(w) = \NormalE{\mu_W}{\Sigma_W}{w}
\end{equation*}
with
\begin{equation*}
  \mu_W = \begin{bmatrix} \mu_U\\\mu_V \end{bmatrix} \text{ and }
  \Sigma_W =   \begin{bmatrix} \Sigma_{UU} & \Sigma_{UV}\\ \Sigma_{VU}
  & \Sigma_{VV} \end{bmatrix} \equiv \begin{bmatrix} A & C \\
  C\transpose & B \end{bmatrix},
\end{equation*}
where we have introduced $A \equiv \Sigma_{UU}$, $B \equiv
\Sigma_{VV}$, and $C \equiv \Sigma_{UV}$ to shorten the notation.
If we denote
\begin{equation*}
    \Sigma_W^{-1} =   \begin{bmatrix} D & F \\ F\transpose & E
  \end{bmatrix},
\end{equation*}
then from Eqns.~\eqref{eq:MatrixInverse} and ~\eqref{eq:SMW}
\begin{subequations}
  \label{eq:MI:ABCDEF}
  \begin{align}
    \label{eq:MI:D}
    D &= (A - CB^{-1}C\transpose)^{-1} & &= A^{-1} + A^{-1}C E
    C\transpose A^{-1}\\
    \label{eq:MI:E}
    E &= (B - C\transpose A^{-1}C)^{-1} & &= B^{-1} + B^{-1}C\transpose
    D C B^{-1} \\
    \label{eq:MI:F}
    F &= -A^{-1}C E & &= -DCB^{-1}.
  \end{align}
\end{subequations}
In this notation, the marginal distributions are
\begin{subequations}
  \label{eq:Gauss-Marginal}
  \begin{align}
    P\left(u \right) &= \int P\left(u,v \right) dv \\
    &= \NormalE{\mu_U}{A}{u} \\
    P\left(v \right) &= \int P\left(u,v \right) du \\
    &= \NormalE{\mu_V}{B}{v},
  \end{align}
\end{subequations}
and the conditional distributions are
\begin{subequations}
  \label{eq:Gauss-Conditional}
  \begin{align}
    P\left(u |v \right) &= \frac{ P\left(u,v \right)}{P\left(v \right)} \\
    &= \NormalE{\mu_U + CB^{-1}(v-\mu_v)}{D^{-1}}{u} \\
    P\left(v |u \right) &= \frac{ P\left(v,u \right)}{P\left(u \right)} \\
    &= \NormalE{\mu_V + C\transpose A^{-1} (u - \mu_U)}{E^{-1}}{v}
  \end{align}
\end{subequations}
Notice that the covariance of the \emph{marginal} distribution of $U$
is given by the $UU$ block of $\Sigma_W$, but that the inverse
covariance of the \emph{conditional} distribution of $U$ is given by
the $UU$ block of $\Sigma_W^{-1}$.

As a check of these formulas, we examine $P\left(u |v \right) P\left(v
\right)$ and find
\begin{align*}
  P\left(u |v \right) P\left(v \right) &= \frac{\sqrt{\left| D
      \right|}} {\sqrt{(2\pi)^n}} e^{ -\frac{1}{2} \left(u - \mu_U -
      CB^{-1}(v - \mu_V) \right)\transpose D \left(u - \mu_U -
      CB^{-1}(v - \mu_V) \right)
  } \\
  &\quad \times \frac{1}{\sqrt{(2\pi)^m \left| B \right|}} e^{ -\frac{1}{2}
    (v-\mu_V)\transpose B^{-1} (v-\mu_V)}\\
                                %
    &= \frac{1} {\sqrt{(2\pi)^{n+m} \left| \Sigma_W
      \right| }} \exp \bigg( -\frac{1}{2} \Big[ \\
  &\quad \left( u - \mu_U - CB^{-1}(v - \mu_V) \right)\transpose D
       \left(u - \mu_U - CB^{-1}(v - \mu_V) \right)  \\
     &\quad + (v-\mu_V)\transpose B^{-1} (v-\mu_V) \Big]\bigg)\\
                                  %
     &= \frac{1} {\sqrt{(2\pi)^{n+m} \left| \Sigma_W \right| }}\\
     &\quad\times e^{ -\frac{1}{2} \left((u-\mu_U)\transpose D (u-\mu_U) +
         2 (v-\mu_V)\transpose F\transpose (u-\mu_U) +
         (v-\mu_V)\transpose E (v-\mu_V)\right)}\\
     &= P \left(u,v \right)
\end{align*}
all is right with the world.  In the above, Eqn.~\eqref{eq:BlockDet}
implies that $\frac{\sqrt{\det{D}}}{\sqrt{\det{B}}} =
\frac{1}{\sqrt{\det{\Sigma_W}}}$.


\subsubsection{Completing the Square}
\index{completing the square}

Some of the derivations in section~\ref{sec:KDerive} rely on a
procedure called \emph{completing the square}, which we illustrate
with the following example.  Suppose that the function $f(u)$ is the
product of two $n$ dimensional Gaussians,
$\Normal\left(\mu_1,\Sigma_1\right)$ and
$\Normal\left(\mu_2,\Sigma_2\right)$, \ie
\begin{align}
  \label{eq:Qu.a}
  f(u) &= \frac{1} {\sqrt{(2\pi)^n \left| \Sigma_1 \right| }}
  e^{-\frac{1}{2} ( u - \mu_1)\transpose \Sigma_1^{-1} ( u - \mu_1)}
  \frac{1} {\sqrt{(2\pi)^n \left| \Sigma_2 \right| }}
  e^{-\frac{1}{2} ( u - \mu_2)\transpose \Sigma_2^{-1} ( u - \mu_2)}\\
  \label{eq:Qu.b}
  &= \frac{1} {\sqrt{(2\pi)^{2n} \left| \Sigma_1 \right| \left|
        \Sigma_2 \right| }} e^{-\frac{1}{2}\big[ ( u -
    \mu_1)\transpose \Sigma_1^{-1} ( u - \mu_1) + ( u -
    \mu_2)\transpose \Sigma_2^{-1} ( u - \mu_2)\big]}\\
  \label{eq:Qu.c}
  &\equiv \frac{1} {\sqrt{(2\pi)^{2n} \left| \Sigma_1 \right| \left|
        \Sigma_2 \right| }} e^{-\frac{1}{2}\big[Q(u)\big]}.
\end{align}
By expanding the function $Q(u)$ in the exponent, we find:
\begin{align}
  \label{eq:Qu.d}
  Q(u) &= u\transpose \left( \Sigma_1^{-1} + \Sigma_2^{-1} \right) u -
  2 u\transpose  \left( \Sigma_1^{-1} \mu_1 + \Sigma_2^{-1} \mu_2
  \right) + \mu_1\transpose \Sigma_1^{-1} \mu_1 + \mu_2\transpose
  \Sigma_2^{-1} \mu_2 \\
  \label{eq:Qu.e}
  &= u\transpose q u - 2 u\transpose l + s
\end{align}
where the quadratic, linear, and scalar terms are
\begin{align*}
  q &= \left( \Sigma_1^{-1} + \Sigma_2^{-1} \right) \\
  l &= \left( \Sigma_1^{-1} \mu_1 + \Sigma_2^{-1} \mu_2 \right) \\
  s &= \mu_1\transpose \Sigma_1^{-1} \mu_1 + \mu_2\transpose
  \Sigma_2^{-1} \mu_2
\end{align*}
respectively.

\emph{Completing the square} means finding values $\mu$, $\Sigma$, and
$R$ for which Eqn.~\eqref{eq:Qu.e} takes the form
\begin{equation}
  \label{eq:Qu.f}
  Q(u) = (u - \mu)\transpose \Sigma^{-1} (u - \mu) + R,
\end{equation}
where $R$ is not a function of $u$.  One can verify by substitution
that the solution is
\begin{align*}
  \Sigma^{-1} &= q\\
  \mu &= \Sigma l\\
  R &= s -  \mu\transpose \Sigma^{-1} \mu.
\end{align*}
For the product of Gaussians example \eqref{eq:Qu.a},
\begin{align*}
  \Sigma^{-1} &= \Sigma_1^{-1} + \Sigma_2^{-1} \\
  \mu &= \Sigma \left( \Sigma_1^{-1} \mu_1 + \Sigma_2^{-1} \mu_2
  \right) \\
  &= \left( \Sigma_1^{-1} + \Sigma_2^{-1} \right)^{-1} \left(
  \Sigma_1^{-1} \mu_1 + \Sigma_2^{-1} \mu_2  \right) \\
  R &= \mu_1\transpose \Sigma_1^{-1} \mu_1 + \mu_2\transpose
  \Sigma_2^{-1} \mu_2 -  \mu\transpose \Sigma^{-1} \mu \\
  &= \mu_1\transpose \Sigma_1^{-1} \mu_1 + \mu_2\transpose
  \Sigma_2^{-1} \mu_2 - \left( \Sigma_1^{-1} \mu_1 + \Sigma_2^{-1}
  \mu_2  \right) \transpose \left( \Sigma_1^{-1} + \Sigma_2^{-1}
  \right)^{-1} \left( \Sigma_1^{-1} \mu_1 + \Sigma_2^{-1} \mu_2
  \right).
\end{align*}
In words the product of two Gaussian density functions is an
unnormalized Gaussian density function in which the inverse covariance
is the sum of the inverse covariances of the factors and the mean is
the average of the factor means weighted by the inverse covariances.

\chapter{Notes on Software}
\label{cha:Software}

All of the software we've used (both the code we've written and the
software that code depends on) for this book is free software.  After
fetching our code, typing ``make'' in the top level directory of the
hmmds project will create a copy of the book in a file called
\emph{main.pdf} after some hours of computation.

\section{Dependencies}
\label{sec:SWdep}

Our code relies on many free software packages.  Although we developed
the code on NixOS systems, the code will run on any of the many other
platforms for which the dependencies are available.  In addition to
the standard gnu development tools, we require at least the following
and their dependencies:
  \begin{description}
  \item[python] An interpreted, interactive, object-oriented,
    extensible programming language.  The following python packages
    \begin{description}
    \item[numpy] 
    \item[scipy] 
    \item[matplotlib]
    \item[cython] 
    \item[pint] 
    \item[pathlib2] For py-ecg-detectors
    \item[pywavelets] For py-ecg-detectors
    \item[pygraphviz] For making pdf graph of HMM
    \item[pandas] For wfdb
    \item[pyqtgraph] For GUIs
    \item[pyqt5] GUIs import PyQt5 
    \item[sortedcontainers] FixMe: Only used by ECG.develop.ItemScoreT
      that may be dead code.
    \end{description}
  \item[bash] 
  \item[gnumake] 
  \item[xfig]
  \item[fig2dev] 
  \item[wget]
  \item[graphviz] 
  \item[texlive] TeX, LaTeX and much supporting software.  We use the
    following packages:
    \begin{description}
    \item[tcolorbox] 
    \item[environ] 
    \item[trimspaces] 
    \item[listingssutf8] 
    \item[latexmk]
    \item[type1cm] 
    \item[showlabels] 
    \item[xstring] 
    \item[framed] 
    \end{description}
  \end{description}

\section{Data}
\label{sec:SWdata}

We used the following sets of data for examples:
\begin{description}
\item[Tang's laser data] Carl Otto Weiss \index{Weiss, C.\ O.} mailed
  us a CD full of data from various experiments that he and Tang did
  in the 1980s and 1990s.  Although we analyzed many of the files, we
  finally used only a file called \emph{LP5.DAT} in the book (see
  Section~\ref{sec:laser}).  The file \emph{LP5.DAT} is included in
  \emph{hmmdsbook}.
\item[H.~L.~Mencken's \emph{A Book of Prefaces}] We used Mencken's
  book for the parts of speech example in Section~\ref{sec:POSpeech}.
  Although the code fetches the book from www.gutenberg.org as of
  April, 2007, we planned to include the parsed text in
  \emph{hmmdsbook}.
\item[CINC2000 ECG data] We used Dr. Thomas Penzel's ECG measurements
  throughout Chapter~\ref{chap:apnea}.  Although the code fetches the
  the data from\\
  www.physionet.org/physiobank/database/apnea-ecg as of April, 2007,
  we planned to include much smaller files that contain estimates of
  the timing of heart beats in \emph{hmmdsbook}.
\end{description}

\section{Clarity and Efficiency}
\label{sec:Clarity}

%%% Local Variables:
%%% TeX-master: "main"
%%% eval: (load-file "hmmkeys")
%%% End:

% LocalWords:  Welch Viterbi HMM
