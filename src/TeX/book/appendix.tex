\appendix

\chapter{Formulas for Matrices and Gaussians}
\label{cha:MatrixFormulas}

Here we review some material necessary for deriving
Eqns.~\eqref{eq:KUpdate}-\eqref{eq:smoothing} on page
\pageref{eq:KUpdate}.  Similar material appears in Appendix A of
Kailath et al.\cite{KSH00}.

\subsubsection{Block Matrix Inverse}
\index{block matrix inverse}%
\index{matrix inverse|see{block matrix inverse}}%

If $G$ is an $n\times n$ invertible matrix, $K$ is an $m\times m$
invertible matrix, and $H$ and $J$ are $n\times m$ and $m\times n$
respectively, then direct matrix multiplication verifies that
\begin{subequations}
  \label{eq:MatrixInverse}
  \begin{align}
    \begin{bmatrix}
      (G-HK^{-1}J)^{-1} & -(G-HK^{-1}J)^{-1}HK^{-1} \\
      -(K-JG^{-1}H)^{-1}JG^{-1} & (K-JG^{-1}H)^{-1}
    \end{bmatrix}
    \begin{bmatrix}
      G & H \\ J & K
    \end{bmatrix}
    &=
    \begin{bmatrix}
      \id & 0 \\ 0 & \id
    \end{bmatrix}\\ \nonumber \\
    \begin{bmatrix}
      G & H \\ J & K
    \end{bmatrix}
    \begin{bmatrix}
      (G-HK^{-1}J)^{-1} & -G^{-1}H(K-JG^{-1}H)^{-1} \\
      -K^{-1}J(G-HK^{-1}J)^{-1} & (K-JG^{-1}H)^{-1}
    \end{bmatrix}
    &=
    \begin{bmatrix}
      \id & 0 \\ 0 & \id
    \end{bmatrix},
  \end{align}
\end{subequations}
assuming that $(G-HK^{-1}J)^{-1}$ and $(K-JG^{-1}H)^{-1}$ exist.
One can derive other expressions for the inverse by using the Sherman
Morrison Woodbury formula (Eqn.~\eqref{eq:SMW}) to expand terms in
Eqn.~\eqref{eq:MatrixInverse}.

By noting
\begin{multline*}
  \begin{bmatrix}
    (G-HK^{-1}J)^{-1} & -(G-HK^{-1}J)^{-1}HK^{-1} \\
    0 & (K-JG^{-1}H)^{-1}
  \end{bmatrix}
  \begin{bmatrix}
    G & H \\ J & K
  \end{bmatrix}\\
  =
  \begin{bmatrix}
    \id & 0 \\
     (K-JG^{-1}H)^{-1}J & (K-JG^{-1}H)^{-1} K
  \end{bmatrix}
\end{multline*}
and taking the determinant of both sides
\renewcommand{\det}[1]{\left| #1 \right|}
\begin{equation*}
  \det{(G-HK^{-1}J)^{-1}} \cdot \det{(K-JG^{-1}H)^{-1}} \cdot 
  \det{\begin{bmatrix}
    G & H \\ J & K
  \end{bmatrix}} = \det{(K-JG^{-1}H)^{-1}} \cdot \det {K}
\end{equation*}
one finds the following formula for determinants
\index{block matrix determinant}
\index{determinant|see{block matrix determinant}}
\begin{equation}
  \label{eq:BlockDet}
  \det{ \begin{bmatrix} G & H \\ J & K \end{bmatrix}} = \det{K}
  \det{(G-HK^{-1}J)}
\end{equation}
\subsubsection{Sherman Morrison Woodbury Formula}
\index{Sherman Morrison Woodbury formula}

If $G$ and $ K $ are invertible matrices, $H$ and $ J $ have
dimensions so that $\left( G + H K J \right)^{-1}$ makes sense and
exists, and $\left( J G^{-1}H + K ^{-1} \right)^{-1}$ exists, then
\begin{equation}
  \label{eq:SMW}
  \left( G + H K  J  \right)^{-1} = G^{-1} - G^{-1}H\left( J G^{-1}H +
   K ^{-1}\right)^{-1} J G^{-1}.
\end{equation}
Multiplying both sides by $\left( G + H K  J  \right)$ verifies the
formula.  Equation~\eqref{eq:SMW} is called the Sherman Morrison
Woodbury formula.

To invert $(A^{-1} + C\transpose B^{-1} C)$, when $A$ is an $n\times
n$ matrix and $B$ is an $m \times m$ matrix, if $n > m$ one can use
\eqref{eq:SMW} to write
\begin{equation}
  \label{eq:MatrixInversion}
  \left(A^{-1} + C\transpose  B ^{-1}C \right)^{-1} = A - AC\transpose
  \left( CAC\transpose +  B  \right)^{-1}CA.
\end{equation}
The right hand side requires inverting an $m \times m$ matrix while
the left hand side requires inverting an $n \times n$ matrix.

\subsubsection{Marginal and Conditional Distributions of a Gaussian}
\index{marginal distribution, Gaussian}
\index{conditional distribution, Gaussian}

Suppose that $W = \begin{bmatrix} U \\ V \end{bmatrix}$ is a Gaussian
random variable with an $n$ dimensional component $U$ and an $m$
dimensional component $V$.  We write its distribution
%
\nomenclature[rNormal]{$\Normal \left( \mu,\Sigma \right)$}{A
  \emph{normal} or Gaussian distribution function; $\mu$ is an $n$
  dimensional vector and $\Sigma$ is an $n\times n$ matrix.  Writing
  $X\sim \Normal \left( \mu,\Sigma \right)$ means $X$ is distributed
  normally with mean $\mu$ and covariance $\Sigma$ and the
  probability density at any particular vector $x$ is
  $\NormalE{\mu}{\Sigma}{x}$.
} % end \nomenclature
%
\nomenclature[rNormalE]{$\NormalE{\mu}{\Sigma}{x}$}{The value of a
  Gaussian probability density function evaluated at the $n$
  dimensional vector $x$, \ie
  \begin{equation*}
    \NormalE{\mu}{\Sigma}{x} \equiv \frac{1}
    {\sqrt{(2\pi)^n\left"|\Sigma\right"|}} e^{ -\frac{1}{2}
      (x-\mu)\transpose \Sigma^{-1} (x-\mu)}
  \end{equation*}
  where $\left"|\Sigma\right"|$ is the determinant of the covariance
  matrix $\Sigma$.
} % end \nomenclature
\begin{equation*}
  W \sim \Normal \left( \mu_W,\Sigma_W \right) \text{ or equivalently
  } P(w) = \NormalE{\mu_W}{\Sigma_W}{w}
\end{equation*}
with
\begin{equation*}
  \mu_W = \begin{bmatrix} \mu_U\\\mu_V \end{bmatrix} \text{ and }
  \Sigma_W =   \begin{bmatrix} \Sigma_{UU} & \Sigma_{UV}\\ \Sigma_{VU}
  & \Sigma_{VV} \end{bmatrix} \equiv \begin{bmatrix} A & C \\
  C\transpose & B \end{bmatrix},
\end{equation*}
where we have introduced $A \equiv \Sigma_{UU}$, $B \equiv
\Sigma_{VV}$, and $C \equiv \Sigma_{UV}$ to shorten the notation.
If we denote
\begin{equation*}
    \Sigma_W^{-1} =   \begin{bmatrix} D & F \\ F\transpose & E
  \end{bmatrix},
\end{equation*}
then from Eqns.~\eqref{eq:MatrixInverse} and ~\eqref{eq:SMW}
\begin{subequations}
  \label{eq:MI:ABCDEF}
  \begin{align}
    \label{eq:MI:D}
    D &= (A - CB^{-1}C\transpose)^{-1} & &= A^{-1} + A^{-1}C E
    C\transpose A^{-1}\\
    \label{eq:MI:E}
    E &= (B - C\transpose A^{-1}C)^{-1} & &= B^{-1} + B^{-1}C\transpose
    D C B^{-1} \\
    \label{eq:MI:F}
    F &= -A^{-1}C E & &= -DCB^{-1}.
  \end{align}
\end{subequations}
In this notation, the marginal distributions are
\begin{subequations}
  \label{eq:Gauss-Marginal}
  \begin{align}
    P\left(u \right) &= \int P\left(u,v \right) dv \\
    &= \NormalE{\mu_U}{A}{u} \\
    P\left(v \right) &= \int P\left(u,v \right) du \\
    &= \NormalE{\mu_V}{B}{v},
  \end{align}
\end{subequations}
and the conditional distributions are
\begin{subequations}
  \label{eq:Gauss-Conditional}
  \begin{align}
    P\left(u \given v \right) &= \frac{ P\left(u,v \right)}{P\left(v \right)} \\
    &= \NormalE{\mu_U + CB^{-1}(v-\mu_v)}{D^{-1}}{u} \\
    P\left(v \given u \right) &= \frac{ P\left(v,u \right)}{P\left(u \right)} \\
    &= \NormalE{\mu_V + C\transpose A^{-1} (u - \mu_U)}{E^{-1}}{v}
  \end{align}
\end{subequations}
Notice that the covariance of the \emph{marginal} distribution of $U$
is given by the $UU$ block of $\Sigma_W$, but that the inverse
covariance of the \emph{conditional} distribution of $U$ is given by
the $UU$ block of $\Sigma_W^{-1}$.

As a check of these formulas, we examine $P\left(u \given v \right) P\left(v
\right)$ and find
\begin{align*}
  P\left(u \given v \right) P\left(v \right) &= \frac{\sqrt{\left| D
      \right|}} {\sqrt{(2\pi)^n}} e^{ -\frac{1}{2} \left(u - \mu_U -
      CB^{-1}(v - \mu_V) \right)\transpose D \left(u - \mu_U -
      CB^{-1}(v - \mu_V) \right)
  } \\
  &\quad \times \frac{1}{\sqrt{(2\pi)^m \left| B \right|}} e^{ -\frac{1}{2}
    (v-\mu_V)\transpose B^{-1} (v-\mu_V)}\\
                                %
    &= \frac{1} {\sqrt{(2\pi)^{n+m} \left| \Sigma_W
      \right| }} \exp \bigg( -\frac{1}{2} \Big[ \\
  &\quad \left( u - \mu_U - CB^{-1}(v - \mu_V) \right)\transpose D
       \left(u - \mu_U - CB^{-1}(v - \mu_V) \right)  \\
     &\quad + (v-\mu_V)\transpose B^{-1} (v-\mu_V) \Big]\bigg)\\
                                  %
     &= \frac{1} {\sqrt{(2\pi)^{n+m} \left| \Sigma_W \right| }}\\
     &\quad\times e^{ -\frac{1}{2} \left((u-\mu_U)\transpose D (u-\mu_U) +
         2 (v-\mu_V)\transpose F\transpose (u-\mu_U) +
         (v-\mu_V)\transpose E (v-\mu_V)\right)}\\
     &= P \left(u,v \right)
\end{align*}
all is right with the world.  In the above, Eqn.~\eqref{eq:BlockDet}
implies that $\frac{\sqrt{\det{D}}}{\sqrt{\det{B}}} =
\frac{1}{\sqrt{\det{\Sigma_W}}}$.


\subsubsection{Completing the Square}
\index{completing the square}

Some of the derivations in section~\ref{sec:KDerive} rely on a
procedure called \emph{completing the square}, which we illustrate
with the following example.  Suppose that the function $f(u)$ is the
product of two $n$ dimensional Gaussians,
$\Normal\left(\mu_1,\Sigma_1\right)$ and
$\Normal\left(\mu_2,\Sigma_2\right)$, \ie
\begin{align}
  \label{eq:Qu.a}
  f(u) &= \frac{1} {\sqrt{(2\pi)^n \left| \Sigma_1 \right| }}
  e^{-\frac{1}{2} ( u - \mu_1)\transpose \Sigma_1^{-1} ( u - \mu_1)}
  \frac{1} {\sqrt{(2\pi)^n \left| \Sigma_2 \right| }}
  e^{-\frac{1}{2} ( u - \mu_2)\transpose \Sigma_2^{-1} ( u - \mu_2)}\\
  \nonumber
       &= \frac{1} {\sqrt{(2\pi)^{2n} \left| \Sigma_1 \right| \left|
        \Sigma_2 \right| }} e^{-\frac{1}{2}\big[ ( u -
    \mu_1)\transpose \Sigma_1^{-1} ( u - \mu_1) + ( u -
    \mu_2)\transpose \Sigma_2^{-1} ( u - \mu_2)\big]}\\
  \nonumber
  &\equiv \frac{1} {\sqrt{(2\pi)^{2n} \left| \Sigma_1 \right| \left|
        \Sigma_2 \right| }} e^{-\frac{1}{2}\big[Q(u)\big]}.
\end{align}
By expanding the function $Q(u)$ in the exponent, we find:
\begin{align}
  \nonumber
  Q(u) &= u\transpose \left( \Sigma_1^{-1} + \Sigma_2^{-1} \right) u -
  2 u\transpose  \left( \Sigma_1^{-1} \mu_1 + \Sigma_2^{-1} \mu_2
  \right) + \mu_1\transpose \Sigma_1^{-1} \mu_1 + \mu_2\transpose
  \Sigma_2^{-1} \mu_2 \\
  \label{eq:Qu.e}
  &= u\transpose q u - 2 u\transpose l + s
\end{align}
where the quadratic, linear, and scalar terms are
\begin{align*}
  q &= \left( \Sigma_1^{-1} + \Sigma_2^{-1} \right) \\
  l &= \left( \Sigma_1^{-1} \mu_1 + \Sigma_2^{-1} \mu_2 \right) \\
  s &= \mu_1\transpose \Sigma_1^{-1} \mu_1 + \mu_2\transpose
  \Sigma_2^{-1} \mu_2
\end{align*}
respectively.

\emph{Completing the square} means finding values $\mu$, $\Sigma$, and
$R$ for which Eqn.~\eqref{eq:Qu.e} takes the form
\begin{equation*}
  Q(u) = (u - \mu)\transpose \Sigma^{-1} (u - \mu) + R,
\end{equation*}
where $R$ is not a function of $u$.  One can verify by substitution
that the solution is
\begin{align*}
  \Sigma^{-1} &= q\\
  \mu &= \Sigma l\\
  R &= s -  \mu\transpose \Sigma^{-1} \mu.
\end{align*}
For the product of Gaussians example \eqref{eq:Qu.a},
\begin{subequations}
  \label{eq:GaussianProduct}
  \begin{align}
  \Sigma^{-1} &= \Sigma_1^{-1} + \Sigma_2^{-1} \\
  \mu &= \Sigma \left( \Sigma_1^{-1} \mu_1 + \Sigma_2^{-1} \mu_2
  \right) \\
  &= \left( \Sigma_1^{-1} + \Sigma_2^{-1} \right)^{-1} \left(
  \Sigma_1^{-1} \mu_1 + \Sigma_2^{-1} \mu_2  \right) \\
  R &= \mu_1\transpose \Sigma_1^{-1} \mu_1 + \mu_2\transpose
  \Sigma_2^{-1} \mu_2 -  \mu\transpose \Sigma^{-1} \mu \\
  &= \mu_1\transpose \Sigma_1^{-1} \mu_1 + \mu_2\transpose
  \Sigma_2^{-1} \mu_2 - \left( \Sigma_1^{-1} \mu_1 + \Sigma_2^{-1}
  \mu_2  \right) \transpose \left( \Sigma_1^{-1} + \Sigma_2^{-1}
  \right)^{-1} \left( \Sigma_1^{-1} \mu_1 + \Sigma_2^{-1} \mu_2
  \right).
\end{align}
\end{subequations}
In words the product of two Gaussian density functions is an
unnormalized Gaussian density function in which the inverse covariance
is the sum of the inverse covariances of the factors and the mean is
the average of the factor means weighted by the inverse covariances.

\chapter{EM Convergence Rate}
\label{chap:em_appendix}

\newcommand{\OldParameters}{\parameters}
\newcommand{\NewParameters}{\parameters'} %
Here we calculate the linear approximation of the behavior of
$\EMmap$, the map implemented by the EM algorithm, in the neighborhood
of a fixed point, $\EMfixedPoint$.  We show that the linear stability
properties of $\EMmap$ are the same as the linear stability properties
of the flow of solutions of the ordinary differential
equation\footnote{In \eqref{eq:LogDynamics} we use the log of the
  likelihood as a potential function while in
  \eqref{eq:GradientDynamics}, we simply used the likelihood.  The two
  equations have the same fixed points, and the linear stability of
  those fixed points are also the same.}
\begin{equation}
  \label{eq:LogDynamics}
  \dot \parameters = \frac{d}{d \parameters} 
  \LogLikelihood(\parameters) ,
\end{equation}
where $\LogLikelihood(\parameters) \equiv \log \left( P \left( y \given
    \parameters \right) \right)$ is the log likelihood. Like trajectories
defined by the differential equation, trajectories of $\EMmap$
generically converge to local maxima of $\LogLikelihood$, and they
avoid local minima and saddles.

The manipulations here require that the probability and likelihood
functions have continuous first and second derivatives and that some
second derivatives have full rank which we implicitly assume.  We let
$\EMmap$ denote the action of one iteration of the algorithm and let
$\EMfixedPoint$ denote a fixed point with
\begin{align*}
  \ti{\parameters}{n+1} &= \EMmap\left( \ti{\parameters}{n} \right) \\
  \EMmap(\EMfixedPoint) &= \EMfixedPoint && \text{with Taylor series} \\
  \EMmap(\parameters) &= \EMfixedPoint + \left[ \frac{\partial
                 \EMmap(\parameters)}{\partial \parameters} \right]_{\EMfixedPoint} (\parameters - \EMfixedPoint) + \text{Remainder}.
\end{align*}
Recall that the EM map depends on the auxiliary function, $Q$, with
\begin{align*}
  Q(\NewParameters, \OldParameters) &\equiv \EV_{S \given y,\OldParameters} \log\left(P(y,S \given
  \NewParameters \right) \\
  \EMmap(\OldParameters) &= \argmax_{\NewParameters} Q(\NewParameters, \OldParameters). 
\end{align*}
For a given value of $\OldParameters$, the derivative of $Q(\NewParameters,
\OldParameters)$ at a maximum is zero, and we write
\begin{align}
  \nonumber
  \Psi(\NewParameters,\OldParameters)
  &\equiv \frac{\partial Q(\NewParameters,
    \OldParameters)}{\partial \NewParameters} \\
  \nonumber
  \Psi(\EMmap(\OldParameters), \OldParameters)
  &= 0 \\
  \nonumber
  \frac{d\Psi(\EMmap(\OldParameters), \OldParameters)}{d \OldParameters}
  &= 0 \\
  \nonumber
  \frac{d\Psi(\EMmap(\OldParameters), \OldParameters)}{d \OldParameters}
  &= \left. \frac{\partial \Psi(\NewParameters, \OldParameters)}{\partial \NewParameters}
    \right|_{\EMmap(\OldParameters), \OldParameters} \left.\frac{\partial
    \EMmap(\parameters)}{\partial \parameters} \right|_{\OldParameters} +
    \left. \frac{\partial \Psi(\NewParameters, \OldParameters)}{\partial \OldParameters}
    \right|_{\EMmap(\OldParameters), \OldParameters} \\
  \nonumber
  &= \left. \frac{\partial^2 Q(\NewParameters, \OldParameters)}{\partial \NewParameters^2}
    \right|_{\EMmap(\OldParameters), \OldParameters} \left.\frac{\partial
    \EMmap(\parameters)}{\partial \parameters} \right|_{\OldParameters} +
    \left. \frac{\partial^2 Q(\NewParameters, \OldParameters)}{\partial \OldParameters
    \partial \NewParameters} \right|_{\EMmap(\OldParameters), \OldParameters} =0 \\
  \label{eq:two_second_derivatives}
  \left.\frac{\partial \EMmap(\parameters)}{\partial \parameters}
  \right|_{\OldParameters}
  &= - \left[ \left. \frac{\partial^2 Q(\NewParameters,  \OldParameters)}{\partial
    \NewParameters^2} \right|_{\EMmap(\OldParameters), \OldParameters} \right]^{-1} \left[
    \left. \frac{\partial^2 Q(\NewParameters, \OldParameters)}{\partial \OldParameters
    \partial \NewParameters} \right|_{\EMmap(\OldParameters), \OldParameters} \right].
\end{align}
Manipulating the first of the two second derivatives in
\eqref{eq:two_second_derivatives} we find
\begin{align}
  \nonumber
  \frac{\partial^2 Q(\NewParameters,  \OldParameters)}{\partial \NewParameters^2}
  &= \frac{\partial^2}{\partial \NewParameters^2} \EV_{S \given y, \OldParameters}
    \log \left( P(y, S \given \NewParameters \right) \\
  \nonumber
  &= \frac{\partial^2}{\partial \NewParameters^2} \EV_{S \given y, \OldParameters}
    \left( \log \left( P(y \given \NewParameters \right) + \log \left( P(S
    \given y, \NewParameters \right) \right)\\
  \nonumber
  &=  \frac{\partial^2}{\partial \NewParameters^2} \log \left( P(y \given
    \NewParameters \right) + \EV_{S \given y, \OldParameters} \left(
    \frac{\partial^2}{\partial \NewParameters^2} \log \left( P(S \given y,
    \NewParameters \right) \right) \\
  \label{eq:second_partial_information}
  &= -J_y - I_{S \given y},
\end{align}
where
\begin{equation*}
  J_y\equiv-\frac{\partial^2}{\partial \parameters^2} \log \left( P(y \given
  \parameters \right)
\end{equation*}
is called the \emph{observed information} that $y$ provides about
$\NewParameters$, and
\begin{equation}
  \label{eq:second_partial_fisher}
  I_{S \given y} = -\EV_{S \given y, \OldParameters} \left(
    \frac{\partial^2}{\partial \NewParameters^2} \log \left( P(S \given y,
    \NewParameters \right) \right)
\end{equation}
is the \emph{Fisher information}\footnote{Fisher Information is
  defined in terms of the squared first derivative, see
  \eqref{eq:score_fisher}.  In many cases a calculation in terms of
  the second derivative like \eqref{eq:second_partial_fisher} is
  equivalent.}  of the
  unobserved data.  Now manipulating the second of the two second
  derivatives in \eqref{eq:two_second_derivatives} we find
\begin{align}
  \nonumber
  \frac{\partial^2 Q(\NewParameters,  \OldParameters)}{\partial \NewParameters \partial
  \OldParameters}
  &= \frac{\partial^2}{\partial \NewParameters \partial \OldParameters} \EV_{S \given y, \OldParameters}
    \log \left( P(y, S \given \NewParameters \right) \\
  \nonumber
  &= \frac{\partial^2}{\partial \NewParameters \partial \OldParameters} \EV_{S \given y, \OldParameters}
    \left( \log \left( P(y \given \NewParameters \right) + \log \left( P(S
    \given y, \NewParameters \right) \right)\\
  \nonumber
  &= \frac{\partial^2}{\partial \NewParameters \partial \OldParameters} \log \left( P(y \given
    \NewParameters \right) + \frac{\partial}{\partial \OldParameters} \EV_{S \given y, \OldParameters} \left(
    \frac{\partial}{\partial \NewParameters} \log \left( P(S \given y,
    \NewParameters \right) \right) \\
  \nonumber
  &=  \frac{\partial}{\partial \OldParameters} \EV_{S \given y, \OldParameters} \left(
    \frac{\partial}{\partial \NewParameters} \log \left( P(S \given y,
    \NewParameters \right) \right) \\
  \nonumber
  &= \frac{\partial}{\partial \OldParameters} \sum_s P(s \given y, \OldParameters) %
    \frac{ %
    \frac{\partial P(s \given y,\NewParameters )}{\partial \NewParameters}
    }{P(s \given y,\NewParameters )}  \\
  \nonumber
  &= \sum_s P(s \given y, \OldParameters) %
    \frac{ %
    \frac{\partial P(s \given y,\OldParameters )}{\partial \OldParameters}
    }{P(s \given y,\OldParameters )}
    \frac{ %
    \frac{\partial P(s \given y,\NewParameters )}{\partial \NewParameters}
    }{P(s \given y,\NewParameters )}
\end{align}
At the fixed point $\EMfixedPoint = \NewParameters = \OldParameters$
and
\begin{align}
  \label{eq:score_fisher}
  \left. \frac{\partial^2 Q(\NewParameters,  \OldParameters)}{\partial \NewParameters
  \partial \OldParameters} \right|_{\EMfixedPoint, \EMfixedPoint}
  &= \EV_{S \given y, \EMfixedPoint} \left( \frac{\partial}{\partial
    \NewParameters} \log \left( P(S \given y, \NewParameters )\right) \right)^2\\
  \label{eq:mixed_partial_information}
  &\equiv I_{S \given y}.
\end{align}
Combining \eqref{eq:mixed_partial_information} and
\eqref{eq:second_partial_information} with
\eqref{eq:two_second_derivatives} we write
\begin{equation}
  \label{eq:information_em_derivative}
  \left.\frac{\partial \EMmap(\parameters)}{\partial \parameters}
  \right|_{\EMfixedPoint} = \left[J_y + I_{S \given y}  \right]^{-1}
  I_{S \given y}.
\end{equation}
Equation~\eqref{eq:information_em_derivative} matches our intuition.
If the observed information, $J_y$, is much larger than the unobserved
information, $I_{S \given y}$, the derivative is small and the
convergence is fast.  Alternatively, if the unobserved information
dominates, then the derivative is close to one and the convergence is
slow\footnote{While Dempster Laird and Rubin\cite{Dempster77} make a
  similar comment about their Theorem 4, Alexis Roche's unpublished
  ``EM algorithm and variants: an informal tutorial'' inspired our
  derivation.}.

\subsection*{Linear Stability of EM}
\label{em_stabiltiy}

Here we use an idea inspired by Sylvester's law of
inertia\footnote{Sylvester's law of inertia is: If $B$ is a symmetric
  matrix, then for any invertible matrix $A$, the number of positive,
  negative and zero eigenvalues (called the inertia of the matrix) of
  $C = A B A\transpose$ is constant.} to show that if at
$\EMfixedPoint$ $J_y$ is positive definite then $\EMmap$ is linearly
stable.

We need the following lemma: If $A$ is positive definite and symmetric
and $B$ is positive definite and symmetric then the eigenvalues of
their product $C = AB$ are positive.  Because $A$ is positive
definite and symmetric, there is an $X$ with
\begin{align*}
  A &= X X\transpose  &&\text{and we can define}\\
  \Gamma &\equiv X^{-1} C X = X^{-1} X X\transpose B X \\
  \Gamma &= X\transpose B X \\
\end{align*}
By assumption $B$ is positive definite so
$y\transpose B y > 0 ~ \forall y \neq 0$.  Now
$\forall z \neq 0, ~ z\transpose X\transpose B X z > 0$ because $Xz$
is a $y$.  So $\Gamma$ is positive definite.  The symmetry of $B$
implies that $\Gamma$ is also symmetric.  Thus $\Gamma$ is positive
definite and symmetric\footnote{The symmetry of $\Gamma$ ensures that
  its eigenvalues are real.}, and all of its eigenvalues are positive.
Because they are related by a similarity transformation, $\Gamma$ and
$C$ have the same eigenvalues, and we know that all of the eigenvalues
of $AB = C$ are positive.

At a fixed point $\EMfixedPoint$, $\EMmap$ is linearly stable if and
only if $\left| \lambda \right| < 1$ for all eigenvalues $\lambda$ of
its derivative $D$.  If $I_{S \given y}$ is full rank, we can rewrite
\eqref{eq:information_em_derivative} as
\begin{align}
  \nonumber
  D &= \left[I_{S \given y}^{-1} J_y + 1 \right]^{-1} \\
  \nonumber
  D^{-1} &= I_{S \given y}^{-1} J_y + 1 \\
  \label{eq:dIJY}
  D^{-1} -1  &= I_{S \given y}^{-1} J_y
\end{align}
Since the right hand side of \eqref{eq:dIJY} satisfies the premises of
the lemma, each of its eigenvalues $\lambda_R$ is positive.  Now for
each eigenvalue of the right hand side there is an eigenvalue of $D$
with
\begin{align}
  \nonumber
  \frac{1}{\lambda_D} - 1 &= \lambda_R  \\
  \nonumber
  \lambda_D &= \frac{1}{1+\lambda_R} &&\text{and since } \lambda_R >0 \\
  \label{eq:D_is_stable}
  0 &< \lambda_D < 1.
\end{align}
Thus if the eigenvalues of $J_y$ are positive (because it is
symmetric, this is equivalent to it being positive definite) then
$\EMmap$ is linearly stable.  A similar argument shows that if
$J_y\equiv-\frac{\partial^2}{\partial \parameters^2} \log \left( P(y \given
  \parameters \right)$ has negative eigenvalues then $\EMmap$ is linearly
unstable.

In summary: Qualitatively $\EMmap$ acts like a gradient flow on the
log likelihood,
$\LogLikelihood(\parameters) \equiv\log \left( P(y \given \parameters
\right)$; convergence to fixed point of $\EMmap$ that is a local
maximum of the likelihood is generic and convergence to a saddle point
of the likelihood is not generic.

\subsubsection{Comments}
\label{sec:ConvergenceRate}

The analysis of this appendix shows that in a neighborhood of a local
maximum of the likelihood the largest eigenvalue,
$\lambda_{\text{max}}$, of $D$ is between 0 and 1.  In the limit of
large $n$ the convergence of the EM algorithm is linear, with an error
$\ti{\epsilon}{n} \equiv \ti{\parameters}{n} - \EMfixedPoint$
dominated by the largest eigenvalue, $\lambda_{\text{max}}$ of $D$,
with
\begin{align*}
  \ti{\epsilon}{n+1} \approx \lambda_{\text{max}} \ti{\epsilon}{n}
  &&\text{for large } n \text{ and }\\
  \lim_{n \rightarrow \infty} \frac{1}{n}\log(\ti{\epsilon}{n}) &= \log(\lambda_{\text{max}}).
\end{align*}

One generally uses the EM algorithm when maximizing the auxiliary
function $Q$ is much easier than estimating derivatives such as
$\delta(\parameters) \equiv \frac{\partial}{\partial \parameters} \log
\left( P(y \given \parameters )\right)$ or
$H(\parameters) \equiv \frac{\partial^2}{\partial \parameters^2} \log
\left( P(y \given \parameters \right)$.  If it is easy to estimate
$\delta$ and $H$, one should use the Newton scheme
\begin{equation*}
  \ti{\parameters}{n+1} = \ti{\parameters}{n} - \left(H(\ti{\parameters}{n})\right)^{-1} \delta(\ti{\parameters}{n})
\end{equation*}
which converges quadratically.  In
Equation~\eqref{eq:information_em_derivative} we have expressed $D$ in
terms of the second derivative $J_y\equiv -H$ and the term $I_{S|y}$
which would be even more difficult to calculate.  We do not suggest
calculating $J_y$, $I_{S|y}$, or $D$, because if you could do that,
you should use a Newton method.  The analysis here both assures that
generic trajectories of the EM algorithm only converge to local maxima
of the likelihood and in Equation~\eqref{eq:information_em_derivative}
gives one a sense of how the convergence rate depends on
characteristics of the likelihood and the unobserved data.

\chapter{Notes on Data and Software}
\label{cha:Software}

We found that writing the text for this book took less time than
writing the supporting software.  All of the software we've used (both
the code we've written and the software that that code depends on) for
this book is free software.  After fetching our code, typing ``make
book'' in the top level directory of the hmmds project will create a
copy of the book in a file called \emph{main.pdf} after some hours of
computation.  \ToDo{For now (2025-02-23) the code specifically for the
  book is at gitlab.com:fraserphysics/hmmds.git, and the more general
  code which that relies on is at gitlab.com:fraserphysics1/hmm.git.
  I develop on NixOS systems.  I intend to repackage the code to make
  it easy for others to use.}

\section*{Data}

We used the following sets of data for examples:
\begin{description}
\item[Tang's laser data] Carl Otto Weiss \index{Weiss, C.\ O.} mailed
  us a CD full of data from various experiments that he and Tang did
  in the 1980s and 1990s.  Although we analyzed many of the files, we
  finally used only a file called \emph{LP5.DAT} in the book (see
  Section~\ref{sec:laser}).  The file \emph{LP5.DAT} is included in
  \emph{hmmdsbook}.
\item[H.~L.~Mencken's \emph{A Book of Prefaces}] We used Mencken's
  book for the parts of speech example in Section~\ref{sec:POSpeech}.
  Although the code fetches the book from www.gutenberg.org as of
  April, 2007, we planned to include the parsed text in
  \emph{hmmdsbook}.
\item[CINC2000 ECG data] We used Dr. Thomas Penzel's ECG measurements
  throughout Chapter~\ref{chap:apnea}.  Although the code fetches the
  the data from\\
  www.physionet.org/physiobank/database/apnea-ecg as of April, 2007,
  we planned to include much smaller files that contain estimates of
  the timing of heart beats in \emph{hmmdsbook}.
\end{description}

\section*{Clarity and Efficiency}

Before the SciPy or NumPy packages existed, we wrote early versions of
the code for this book in C to make it run fast.  Since we wrote those
early versions, SciPy and NumPy have made most of that old C code
obsolete.  Now we have Python code for all of the algorithms described
in the book.

We also provide Cython code for a few of the algorithms.  While the
Cython code is faster, it is harder to read and debug.  The interfaces
to call Cython code match the interfaces to the Python code.  We
recommend developing with the Python code and after that if you need
the speed, try the Cython versions.

Here is the heart of our simple Python implementation of the forward
algorithm described in about 4 pages in Section~\ref{sec:forward}.  It
looks pretty simple here.
\label{code:forward}
\begin{verbatim}
        # last is a conditional distribution of state probabilities.
        # What it is conditioned on changes as the calculations
        # progress.
        last = numpy.copy(self.p_state_initial.reshape(-1))
        for t, likelihood_t in enumerate(self.state_likelihood):
            last *= likelihood_t  # Element-wise multiply
            self.gamma_inv[t] = 1 / last.sum()
            last *= self.gamma_inv[t]
            self.alpha[t, :] = last
            last[:] = numpy.dot(last, self.p_state2state)
\end{verbatim}
And here is the heart of the corresponding implementation of the
backward algorithm.  It is even simpler.
\label{code:backward}
\begin{verbatim}
        # last and beta are analogous to last and alpha in forward(),
        # but the precise interpretations are more complicated.
        last = numpy.ones(self.n_states)
        for t in range(len(self.state_likelihood) - 1, -1, -1):
            self.beta[t, :] = last
            last *= self.state_likelihood[t] * self.gamma_inv[t]
            last[:] = numpy.dot(self.p_state2state, last)
\end{verbatim}

%%% Local Variables:
%%% TeX-master: "main"
%%% eval: (load-file "hmmkeys")
%%% mode: LaTeX
%%% End:

% LocalWords:  Welch Viterbi HMM
