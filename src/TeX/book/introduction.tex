\def\boundaryA{-10} % lorenz.get_bins()
\def\boundaryB{0}
\def\boundaryC{10}
\def\LorenzTrainingIterations{1,000} % synthetic/Rules.mk
\def\NMoreThanTwice{2,759}
\def\NMoreThanTwicePlusOne{2,760}
\def\NTokens{68,857}
\def\NUniqueTokens{9,772}
\def\TextTrainingIterations{100}
\def\NLorenzBig{40,000} % synthetic/Rules.mk
% FixMe: calculate and input these defs
\chapter{Introduction}
\label{chap:introduction}

In a dynamical system, a rule maps states $x\in\cal X$ forward in
time.  Familiar examples include discrete time maps
%% \begin{equation*}
%%   \ti{x}{t+1} = F(\ti{x}{t}) ~~ t \in \INTEGER
%% \end{equation*}
and differential equations
%% \begin{equation*}
%%   \dot x = F(x),
%% \end{equation*}
in which the states are elements of $\REAL^n$.  At time $t$, the
current state $\ti{x}{t}$ of a dynamical system provides all the
information necessary to calculate future states and information about
past states is redundant.  Thus the sequence of states in a dynamical
system satisfies the \emph{Markov property}, which we will more
formally define in Eqn.~\eqref{eq:MarkovChain}.  In applications, we
often think of measured data as being a function of states of a
dynamical system with $ \ti{y}{t} = G(\ti{x}{t})$.  The function $F$
that maps states forward in time and the function $G$ that maps states
to observations make up a \emph{\index*{state space model}} for
sequences of observations.  If the observation function is not
invertible, then knowledge of the measurement $\ti{y}{t}$ at a single
time $t$ is not sufficient to specify the state $\ti{x}{t}$ uniquely,
and one could say that $\ti{x}{t}$ is \emph{hidden}.  That is the
sense of \emph{hidden} in the term ``hidden Markov model''.

Given a short sequence of observations, say $\left(
  \ti{y}{1},\ti{y}{2}, \ti{y}{3}\right)$, one might hope to
\emph{reveal} the state $\ti{x}{3}$ by looking for all initial states
that are consistent with the observations.  The strategy will only
work if the images of all initial states that are consistent with the
observations all fall on the same state, \ie, if for all $x$ such that
$ G(x) = \ti{y}{1}, G\circ F(x) = \ti{y}{2},$ and $G\circ F\circ F(x)
= \ti{y}{3}$, we find that $F\circ F(x) = \hat x$, then the
measurements are sufficient to identify $\ti{x}{3} = \hat x$ uniquely.
If such a revelation procedure works, then one can use it to map long
sequences of observations to long sequences of states and from there
to do forecasting of both states and observations.

For most of the state space models that we consider, the function that
governs the state dynamics and the observation function both have
random elements.  Only imagination limits what constitutes the set of
states in a state space model.  We will consider discrete state spaces
that are sets with a finite number of elements and state spaces that
are real vector spaces.  The sets of observations are similarly
varied.  As a prelude, we look at some measurements of a laser system
that is ``Lorenz like''.

\section{Laser Example}
\label{sec:laser}

In 1963 E.N.~Lorenz\cite{Lorenz63} %
\index{Lorenz system}%
reported interesting behavior in numerical solutions of the system of
equations
\begin{subequations}
  \label{eq:Lorenz}
  \begin{align}
    \dot x_1 &= s x_2 -s x_1\\
    \dot x_2 &= -x_1 x_3 + r x_1 - x_2 \\
    \dot x_3 &= x_1 x_2 - b x_3
  \end{align}
\end{subequations}
which he had proposed as an approximation for fluid convection.  In
Eqn.~\eqref{eq:Lorenz}, $x = (x_1,x_2,x_3)$ is a vector of mode
amplitudes, and the parameters $s$, $r$, and $b$ describe properties
of the fluid.  The paper is widely cited, not because it is a good
model for convection, but because the interesting behavior of the
solutions has characteristics that are typical of what is now called
\emph{\index*{chaos}}.  The Lorenz system has been used countless
times as an example of a system whose solution trajectories are
unstable, aperiodic and bounded, \ie, \emph{chaotic}.  We will use
numerical simulations of the system as illustrations throughout this
book.

In 1975 Haken\cite{Haken75} observed that under certain conditions a
\index*{laser} should obey the same equations.  For a laser, one
interprets the components of the state $x$ as the electric field, the
polarization, and the population inversion in the laser medium.  In
1992 %
\index{Tang, D.\ Y.}%
Tang \etal\cite{Tang92,Tang94}\ reported measurements of the time %
dependent intensity of the electric field for such a laser.  The
measured quantity corresponds to $(\ti{x_1}{t})^2$.
Figure~\ref{fig:LaserLP5} shows a sequence of Tang's measurements.  We
produced the second trace in the figure by numerically integrating
Eqn.~\eqref{eq:Lorenz} with initial conditions and parameters selected
to optimize the match.  The similarity of the two traces convincingly
supports the claim that the laser system is like the Lorenz system.
\begin{figure}[htbp]
  \centering{
    \includegraphics[width=1.0\textwidth]{LaserLP5.pdf}}
  \caption[\comment{fig:LaserLP5 }Laser intensity measurements.]%
  {Laser intensity measurements.  The trace labeled \emph{Laser} is a
    plot of laser intensity measurements provided by Tang \etal  The
    trace labeled \emph{Simulation} plots a numerical simulation of
    the Lorenz system \eqref{eq:Lorenz} with parameters  that optimize
    the match.}
  \label{fig:LaserLP5}
\end{figure}

In working with Tang's laser data, we used a stochastic state space
model with the form
\begin{subequations}
  \label{eq:FandG}
  \begin{align}
    \ti{x}{t+1} &= F(\ti{x}{t}) + \ti{\eta}{t}\\
    \ti{y}{t}   &= G(\ti{x}{t}) + \ti{\epsilon}{t}.
  \end{align}
\end{subequations}
We implemented the function $F$ by integrating the Lorenz system for
an interval $\Delta \tau$, and we used independently identically
distributed Gaussian noise with mean zero and covariance
$\id\sigma_\eta^2$ to implement the state noise $\ti{\eta}{t}$.
\nomenclature[rid]{$\id$}{The \emph{identity} operator; a diagonal
  matrix of ones.}  Our measurement model is $G(\ti{x}{t}) = S_g \cdot
(\ti{x_1}{t})^2 + O_g$ where $S_g$ and $O_g$ are fixed scale and
offset parameters, and the measurement noise $\ti{\epsilon}{t}$ is
independently identically distributed Gaussian noise with mean zero
and covariance $\id\sigma_\epsilon^2$.  The model has the following
eleven free parameters:
\begin{description}
\item[Lorenz system parameters] The values of $r$, $s$, and $b$ in
  \eqref{eq:Lorenz} constitute three free parameters.
\item[Integration time] The single parameter $\Delta \tau$.
\item[Offset and scale] The pair of parameters $O_g$ and $S_g$.
\item[Measurement noise] The single parameter $\sigma_\epsilon$.
\item[State noise] The single parameter $\sigma_\eta$.
\item[Initial state distribution] We model the distribution of the
  initial state as a Gaussian.  We treat the mean as three
  parameters and set the covariance to $\sigma_\eta$.
\end{description}

Using this parameterization, we wrote a routine based on the
\emph{extended Kalman filter} %
\index{Kalman filter|see{extended Kalman filter}}%
\index{extended Kalman filter}%
techniques described in Chapter~\ref{chap:continuous} to calculate
approximate probabilities which we write as $P_{*|\theta}$ where
$\theta$ denotes the collection of parameters.  By passing that
routine and Tang's data to the scipy optimization package, we found a
parameter vector that satisfies
\begin{equation}
  \label{eq:theta-hat-laser}
  \hat \theta = \argmax_{\theta} ~ P(\ts{y}{1}{250}|\theta),
\end{equation}
where $P(\ts{y}{1}{250}|\theta)$ is the conditional probability that a
sequence of 250 observations will have the values
$\ti{y}{1},\ti{y}{2}, \ldots,\ti{y}{250}$ given the parameters
$\theta$.  The parameter vector $\hat \theta$ is called the
\emph{maximum likelihood estimate} %
\index{maximum likelihood estimate (MLE)}%
\index{MLE|see{maximum likelihood estimate}}%
of the parameter vector.  Figure~\ref{fig:LaserLogLike} sketches a
piece of the log-likelihood function.

\begin{figure}[htbp]
  \centering{
    \includegraphics[width=1.0\textwidth]{LaserLogLike.pdf}}
  \caption[\comment{fig:LaserLogLike }Log likelihood as function of $s$ and $b$.]%
  {Log likelihood as function of $s$ and $b$.  Other parameters were
    taken from the vector $\hat \theta$ that maximizes the likelihood
    $P(\ts{y}{1}{250}|\theta)$ (see Eqn.~\eqref{eq:theta-hat-laser}).}
  \label{fig:LaserLogLike}
\end{figure}

Given $\hat \theta$, the maximum likelihood parameters, and the
observations, we can calculate many interesting quantities.  For
example, in Fig.~\ref{fig:LaserStates} we have plotted the sequence of
states that has the highest probability, \ie,
\begin{equation}
  \label{eq:xhatSeq}
  \ts{\hat x}{1}{250} = \argmax_{\ts{x}{1}{250}}
  P(\ts{x}{1}{250}|\ts{y}{1}{250},\hat \theta),
\end{equation}
and in Fig.~\ref{fig:LaserForecast} we have plotted a forecast that we
made by iterating the function $F$ on the state $\hat x$ that has
highest probability given the first 250 observations, \ie,
\begin{equation}
  \label{eq:xhat250}
  \hat x = \argmax_{\ti{x}{250}} P(\ti{x}{250}|\ts{y}{1}{250},\hat \theta).
\end{equation}

\begin{figure}[htbp]
  \centering{\plotsize%
    \includegraphics[width=1.0\textwidth]{LaserStates.pdf}}
  \caption[\comment{fig:LaserStates }State trajectory $\ts{\hat x}{1}{250}$.]%
  {State trajectory $\ts{\hat x}{1}{250}$ estimated from
    observation sequence $\ts{y}{1}{250}$. (see
    Eqn.~\eqref{eq:xhatSeq}.)  Components $x_1$ and $x_3$ of the
    Lorenz system (see Eqn.~\eqref{eq:Lorenz}) are plotted.}
  %\ToDo{Plot interpolated trajectory with dots at sample points.}
  \label{fig:LaserStates}
\end{figure}

\begin{figure}[htbp]
  \centering{
    \includegraphics[width=1.0\textwidth]{LaserForecast.pdf}}
  \caption[\comment{fig:LaserForecast }Forecast observation
  sequence.]%
  {Forecast observation sequence.  We set the noise terms $\eta$ and
    $\epsilon$ to zero and iterated Eqn.~\eqref{eq:FandG} 250 times to
    generate the forecast $\ts{\hat y}{251}{500}$.  We started with
    the initial condition $\hat x$ defined by Eqn.~\eqref{eq:xhat250}.
    The forecast begins to fail noticeably around $t=450$.  The failure
    suggests that the period five cycle in the model is unstable.  The
    period five cycle must have been stable in the actual laser system
    to appear in the data.  Thus an essential characteristic of the
    model is wrong.}
  \label{fig:LaserForecast}
\end{figure}


\section{State Space Models}
\label{sec:formal_ssm}

To get \index*{state space model}s that are more general than the form
(Eqn.~\eqref{eq:FandG}) that we used to describe the laser data, we
suppose only that a conditional probability distribution
$P_{\ti{X}{t+1}|\ti{X}{t}}$ governs evolution in state space and
another conditional distribution $P_{\ti{Y}{t}|\ti{X}{t}}$ governs the
observations $\ti{Y}{t}$.  Combining these two conditional
distribution functions with a distribution $P_{\ti{X}{1}}$ of initial
states defines probabilities for any collection of states and
observations.  In particular, it defines the joint probabilities of
the \emph{stochastic process} or \emph{information source} consisting
of sequences of observations.  We refer to such a combination as a
\emph{model} and denote it as $P_{*|\theta}$.  So defined, the class
of state space models is so broad that to do anything useful, we must
use smaller subclasses.  Typically, we assume that the conditional
distributions are time invariant and that a finite set of parameters
$\theta$ specifies the model.  Notice that we have not specified the
sets from which we draw the states or observations; they could be
discrete, real scalars, real vectors, or something else.


\subsection{Tasks}
\label{sec:tasks}

One can use a parameterized class of state space models $\left\{
  P_{*|\theta} \right\}$ in many ways including the following:
\begin{description}
\item[Model Parameter Estimation] Given a model class $\left\{
    P_{*|\theta} \right\}$ and a sequence of observations $\ts{y}{1}{T}$,
  we often use the maximum likelihood estimate
  \begin{equation}
    \label{eq:IntroMLE}
    \hat \theta_{MLE} \equiv \argmax_{\theta} P(\ts{y}{1}{T}|\theta)
  \end{equation}
  to characterize the source $Y$.
\item[Trajectory Estimation] Given a particular model $P_{*|\theta}$ and
  a sequence of observations $\ts{y}{1}{T}$, one can calculate the
  conditional distribution of states
  $P(\ts{x}{1}{T}|\ts{y}{1}{T},\theta)$.  For example,
  Fig.~\ref{fig:LaserStates} plots the result of a calculation of
  \begin{equation*}
    \ts{\hat x}{1}{250} = \argmax_{\ts{x}{1}{250} \in {\cal X}^{250}}
    P(\ts{x}{1}{250}|\ts{y}{1}{250},\theta).    
  \end{equation*}
\item[Short Term Forecasting] Given a model $P_{*|\theta}$ and a
  distribution of states at time $t$, $P_{\ti{X}{t}}$, one can
  calculate the conditional distribution of future states or
  observations.  For example, Fig.~\ref{fig:LaserForecast} plots the
  result of a calculation of
  \begin{equation*}
    \ts{\hat y}{251}{500} = \argmax_{\ts{y}{251}{500}}
    P(\ts{y}{251}{500}|\ts{y}{1}{250},\theta).
  \end{equation*}
\item[Simulation] Given a model $P_{*|\theta}$, one can characterize its
  long term behavior, answering questions like ``What is a hundred
  year flood?''.  We often find that models that we fit are not good
  for such long term extrapolation.  For example, the laser data that
  we described in the previous section seems to come from a stable
  period five orbit, but the periodic orbit that the trajectory in
  Fig.~\ref{fig:LaserStates} approximates is linearly unstable.  Thus
  the long term behavior of our estimated model is very different from
  the actual laser system.
\item[Classification] Given sample signals like $\ts{y}{1}{T}$ and two
  possible signal sources, $\alpha$ and $\beta$ where $P_{*|\alpha}$
  characterizes healthy units and $P_{*|\beta}$ characterizes defective
  units, one can can classify a unit on the basis of the
  \emph{likelihood ratio}
  \begin{equation*}
    R(\ts{y}{1}{T}) = \frac{P(\ts{y}{1}{T}|\beta)}{P(\ts{y}{1}{T}|\alpha)}.
  \end{equation*}
  If $R(\ts{y}{1}{T})$ is above some threshold, it is classified as defective.
\end{description}


\section{Discrete Hidden Markov Models}
\label{sec:intro_hmm}
\index{discrete hidden Markov model}%

In this section we describe the simplest state space models: those
that are discrete in time, state, and observation.  We begin with a
couple of definitions.  Three random variables $\ti{X}{1}$,
$\ti{X}{2}$, and $\ti{X}{3}$ constitute a \emph{\index*{Markov chain}}
if
\begin{equation}
  \label{eq:MarkovChain}
  P_{\ti{X}{3}|\ti{X}{1},\ti{X}{2}} = P_{\ti{X}{3}|\ti{X}{2}},
\end{equation}
which is equivalent to $\ti{X}{1}$ and $\ti{X}{3}$ being conditionally
independent given $\ti{X}{2}$, \ie,
\begin{equation*}
  P_{\ti{X}{3},\ti{X}{1}|\ti{X}{2}} = P_{\ti{X}{3}|\ti{X}{2}}   P_{\ti{X}{1}|\ti{X}{2}}.
\end{equation*}
An indexed sequence of random variables $\ts{X}{1}{T}$ is a
\emph{Markov process} \index{Markov process} if
for any $t: 1 < t < T$ the variables before and after $t$ are
conditionally independent given $\ti{X}{t}$, \ie,
\begin{equation}
  \label{eq:MarkovProcess}
  P_{\ts{X}{1}{t-1},\ts{X}{t+1}{T}|\ti{X}{t}} =
  P_{\ts{X}{1}{t-1}|\ti{X}{t}} P_{\ts{X}{t+1}{T}|\ti{X}{t}}.
\end{equation}
We will restrict our attention to time invariant models, \ie, those
for which the transition probabilities are constant over time.
Begin by considering the ordinary (\emph{unhidden})
Markov model or process sketched in Fig.~\ref{fig:mm}.  The set of
states $\states = \left\{u,v,w\right\}$, the probability distribution
for the initial state
\begin{equation}
  \label{eq:InitialProbabilites}
P_{\ti{S}{1}} =
\begin{bmatrix}
  \frac{1}{3}, & \frac{1}{3}, & \frac{1}{3}
\end{bmatrix},  
\end{equation}
and the \index*{transition matrix} %
\index{matrix!transition}%
\begin{equation}
  \label{eq:TransitionMatrix}
\begin{array}{rr|ccc}
  && \multicolumn{3}{c}{\ti{S}{t+1}} \\
  \multicolumn{2}{c|}{P(\ti{s}{t+1}|\ti{s}{t})} & u & v & w \\ \hline
  & u & 0 & 1 & 0 \\  && \vspace{-1 em} \\
  \ti{S}{t} & v & 0 & \frac{1}{2} & \frac{1}{2} \\  && \vspace{-0.98 em} \\
  & w & \frac{1}{2} & \frac{1}{2} & 0
\end{array}
\end{equation}
define the model, and the model determines the probability of any
sequence of states $\ts{s}{1}{T}$, which we write\footnote{We use
  upper case letters to denote random variables and $P$ to denote
  probability distribution functions.  A random variable used as
  subscript on $P$ specifies that we mean the distribution of that
  random variable.  We can give $P$ an argument to specify the value
  of the distribution function at that value, \eg $P_X(3)$ is the
  probability that the random variable $X$ has the value 3 and
  $P_X(x)$ is the probability that the random variable $X$ has the
  value $x$.  We usually drop subscripts on $P$ when the context or
  argument resolves ambiguity as to which probability function we
  mean.} as $P_{\ts{S}{1}{T}}\left( \ts{s}{1}{T} \right)$.  For
example we calculate the probability that a sequence of 4 states has
the values $\ts{s}{1}{4} = (u,v,w,v)$, (\ie, $\ti{s}{1} = u$,
$\ti{s}{2} = v$, $\ti{s}{3} = w$, and $\ti{s}{4} = v$) as follows:

\begin{align}
  \label{eq:intro_hmm1}
  P(\ts{s}{1}{4}) &= P(\ti{s}{1}) \prod_{\tau=2}^4 %
                     P(\ti{s}{\tau}|\ts{s}{1}{\tau-1})\\
  \label{eq:intro_hmm2}
                  &= P(\ti{s}{1}) \prod_{\tau=2}^4 %
                     P(\ti{s}{\tau}|\ti{s}{\tau-1}) \\
  \label{eq:intro_hmm3}
  P(u,v,w,v)      &= P(v|u,v,w) \cdot P(w|u,v) \cdot  P(v|u) \cdot P(u)\\
  \label{eq:intro_hmm4}
        &= P(v|w) \cdot P(w|v) \cdot  P(v|u) \cdot P(u)\\
  \label{eq:intro_hmm5}
        &= \frac{1}{2} \cdot \frac{1}{2} \cdot 1 %
                     \cdot \frac{1}{3} = \frac{1}{12}.
\end{align}

Applying \index*{Bayes rule} $\left( P_{A|B} P_B = P_{A,B} \right)$
recursively, yields Eqn.~\eqref{eq:intro_hmm1} and the special case,
Eqn.~\eqref{eq:intro_hmm3}.  Equations \eqref{eq:intro_hmm2} and
\eqref{eq:intro_hmm4} follow from Eqns.~\eqref{eq:intro_hmm1} and
\eqref{eq:intro_hmm3} respectively by the %
\emph{\index*{Markov assumption}}, Eqn.~\eqref{eq:MarkovProcess}.
which says that in determining the probability of the $t^\text{th}$
state given any sequence of previous states only the $(t-1)^\text{th}$
state is relevant. %
%%%


A common exercise is to find a \emph{\index*{stationary}} probability
distribution, \ie, given a transition matrix $T$ find the probability
vector $V$ (nonnegative entries that sum to one) that satisfies
\begin{equation}
  \label{eq:statCond}
  VT = V.  
\end{equation}
If \eqref{eq:statCond} holds for $V = P_{\ti{S}{1}}$, then
\begin{equation*}
  P_{\ti{S}{2}}  = P_{\ti{S}{1}}T = P_{\ti{S}{1}} = P_{\ti{S}{t}}
  \forall t,
\end{equation*}
and in fact all probabilities are independent of shifts in time, \ie,
\begin{equation*}
  P_{\ts{S}{1}{t}} = P_{\ts{S}{1+\tau}{t+\tau}} \forall (t,\tau),
\end{equation*}
which is the definition of a stationary process.  Quick calculations
verify that the initial probability and transition matrix in
\eqref{eq:InitialProbabilites} and \eqref{eq:TransitionMatrix} do not
satisfy \eqref{eq:statCond} but that the distribution $V =
\begin{bmatrix} \frac{1}{7}, & \frac{4}{7}, & \frac{2}{7}
\end{bmatrix}$ does.  Although our example is not a stationary
stochastic process, it relaxes towards such a process in the sense
that
\begin{equation*}
  \lim_{t \rightarrow \infty} P_{\ti{S}{t}} =  \lim_{t \rightarrow
    \infty} P_{\ti{S}{1}} T^t = \begin{bmatrix} \frac{1}{7}, & \frac{4}{7}, & \frac{2}{7}
\end{bmatrix}.
\end{equation*}

\begin{figure}[htbp]
  \centering{\plotsize%
    \input{Markov_mm.pdf_t}}
  \caption[\comment{fig:Markov }A Markov model.]{A Markov model}
  \label{fig:mm}
\end{figure}

The important points about a Markov model also apply to hidden
Markov models, namely that:
\begin{itemize}
\item The model determines the probability of arbitrary sequences of
  observations,
\item and the assumptions about independence and time invariance
  permit specification of the model by a small number of parameters.
\end{itemize}

% karlheg does not like the "," after the display-math array.  The
% punctuation here is not in any style manual, but karlheg thinks it
% seems right:
%
Now suppose, as sketched in Fig.~\ref{fig:dhmm}, that when the system
arrives in a state that rather than observing the state directly, one
observes a random variable that depends on the state.  The matrix that
specifies the random map from states to observations, \ie%
:
\begin{equation*}
  \begin{array}{cr|ccc}
      &      &\multicolumn{3}{c}{Y} \\
      \multicolumn{2}{r|}{P(y|s)} & d & e           & f \\ %
      \hline%
      & u      & 1 & 0           & 0 \\
      &        & \vspace{-1 em} \\
    S & v      & 0 & \frac{1}{3} & \frac{2}{3} \\
      &        & \vspace{-1 em} \\
      & w      & 0 & \frac{2}{3} & \frac{1}{3}
  \end{array}%,
\end{equation*}
combined with the distribution of initial states
\eqref{eq:InitialProbabilites} and transition matrix
\eqref{eq:TransitionMatrix} specifies this hidden Markov model.  The
notion is that the underlying Markov process chugs along unaware of
the observations, and that when the process arrives at each successive
state $\ti{s}{t}$, an observation $\ti{y}{t}$ is produced in a fashion
that depends only on the state $\ti{s}{t}$.

\begin{figure}[htbp]
  \centering{\plotsize%
    \input{Markov_dhmm.pdf_t}}
  \caption[\comment{fig:Markov_dhmm }A hidden Markov model.]{A hidden Markov model}
  \label{fig:dhmm}
\end{figure}

Let us calculate the probability that a sequence of four observations
from this process would have the values $\ts{y}{1}{4} = (d,e,f,e)$.
As an intermediate step we calculate $P(\ts{y}{1}{4},\ts{s}{1}{4})$
for the given observation sequence and all possible state sequences.
Then we add to obtain
\begin{equation}
  \label{eq:dhmm_sum}
  \sum_{\ts{s}{1}{4}} P(\ts{y}{1}{4},\ts{s}{1}{4}) = P(\ts{y}{1}{4}).
\end{equation}
It is convenient that the only state sequences that could have
produced the observation sequence are $(u,v,v,v)$, $(u,v,v,w)$, and
$(u,v,w,v)$.  For any other state sequence
$P(\ts{y}{1}{4},\ts{s}{1}{4}) = 0$.

\begin{subequations}
\label{eq:pcalc}
 \begin{align}
  \ts{s}{1}{4} & & P(\ts{s}{1}{4}) %
                & & P(\ts{y}{1}{4}|\ts{s}{1}{4}) %
                 & & P(\ts{y}{1}{4},\ts{s}{1}{4})\nonumber\\[1ex]
  \hline\nonumber\\[-2ex]
  uvvv         & & \frac{1}{3} \cdot 1 \cdot \frac{1}{2} \cdot \frac{1}{2} %
                & & 1 \cdot \frac{1}{3} \cdot \frac{2}{3} \cdot \frac{1}{3} %
                 & & \frac{2}{324}\\[1ex]
%%%
  uvvw         & & \frac{1}{3} \cdot 1 \cdot \frac{1}{2} \cdot \frac{1}{2} % 
                & & 1 \cdot \frac{1}{3} \cdot \frac{2}{3} \cdot \frac{2}{3} %
                 & & \frac{4}{324}\\[1ex]
%%%
  uvwv         & & \frac{1}{3} \cdot 1 \cdot \frac{1}{2} \cdot \frac{1}{2} %
                & & 1 \cdot \frac{1}{3} \cdot \frac{1}{3} \cdot \frac{1}{3} %
                 & & \frac{1}{324}
 \end{align}
\end{subequations}

Adding the fractions in the right hand column yields %
\begin{equation*}
  P(d,e,f,e) = \frac{7}{324}.
\end{equation*}

Now examine this calculation more carefully beginning with a statement
of the model assumptions.
\begin{description}
\item[The state process is Markov:] Given the current state, the
  probability of the next state is independent of earlier states and
  observations, \ie,
  \begin{equation}
    \label{eq:assume_markov}
    P_{\ti{S}{t+1}|\ts{S}{1}{t},\ts{Y}{1}{t}} = %
        P_{\ti{S}{t+1}|\ti{S}{t}}.
  \end{equation}
\item[The observations are conditionally independent given the states:]
  Given the current state, the probability of the current observation is
  independent of states and observations at all earlier times, \ie,
  \begin{equation}
    \label{eq:assume_output}
    P_{\ti{Y}{t}|\ts{S}{1}{t},\ts{Y}{1}{t-1}} = %
        P_{\ti{Y}{t}|\ti{S}{t}}.
  \end{equation}
\end{description}
Though the assumptions appear asymmetric in time, they are
not\footnote{We often use the following facts about independence
  relations:
  \begin{align}
    P(A|B,C) = P(A|B)   &\iff     P(A,C|B) = P(A|B) \cdot P(C|B)\nonumber\\
    \label{eq:markov-symmetry}%
                        &\iff     P(C|A,B) = P(C|B)\\
    P(A|B,C,D) = P(A|B) &\iff     P(A,C,D|B) = P(A|B) \cdot P(C,D|B)\nonumber\\
                        &\implies P(A,C|B) = P(A|B) \cdot P(C|B)\nonumber\\
    \label{eq:independence-groups}%
                        &\iff     P(A|B,C) = P(A|B).
  \end{align}
  The first chain of implications, \eqref{eq:markov-symmetry}, says
  that if a process is Markov with time going forward, then it is also
  Markov with time going backwards.  The second chain,
  \eqref{eq:independence-groups}, says that if $A$ is conditionally
  independent of $C$ and $D$ given $B$, then $A$ is conditionally
  independent of $C$ alone given $B$.  By symmetry, $A$ is also
  conditionally independent of $D$ given $B$.}.  From the assumptions,
one can derive that
\begin{description}
\item[The joint process is Markov:] 
  \begin{equation*}
    P_{\ts{Y}{t+1}{T},\ts{S}{t+1}{T}|\ts{Y}{1}{t},\ts{S}{1}{t}}
      = P_{\ts{Y}{t+1}{T},\ts{S}{t+1}{T}|\ti{Y}{t},\ti{S}{t}}
  \end{equation*}
\item[Given $\bm{\ti{S}{t}}$, $\bm{\ti{Y}{t}}$ is conditionally independent of everything else:]
  \begin{equation*}
    P_{\ti{Y}{t} |\ts{Y}{1}{t-1},\ts{Y}{t+1}{T}, \ts{S}{1}{T}} = P_{\ti{Y}{t} |\ti{S}{t}}
  \end{equation*}
\end{description}
Equations \eqref{eq:assume_markov} and \eqref{eq:assume_output} are
assumptions about \emph{conditional independence} relations.  Figure
\ref{fig:dhmm_net} represents these relations as a %
\emph{\index*{Bayes net}}\cite{Pearl91a}.

\begin{figure}[htbp]
  \centering{\plotsize%
    \input{Markov_dhmm_net.pdf_t}}
  \caption[\comment{fig:dhmm-net }Bayes net schematic for a hidden Markov model.]%
  {Bayes net schematic for a hidden Markov model.  The drawn edges
    indicate the dependence and independence relations: Given
    $\ti{S}{t}$, $\ti{Y}{t}$ is conditionally independent of
    everything else, and given $\ti{S}{t-1}$, $\ti{S}{t+1}$, and
    $\ti{Y}{t}$, $\ti{S}{t}$ is conditionally independent of
    everything else.}
  \label{fig:dhmm_net}
\end{figure}

Bayes rule and the assumptions justify
\begin{align}
  P_{\ts{Y}{1}{T},\ts{S}{1}{T}} &= P_{\ts{S}{1}{T}} \, %
                                   P_{\ts{Y}{1}{T}|\ts{S}{1}{T}},\notag\\
  \label{eq:sseqprob}%
  P_{\ts{S}{1}{T}}              &= P_{\ti{S}{1}} %
                                   \prod_{t=2}^T P_{\ti{S}{t}|\ti{S}{t-1}}\\
  \label{eq:condyseqprob}%
  P_{\ts{Y}{1}{T}|\ts{S}{1}{T}} &= \prod_{t=1}^T P_{\ti{Y}{t}|\ti{S}{t}}\\
  \intertext{and we conclude}
  P_{\ts{Y}{1}{T},\ts{S}{1}{T}} &= P_{\ti{S}{1}} %
                                   \prod_{t=2}^T P_{\ti{S}{t}|\ti{S}{t-1}}\, %
                                   \prod_{t=1}^T P_{\ti{Y}{t}|\ti{S}{t}}.\notag
\end{align}

Since the state $u$ produces the observation $d$ exclusively and no
other state can produce $d$, the observation sequence $(d,f,e,f)$ is
only possible if the state sequence begins with $u$ and does not
return to $u$.  That constraint reduces the number of possible state
sequences to eight.  The impossibility of state $w$ following itself,
further constrains the possible state sequences to the three listed in
the calculations of Eqn.~\eqref{eq:pcalc}.  One can verify the values
for $P(\ts{s}{1}{T})$ and $P(\ts{y}{1}{T}|\ts{s}{1}{T})$ in those
calculations by applying Eqns.~\eqref{eq:sseqprob} and
\eqref{eq:condyseqprob}.

The calculation of $P(\ts{y}{1}{4})$ in Eqn.~\eqref{eq:pcalc} is easy
because, of the $3^4 = 81$ conceivable state sequences, only three are
consistent with the observations and model structure.  In general
however, if there are $N_S$ states and an observation sequence with
length $T$, then implementing
\begin{equation*}
  P(\ts{y}{1}{T}) = \sum_{\ts{s}{1}{T}} P\left(\ts{s}{1}{T}
    ,\ts{y}{1}{T} \right)
\end{equation*}
naively requires order $\left(N_S\right)^T$ calculations.  If $N_S$
and $T$ are as large as one hundred, $\left(N_S\right)^T$ is too many
calculations for any conceivable computer.

There is a family of algorithms whose complexities are linear in the
length $T$ that make it possible to use HMMs with interestingly long
time series.  The details of these algorithms constitute Chapter
\ref{chap:algorithms}; here we only list their names and objectives.
In these descriptions, we denote by $\parameters$ the vector of
parameters that define an HMM, namely the state transition
probabilities, the initial state probabilities, and the conditional
observation probabilities,
\begin{equation*}
   \parameters \equiv \left\{
     \begin{aligned}
       &\left\{\; P_{\ti{S}{t+1}|\ti{S}{t}} \left(s'|s
       \right)~\forall s,s' \;\right\},\\
       & \left\{\; P_{\ti{S}{1}} \left(s \right)~\forall s
       \;\right\},\\
       &  \left\{\; P_{\ti{Y}{t}|\ti{S}{t}} \left(y_i|s' \right)
       ~ \forall y_i,s' \;\right\}
     \end{aligned}
   \right\}.
\end{equation*}
\begin{description}
\item[The Viterbi Algorithm:] \index{Viterbi algorithm} Given a model
  $\parameters$ and a sequence of observations $\ts{y}{1}{T}$, the
  Viterbi algorithm finds the most probable state sequence $\ts{\hat
    s}{1}{T}$, \ie,
  \begin{equation}
    \label{eq:intro-viterbi}
    \ts{\hat s}{1}{T} = \argmax_{\ts{s}{1}{T}} P
    \left(\ts{s}{1}{T}|\ts{y}{1}{T},\parameters\right).
  \end{equation}
\item[The Baum-Welch Algorithm:] \index{Baum-Welch algorithm}
  \index{forward backward algorithm|see{Baum-Welch algorithm}}(Often
  called the \emph{Forward Backward Algorithm}) Given a sequence of
  observations $\ts{y}{1}{T}$ and an initial set of model parameters
  $\parameters_0$, a single pass of the Baum-Welch algorithm
  calculates a new set of parameters $\parameters_1$ that has higher
  likelihood
  \begin{equation}
    \label{eq:intro-fba}
    P\left( \ts{y}{1}{T}|\parameters_1 \right) \geq
    P\left( \ts{y}{1}{T}|\parameters_0 \right).
  \end{equation}
  Equality can only occur at critical points of the likelihood
  function (where $\partial_\parameters P\left(
    \ts{y}{1}{T}|\parameters \right) = 0$).  In generic cases,
running many iterations of the Baum-Welch algorithm yields a sequence
$\ts{\parameters}{0}{n}$ that approaches a local maximum of the
likelihood.
\item[The Forward Algorithm:] For each time step $t$ and each state
  $s$, the \index*{forward algorithm} calculates the conditional
  probability of being in state $s$ at time $t$ given all of the
  observations up to that time, \ie,
  $P_{\ti{S}{t}|\ts{Y}{1}{t},\parameters} \left(s|\ts{y}{1}{t},
    \parameters \right)$. It also calculates
  $P\left(\ti{y}{t}|\ts{y}{1}{t-1}, \parameters \right)$, the
  conditional probability of each observation given previous
  observations.  Using these terms it calculates the probability of
  the entire data sequence given the model,
  \begin{equation*}
    P \left(\ts{y}{1}{T}|\parameters \right) =  P \left(\ti{y}{1}
      |\parameters \right) \cdot  \prod_{t=2}^T P
    \left(\ti{y}{t}|\ts{y}{1}{t-1}, \parameters \right).
  \end{equation*}
  The forward algorithm is the first phase of the Baum-Welch algorithm.
\end{description}

\subsection{Example: Quantized Lorenz Time Series}
\label{sec:QuantizedLorenz}

%\ToDo{Train on one set and decode another}

To illustrate these algorithms we have applied them to data that we
synthesized by numerically integrating the Lorenz system
(Eqn.~\eqref{eq:Lorenz} with parameter values $r=28$, $s=10$, and
$b=\frac{8}{3}$) and recording \NLorenzBig vectors $x(\tau)$ with a
sampling interval $\Delta \tau = 0.15$.  Then we produced a sequence
of integer valued observations $y_1^{\NLorenzBig}$ by binning $x_1$
with boundaries at $\boundaryA,\boundaryB$ and $\boundaryC$.  The
result is that for each integer $1\leq t \leq \NLorenzBig$,
$\ti{x_1}{t\cdot0.15}$ yields $\ti{y}{t} \in \left\{1,2,3,4\right\}$.
Figure~\ref{fig:TSintro} depicts the first few observations.

\begin{figure}[htbp]
  \centering{\plotsize%
    \includegraphics[width=1.0\textwidth]{TSintro.pdf}}
  \caption[\comment{fig:TSintro }Generating the observations
  $\ts{y}{1}{40}$.]%
  {Generating the observations $\ts{y}{1}{40}$.  The curve in the
    upper plot depicts the first component $\ti{x_1}{\tau}$ of an
    orbit of the Lorenz system (Eqn.~\ref{eq:Lorenz}), and the points
    marked with red dots indicate the values sampled with an interval
    $\Delta \tau = 0.15$.  The points in the lower plot are the
    $\ti{y}{t}$ values quantized with boundaries
    $\left\{\boundaryA, \boundaryB, \boundaryC\right\}$.  }
  \label{fig:TSintro} 
\end{figure}
 
We randomly generated an HMM with twelve hidden states\footnote{We
  chose the number of hidden state to be twelve capriciously so that
  we could organize Fig.~\ref{fig:Statesintro} on a $4\times 4$ grid.}
and 4 possible observations, and then used \LorenzTrainingIterations
iterations of the Baum-Welch algorithm to select a set of parameters
$\hat \parameters$ with high likelihood for the data.  Finally we used
the Viterbi algorithm to find the most likely state sequence
 \begin{equation*}
   {\ts{\hat s}{1}{T}} = \argmax_{\ts{s}{1}{T}}P
   \left(\ts{s}{1}{T}|\ts{y}{1}{T}, \hat \parameters \right).
 \end{equation*}
 
 Although the plot of \emph{decoded} \index{decoded state sequence}
 state values in Fig.~\ref{fig:STSintro} is not very enlightening, we
 can illustrate that there is a relationship between the learned
 decoded states and the original Lorenz system states by going back to
 the original data.  For each state $s$, we identify the set of
 integer times $t$ such that the decoded state is $s$, \ie, $\left\{ t
   :\ti{\hat s}{t} = s \right\}$, and then we find what the Lorenz
 system state was at each of these times and plot that set of points.
 In the upper right box of Fig.~\ref{fig:Statesintro} we have plotted
 points in the original state space that correspond to hidden state
 number one, \ie, the set of pairs $\left\{
   (\ti{x_1}{t\cdot\Delta\tau},\ti{x_3}{t\cdot\Delta\tau}) :\ti{\hat
     s}{t} = 1 \right\}$.  In searching for model parameters that give
 the observed data high likelihood, the Baum-Welch algorithm
 ``discovers'' a discrete hidden state structure, and
 Fig.~\ref{fig:Statesintro} shows that the discrete hidden state
 structure is an approximation of the continuous state space that
 generated the data.
 %%
 \begin{figure}[htbp]
   \centering{\plotsize%
     \includegraphics[width=1.0\textwidth]{STSintro.pdf}}
   \caption[\comment{fig:STSintro }A plot of a state sequence found by Viterbi decoding.]%
   {A plot of the state sequence found by Viterbi decoding a quantized
     time series from the Lorenz system.  Here the number of the
     decoded state $\ti{s}{t}$ is plotted against time $t$.  Although
     it is hard to see any structure in the plot because the numbers
     assigned to the states are not significant,
     Fig.~\ref{fig:Statesintro} illustrates that the decoded states
     are closely related to positions in the generating state space.}
   \label{fig:STSintro}
 \end{figure}
%%%\afterpage{\clearpage}%% Print this right here or let it float to end of chapter?
%%%
%%% ToDo: fig:Statesintro -- Do these color figures need to be in
%%% ToDo: their own "Chapter"-like section so that they can be printed
%%% ToDo: together on a set of color plates?
%%%
%%% This is a large color figure on a page by itself. (butterfly)
%%%
 \begin{figure}[p]
   \centering{\plotsize%
     \includegraphics[width=1.0\textwidth]{Statesintro}}%
   \caption[\comment{fig:Statesintro }Relationship between states of HMM and Lorenz system.]%
   {The relationship between the hidden states of an HMM and the
     original coordinates of the Lorenz system.}
   \label{fig:Statesintro}
 \end{figure}
%%%\afterpage{\clearpage}%% Print this right here or let it float to end of chapter?
\ToDo{Transition diagram for back cover}


\subsection{Example: Hidden States as Parts of Speech}
\label{sec:POSpeech}

Hidden Markov models were developed for speech and text processing,
and for unscientific audiences, we find the application to language
modeling the easiest way to motivate HMMs.  Consider for example the
sentence, ``The dog ate a biscuit.'' and its reduction to a sequence
of parts of speech: \emph{article noun verb article noun}.  By
choosing different particular articles nouns and verbs and placing
them in the order specified by the sequence of parts of speech, we can
produce many other sentences such as, ``An equation describes the
dynamics.''  The parts of speech are like hidden states and the
particular words are like observations.

Rather than using a dictionary or our own knowledge to build a model
of language, here we describe the experiment of applying the
Baum-Welch algorithm to some sample text to create an HMM.  We hope
that the experiment will \emph{discover} parts of speech.  We fetched
the fourth edition of \emph{A Book of Prefaces} by H.~L.~Mencken from
www.gutenberg.org, fit an HMM with the Baum-Welch algorithm, and
decoded a state sequence with the Viterbi algorithm.  We chose a book
of essays rather than a novel because we expected that it would not
have as much dialog.  We feared that different speakers in a novel
would require different models.

The experiment consisted of the following steps:
\begin{description}
\item[Parse the text:] We reduced the text to a sequence of tokens.
  Each token is a word, a number, or a special character such as
  punctuation.  We retained distinctions between lower and upper case.
  The length of the resulting sequence was \NTokens tokens, \ie,
  $\ts{w}{1}{T}$ with $T=\NTokens$.
\item[Identify unique tokens:] There were \NUniqueTokens unique tokens of which
  2,759 appear in the text more than twice.
\item[Create a map from tokens to rank:] We sorted the tokens by the
  frequency of their occurrence so that for the most frequent token
  $w'$, $R(w')=1$ and for the most infrequent token, $\bar w$, $R(\bar
  w)=\NUniqueTokens$.
\item[Map tokens to integers:] We created a sequence of integers
  $\ts{y}{1}{T}$ where $\ti{y}{t} \in \{0,\ldots,\NMoreThanTwicePlusOne\} \forall
  t$.  If the token $\ti{w}{t}$ appeared in the text less than three
  times, we set $\ti{y}{t}=\NMoreThanTwicePlusOne$.  Otherwise, we set  $\ti{y}{t}$ to
  $R(\ti{w}{t})$.
\item[Train an HMM:] Starting from a fifteen state model with random
  parameters, we used \TextTrainingIterations iterations of the
  Baum-Welch algorithm to obtain a trained model.
\item[Decode a sequence of states:] By applying the Viterbi algorithm
  to $\ts{y}{1}{T}$ we obtained $\ts{s}{1}{T}$ where $\ti{s}{t}\in
  \{1,\ldots,12\}$.
\item[Print the most frequent words for each state:] For each state
  $s$, count the number of times each integer $y$ occurs, \ie, $c(y,s)
  = \sum_{t:\ti{s}{t}=s} \delta(y,\ti{y}{t})$.  Then print the words
  corresponding the ten most frequently occurring integers (excluding
  the special value $y=\NMoreThanTwicePlusOne$).
\end{description}
The results appear in Table~\ref{tab:POS}.

\begin{table}[htb]
  \caption[Words most frequently associated with each state.]%
  {Words most frequently associated with each state.  While we have no
    interpretation for some of the states, the following
    interpretations of other states are plausible.}
  \begin{center}{\plotsize%
      \fbox{%
      \begin{tabular}[t]{r@{\hspace{0.28em}}p{19em}}
        1  -- & \rule{0pt}{2.5ex}Nominiative pronouns \\
        2  -- & \rule{0pt}{2.5ex}Prepositions \\
        3  -- & \rule{0pt}{2.5ex}Helping verbs \\
        5  -- & \rule{0pt}{2.5ex}Relative pronouns \\
        9  -- & \rule{0pt}{2.5ex}Articles
      \end{tabular}\qquad%
      \begin{tabular}[t]{r@{\hspace{0.28em}}p{10em}}
        10 -- & \rule{0pt}{2.5ex}Prepositions \\
        11 -- & \rule{0pt}{2.5ex}Nouns \\
        13 -- & \rule{0pt}{2.5ex}Nouns \\
        15 -- & \rule{0pt}{2.5ex}Adjectives
      \end{tabular}}\\[2.0ex]
      \input{decoded_menken.tex}
    }\end{center}
  \label{tab:POS}
\end{table}

\subsection{Remarks}
\label{sec:DHMMRemarks}

One might imagine that HMMs are simply higher order Markov processes.
For example, consider the suggestion that the states depicted in
Fig.~\ref{fig:Statesintro} correspond to sequential pairs of
observations and that the model is a second order Markov model that is
characterized by $P_{\ti{Y}{t+1}|\ti{Y}{t},\ti{Y}{t-1}}$, the set of
observation probabilities conditioned on the \emph{two} previous
observations.  Although the number of unique sequential pairs
$\ts{y}{t}{t+1}$ that occur in our data is in fact twelve, the fact
that some of the states in Fig.~\ref{fig:Statesintro} straddle the
quantization boundaries at $x = \boundaryA$ and $x=\boundaryC$ belies the suggestion.
In general, the class of HMMs is more powerful than the class of
simple Markov models in the sense that the former includes the later
but not vice versa.

Let us emphasize the following points about discrete HMMs:
\begin{enumerate}
\item Although the hidden process is first order Markov, the observation
process may not be a Markov process (of any order). \label{noX}
\item Any Markov model of any order can be represented by an HMM.
\item Even if the functions governing the dynamics and observations of
  a continuous state space system are nonlinear, a discrete HMM can
  approximate the system arbitrarily well\footnote{By using a discrete
    HMM to approximate dynamics governed by a continuous function one
    sacrifices the opportunity to exploit continuity.  That sacrifice
    will degrade model performance in many applications.} by
  using large numbers of states $N_S$ and possible observation values
  $N_Y$.
\item For estimating model parameters, larger numbers of training data
  are required as $N_S$ and $N_Y$ are increased. \label{point:large}
\end{enumerate}

As an illustration of Point \ref{noX},\ consider the process depicted in
Fig.~\ref{fig:nonmm}, which produces observation strings with runs of about
seven $a$'s interspersed with occasional $b$'s and $c$'s.  In the
observation stream, the $b$'s and $c$'s alternate no matter how many $a$'s
fall in between.  Such behavior can not be captured by a simple Markov
process of any order.

\begin{figure}[htbp]
  \centering{\plotsize%
    \input{nonmm.pdf_t}}
  \caption[\comment{fig:nonmm }An HMM that cannot be represented by a Markov model.]%
  {An HMM that cannot be represented by a Markov model of any order.
    Consider the string of observations ``$b,a,a,\ldots,a,a,a$''.
    Since the previous non-''$a$'' observation was ``$b$'' and the
    model will not produce another ``$b$'' before it produces a
    ``$c$'', the next observation can be either a ``$c$'' or another
    ``$a$'', but not a ``$b$''.  Because there is no limit on the
    number of consecutive ``$a$'s'' that can appear, there is no limit
    on how far back in the observation sequence you might have to look
    to know the probabilities of the next observation.}
  \label{fig:nonmm}
\end{figure}
%%%\afterpage{\clearpage}%% Print this right here or let it float to end of chapter?

The possibility of long term memory makes state space models, \eg,
HMMs, more powerful than Markov models.  That observation suggests
that if there is noise, then the \emph{\index*{delay vector} reconstructions}
described in the chaos
literature\cite{Packard80,Takens81,Fraser86,Sauer91} are suboptimal
because they discard information from earlier observations that could
be used to more accurately specify the state.

%%% Local Variables:
%%% TeX-master: "main"
%%% eval: (load-file "hmmkeys.el")
%%% End:
