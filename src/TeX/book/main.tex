%%%
%%% "hmmdsbook" is a subclass of "newsiambook", unless you use the
%%% 'ltxbook' option, in which case it is a subclass of the standard
%%% LaTeX "book".
%%%
%%% Note: The 'nottoc' option should be passed if we do not want the
%%%       TOC to appear in the TOC.  It will be passed along to
%%%       tocbibind.sty by hmmdsbook.cls.
%%%

%\documentclass[]{hmmdsbook} % For old siam style
\documentclass[ltxbook,showlabels]{hmmdsbook}

%%% Note: The following settings will cause many overfull hbox false alarms:
% \documentclass[dvips,prelim,showlabels]{hmmdsbook}
% %%% Show index entries in margin
% \proofmodetrue

%%% I'm putting this in here for our own runs, since I think that it
%%% provides a reasonable facimile of the font metrics that the final
%%% document will have, at least for the plot labels.  This should
%%% probably be removed or commented off before the final run.
%%%
\usepackage{type1cm}
\renewcommand{\plotsize}{%
  \fontsize{8.5}{8.5}%
  \selectfont%
}


%%% I found that the crop.sty included with TeXlive works a lot better
%%% than the one included with the SIAM book macro package.  I've
%%% moved theirs aside so the TeXlive version is used.
%%%
\usepackage[letter,center]{crop}
\crop

\usepackage{fancybox,framed}
\usepackage{rotating}

\author{\copyright Andrew M. Fraser}
\title{Hidden Markov Models and Dynamical Systems}


\begin{document}
\frontmatter
\maketitle

\chapter*{Preface}
\addcontentsline{toc}{chapter}{Preface}
\label{chap:preface}
This book arose from a pair of symposia on hidden Markov models that
Kevin Vixie organized at the 2001 SIAM Dynamical Systems meeting.  At
the end of the first symposium someone asked for a simple reference
that explains the basic ideas and algorithms for applying HMMs.  We
were stumped\footnote{In subsequent years several FixMe.}.  A group of
the participants suggested writing this book to answer the question.
Two years later, I delivered a proposal for the book that included
about two chapters to SIAM at the 2003 Dynamical Systems meeting.
Sometime after SIAM published the first edition in 2008, I began
working on updating the software I used to generate the book.  In an
effort to use better software practices, I wrote tests, and one of
those tests revealed a conceptual error in the last chapter of the
book.  In the current edition Chapter~\ref{chap:apnea} is a thorough
revision that addresses the same tasks with a simpler approach.

The book is intended for readers who have backgrounds and interests
typical of those who attend the SIAM Dynamical Systems meetings.  In
particular, my view is that HMMs are discrete state discrete time
stochastic dynamical systems, and that they are often used to
approximate dynamical systems with continuous state spaces operating
in continuous time.  Thus, by using familiar analogies, it should be
easy to explain HMMs to readers who have studied dynamical systems.

The basic techniques were well developed in the 1970's for work on
speech and language processing.  Many in speech research learned about
the techniques at a symposium held at the Institute for Defense
Analysis \index{Institute for Defense Analysis} in Princeton.  A small
number of proceedings of that symposium \cite{ida80} were printed, and
the volume was called \emph{the blue book} by workers in the field.  I
was not part of that community, but I have a copy of the blue book.
It explains the basic algorithms and illustrates them with simple
concrete examples.  I hope my book is as simple, useful, and clear.

Although there are now other books and papers that are about HMMs
exclusively or in part, I believe that the present volume is unique in
that:
\begin{description}
\item[It is introductory] Sophomore or Junior study in engineering,
  mathematics, or science provides the prerequisites for most of the
  book.  The exceptions are ideas from dynamical systems and
  information theory.  In particular, I use the Gibbs inequality
  (Eqn.~\eqref{eq:GibbsIE}) in developing the EM algorithm in Chapter
  \ref{chap:algorithms}.  Although Chapter \ref{chap:toys} \emph{Toy
    Problems and Performance Bounds} deals with Lyapunov exponents and
  entropies, it is not a prerequisite for any other chapter.
\item[Algorithms are explained and justified] I present enough of the
  theory behind the basic algorithms in Chapter \ref{chap:algorithms}
  so that a reader can use it as a guide to developing his own
  variants.
\item[I provide code implementing the algorithms and data for the
  examples] Although algorithms are given in pseudo code in the text,
  at \url{http://www.fraserphysics.com/~andy/hmmdsbook} and % FixMe
  \url{www.siam.org/~fraser/XXXX} I provide a working implementation
  of each of the algorithms I describe.  I have chosen to write the
  programs in the python language because it is easy to read and the
  interpreter is free software.  I have written the programs to follow
  the descriptions and notation in the text.  I provide data and
  scripts (makefiles, shell scripts, etc.)  that make all of the
  figures in the text.  On a gnu system, % FixMe What systems?
  issuing ``make book.pdf''
  from a command line compiles the software, runs the numerical
  experiments, makes the figures, and formats the entire book.
\item[It uses analogies to dynamical systems] For example, I
  demonstrate the HMM training algorithm by applying it to data
  derived from the Lorenz system.  The result, as
  Fig.~\ref{fig:Statesintro} illustrates, is that the algorithm
  estimates a discrete state generating mechanism that is an
  approximation to the state space of the original Lorenz system.
\item[I illustrate with a practical example] In Chapter \ref{chap:apnea} I
  present an application to experimental measurements of
  electrocardiograms (ECGs).
\end{description}

\ToDo{ Point to review papers and material (I've collected some of the
  kind of stuff I have in mind at
  \url{http://fraserphysics.com/~andy/HMMs/} ).  Perhaps organize this
  stuff under the following topics:
  \begin{itemize}
  \item MCMC annealing and optimization
  \item Other applications
  \item Other papers books and web sites
  \item Bayes nets
  \end{itemize}
}
\section*{Acknowledgments}
\addcontentsline{toc}{section}{Acknowledgments}%
\label{sec:ack}
\longpage%%% Maybe not required for MathTime... look for partial
         %%% paragraph on page by itself before TOC begins.

In writing this book, I have used much that I learned from colleagues
in Portland.  In particular, Todd Leen kept me informed about
developments in the machine learning community.  I've relied on the
review of the ergodic theory of dynamical systems in Kevin Vixie's
dissertation\cite{vixie02} for Chapter~\ref{chap:toys} of this book.
Shari Matzner and Gerardo Lafferriere helped me understand the
convergence properties of the EM algorithm.  Also the following
colleagues have given me constructive comments on drafts of the book:
% 
Patrick Campbell, %
Ralf Juengling, %
Shari Matzner, %
Cosma Shalizi, %
and %
Rudolph Van der Merwe. %

I thank the following people and organizations for providing the data
that I used for examples:
\begin{description}
\item[Carl Otto Weiss], for providing Tang's\cite{Tang92} laser data
\item[PhysioNet] For providing Penzel's\cite{Penzel02} ECG data and
  % FixMe
  https://physionet.org/content/wfdb-python/4.1.0/ \cite{PhysioToolkit}
\item[Project Gutenberg] For digitizing and distributing \emph{A
    Book of Prefaces},\\ by H.~L.~Mencken
%%% I hate to put a line break in there like that, but it was hanging
%%% out into the margin without it. -- karlheg
\end{description}

I was fortunate to meet the late Karl Hegbloom in Portland.  Karl
contributed to several free software projects and he helped me with
the figures and software for the fist edition.  In addition to
personal help with software, I relied on free software written by too
many contributers to list.  I used the NixOS distribution of Linux to
organize the software.

For writing the first edition, I acknowledge Portland State University
for support of a sabbatical leave
%


\tableofcontents
%%%\addcontentsline{toc}{chapter}{Table of Contents}%
%%% Enable these to make it easier to find them during editing phase
% \listoffigures% Not unless required by SIAM
% \listoftables% Not unless required by SIAM
%%% \listofalgorithms%% There are none, since at this point they are done as figures.

\mainmatter
\chapter{Introduction}
\label{chap:introduction}

In a dynamical system, a rule maps states $x\in\cal X$ forward in
time.  Familiar examples include discrete time maps
%% \begin{equation*}
%%   \ti{x}{t+1} = F(\ti{x}{t}) ~~ t \in \INTEGER
%% \end{equation*}
and differential equations
%% \begin{equation*}
%%   \dot x = F(x),
%% \end{equation*}
in which the states are elements of $\REAL^n$.  At time $t$, the
current state $\ti{x}{t}$ of a dynamical system provides all the
information necessary to calculate future states and information about
past states is redundant.  Thus the sequence of states in a dynamical
system satisfies the \emph{Markov property}, which we will more
formally define in Eqn.~\eqref{eq:MarkovChain}.  In applications, we
often think of measured data as being a function of states of a
dynamical system with $ \ti{y}{t} = G(\ti{x}{t})$.  The function $F$
that maps states forward in time and the function $G$ that maps states
to observations make up a \emph{\index*{state space model}} for
sequences of observations.  If the observation function is not
invertible, then knowledge of the measurement $\ti{y}{t}$ at a single
time $t$ is not sufficient to specify the state $\ti{x}{t}$ uniquely,
and one could say that $\ti{x}{t}$ is \emph{hidden}.  That is the
sense of \emph{hidden} in the term ``hidden Markov model''.

Given a short sequence of observations, say $\left(
  \ti{y}{1},\ti{y}{2}, \ti{y}{3}\right)$, one might hope to
\emph{reveal} the state $\ti{x}{3}$ by looking for all initial states
that are consistent with the observations.  The strategy will only
work if the images of all initial states that are consistent with the
observations all fall on the same state, \ie, if for all $x$ such that
$ G(x) = \ti{y}{1}, G\circ F(x) = \ti{y}{2},$ and $G\circ F\circ F(x)
= \ti{y}{3}$, we find that $F\circ F(x) = \hat x$, then the
measurements are sufficient to identify $\ti{x}{3} = \hat x$ uniquely.
If such a revelation procedure works, then one can use it to map long
sequences of observations to long sequences of states and from there
to do forecasting of both states and observations.
%%
%%  http://en.wikipedia.org/wiki/Inductive_reasoning 
%%
%%  "Induction is sometimes framed as reasoning about the future from
%%  the past, but in its broadest sense it involves reaching
%%  conclusions about unobserved things on the basis of what has been
%%  observed.  Inferences about the past from present evidence for
%%  instance, as in archaeology, count as induction. Induction could
%%  also be across space rather than time, for instance as in physical
%%  cosmology where conclusions about the whole universe are drawn
%%  from the limited perspective we are able to observe (see cosmic
%%  variance); or in economics, where national economic policy is
%%  derived from local economic performance."

For most of the state space models that we consider, the function that
governs the state dynamics and the observation function both have
random elements.  Only imagination limits what constitutes the set of
states in a state space model.  We will consider discrete state spaces
that are sets with a finite number of elements and state spaces that
are real vector spaces.  The sets of observations are similarly
varied.  As a prelude, we look at some measurements of a laser system
that is ``Lorenz like''.

\section{Laser Example}
\label{sec:laser}

In 1963 E.N.~Lorenz\cite{Lorenz63} %
\index{Lorenz system}%
reported interesting behavior in numerical solutions of the system of
equations
\begin{subequations}
  \label{eq:Lorenz}
  \begin{align}
    \dot x_1 &= s x_2 -s x_1\\
    \dot x_2 &= -x_1 x_3 + r x_1 - x_2 \\
    \dot x_3 &= x_1 x_2 - b x_3
  \end{align}
\end{subequations}
which he had proposed as an approximation for fluid convection.  In
Eqn.~\eqref{eq:Lorenz}, $x = (x_1,x_2,x_3)$ is a vector of mode
amplitudes, and the parameters $s$, $r$, and $b$ describe properties
of the fluid.  The paper is widely cited, not because it is a good
model for convection, but because the interesting behavior of the
solutions has characteristics that are typical of what is now called
\emph{\index*{chaos}}.  The Lorenz system has been used countless
times as an example of a system whose solution trajectories are
unstable, aperiodic and bounded, \ie, \emph{chaotic}.  We will use
numerical simulations of the system as illustrations throughout this
book.

In 1975 Haken\cite{Haken75} observed that under certain conditions a
\index*{laser} should obey the same equations.  For a laser, one
interprets the components of the state $x$ as the electric field, the
polarization, and the population inversion in the laser medium.  In
1992 %
\index{Tang, D.\ Y.}%
Tang \etal\cite{Tang92}\ reported measurements of the time %
% @Article{PhysRevA.49.1296,
%   title = {Uniqueness of the chaotic attractor of a single-mode laser},
%   author = {Tang, D. Y. and Weiss, C. O.},
%   journal = {Phys. Rev. A},
%   volume = {49},
%   number = {2},
%   pages = {1296--1300},
%   numpages = {4},
%   year = {1994},
%   month = {Feb},
%   doi = {10.1103/PhysRevA.49.1296},
%   publisher = {American Physical Society}
% }
dependent intensity of the electric field for such a laser.  The
measured quantity corresponds to $(\ti{x_1}{t})^2$.
Figure~\ref{fig:LaserLP5} shows a sequence of Tang's measurements.  We
produced the second trace in the figure by numerically integrating
Eqn.~\eqref{eq:Lorenz} with initial conditions and parameters selected
to optimize the match.  The similarity of the two traces convincingly
supports the claim that the laser system is like the Lorenz system.
\begin{figure}[htbp]
  \centering{
    \includegraphics[width=1.0\textwidth]{LaserLP5.pdf}}
  \caption[\comment{fig:LaserLP5 }Laser intensity measurements.]%
  {Laser intensity measurements.  The trace labeled \emph{Laser} is a
    plot of laser intensity measurements provided by Tang \etal  The
    trace labeled \emph{Simulation} plots a numerical simulation of
    the Lorenz system \eqref{eq:Lorenz} with parameters  that optimize
    the match.}
  \label{fig:LaserLP5}
\end{figure}

In working with Tang's laser data, we used a stochastic state space
model with the form
\begin{subequations}
  \label{eq:FandG}
  \begin{align}
    \ti{x}{t+1} &= F(\ti{x}{t}) + \ti{\eta}{t}\\
    \ti{y}{t}   &= G(\ti{x}{t}) + \ti{\epsilon}{t}.
  \end{align}
\end{subequations}
We implemented the function $F$ by integrating the Lorenz system for
an interval $\Delta \tau$, and we used independently identically
distributed Gaussian noise with mean zero and covariance
$\id\sigma_\eta^2$ to implement the state noise $\ti{\eta}{t}$.
\nomenclature[rid]{$\id$}{The \emph{identity} operator; a diagonal
  matrix of ones.}  Our measurement model is $G(\ti{x}{t}) = S_g \cdot
(\ti{x_1}{t})^2 + O_g$ where $S_g$ and $O_g$ are fixed scale and
offset parameters, and the measurement noise $\ti{\epsilon}{t}$ is
independently identically distributed Gaussian noise with mean zero
and covariance $\id\sigma_\epsilon^2$.  The model has the following
eleven free parameters:
\begin{description}
\item[Lorenz system parameters] The values of $r$, $s$, and $b$ in
  \eqref{eq:Lorenz} constitute three free parameters.
\item[Integration time] The single parameter $\Delta \tau$.
\item[Offset and scale] The pair of parameters $O_g$ and $S_g$.
\item[Measurement noise] The single parameter $\sigma_\epsilon$.
\item[State noise] The single parameter $\sigma_\eta$.
\item[Initial state distribution] We model the distribution of the
  initial state as a Gaussian.  We treat the mean as three
  parameters and set the covariance to $\sigma_\eta$.
\end{description}

Using this parameterization, we wrote a routine based on the
\emph{extended Kalman filter} %
\index{Kalman filter|see{extended Kalman filter}}%
\index{extended Kalman filter}%
techniques described in Chapter~\ref{chap:continuous} to calculate
approximate probabilities which we write as $P_{*|\theta}$ where
$\theta$ denotes the collection of parameters.  By passing that
routine and Tang's data to the scipy optimization package, we found a
parameter vector that satisfies
\begin{equation}
  \label{eq:theta-hat-laser}
  \hat \theta = \argmax_{\theta} ~ P(\ts{y}{1}{250}|\theta),
\end{equation}
where $P(\ts{y}{1}{250}|\theta)$ is the conditional probability that a
sequence of 250 observations will have the values
$\ti{y}{1},\ti{y}{2}, \ldots,\ti{y}{250}$ given the parameters
$\theta$.  The parameter vector $\hat \theta$ is called the
\emph{maximum likelihood estimate} %
\index{maximum likelihood estimate (MLE)}%
\index{MLE|see{maximum likelihood estimate}}%
of the parameter vector.  Figure~\ref{fig:LaserLogLike} sketches a
piece of the log-likelihood function.

\begin{figure}[htbp]
  \centering{
    \includegraphics[width=1.0\textwidth]{LaserLogLike.pdf}}
  \caption[\comment{fig:LaserLogLike }Log likelihood as function of $s$ and $b$.]%
  {Log likelihood as function of $s$ and $b$.  Other parameters were
    taken from the vector $\hat \theta$ that maximizes the likelihood
    $P(\ts{y}{1}{250}|\theta)$ (see Eqn.~\eqref{eq:theta-hat-laser}).}
  \label{fig:LaserLogLike}
\end{figure}

Given $\hat \theta$, the maximum likelihood parameters, and the
observations, we can calculate many interesting quantities.  For
example, in Fig.~\ref{fig:LaserStates} we have plotted the sequence of
states that has the highest probability, \ie,
\begin{equation}
  \label{eq:xhatSeq}
  \ts{\hat x}{1}{250} = \argmax_{\ts{x}{1}{250}}
  P(\ts{x}{1}{250}|\ts{y}{1}{250},\hat \theta),
\end{equation}
and in Fig.~\ref{fig:LaserForecast} we have plotted a forecast that we
made by iterating the function $F$ on the state $\hat x$ that has
highest probability given the first 250 observations, \ie,
\begin{equation}
  \label{eq:xhat250}
  \hat x = \argmax_{\ti{x}{250}} P(\ti{x}{250}|\ts{y}{1}{250},\hat \theta).
\end{equation}

\begin{figure}[htbp]
  \centering{\plotsize%
    \includegraphics[width=1.0\textwidth]{LaserStates.pdf}}
  \caption[\comment{fig:LaserStates }State trajectory $\ts{\hat x}{1}{250}$.]%
  {State trajectory $\ts{\hat x}{1}{250}$ estimated from
    observation sequence $\ts{y}{1}{250}$. (see
    Eqn.~\eqref{eq:xhatSeq}.)  Components $x_1$ and $x_3$ of the
    Lorenz system (see Eqn.~\eqref{eq:Lorenz}) are plotted.}
  %\ToDo{Plot interpolated trajectory with dots at sample points.}
  \label{fig:LaserStates}
\end{figure}

\begin{figure}[htbp]
  \centering{
    \includegraphics[width=1.0\textwidth]{LaserForecast.pdf}}
  \caption[\comment{fig:LaserForecast }Forecast observation
  sequence.]%
  {Forecast observation sequence.  We set the noise terms $\eta$ and
    $\epsilon$ to zero and iterated Eqn.~\eqref{eq:FandG} 250 times to
    generate the forecast $\ts{\hat y}{251}{500}$.  We started with
    the initial condition $\hat x$ defined by Eqn.~\eqref{eq:xhat250}.
    The forecast begins to fail noticeably around $t=450$.  The failure
    suggests that the period five cycle in the model is unstable.  The
    period five cycle must have been stable in the actual laser system
    to appear in the data.  Thus an essential characteristic of the
    model is wrong.}
  \label{fig:LaserForecast}
\end{figure}


\section{State Space Models}
\label{sec:formal_ssm}

To get \index*{state space model}s that are more general than the form
(Eqn.~\eqref{eq:FandG}) that we used to describe the laser data, we
suppose only that a conditional probability distribution
$P_{\ti{X}{t+1}|\ti{X}{t}}$ governs evolution in state space and
another conditional distribution $P_{\ti{Y}{t}|\ti{X}{t}}$ governs the
observations $\ti{Y}{t}$.  Combining these two conditional
distribution functions with a distribution $P_{\ti{X}{1}}$ of initial
states defines probabilities for any collection of states and
observations.  In particular, it defines the joint probabilities of
the \emph{stochastic process} or \emph{information source} consisting
of sequences of observations.  We refer to such a combination as a
\emph{model} and denote it as $P_{*|\theta}$.  So defined, the class
of state space models is so broad that to do anything useful, we must
use smaller subclasses.  Typically, we assume that the conditional
distributions are time invariant and that a finite set of parameters
$\theta$ specifies the model.  Notice that we have not specified the
sets from which we draw the states or observations; they could be
discrete, real scalars, real vectors, or something else.


\subsection{Tasks}
\label{sec:tasks}

One can use a parameterized class of state space models $\left\{
  P_{*|\theta} \right\}$ in many ways including the following:
\begin{description}
\item[Model Parameter Estimation] Given a model class $\left\{
    P_{*|\theta} \right\}$ and a sequence of observations $\ts{y}{1}{T}$,
  we often use the maximum likelihood estimate
  \begin{equation}
    \label{eq:IntroMLE}
    \hat \theta_{MLE} \equiv \argmax_{\theta} P(\ts{y}{1}{T}|\theta)
  \end{equation}
  to characterize the source $Y$.
\item[Trajectory Estimation] Given a particular model $P_{*|\theta}$ and
  a sequence of observations $\ts{y}{1}{T}$, one can calculate the
  conditional distribution of states
  $P(\ts{x}{1}{T}|\ts{y}{1}{T},\theta)$.  For example,
  Fig.~\ref{fig:LaserStates} plots the result of a calculation of
  \begin{equation*}
    \ts{\hat x}{1}{250} = \argmax_{\ts{x}{1}{250} \in {\cal X}^{250}}
    P(\ts{x}{1}{250}|\ts{y}{1}{250},\theta).    
  \end{equation*}
\item[Short Term Forecasting] Given a model $P_{*|\theta}$ and a
  distribution of states at time $t$, $P_{\ti{X}{t}}$, one can
  calculate the conditional distribution of future states or
  observations.  For example, Fig.~\ref{fig:LaserForecast} plots the
  result of a calculation of
  \begin{equation*}
    \ts{\hat y}{251}{500} = \argmax_{\ts{y}{251}{500}}
    P(\ts{y}{251}{500}|\ts{y}{1}{250},\theta).
  \end{equation*}
\item[Simulation] Given a model $P_{*|\theta}$, one can characterize its
  long term behavior, answering questions like ``What is a hundred
  year flood?''.  We often find that models that we fit are not good
  for such long term extrapolation.  For example, the laser data that
  we described in the previous section seems to come from a stable
  period five orbit, but the periodic orbit that the trajectory in
  Fig.~\ref{fig:LaserStates} approximates is linearly unstable.  Thus
  the long term behavior of our estimated model is very different from
  the actual laser system.
\item[Classification] Given sample signals like $\ts{y}{1}{T}$ and two
  possible signal sources, $\alpha$ and $\beta$ where $P_{*|\alpha}$
  characterizes healthy units and $P_{*|\beta}$ characterizes defective
  units, one can can classify a unit on the basis of the
  \emph{likelihood ratio}
  \begin{equation*}
    R(\ts{y}{1}{T}) = \frac{P(\ts{y}{1}{T}|\beta)}{P(\ts{y}{1}{T}|\alpha)}.
  \end{equation*}
  If $R(\ts{y}{1}{T})$ is above some threshold, it is classified as defective.
\end{description}


\section{Discrete Hidden Markov Models}
\label{sec:intro_hmm}
\index{discrete hidden Markov model}%

In this section we describe the simplest state space models: those
that are discrete in time, state, and observation.  We begin with a
couple of definitions.  Three random variables $\ti{X}{1}$,
$\ti{X}{2}$, and $\ti{X}{3}$ constitute a \emph{\index*{Markov chain}}
if
\begin{equation}
  \label{eq:MarkovChain}
  P_{\ti{X}{3}|\ti{X}{1},\ti{X}{2}} = P_{\ti{X}{3}|\ti{X}{2}},
\end{equation}
which is equivalent to $\ti{X}{1}$ and $\ti{X}{3}$ being conditionally
independent given $\ti{X}{2}$, \ie,
\begin{equation*}
  P_{\ti{X}{3},\ti{X}{1}|\ti{X}{2}} = P_{\ti{X}{3}|\ti{X}{2}}   P_{\ti{X}{1}|\ti{X}{2}}.
\end{equation*}
An indexed sequence of random variables $\ts{X}{1}{T}$ is a
\emph{Markov process} \index{Markov process} if
for any $t: 1 < t < T$ the variables before and after $t$ are
conditionally independent given $\ti{X}{t}$, \ie,
\begin{equation}
  \label{eq:MarkovProcess}
  P_{\ts{X}{1}{t-1},\ts{X}{t+1}{T}|\ti{X}{t}} =
  P_{\ts{X}{1}{t-1}|\ti{X}{t}} P_{\ts{X}{t+1}{T}|\ti{X}{t}}.
\end{equation}
We will restrict our attention to time invariant models, \ie, those
for which the transition probabilities are constant over time.
Begin by considering the ordinary (\emph{unhidden})
Markov model or process sketched in Fig.~\ref{fig:mm}.  The set of
states $\states = \left\{u,v,w\right\}$, the probability distribution
for the initial state
\begin{equation}
  \label{eq:InitialProbabilites}
P_{\ti{S}{1}} =
\begin{bmatrix}
  \frac{1}{3}, & \frac{1}{3}, & \frac{1}{3}
\end{bmatrix},  
\end{equation}
and the \index*{transition matrix} %
\index{matrix!transition}%
\begin{equation}
  \label{eq:TransitionMatrix}
\begin{array}{rr|ccc}
  && \multicolumn{3}{c}{\ti{S}{t+1}} \\
  \multicolumn{2}{c|}{P(\ti{s}{t+1}|\ti{s}{t})} & u & v & w \\ \hline
  & u & 0 & 1 & 0 \\  && \vspace{-1 em} \\
  \ti{S}{t} & v & 0 & \frac{1}{2} & \frac{1}{2} \\  && \vspace{-0.98 em} \\
  & w & \frac{1}{2} & \frac{1}{2} & 0
\end{array}
\end{equation}
define the model, and the model determines the probability of any
sequence of states $\ts{s}{1}{T}$, which we write\footnote{We use
  upper case letters to denote random variables and $P$ to denote
  probability distribution functions.  A random variable used as
  subscript on $P$ specifies that we mean the distribution of that
  random variable.  We can give $P$ an argument to specify the value
  of the distribution function at that value, \eg $P_X(3)$ is the
  probability that the random variable $X$ has the value 3 and
  $P_X(x)$ is the probability that the random variable $X$ has the
  value $x$.  We usually drop subscripts on $P$ when the context or
  argument resolves ambiguity as to which probability function we
  mean.} as $P_{\ts{S}{1}{T}}\left( \ts{s}{1}{T} \right)$.  For
example we calculate the probability that a sequence of 4 states has
the values $\ts{s}{1}{4} = (u,v,w,v)$, (\ie, $\ti{s}{1} = u$,
$\ti{s}{2} = v$, $\ti{s}{3} = w$, and $\ti{s}{4} = v$) as follows:

\begin{align}
  \label{eq:intro_hmm1}
  P(\ts{s}{1}{4}) &= P(\ti{s}{1}) \prod_{\tau=2}^4 %
                     P(\ti{s}{\tau}|\ts{s}{1}{\tau-1})\\
  \label{eq:intro_hmm2}
                  &= P(\ti{s}{1}) \prod_{\tau=2}^4 %
                     P(\ti{s}{\tau}|\ti{s}{\tau-1}) \\
  \label{eq:intro_hmm3}
  P(u,v,w,v)      &= P(v|u,v,w) \cdot P(w|u,v) \cdot  P(v|u) \cdot P(u)\\
  \label{eq:intro_hmm4}
        &= P(v|w) \cdot P(w|v) \cdot  P(v|u) \cdot P(u)\\
  \label{eq:intro_hmm5}
        &= \frac{1}{2} \cdot \frac{1}{2} \cdot 1 %
                     \cdot \frac{1}{3} = \frac{1}{12}.
\end{align}

Applying \index*{Bayes rule} $\left( P_{A|B} P_B = P_{A,B} \right)$
recursively, yields Eqn.~\eqref{eq:intro_hmm1} and the special case,
Eqn.~\eqref{eq:intro_hmm3}.  Equations \eqref{eq:intro_hmm2} and
\eqref{eq:intro_hmm4} follow from Eqns.~\eqref{eq:intro_hmm1} and
\eqref{eq:intro_hmm3} respectively by the %
\emph{\index*{Markov assumption}}, Eqn.~\eqref{eq:MarkovProcess}.
which says that in determining the probability of the $t^\text{th}$
state given any sequence of previous states only the $(t-1)^\text{th}$
state is relevant. %
%%%


A common exercise is to find a \emph{\index*{stationary}} probability
distribution, \ie, given a transition matrix $T$ find the probability
vector $V$ (nonnegative entries that sum to one) that satisfies
\begin{equation}
  \label{eq:statCond}
  VT = V.  
\end{equation}
If \eqref{eq:statCond} holds for $V = P_{\ti{S}{1}}$, then
\begin{equation*}
  P_{\ti{S}{2}}  = P_{\ti{S}{1}}T = P_{\ti{S}{1}} = P_{\ti{S}{t}}
  \forall t,
\end{equation*}
and in fact all probabilities are independent of shifts in time, \ie,
\begin{equation*}
  P_{\ts{S}{1}{t}} = P_{\ts{S}{1+\tau}{t+\tau}} \forall (t,\tau),
\end{equation*}
which is the definition of a stationary process.  Quick calculations
verify that the initial probability and transition matrix in
\eqref{eq:InitialProbabilites} and \eqref{eq:TransitionMatrix} do not
satisfy \eqref{eq:statCond} but that the distribution $V =
\begin{bmatrix} \frac{1}{7}, & \frac{4}{7}, & \frac{2}{7}
\end{bmatrix}$ does.  Although our example is not a stationary
stochastic process, it relaxes towards such a process in the sense
that
\begin{equation*}
  \lim_{t \rightarrow \infty} P_{\ti{S}{t}} =  \lim_{t \rightarrow
    \infty} P_{\ti{S}{1}} T^t = \begin{bmatrix} \frac{1}{7}, & \frac{4}{7}, & \frac{2}{7}
\end{bmatrix}.
\end{equation*}

\begin{figure}[htbp]
  \centering{\plotsize%
    \input{Markov_mm.pdf_t}}
  \caption[\comment{fig:Markov }A Markov model.]{A Markov model}
  \label{fig:mm}
\end{figure}

The important points about a Markov model also apply to hidden
Markov models, namely that:
\begin{itemize}
\item The model determines the probability of arbitrary sequences of
  observations,
\item and the assumptions about independence and time invariance
  permit specification of the model by a small number of parameters.
\end{itemize}

% karlheg does not like the "," after the display-math array.  The
% punctuation here is not in any style manual, but karlheg thinks it
% seems right:
%
Now suppose, as sketched in Fig.~\ref{fig:dhmm}, that when the system
arrives in a state that rather than observing the state directly, one
observes a random variable that depends on the state.  The matrix that
specifies the random map from states to observations, \ie%
:
\begin{equation*}
  \begin{array}{cr|ccc}
      &      &\multicolumn{3}{c}{Y} \\
      \multicolumn{2}{r|}{P(y|s)} & d & e           & f \\ %
      \hline%
      & u      & 1 & 0           & 0 \\
      &        & \vspace{-1 em} \\
    S & v      & 0 & \frac{1}{3} & \frac{2}{3} \\
      &        & \vspace{-1 em} \\
      & w      & 0 & \frac{2}{3} & \frac{1}{3}
  \end{array}%,
\end{equation*}
combined with the distribution of initial states
\eqref{eq:InitialProbabilites} and transition matrix
\eqref{eq:TransitionMatrix} specifies this hidden Markov model.  The
notion is that the underlying Markov process chugs along unaware of
the observations, and that when the process arrives at each successive
state $\ti{s}{t}$, an observation $\ti{y}{t}$ is produced in a fashion
that depends only on the state $\ti{s}{t}$.

\begin{figure}[htbp]
  \centering{\plotsize%
    \input{Markov_dhmm.pdf_t}}
  \caption[\comment{fig:Markov_dhmm }A hidden Markov model.]{A hidden Markov model}
  \label{fig:dhmm}
\end{figure}

Let us calculate the probability that a sequence of four observations
from this process would have the values $\ts{y}{1}{4} = (d,e,f,e)$.
As an intermediate step we calculate $P(\ts{y}{1}{4},\ts{s}{1}{4})$
for the given observation sequence and all possible state sequences.
Then we add to obtain
\begin{equation}
  \label{eq:dhmm_sum}
  \sum_{\ts{s}{1}{4}} P(\ts{y}{1}{4},\ts{s}{1}{4}) = P(\ts{y}{1}{4}).
\end{equation}
It is convenient that the only state sequences that could have
produced the observation sequence are $(u,v,v,v)$, $(u,v,v,w)$, and
$(u,v,w,v)$.  For any other state sequence
$P(\ts{y}{1}{4},\ts{s}{1}{4}) = 0$.

\begin{subequations}
\label{eq:pcalc}
 \begin{align}
  \ts{s}{1}{4} & & P(\ts{s}{1}{4}) %
                & & P(\ts{y}{1}{4}|\ts{s}{1}{4}) %
                 & & P(\ts{y}{1}{4},\ts{s}{1}{4})\nonumber\\[1ex]
  \hline\nonumber\\[-2ex]
  uvvv         & & \frac{1}{3} \cdot 1 \cdot \frac{1}{2} \cdot \frac{1}{2} %
                & & 1 \cdot \frac{1}{3} \cdot \frac{2}{3} \cdot \frac{1}{3} %
                 & & \frac{2}{324}\\[1ex]
%%%
  uvvw         & & \frac{1}{3} \cdot 1 \cdot \frac{1}{2} \cdot \frac{1}{2} % 
                & & 1 \cdot \frac{1}{3} \cdot \frac{2}{3} \cdot \frac{2}{3} %
                 & & \frac{4}{324}\\[1ex]
%%%
  uvwv         & & \frac{1}{3} \cdot 1 \cdot \frac{1}{2} \cdot \frac{1}{2} %
                & & 1 \cdot \frac{1}{3} \cdot \frac{1}{3} \cdot \frac{1}{3} %
                 & & \frac{1}{324}
 \end{align}
\end{subequations}

Adding the fractions in the right hand column yields %
\begin{equation*}
  P(d,e,f,e) = \frac{7}{324}.
\end{equation*}

Now examine this calculation more carefully beginning with a statement
of the model assumptions.
\begin{description}
\item[The state process is Markov:] Given the current state, the
  probability of the next state is independent of earlier states and
  observations, \ie,
  \begin{equation}
    \label{eq:assume_markov}
    P_{\ti{S}{t+1}|\ts{S}{1}{t},\ts{Y}{1}{t}} = %
        P_{\ti{S}{t+1}|\ti{S}{t}}.
  \end{equation}
\item[The observations are conditionally independent given the states:]
  Given the current state, the probability of the current observation is
  independent of states and observations at all earlier times, \ie,
  \begin{equation}
    \label{eq:assume_output}
    P_{\ti{Y}{t}|\ts{S}{1}{t},\ts{Y}{1}{t-1}} = %
        P_{\ti{Y}{t}|\ti{S}{t}}.
  \end{equation}
\end{description}
Though the assumptions appear asymmetric in time, they are
not\footnote{We often use the following facts about independence
  relations:
  \begin{align}
    P(A|B,C) = P(A|B)   &\iff     P(A,C|B) = P(A|B) \cdot P(C|B)\nonumber\\
    \label{eq:markov-symmetry}%
                        &\iff     P(C|A,B) = P(C|B)\\
    P(A|B,C,D) = P(A|B) &\iff     P(A,C,D|B) = P(A|B) \cdot P(C,D|B)\nonumber\\
                        &\implies P(A,C|B) = P(A|B) \cdot P(C|B)\nonumber\\
    \label{eq:independence-groups}%
                        &\iff     P(A|B,C) = P(A|B).
  \end{align}
  The first chain of implications, \eqref{eq:markov-symmetry}, says
  that if a process is Markov with time going forward, then it is also
  Markov with time going backwards.  The second chain,
  \eqref{eq:independence-groups}, says that if $A$ is conditionally
  independent of $C$ and $D$ given $B$, then $A$ is conditionally
  independent of $C$ alone given $B$.  By symmetry, $A$ is also
  conditionally independent of $D$ given $B$.}.  From the assumptions,
one can derive that
\begin{description}
\item[The joint process is Markov:] 
  \begin{equation*}
    P_{\ts{Y}{t+1}{T},\ts{S}{t+1}{T}|\ts{Y}{1}{t},\ts{S}{1}{t}}
      = P_{\ts{Y}{t+1}{T},\ts{S}{t+1}{T}|\ti{Y}{t},\ti{S}{t}}
  \end{equation*}
\item[Given $\bm{\ti{S}{t}}$, $\bm{\ti{Y}{t}}$ is conditionally independent of everything else:]
  \begin{equation*}
    P_{\ti{Y}{t} |\ts{Y}{1}{t-1},\ts{Y}{t+1}{T}, \ts{S}{1}{T}} = P_{\ti{Y}{t} |\ti{S}{t}}
  \end{equation*}
\end{description}
Equations \eqref{eq:assume_markov} and \eqref{eq:assume_output} are
assumptions about \emph{conditional independence} relations.  Figure
\ref{fig:dhmm_net} represents these relations as a %
\emph{\index*{Bayes net}}\cite{Pearl91a}.

\begin{figure}[htbp]
  \centering{\plotsize%
    \input{Markov_dhmm_net.pdf_t}}
  \caption[\comment{fig:dhmm-net }Bayes net schematic for a hidden Markov model.]%
  {Bayes net schematic for a hidden Markov model.  The drawn edges
    indicate the dependence and independence relations: Given
    $\ti{S}{t}$, $\ti{Y}{t}$ is conditionally independent of
    everything else, and given $\ti{S}{t-1}$, $\ti{S}{t+1}$, and
    $\ti{Y}{t}$, $\ti{S}{t}$ is conditionally independent of
    everything else.}
  \label{fig:dhmm_net}
\end{figure}

Bayes rule and the assumptions justify\marginpar{FixMe: $P_{X}$ not
  $P(X)$?}
\begin{align}
  P(\ts{Y}{1}{T},\ts{S}{1}{T}) &= P(\ts{S}{1}{T}) \, %
                                   P(\ts{Y}{1}{T}|\ts{S}{1}{T}),\notag\\
  \label{eq:sseqprob}%
  P(\ts{S}{1}{T})              &= P(\ti{S}{1}) %
                                   \prod_{t=2}^T P(\ti{S}{t}|\ti{S}{t-1})\\
  \label{eq:condyseqprob}%
  P(\ts{Y}{1}{T}|\ts{S}{1}{T}) &= \prod_{t=1}^T P(\ti{Y}{t}|\ti{S}{t})\\
  \intertext{and we conclude}
  P(\ts{Y}{1}{T},\ts{S}{1}{T}) &= P(\ti{S}{1}) %
                                   \prod_{t=2}^T P(\ti{S}{t}|\ti{S}{t-1})\, %
                                   \prod_{t=1}^T P(\ti{Y}{t}|\ti{S}{t}).\notag
\end{align}

Since the state $u$ produces the observation $d$ exclusively and no
other state can produce $d$, the observation sequence $(d,f,e,f)$ is
only possible if the state sequence begins with $u$ and does not
return to $u$.  That constraint reduces the number of possible state
sequences to eight.  The impossibility of state $w$ following itself,
further constrains the possible state sequences to the three listed in
the calculations of Eqn.~\eqref{eq:pcalc}.  One can verify the values
for $P(\ts{s}{1}{T})$ and $P(\ts{y}{1}{T}|\ts{s}{1}{T})$ in those
calculations by applying Eqns.~\eqref{eq:sseqprob} and
\eqref{eq:condyseqprob}.

The calculation of $P(\ts{y}{1}{4})$ in Eqn.~\eqref{eq:pcalc} is easy
because, of the $3^4 = 81$ conceivable state sequences, only three are
consistent with the observations and model structure.  In general
however, if there are $N_S$ states and an observation sequence with
length $T$, then implementing
\begin{equation*}
  P(\ts{y}{1}{T}) = \sum_{\ts{s}{1}{T}} P\left(\ts{s}{1}{T}
    ,\ts{y}{1}{T} \right)
\end{equation*}
naively requires order $\left(N_S\right)^T$ calculations.  If $N_S$
and $T$ are as large as one hundred, $\left(N_S\right)^T$ is too many
calculations for any conceivable computer.

There is a family of algorithms whose complexities are linear in the
length $T$ that make it possible to use HMMs with interestingly long
time series.  The details of these algorithms constitute Chapter
\ref{chap:algorithms}; here we only list their names and objectives.
In these descriptions, we denote by $\parameters$ the vector of
parameters that define an HMM, namely the state transition
probabilities, the initial state probabilities, and the conditional
observation probabilities,
\begin{equation*}
   \parameters \equiv \left\{
     \begin{aligned}
       &\left\{\; P_{\ti{S}{t+1}|\ti{S}{t}} \left(s'|s
       \right)~\forall s,s' \;\right\},\\
       & \left\{\; P_{\ti{S}{1}} \left(s \right)~\forall s
       \;\right\},\\
       &  \left\{\; P_{\ti{Y}{t}|\ti{S}{t}} \left(y_i|s' \right)
       ~ \forall y_i,s' \;\right\}
     \end{aligned}
   \right\}.
\end{equation*}
\begin{description}
\item[The Viterbi Algorithm:] \index{Viterbi algorithm} Given a model
  $\parameters$ and a sequence of observations $\ts{y}{1}{T}$, the
  Viterbi algorithm finds the most probable state sequence $\ts{\hat
    s}{1}{T}$, \ie,
  \begin{equation}
    \label{eq:intro-viterbi}
    \ts{\hat s}{1}{T} = \argmax_{\ts{s}{1}{T}} P
    \left(\ts{s}{1}{T}|\ts{y}{1}{T},\parameters\right).
  \end{equation}
\item[The Baum-Welch Algorithm:] \index{Baum-Welch algorithm}
  \index{forward backward algorithm|see{Baum-Welch algorithm}}(Often
  called the \emph{Forward Backward Algorithm}) Given a sequence of
  observations $\ts{y}{1}{T}$ and an initial set of model parameters
  $\parameters_0$, a single pass of the Baum-Welch algorithm
  calculates a new set of parameters $\parameters_1$ that has higher
  likelihood
  \begin{equation}
    \label{eq:intro-fba}
    P\left( \ts{y}{1}{T}|\parameters_1 \right) \geq
    P\left( \ts{y}{1}{T}|\parameters_0 \right).
  \end{equation}
  Equality can only occur at critical points of the likelihood
  function (where $\partial_\parameters P\left(
    \ts{y}{1}{T}|\parameters \right) = 0$).  In generic cases,
running many iterations of the Baum-Welch algorithm yields a sequence
$\ts{\parameters}{0}{n}$ that approaches a local maximum of the
likelihood.
\item[The Forward Algorithm:] For each time step $t$ and each state
  $s$, the \index*{forward algorithm} calculates the conditional
  probability of being in state $s$ at time $t$ given all of the
  observations up to that time, \ie,
  $P_{\ti{S}{t}|\ts{Y}{1}{t},\parameters} \left(s|\ts{y}{1}{t},
    \parameters \right)$. It also calculates
  $P\left(\ti{y}{t}|\ts{y}{1}{t-1}, \parameters \right)$, the
  conditional probability of each observation given previous
  observations.  Using these terms it calculates the probability of
  the entire data sequence given the model,
  \begin{equation*}
    P \left(\ts{y}{1}{T}|\parameters \right) =  P \left(\ti{y}{1}
      |\parameters \right) \cdot  \prod_{t=2}^T P
    \left(\ti{y}{t}|\ts{y}{1}{t-1}, \parameters \right).
  \end{equation*}
  The forward algorithm is the first phase of the Baum-Welch algorithm.
\end{description}

\subsection{Example: Quantized Lorenz Time Series}
\label{sec:QuantizedLorenz}

%\ToDo{Train on one set and decode another}

To illustrate these algorithms we have applied them to data that we
synthesized by numerically integrating the Lorenz system
(Eqn.~\eqref{eq:Lorenz} with parameter values $r=28$, $s=10$, and
$b=\frac{8}{3}$) and recording 40,000 vectors $x(\tau)$ with a
sampling interval $\Delta \tau = 0.15$.  Then we produced a sequence
of integer valued observations $y_1^{40,000}$ by dividing the $x_1$
values by 10 adding 2 and keeping the integer part.\marginpar{FixMe:
  May need a generating partition in Chap.~5}  The result is
that for each integer $1\leq t \leq 40,000$, $\ti{x_1}{t\cdot0.15}$
yields $\ti{y}{t} \in \left\{0,1,2,3\right\}$.
Figure~\ref{fig:TSintro} depicts the first few observations.

\begin{figure}[htbp]
  \centering{\plotsize%
    \includegraphics[width=1.0\textwidth]{TSintro.pdf}}
  \caption[\comment{fig:TSintro }Generating the observations
  $\ts{y}{1}{40}$.]%
  {Generating the observations $\ts{y}{1}{40}$.  The curve in the
    upper plot depicts the first component $\ti{x_1}{\tau}$ of an
    orbit of the Lorenz system (Eqn.~\ref{eq:Lorenz}), and the points
    marked with red dots indicate the values sampled with an interval
    $\Delta \tau = 0.15$.  The points in the lower plot are the
    quantized values
    $\ti{y}{t} \equiv \bceil{\frac{\ti{x_1}{t \cdot \Delta \tau}}{10}
      + 2}$, where $\ceil{u}$ is the least integer greater than or
    equal to $u$.  }
  \label{fig:TSintro} 
\end{figure}
 
We randomly generated an HMM with twelve hidden states\footnote{We
  chose the number of hidden state to be twelve capriciously so that
  we could organize Fig.~\ref{fig:Statesintro} on a $4\times 4$ grid.}
and 4 possible observations, and then used 1,000 iterations of the
Baum-Welch algorithm to select a set of parameters $\hat \parameters$
with high likelihood for the data.  Finally we used the Viterbi
algorithm to find the most likely state sequence
 \begin{equation*}
   {\ts{\hat s}{1}{T}} = \argmax_{\ts{s}{1}{T}}P
   \left(\ts{s}{1}{T}|\ts{y}{1}{T}, \hat \parameters \right).
 \end{equation*}
 
 Although the plot of \emph{decoded} \index{decoded state sequence}
 state values in Fig.~\ref{fig:STSintro} is not very enlightening, we
 can illustrate that there is a relationship between the learned
 decoded states and the original Lorenz system states by going back to
 the original data.  For each state $s$, we identify the set of
 integer times $t$ such that the decoded state is $s$, \ie, $\left\{ t
   :\ti{\hat s}{t} = s \right\}$, and then we find what the Lorenz
 system state was at each of these times and plot that set of points.
 In the upper right box of Fig.~\ref{fig:Statesintro} we have plotted
 points in the original state space that correspond to hidden state
 number one, \ie, the set of pairs $\left\{
   (\ti{x_1}{t\cdot\Delta\tau},\ti{x_3}{t\cdot\Delta\tau}) :\ti{\hat
     s}{t} = 1 \right\}$.  In searching for model parameters that give
 the observed data high likelihood, the Baum-Welch algorithm
 ``discovers'' a discrete hidden state structure, and
 Fig.~\ref{fig:Statesintro} shows that the discrete hidden state
 structure is an approximation of the continuous state space that
 generated the data.
 %%
 \begin{figure}[htbp]
   \centering{\plotsize%
     \includegraphics[width=1.0\textwidth]{STSintro.pdf}}
   \caption[\comment{fig:STSintro }A plot of a state sequence found by Viterbi decoding.]%
   {A plot of the state sequence found by Viterbi decoding a quantized
     time series from the Lorenz system.  Here the number of the
     decoded state $\ti{s}{t}$ is plotted against time $t$.  Although
     it is hard to see any structure in the plot because the numbers
     assigned to the states are not significant,
     Fig.~\ref{fig:Statesintro} illustrates that the decoded states
     are closely related to positions in the generating state space.}
   \label{fig:STSintro}
 \end{figure}
%%%\afterpage{\clearpage}%% Print this right here or let it float to end of chapter?
%%%
%%% ToDo: fig:Statesintro -- Do these color figures need to be in
%%% ToDo: their own "Chapter"-like section so that they can be printed
%%% ToDo: together on a set of color plates?
%%%
%%% This is a large color figure on a page by itself. (butterfly)
%%%
 \begin{figure}[p]
   \centering{\plotsize%
     \includegraphics[width=1.0\textwidth]{Statesintro}}%
   \caption[\comment{fig:Statesintro }Relationship between states of HMM and Lorenz system.]%
   {The relationship between the hidden states of an HMM and the
     original coordinates of the Lorenz system.}
   \label{fig:Statesintro}
 \end{figure}
%%%\afterpage{\clearpage}%% Print this right here or let it float to end of chapter?
\ToDo{Transition diagram for back cover}


\subsection{Example: Hidden States as Parts of Speech}
\label{sec:POSpeech}

Hidden Markov models were developed for speech and text processing,
and for unscientific audiences, we find the application to language
modeling the easiest way to motivate HMMs.  Consider for example the
sentence, ``The dog ate a biscuit.'' and its reduction to a sequence
of parts of speech: \emph{article noun verb article noun}.  By
choosing different particular articles nouns and verbs and placing
them in the order specified by the sequence of parts of speech, we can
produce many other sentences such as, ``An equation describes the
dynamics.''  The parts of speech are like hidden states and the
particular words are like observations.

Rather than using a dictionary or our own knowledge to build a model
of language, here we describe the experiment of applying the
Baum-Welch algorithm to some sample text to create an HMM.  We hope
that the experiment will \emph{discover} parts of speech.  We fetched
the fourth edition of \emph{A Book of Prefaces} by H.~L.~Mencken from
www.gutenberg.org, fit an HMM with the Baum-Welch algorithm, and
decoded a state sequence with the Viterbi algorithm.  We chose a book
of essays rather than a novel because we expected that it would not
have as much dialog.  We feared that different speakers in a novel
would require different models.

The experiment consisted of the following steps:
\begin{description}
\item[Parse the text:] We reduced the text to a sequence of tokens.
  Each token is a word, a number, or a special character such as
  punctuation.  We retained distinctions between lower and upper case.
  The length of the resulting sequence was 68,857 tokens, \ie,
  $\ts{w}{1}{T}$ with $T=68,857$.
\item[Identify unique tokens:] There were 9,772 unique tokens of which
  2,759 appear in the text more than twice.
\item[Create a map from tokens to rank:] We sorted the tokens by the
  frequency of their occurrence so that for the most frequent token
  $w'$, $R(w')=1$ and for the most infrequent token, $\bar w$, $R(\bar
  w)=9,772$.
\item[Map tokens to integers:] We created a sequence of integers
  $\ts{y}{1}{T}$ where $\ti{y}{t} \in \{0,\ldots,2,760\} \forall
  t$.  If the token $\ti{w}{t}$ appeared in the text less than three
  times, we set $\ti{y}{t}=2,760$.  Otherwise, we set  $\ti{y}{t}$ to
  $R(\ti{w}{t})$.
\item[Train an HMM:] Starting from a fifteen state model with random
  parameters, we used 100 iterations of the Baum-Welch algorithm to
  obtain a trained model.
\item[Decode a sequence of states:] By applying the Viterbi algorithm
  to $\ts{y}{1}{T}$ we obtained $\ts{s}{1}{T}$ where $\ti{s}{t}\in
  \{1,\ldots,12\}$.
\item[Print the most frequent words for each state:] For each state
  $s$, count the number of times each integer $y$ occurs, \ie, $c(y,s)
  = \sum_{t:\ti{s}{t}=s} \delta(y,\ti{y}{t})$.  Then print the words
  corresponding the ten most frequently occurring integers (excluding
  the special value $y=2,760$).
\end{description}
The results appear in Table~\ref{tab:POS}.

\begin{table}[htb]
  \caption[Words most frequently associated with each state.]%
  {Words most frequently associated with each state.  While we have no
    interpretation for three of the states, some of the following
    interpretations of the other states are strikingly successful.}
  \begin{center}{\plotsize%
      \fbox{%
      \begin{tabular}[t]{r@{\hspace{0.28em}}p{19em}}
        1  -- & \rule{0pt}{2.5ex}Adjectives \\
        2  -- & \rule{0pt}{2.5ex}Punctuation and other tokens that appear at the end of phrases \\
        6  -- & \rule{0pt}{2.5ex}Capitalized articles and other tokens that appear at the beginning of phrases \\
        7  -- & \rule{0pt}{2.5ex}Objective pronouns and other words with similar functions \\
        8  -- & \rule{0pt}{2.5ex}Nouns
      \end{tabular}\qquad%
      \begin{tabular}[t]{r@{\hspace{0.28em}}p{10em}}
        9  -- & \rule{0pt}{2.5ex}Nouns \\
        10 -- & \rule{0pt}{2.5ex}Helping verbs \\
        11 -- & \rule{0pt}{2.5ex}Nominative pronouns \\
        12 -- & \rule{0pt}{2.5ex}Articles \\
        13 -- & \rule{0pt}{2.5ex}Conjunctions \\
        14 -- & \rule{0pt}{2.5ex}Prepositions \\
        15 -- & \rule{0pt}{2.5ex}Relative pronouns
      \end{tabular}}\\[2.0ex]
      \begin{tabular}{|@{\hspace{0.10em}}r@{\hspace{0.40em}}|*{10}{@{\hspace{0.28em}}l@{\hspace{0.28em}}}|}
        \hline
        \rule{0pt}{2.0ex} 1  & other & first  & American & same   & new   & own        & whole    & great  & Puritan & very    \\
        \rule{0pt}{2.0ex} 2  & .     & ,      & ;        & that   & ]     & ?          & "        & and    & -       & !       \\
        \rule{0pt}{2.0ex} 3  & more  & be     & out      & not    & X     & only       & on       & much   & up      & make    \\
        \rule{0pt}{2.0ex} 4  & "     & indeed & Carrie   & York   & Sec   & Titan      & Gerhardt & Here   & )       & W       \\
        \rule{0pt}{2.0ex} 5  & to    & -      & not      & no     & as    & so         & a        & be     & and     & been    \\
        \rule{0pt}{2.0ex} 6  & "     & The    & [        & and    & A     & In         & New      & His    & in      & ,       \\
        \rule{0pt}{2.0ex} 7  & it    & all    & them     & him    & life  & us         & which    & course & what    & Dreiser \\
        \rule{0pt}{2.0ex} 8  & man   & hand   & book     & artist & men   & literature & books    & end    & story   & work    \\
        \rule{0pt}{2.0ex} 9  & book  & sort   & work     & sense  & man   & way        & thing    & story  & books   & out     \\
        \rule{0pt}{2.0ex} 10 & is    & was    & are      & has    & have  & had        & were     & would  & may     & could   \\
        \rule{0pt}{2.0ex} 11 & he    & it     & "        & He     & there & they       & It       & But    & I       & we      \\
        \rule{0pt}{2.0ex} 12 & the   & a      & his      & its    & an    & their      & that     & this   & any     & such    \\
        \rule{0pt}{2.0ex} 13 & and   & but    & as       & that   & or    & "          & even     & not    & so      & if      \\
        \rule{0pt}{2.0ex} 14 & of    & in     & and      & to     & for   & with       & by       & as     & at      & from    \\
        \rule{0pt}{2.0ex} 15 & ,     & --     & that     & ;      & "     & which      & who      & and    & as      & (       \\[0.5ex]
        \hline
      \end{tabular}
    }\end{center}
  \label{tab:POS}
\end{table}

\subsection{Remarks}
\label{sec:DHMMRemarks}

One might imagine that HMMs are simply higher order Markov processes.
For example, consider the suggestion that the states depicted in
Fig.~\ref{fig:Statesintro} correspond to sequential pairs of
observations and that the model is a second order Markov model that is
characterized by $P_{\ti{Y}{t+1}|\ti{Y}{t},\ti{Y}{t-1}}$, the set of
observation probabilities conditioned on the \emph{two} previous
observations.  Although the number of unique sequential pairs
$\ts{y}{t}{t+1}$ that occur in our data is in fact twelve, the fact
that some of the states in Fig.~\ref{fig:Statesintro} straddle the
quantization boundaries at $x = -10$ and $x=10$ belies the suggestion.
In general, the class of HMMs is more powerful than the class of
simple Markov models in the sense that the former includes the later
but not vice versa.

Let us emphasize the following points about discrete HMMs:
\begin{enumerate}
\item Although the hidden process is first order Markov, the observation
process may not be a Markov process (of any order). \label{noX}
\item Any Markov model of any order can be represented by an HMM.
\item Even if the functions governing the dynamics and observations of
  a continuous state space system are nonlinear, a discrete HMM can
  approximate the system arbitrarily well\footnote{By using a discrete
    HMM to approximate dynamics governed by a continuous function one
    sacrifices the opportunity to exploit continuity.  That sacrifice
    will degrade model performance in many applications.} by
  using large numbers of states $N_S$ and possible observation values
  $N_Y$.
\item For estimating model parameters, larger numbers of training data
  are required as $N_S$ and $N_Y$ are increased. \label{point:large}
\end{enumerate}

As an illustration of Point \ref{noX},\ consider the process depicted in
Fig.~\ref{fig:nonmm}, which produces observation strings with runs of about
seven $a$'s interspersed with occasional $b$'s and $c$'s.  In the
observation stream, the $b$'s and $c$'s alternate no matter how many $a$'s
fall in between.  Such behavior can not be captured by a simple Markov
process of any order.

\begin{figure}[htbp]
  \centering{\plotsize%
    \input{nonmm.pdf_t}}
  \caption[\comment{fig:nonmm }An HMM that cannot be represented by a Markov model.]%
  {An HMM that cannot be represented by a Markov model of any order.
    Consider the string of observations ``$b,a,a,\ldots,a,a,a$''.
    Since the previous non-''$a$'' observation was ``$b$'' and the
    model will not produce another ``$b$'' before it produces a
    ``$c$'', the next observation can be either a ``$c$'' or another
    ``$a$'', but not a ``$b$''.  Because there is no limit on the
    number of consecutive ``$a$'s'' that can appear, there is no limit
    on how far back in the observation sequence you might have to look
    to know the probabilities of the next observation.}
  \label{fig:nonmm}
\end{figure}
%%%\afterpage{\clearpage}%% Print this right here or let it float to end of chapter?

The possibility of long term memory makes state space models, \eg,
HMMs, more powerful than Markov models.  That observation suggests
that if there is noise, then the \emph{\index*{delay vector} reconstructions}
described in the chaos
literature\cite{Packard80,Takens81,Fraser86,Sauer91} are suboptimal
because they discard information from earlier observations that could
be used to more accurately specify the state.

%%% Local Variables:
%%% TeX-master: "main"
%%% eval: (load-file "hmmkeys.el")
%%% End:

\chapter{Basic Algorithms}
\label{chap:algorithms}

In Section~\ref{sec:intro_hmm} we mentioned the Viterbi algorithm for
finding the state sequence that has the highest probability given an
observation sequence, the forward algorithm for calculating the
probability of an observation sequence, and the Baum-Welch algorithm
for finding a parameter vector that at least locally maximizes the
likelihood given an observation sequence.  This chapter explains the
details of those algorithms.  Much of the literature on the algorithms
and much of the available computer code uses Ferguson's
\cite{Ferguson80} notation.  In particular, Rabiner's \cite{Rabiner89}
widely cited article follows Ferguson's notation.  Let us begin by
establishing our notation for the model parameters.
\setlength{\nomlabelwidth}{2.7cm}%
\begin{symbdescription}
\item[$\bm{P_{\ti{S}{t+1}|\ti{S}{t}}(s|{\tilde s})}$] The probability
  that at time $t+1$ the system will be in state $s$ given that at
  time $t$ it was in state ${\tilde s}$.  Notice that the parameter is
  independent of time $t$.  Ferguson called these parameters \emph{the
    $A$ matrix} with
  $a_{i,j} = P_{\ti{S}{t+1}|\ti{S}{t}}(s_j|s_i)$.
\item[$\bm{P_{\ti{S}{1}}(s)}$] The probability that the system will
  start in state $s$ at time 1.  Ferguson used $a$ to represent this
  vector of probabilities with $a_i = P_{\ti{S}{1}} \left(s_i \right)$.
\item[$\bm{P_{\ti{Y}{t}|\ti{S}{t}}(y|s)}$] The probability
  that the observation value is $y$ given that the system is in
  state $s$.  Ferguson used $b$ to represent this with
  $b_j(k) = P_{\ti{Y}{t}|\ti{S}{t}} \left(y_k|s_j \right)$.
\item[$\bm{\parameters}$] The entire collection of parameters.  For
  example, iteration of the Baum-Welch algorithm produces a sequence
  of parameter vectors $\ts{\parameters}{0}{N}$ with
  $P\left(\ts{y}{0}{T}|\ti{\parameters}{n+1}\right) \geq
  P\left(\ts{y}{0}{T}|\ti{\parameters}{n}\right)\, : 1 < n < N$.
  Instead of $\parameters$, Ferguson used $\lambda$ to denote the
  entire collection of parameters.
  \nomenclature[gh]{$\bm{\theta}$}{The entire collection of
    parameters that defines a model.}
\end{symbdescription}

\section{The Forward Algorithm}
\label{sec:forward}
\index{forward algorithm|textbf}

Given $\parameters$, the forward algorithm calculates
$P(\ts{y}{0}{T}|\parameters)$, and several useful intermediate terms.
Figure~\ref{fig:forward} sketches the basic recursion's structure.
Since we are considering only one set of parameters, we will drop the
dependence of probabilities on $\parameters$ from the notation for the
remainder of this section.  In the example of Eqn.~\eqref{eq:pcalc} we
found that we could write the right hand terms of
\begin{equation*}
   P\left( \ts{y}{0}{T} \right) = \sum_{\ts{s}{0}{T}}
   P\left( \ts{y}{0}{T},\ts{s}{0}{T} \right)
\end{equation*}
easily from the model assumptions.  Unfortunately, the number of
possible sequences $\ts{s}{0}{T}$ is exponential in $T$, and it is
not feasible to do the sum for even modest lengths $T$.

\begin{sidewaysfigure}[htbp]
  %% The following definitions are invoked by the XFig forward.fig.
  \centering{\plotsize%
    %% Column a
    \def\colaa{$\alpha(s_1,t-1)$}%
    \def\colab{$\alpha(s_2,t-1)$}%
    \def\colac{$\alpha(s_3,t-1)$}%
    %% Column b
    \def\colba{$P \left(s_1|\ts{y}{0}{t} \right)$}%
    \def\colbb{$P \left(s_2|\ts{y}{0}{t} \right)$}%
    \def\colbc{$P \left(s_3|\ts{y}{0}{t} \right)$}%
    \def\sumeqforwardAthree{%
      \begin{minipage}[t]{1.8in}
        \raggedright%
        Weighted\\%
        sum of prior\\%
        $\alpha$'s\\%
        Eqn.~\eqref{eq:forwardA3}
      \end{minipage}}%
    %% Column c
    \def\colca{$P \left(s_1,\ti{y}{t}|\ts{y}{0}{t} \right)$}%
    \def\colcb{$P \left(s_2,\ti{y}{t}|\ts{y}{0}{t} \right)$}%
    \def\colcc{$P \left(s_3,\ti{y}{t}|\ts{y}{0}{t} \right)$}%
    \def\prdeqforwardBtwo{%
      \begin{minipage}[t]{2.3in}
        \raggedright%
        Multiply\\%
        observation\\%
        probability\\%
        Eqn.~\eqref{eq:forwardB2}
      \end{minipage}}%
    %% Column d
    \def\coldb{$P(\ti{y}{t}|\ts{y}{0}{t})$}%
    \def\prdeqforwardC{%
      \begin{minipage}[t]{1.8in}
        \raggedright%
        Add, to get\\%
        $\gamma(t)$, $P \left(\ti{y}{t}|\ts{y}{0}{t} \right)$\\%
        Eqn.~\eqref{eq:forwardC}
      \end{minipage}}%
    %% Column e
    \def\colea{$\alpha(s_1,t)$}%
    \def\coleb{$\alpha(s_2,t)$}%
    \def\colec{$\alpha(s_3,t)$}%
    \def\quoteqforwardD{%
      \begin{minipage}[t]{1.8in}
        \raggedright%
        Normalize\\%
        new $\alpha$'s\\%
        Eqn.~\eqref{eq:forwardD}
      \end{minipage}}%
    \input{forward.pdf_t}}
  \vspace{5 em}
  \caption[\comment{fig:forward }Dependency relations in the forward algorithm.]%
  {Dependency relations in the forward algorithm (See
    Eqns.~\eqref{eq:forwardA}-\eqref{eq:forwardD} in the text).}
  \label{fig:forward}
\end{sidewaysfigure}
\afterpage{\clearpage}%% Print this right here please.

The forward algorithm regroups the terms and produces the desired
result using order $T$ calculations.  For each time $t:\, 1 < t \leq
T$ we calculate $P(\ti{y}{t}|\ts{y}{0}{t})$, and in principle we can
write
\begin{equation*}
  P(\ts{y}{0}{T}) = P(\ti{y}{1}) \prod_{t=2}^T P(\ti{y}{t}|\ts{y}{0}{t}).
\end{equation*}
However the values of $P(\ti{y}{t}|\ts{y}{0}{t})$ are typically
small compared to 1, and the product of many such terms is too small
to be represented even in double precision.  Working with logarithms
avoids underflow:

\begin{equation}
  \mlabel{eq:logP}
  \log\left( P(\ts{y}{0}{T}) \right) = \log\left( P(\ti{y}{1})
 \right) + \sum_{t=2}^T \log\left( P(\ti{y}{t}|\ts{y}{0}{t})\right).
\end{equation}

%%%%%%%%%%%%%%%%%%%%%%%%%%%%%%%%%%%%%%%%%%%%%%%%%%%%%%%%%%%%%%%%%%%%
%%%
%%% #1 = time #2 = state
\newcommand{\alphax}[2]{%
  \colorbox{yellow}{%
    $P_{S(#1)|\ts{Y}{0}{#1}}\left(#2 |\ts{y}{0}{#1} \right)$%
  }%
}%
%%%
%%%
\newcommand{\prealpha}{%
  \colorbox{green}{%
    $P_{\ti{S}{t}|\ts{Y}{0}{t}} \left(s | \ts{y}{0}{t} \right)$%
  }%
}%
%%%
\newcommand{\gammax}{%
  \colorbox{cyan}{%
    $P(\ti{y}{t}|\ts{y}{0}{t})$%
  }%
}%
%%%
%%%
\newcommand{\pregamma}{%
  \colorbox{pink}{%
    $P_{\ti{S}{t},\ti{Y}{t}|\ts{Y}{0}{t}} \left(s,\ti{y}{t}|\ts{y}{0}{t}\right)$%
  }%
}%
%%%
%%%%%%%%%%%%%%%%%%%%%%%%%%%%%%%%%%%%%%%%%%%%%%%%%%%%%%%%%%%%%%%%%%%%

For each time step we do the four calculations specified in the
equations below\footnote{In Section~\ref{sec:Clarity} we present four
  lines of python code that implement these four calculations.}.  In
these equations we use color highlighting to emphasize repeated
instances of the same quantity.  The algorithm saves the following
intermediate results
%
\nomenclature[ga]{$\alpha(s,t)$}{The conditional probability that at time
  $t$ the system is in state $s$ given the observations up to the
  present,
  \begin{equation*}
    \alpha(s,t) \equiv  P_{S(t)"|\ts{Y}{0}{t}}\left(s "|\ts{y}{0}{t}
    \right).
  \end{equation*}
  Also called the forward updated distribution in the Kalman filter
  literature.}
%
\nomenclature[gc]{$\gamma(t)$}{The probability of the present observation
  given the history.}%, $P(y(t)|y_1^{t-1}$.}
\begin{align}
  \label{eq:alpha}
  \alpha(s,t) &\equiv \alphax{t}{s}\\
  \label{eq:gamma}
  \ti{\gamma}{t} &\equiv \gammax .
\end{align}
In words\footnote{In Ferguson's notation $\alpha(s,t) \equiv
  P_{\ti{S}{t},\ts{Y}{0}{t}} \left(s,\ts{y}{0}{t} \right)$.  Our
  notation differs from that by a factor of $P(\ts{y}{0}{t})$.
  Ferguson did not have notation for the term
  $P(\ti{y}{t}|\ts{y}{0}{t})$ which we call $\gamma$, but he did use
  $\gamma$ for quantities that we will denote by $w$ in
  Eqns.~\eqref{eq:wit} and \eqref{eq:wijt}.}, $\alpha(s,t)$ is the
conditional probability of being in state $s$ at time $t$ given all
of the observations up to time $t$ and $\ti{\gamma}{t}$ is the
conditional probability of the observation at time $t$ given all of
the previous observations.  To initialize the algorithm, one assigns
\begin{align*}
  \gamma(1) &= P_{\ti{Y}{1}}(\ti{y}{1}) =
              \sum_{s}P_{\ti{S}{1}}(s)P_{\ti{Y}{t}|\ti{S}{t}}(\ti{y}{1}|s)
  \\
  \alpha(s,1) &= P_{\ti{S}{1}|\ti{Y}{1}}(s|\ti{y}{1}) =
  \frac{P_{\ti{S}{1}}(s)P_{\ti{Y}{t}|\ti{S}{t}}(\ti{y}{1}|s)}
  {\gamma(1)}
~\forall s \in \states
\end{align*}
where the distributions $P_{\ti{S}{1}}$ and $P_{\ti{Y}{t}|\ti{S}{t}}$
are model parameters.  After initialization, the algorithm iterates in
a loop doing the following four calculations: 
\begin{description}
\item[Forecast the Distribution of States] Find for time $t$ and each
  possible state $s$ the conditional probability, given the
  previous observations, of the state being $s$ at time $t$:
  \begin{subequations}
    \label{eq:forwardA}
    \begin{align}
      \mlabel{eq:forwardA1}
      \prealpha & = \sum_{{\tilde s}\in\states}
      P_{\ti{S}{t},\ti{S}{t-1}|\ts{Y}{0}{t}} \left( s, {\tilde s}
      |\ts{y}{0}{t} \right)\\
      \mlabel{eq:forwardA2}
      & = \sum_{{\tilde s}\in\states} \left(
        P_{\ti{S}{t}|\ti{S}{t-1},\ts{Y}{0}{t}}
        \left( s|{\tilde s},\ts{y}{0}{t} \right) \right. \nonumber \\ &
      \left. \times \alphax{t-1}{{\tilde s}} \right) \\
      \mlabel{eq:forwardA3}
      & = \sum_{{\tilde s}\in\states} P_{\ti{S}{t}|\ti{S}{t-1}} \left( s|{\tilde s} \right)
      \cdot \alpha({\tilde s},t-1) .
    \end{align}
  \end{subequations}
  We justify the operations as follows:
  \begin{description}
  \item[\eqref{eq:forwardA1}] $P(a) = \sum_b P(a,b)$
  \item[\eqref{eq:forwardA2}] $P(a|b)\cdot P(b) = P(a,b)$
  \item[\eqref{eq:forwardA3}] Assumption of
    Eqn.~\eqref{eq:assume_markov}: The state process is Markov
  \end{description}
\item[Update the Joint Probability of States and the Current
  Observation] Find for time $t$ and each possible state $s$, the
  conditional probability, given the previous observations, of the
  state being $s$ and the observation being $\ti{y}{t}$ (the value
  actually observed):
  \begin{subequations}
    \label{eq:forwardB}
    \begin{align}
      \mlabel{eq:forwardB1}
      \pregamma &= P_{\ti{Y}{t} |
        \ti{S}{t},\ts{Y}{0}{t}}\left(\ti{y}{t} |s, \ts{y}{0}{t}
      \right) \nonumber \\& \times \prealpha\\
      \mlabel{eq:forwardB2}
      &= P_{\ti{Y}{t} | \ti{S}{t}}\left(\ti{y}{t} |s \right) \cdot
      \prealpha .
    \end{align}
  \end{subequations}
We justify the equations as follows:
\begin{description}
\item[\eqref{eq:forwardB1}]  $P(a|b)\cdot P(b) = P(a,b)$
\item[\eqref{eq:forwardB2}] Assumption of
  Eqn.~\eqref{eq:assume_output}: The observations are conditionally
  independent given the states
\end{description}
\item[Calculate the Conditional Probability of the Current Observation]
  Find for time $t$, the conditional probability, given the previous
  observations, of the observation being $\ti{y}{t}$ (the observation actually
  observed):
  \begin{align}
    \mlabel{eq:forwardC}
    \gammax &= \sum_{s \in \states} \pregamma .
  \end{align}
Equation~\eqref{eq:forwardC} is an application of $P(a) = \sum_b P(a,b)$.
\item[Normalize the Updated Distribution of States] For each possible
  state $s$, find the conditional probability of being in that state
  at time $t$ given all of the observations up to time $t$.  Note that
  this differs from the first calculation, \eqref{eq:forwardA}, in that
  the conditioning event includes $\ti{y}{t}$:%
  \begin{subequations}
    \label{eq:forwardD}
    \begin{align}
      \mlabel{eq:forwardD1}
      \alpha(s,t) &\equiv
      \alphax{t}{s} \\
      \mlabel{eq:forwardD2}
      &= \pregamma \div \gammax
    \end{align}
  \end{subequations}
  Equation \eqref{eq:forwardD2} is an application of Bayes rule, \ie,
  $P(a|b)\cdot P(b) = P(a,b)$.
\end{description}

\section{The Viterbi Algorithm}
\label{sec:viterbi}

For some applications, one must estimate the sequence of states based
on a sequence of observations.  The Viterbi algorithm %
\index{Viterbi algorithm|textbf}%
finds the \emph{best} sequence $\ts{\hat s}{1}{T}$ in the sense of
maximizing the probability $P\left( \ts{s}{0}{T}|\ts{y}{0}{T}
\right)$.  That is equivalent to maximizing $\log \left( P\left(
    \ts{y}{0}{T},\ts{s}{0}{T} \right) \right)$ because $P \left(
  \ts{y}{0}{T} \right)$ is simply a constant, and the $\log$ is
monotonic, ie,
\begin{align*}
  \ts{\hat s}{1}{T} &\equiv \argmax_{\ts{s}{0}{T}}
  P(\ts{s}{0}{T}|\ts{y}{0}{T})\\
  &= \argmax_{\ts{s}{0}{T}} \left( P(\ts{s}{0}{T}|\ts{y}{0}{T}) \cdot
    P(\ts{y}{0}{T}) \right)\\
  &= \argmax_{\ts{s}{0}{T}} \log \left( P(\ts{y}{0}{T},\ts{s}{0}{T})
  \right).
\end{align*}
As in the implementation of the forward algorithm, we use logs to
avoid numerical underflow.  If we define $\log \left( P\left(
    \ts{y}{0}{t}, \ts{s}{0}{t} \right) \right)$ as the
\emph{utility} (negative cost) of the state sequence $\ts{s}{0}{t}$
given the observation sequence $\ts{y}{0}{t}$, then the Viterbi
algorithm finds the maximum utility state sequence.

Initially one calculates
$\log\left(P_{\ti{Y}{1},\ti{S}{1}} \left(\ti{y}{1}, s \right) \right)$
for each state $s \in \states$.  Then for each successive time step
$t: 1 < t \leq T$ one considers each state and determines the best
predecessor for that state and the utility of the best state sequence
ending in that state.  The Viterbi algorithm and the forward algorithm
have similar structures (see Fig.~\ref{fig:forward} and
Fig.~\ref{fig:viterbiB}).  Roughly, the forward algorithm does a sum
over predecessor states, while the Viterbi algorithm finds a maximum
over predecessor states. We use the following notation and equations
to describe and justify the algorithm.

\subsubsection*{Notation:}
\begin{description}
\item[$\bm{\tilde s (t,s)}$ The best $\bm{\ts{s}{0}{t+1}}$ ending in
  $\bm{s}$] Of all length $t+1$ state sequences ending in $s$ at time
  $t$, we define $\tilde s(t,s)$ to be the sequence with the highest
  joint probability with the data, \ie,
  \begin{equation*}
    \tilde s(t,s) \equiv \argmax_{ \ts{s}{0}{t+1}:\ti{s}{t}=s}
    P\left(\ts{y}{0}{t+1}, \ts{s}{0}{t+1} \right).
  \end{equation*}
\item[$\bm{\nu(t,s)}$ The \emph{utility} of the best
  $\bm{\ts{s}{0}{t+1}}$ ending in $\bm{s}$] %
  \nomenclature[gn]{$\nu(t,s)$}{Used in discussing the Viterbi
    algorithm to denote the \emph{utility} of the best sequence ending
    in state $s$.
    \begin{equation*}
      \nu(t,s) \equiv \log\left( P\left(\ts{y}{0}{t+1}, \ts{\tilde s}{0}{t+1}(s)
        \right) \right)
    \end{equation*}} % end nomenclature
  This is simply the log of the joint
  probability of the data with the best state sequence defined above,
  \ie
  \begin{equation*}
    \nu(t,s) = \log\left( P\left(\ts{y}{0}{t+1}, \tilde s(t,s)
      \right) \right)
  \end{equation*}
\item[$\bm{s'(s,t)}$ The immediate predecessor of state $\bm{s}$ in
  $\bm{\tilde s (t,s)}$] In other words, given that the best sequence
  of length $t+1$ that ends in $s_d$ is
  \begin{equation*}
    \tilde s (t,s_d) = \left[ \ti{s}{0}=s_a, \ti{s}{1}=s_b, \ldots, \ti{s}{t-1}=s_c,
      \ti{s}{t}= s_d\right]
  \end{equation*}
  the best predecessor of $s$ is the
  penultimate, \ie at time $t-1$, element of that sequence, \ie $s_c$.
\end{description}

\subsubsection*{Equations:}
Equations~\eqref{eq:cat} and \eqref{eq:nuprop} summarize the Viterbi
algorithm.  When finally deciphered, one finds Eqn.~\eqref{eq:cat} is
the vacuous statement that the best state sequence ending in $s$ at
time $t$ consists of $s$ concatenated with the best sequence ending
in $s'$ at time $t-1$, where $s'$ is the best predecessor of $s$.
\begin{equation}
  \label{eq:cat}
  \tilde s(t,s) = \left[ \tilde s(t-1,s'(s,t),s \right]
\end{equation}
Equation~\eqref{eq:nuprop} says that the total utility of the best
path of length $t+1$ to state $s$ is the utility of the best
predecessor plus two terms, one that accounts for the transition from
the predecessor to $s$ and one that accounts for the probability of
state $s$ producing the observation $\ti{y}{t}$.  From the model
assumptions (Eqn.~\eqref{eq:assume_markov} and
Eqn.~\eqref{eq:assume_output}) we find
\begin{multline*}
  P_{\ti{Y}{t+1},\ts{Y}{0}{t+1},\ti{S}{t+1},\ts{S}{0}{t+1}}
  \left(\ti{y}{t+1},\ts{y}{0}{t+1},s,\ts{s}{0}{t+1} \right) =\\
  P\left(\ts{y}{0}{t+1}, \ts{s}{0}{t+1} \right) \cdot
  P_{\ti{S}{t+1}|\ti{S}{t}} \left(s|\ti{s}{t} \right) \cdot
  P_{\ti{Y}{t+1}|\ti{S}{t+1}} \left(\ti{y}{t}|s \right).
\end{multline*}
Taking logs yields
\begin{equation}
  \label{eq:nuprop}
  \begin{split}
    \nu (t+1,s) = \nu(t,s'(s,t)) +
    \log\left(P_{\ti{S}{t+1}|\ti{S}{}} \left(s|s'(s,t) \right)\right)\\
    \quad + \log \left( P_{\ti{Y}{t+1}|\ti{S}{t+1}}
      \left(\ti{y}{t}|s\right)\right).
  \end{split}
\end{equation}

\subsubsection*{Algorithm:}
Following the steps in Fig.~\ref{fig:viterbi}, note that the algorithm
starts by assigning a utility for each state using the first
observation and then iterates forward through time.  For each time
step $t$ and each possible next state $s_\text{next}$ at time $t+1$,
the algorithm finds and records the best predecessor state $s_\text{best}$
at time $t$.  Then the algorithm calculates the utility of
$s_\text{next}$ at time $t+1$ on the basis of the utility of the best
predecessor, the conditional transition probability, and the
conditional observation probability.  At the final time $T$ the
algorithm selects the highest utility endpoint, \ie,
\begin{equation*}
  \ti{\hat s}{T-1} = \argmax_s \nu(T-1,s),
\end{equation*}
and then backtracks through the optimal predecessor links to produce
the entire highest utility path.
%%% loa: list of algorithms, toc: table of contents, lof: list of
%%% figures, lot: list of tables.
\addcontentsline{loa}{section}{Pseudocode for the Viterbi Algorithm}
%%% 
\begin{figure}[htbp]
  \begin{center}
    \def\tnext{_{\text{next}}}%
    \def\told{_{\text{old}}}%
    \def\tbest{_{\text{best}}}%
    \fbox{
      \begin{minipage}{0.90\textwidth}
        \begin{tabbing}
          XX\=XX\=XX\=XX\=XX\=XX\=XX\=XX\= \kill
          Initialize: \> \+ \\
          for each $s$\\ \> \+
          $\nu\tnext (s) = \log \left( P_{\ti{Y}{0},\ti{S}{0}}
            \left(\ti{y}{0},s \right)\right)$ \\ \\ \< \- \< \-
          Iterate: \> \+ \\
          for $t$ from 1 to $T$\\ \> \+
          Swap $\nu\tnext \leftrightarrow \nu\told$\\
          for each $s\tnext$\\ \> \+
          \\ \# Find best predecessor\\
          $s\tbest = \argmax_{s\told}\left( \nu\told(s\told) + \log\left(
              P_{\ti{S}{t}|\ti{S}{t-1}} \left(s\tnext|s\told \right) \right)\right)$ \\
          \\ \# Update $\nu$\\
          $\nu\tnext(s\tnext) =\,$ \= $\nu\told(s\tbest)$ \\ \> \+
          $+ \log\left( P_{\ti{S}{t}|\ti{S}{t-1}} \left(s\tnext|s\tbest
            \right) \right)$ \\
          $ + \log\left( P_{\ti{Y}{t}|\ti{S}{t}} \left(\ti{y}{t}|s\tnext \right)
          \right)  $ \\ \> \-
          \\ \# Update predecessor array\\
          Predecessor[$s\tnext,t$] = $s\tbest$\\ \\
          \< \- \< \- \< \- %This stuff is for tabs \< \-
          Backtrack: \> \+ \\
          $\ts{s}{0}{T} = \ts{\hat s}{0}{T}(\bar s)$ , where $\bar s =
          \argmax_s \nu\tnext(s)$ at $t=T-1$
        \end{tabbing}
      \end{minipage}
    }
    \caption[\comment{fig:viterbi }Pseudocode for the Viterbi Algorithm.]%
    {Pseudocode for the Viterbi Algorithm}
    \label{fig:viterbi}
  \end{center}
\end{figure}
%%%
%%% fig:viterbiB
%%%
\begin{sidewaysfigure}[htbp]
  \marginpar{Check sequence notation}
  %% 0.75\textheight is approx 5.7 in.  The .eps is exactly that size.
  %% The widths for the minipages below are taken by measuring the
  %% ovals inside of XFig.
  \centering{\plotsize%
    %% Column a
    \def\colaa{$\nu(s_1,t-1)$}%
    \def\colab{$\nu(s_2,t-1)$}%
    \def\colac{$\nu(s_3,t-1)$}%
    %% Column b
    \def\colba{$ \begin{matrix} s'(s_1,t) = \argmax_{{\tilde s}}\\%
        \log\left(P(s_1|{\tilde s}) \right)\\+ \nu({\tilde s},t-1)
      \end{matrix}$}%
    \def\colbb{$ \begin{matrix} s'(s_2,t) = \argmax_{{\tilde s}}\\%
        \log\left(P(s_2|{\tilde s}) \right)\\+ \nu({\tilde s},t-1)
      \end{matrix}$}%
    \def\colbc{$ \begin{matrix} s'(s_3,t) = \argmax_{{\tilde s}}\\%
        \log\left(P(s_3|{\tilde s}) \right)\\+ \nu({\tilde s},t-1)
      \end{matrix}$}%
    \def\bestpred{%
      \begin{minipage}[t]{3.9in}
        \raggedright%
        For each state $s$ find the best\\%
        predecessor $s'(s,t)$, \ie, the\\%
        one that maximizes\\%
        $\log\left(P(s|s'(s,t)) \right) + \nu(s'(s,t),t-1)$.\\%
        The bolder lines indicate best\\%
        predecessors.
      \end{minipage}}%
    %% Column c
    \def\colca{$\nu(s_1,t)$}%
    \def\colcb{$\nu(s_2,t)$}%
    \def\colcc{$\nu(s_3,t)$}%
    \def\newnu{%
      \begin{minipage}[t]{2.7in}
        \raggedright%
        For each state $s$\\%
        calculate $\nu(s,t)$\\%
        by including the\\%
        conditional probability\\%
        of the observation $\ti{y}{t}$,\\%
        \ie, $ \nu(s,t) = \log\left(P(\ti{y}{t}|s) \right) $\\
        \qquad$+ \log\left(P(s|s'(s,t)) \right)$\\
        \qquad$+ \nu(s'(s,t),t-1)$.
      \end{minipage}}
    %%
    \input{viterbiB.pdf_t}}
  \vspace{7em}
  \caption[\comment{fig:viterbiB }Dependency relations in the Viterbi algorithm.]%
  {Dependency relations in the Viterbi algorithm.}
  \label{fig:viterbiB}
\end{sidewaysfigure}


\section{The Baum-Welch Algorithm}
\label{sec:baum_welch}

Given an initial vector of model parameters $\parameters$ for an HMM
and a sequence of observations $\ts{y}{0}{T}$, iteration of the
Baum-Welch algorithm \index{Baum-Welch algorithm|textbf} produces a
sequence of parameter vectors $\ts{\parameters}{1}{N}$ that almost
always converges to a local maximum of the likelihood function
$P_{\parameters} \left( \ts{y}{0}{T} \right)$.  The algorithm was
developed by \index*{Baum} and collaborators \cite{Baum70,Baum67} in
the 1960's at the Institute for Defense Analysis in Princeton.  In
each iteration, it estimates the distribution of the unobserved states
and then maximizes the expected log likelihood with respect to that
estimate.  Although Baum et al.\ limited their attention to HMMs, the
same kind of iteration works on other models that have unobserved
variables.  In 1977, Dempster Laird and Rubin \cite{Dempster77} called
the general procedure the \emph{EM algorithm}.  \index{EM
  (expectation-maximization) algorithm}%
\index{expectation-maximization (EM) algorithm}%

The EM algorithm operates on models $P_{\Y,\bS,\parameters} $ with
parameters $\parameters$ for a mix of data that is observed $(\Y)$ and
data that is unobserved $(\bS)$.  (For our application, $\Y$ is a
sequence of observations $\ts{Y}{0}{T}$ and $\bS$ is a sequence of
discrete hidden states $\ts{S}{0}{T}$.)  The steps in the algorithm
are:
\begin{enumerate}
\item Guess\footnote{Although a symmetric model in which the
    transition probability from each state to every other state is the
    same and the observation probabilities are all uniform is easy to
    describe, such a model is a bad choice for $\ti{\parameters}{1}$
    because the optimization procedure will not break the symmetry of
    the states.}  a starting value of $\ti{\parameters}{1}$, and set
  $n=1$.
\item \label{EM_loop} Choose $\ti{\parameters}{n+1}$ to maximize
  an \emph{auxiliary function} $Q$
                                %
  \begin{equation}
    \label{eq:EMmax}
    \ti{\parameters}{n+1} = \argmax_{\parameters} Q(\parameters,\ti{\parameters}{n})
  \end{equation}
  where
  \begin{equation}
    \label{eq:Qdef}
    Q(\parameters',\parameters) \equiv \EV_{P \left(\bS|\y,\parameters \right)}
    \left( \log P(\y,\bS,\parameters') \right)
  \end{equation}
  (Here the notation $\EV_{P \left(\bS|\y,\parameters
    \right)}(F(\bS))$ means the expected value of $F(\bS)$ over all
  values of $\bS$ using the distribution $P \left(\bS|\y,\parameters
  \right)$.)%
  \nomenclature[rEV]{$\EV_{q(X)}(F(X))$}{The expected value of
    the function $F$ over random variable $X$ with distribution $q$.}
\item If not converged, go to~\ref{loop} with $n \leftarrow n+1.$
\end{enumerate}

We defer further discussion of the general EM algorithm to Section
\ref{sec:EM} and now proceed to the details of its application to
HMMs, \ie, the Baum-Welch algorithm.  The work of an iteration of the
EM algorithm is done in step~\ref{EM_loop}.  To apply it to an HMM, we
first characterize $P \left(\ts{s}{0}{T}|\ts{y}{0}{T}, \parameters
\right)$ by a combination of the \emph{forward algorithm} that we
already described in Section~\ref{sec:forward} and the \emph{backward
  algorithm} which we will describe in the next section.  Using the
characterization of $P \left(\ts{s}{0}{T}|\ts{y}{0}{T}, \parameters
\right)$, we describe the optimization specified by
Eqn.~\eqref{eq:EMmax} in Section~\ref{sec:reestimation}.

\subsection{The Backward Algorithm}
\label{sec:backward}

The \index{backward algorithm} backward algorithm is similar to the
forward algorithm in structure and complexity, but the terms are
neither as easy to interpret nor as clearly useful.  After running
both the forward algorithm and the backward algorithm, one can
calculate $P_{\ti{S}{t}|\ts{Y}{0}{T}} \left(s|\ts{y}{0}{T} \right)$,
the conditional probability of being in any state $s \in \states$ at
any time $t: 1\leq t \leq T$ given the entire sequence of
observations.  The forward algorithm provides the terms $\alpha(s,t)
\equiv P_{\ti{S}{t}|\ts{Y}{0}{t}} \left(s|\ts{y}{0}{t} \right)\,
\forall (s,t)$.  Thus the backward algorithm must provide terms, call
them $\beta(s,t)$, with the values
\begin{equation}
  \label{eq:bw1} \beta(s,t) =
  \frac{P_{\ti{S}{t}|\ts{Y}{0}{T}} \left(s|\ts{y}{0}{T} \right)}
  {\alpha(s,t)} = \frac{P_{\ti{S}{t}|\ts{Y}{0}{T}}
    \left(s|\ts{y}{0}{T} \right)}
  {P_{\ti{S}{t}|\ts{Y}{0}{t}} \left(s|\ts{y}{0}{t} \right)}.
\end{equation}
\nomenclature[gb]{$\beta(s,t)$}{An intermediate quantity calculated in the
  backwards algorithm which is used somewhat like $\alpha$ in the
  forward algorithm.  One may use either of the following two
  equations to define $\beta$
  \begin{align*}
    \beta(s,t) &= \frac{P_{\ts{Y}{t+1}{T}"|\ti{S}{t}}
      \left(\ts{y}{t+1}{T}"|s
      \right)} {P \left(\ts{y}{t+1}{T}"|\ts{y}{0}{t} \right)}\\
    &= \frac{P_{\ti{S}{t}"|\ts{Y}{0}{T}} \left(s"|\ts{y}{0}{T} \right)}
    {\alpha(s,t)} = \frac{P_{\ti{S}{t}"|\ts{Y}{0}{T}}
      \left(s"|\ts{y}{0}{T} \right)} {P_{\ti{S}{t}"|\ts{Y}{0}{t}}
      \left(s"|\ts{y}{0}{t} \right)}.
  \end{align*}
  The interpretation of $\beta$ is less intuitive than the
  interpretation of $\alpha$.  It is also called the backward forecast
  distribution in the Kalman filter literature.}% end nomenclature
Invoking Bayes rule and the model assumptions we find
\begin{align}
  \mlabel{eq:bw2} \beta(s,t) &= \frac{P_{\ts{Y}{0}{T},\ti{S}{t}}
    \left(\ts{y}{0}{T},s \right)\cdot P \left(\ts{y}{0}{t} \right)}
  {P_{\ts{Y}{0}{t},\ti{S}{t}}
    \left(\ts{y}{0}{t},s \right)\cdot P \left(\ts{y}{0}{T} \right)}\\
  \mlabel{eq:bw3} &= \frac{P_{\ts{Y}{t+1}{T}|\ts{Y}{0}{t},\ti{S}{t}}
    \left(\ts{y}{t+1}{T}|\ts{y}{0}{t},s \right)} {P
    \left(\ts{y}{t+1}{T}|\ts{y}{0}{t} \right)}\\
  \label{eq:bw4}
  & = \frac{P_{\ts{Y}{t+1}{T}|\ti{S}{t}} \left(\ts{y}{t+1}{T}|s
    \right)} {P \left(\ts{y}{t+1}{T}|\ts{y}{0}{t} \right)}.
\end{align}
(Note that if $\alpha(s,t)=0$, Eqn.~\eqref{eq:bw1} is undefined, but
we can nonetheless implement Eqn.~\eqref{eq:bw4}.)

The algorithm starts at the final time $T$ with $\beta$ set to
one\footnote{From the premises that $\alpha(s,t) \beta(s,t) =
  P_{\ti{S}{t}|\ts{Y}{0}{T}} \left(s|\ts{y}{0}{T} \right)$ and
  $\alpha(s,T) \equiv P_{S(T)|\ts{Y}{0}{T}}\left(s |\ts{y}{0}{T}
  \right)$, we conclude that $\beta(s,T) = 1 ~\forall s$.} %
for each state, $\beta(s,T) = 1,~\forall s \in \states$, and solves
for $\beta$ at earlier times with the following recursion that goes
\emph{backwards} through time:
\begin{equation}
  \label{eq:backformula}
  \beta({\tilde s},t-1) = \sum_{s\in\states} \beta(s,t)
  \frac{ P_{\ti{Y}{t}|\ti{S}{t}}
    \left(\ti{y}{t}|s \right) \cdot P_{\ti{S}{t}|\ti{S}{t-1}}
    \left(s|{\tilde s} \right)} {\ti{\gamma}{t}}
\end{equation}
Note that ${\ti{\gamma}{t} \equiv P \left( \ti{y}{t} |
    \ts{y}{0}{t}\right)}$ is calculated by the forward algorithm and
that the terms in the numerator are model parameters.  We justify
Eqn.~\eqref{eq:backformula} by using Eqn.~\eqref{eq:bw4}
\begin{align}
  \beta({\tilde s},t-1) & \equiv \frac{P_{\ts{Y}{t}{T}|\ti{S}{t-1}}
    \left(\ts{y}{t}{T}|{\tilde s} \right)}
  {P \left(\ts{y}{t}{T}|\ts{y}{0}{t} \right)} \\
  \mlabel{eq:back2}
  &= \frac{\sum_{s\in\states} P_{\ts{Y}{t}{T},\ti{S}{t}|\ti{S}{t-1}}
    \left(\ts{y}{t}{T},s|{\tilde s} \right)} {P
    \left(\ts{y}{t}{T}|\ts{y}{0}{t} \right)}.
\end{align}
Next, factor each term in the numerator of Eqn.~\eqref{eq:back2} using
Bayes rule twice then apply the model assumptions and the expression
for $\beta(s,t)$ from Eqn.~\eqref{eq:bw4}:
\begin{align*}
  & P_{\ts{Y}{t}{T},\ti{S}{t}|\ti{S}{t-1}}\left( \ts{y}{t}{T},s|{\tilde s} \right)\\
  \begin{split}
    &\quad= P_{\ts{Y}{t+1}{T}|\ti{Y}{t},\ti{S}{t},\ti{S}{t-1}}
       \left(\ts{y}{t+1}{T}|\ti{y}{t},s,{\tilde s} \right)\\
    &\quad\qquad \cdot P_{\ti{Y}{t}|\ti{S}{t},\ti{S}{t-1}} \left(\ti{y}{t}|s,{\tilde s}
       \right)\cdot P_{\ti{S}{t}|\ti{S}{t-1}} \left(s|{\tilde s} \right)
  \end{split}\\
  &\quad= P_{\ts{Y}{t+1}{T}|\ti{S}{t}} \left(\ts{y}{t+1}{T}|s\right)
           \cdot P_{\ti{Y}{t}|\ti{S}{t}} \left(\ti{y}{t}|s \right) \cdot
            P_{\ti{S}{t}|\ti{S}{t-1}} \left(s|{\tilde s} \right) \\
  &\quad= \beta(s,t) \cdot P \left(\ts{y}{t+1}{T}|\ts{y}{0}{t} \right)
           \cdot P_{\ti{Y}{t}|\ti{S}{t}} \left(\ti{y}{t}|s \right) \cdot
            P_{\ti{S}{t}|\ti{S}{t-1}} \left(s|{\tilde s} \right)
\end{align*}
Similarly, simplify the denominator of Eqn.~\eqref{eq:back2} using
Bayes rule:
\begin{equation*}
  P \left(\ts{y}{t}{T}|\ts{y}{0}{t} \right) = P
  \left(\ts{y}{t+1}{T}|\ts{y}{0}{t}  \right) \cdot {P \left(
  \ti{y}{t}|\ts{y}{0}{t}\right)}
\end{equation*}
Finally by substituting these values into the fraction of
\eqref{eq:back2}, we verify the recursion \eqref{eq:backformula}
\begin{align}
  \mlabel{eq:back3} \beta({\tilde s},t-1) &= \frac{\sum_{s\in\states}
    \beta(s,t) \cdot P \left(\ts{y}{t+1}{T}|\ts{y}{0}{t} \right)
    \cdot P_{\ti{Y}{t}|\ti{S}{t}} \left(\ti{y}{t}|s \right) \cdot
    P_{\ti{S}{t}|\ti{S}{t-1}} \left(s|{\tilde s} \right)} {P
    \left(\ts{y}{t+1}{T}|\ts{y}{0}{t} \right) \cdot {P \left(
        \ti{y}{t}|\ts{y}{0}{t}\right)}} \\
  \mlabel{eq:back4} &= \sum_{s\in\states} \beta(s,t) \frac{
    P_{\ti{Y}{t}|\ti{S}{t}} \left(\ti{y}{t}|s \right) \cdot
    P_{\ti{S}{t}|\ti{S}{t-1}} \left(s|{\tilde s} \right)} {\ti{\gamma}{t}}.
\end{align}

\subsection{Weights and Reestimation}
\label{sec:reestimation}

Each pass of the Baum-Welch algorithm consists of the following steps:
Run the forward algorithm described in Section~\ref{sec:forward} to
calculate the values of $\alpha(s,t)$ and $\gamma(t)$ for each time $t
\in [1,\ldots,T]$ and each state $s \in \states$; Run the backward
algorithm described in Section~\ref{sec:backward} to calculate the
values of $\beta({\tilde s},t)$; Reestimate the model parameters using the
formulas in Table~\ref{tab:reestimation}.

We write the reestimation formulas in terms of \emph{weights} which
express the conditional probability of being in specific states at
specific times given the observed data $\ts{y}{0}{T}$.  We denote the
conditional probability of being in state $s$ at time $t$ given all of
the data by
\begin{equation}
  \label{eq:wit}
  w(s,t) \equiv P_{\ti{S}{t}|\ts{Y}{0}{T}} \left(s|\ts{y}{0}{T}
  \right),
\end{equation}
\nomenclature[rwst]{$w(s,t)$}{In the reestimation phase of the Baum-Welch
  algorithm, the \emph{weight} assigned to state $s$ at time $t$,
  \begin{equation*}
    w(s,t) \equiv P_{\ti{S}{t}"|\ts{Y}{0}{T}} \left(s"|\ts{y}{0}{T}
    \right).
  \end{equation*}}% end nomenclature
and we denote the conditional probability, given all of the data, of
being in state $s$ at time $t$ and being in state ${\tilde s}$ at time
$t+1$ by
\begin{equation}
  \label{eq:wijt}
  {\tilde w}({\tilde s},s,t) \equiv P_{\ti{S}{t+1},\ti{S}{t}|\ts{Y}{0}{T}}
  \left({\tilde s},s|\ts{y}{0}{T} \right).
\end{equation}
\nomenclature[rwsst]{$ {\tilde w}({\tilde s},s,t)$}{In the reestimation phase
  of the Baum-Welch algorithm, the \emph{weight} assigned to the
  transition from state $s$ at time $t$ to state $\tilde s$ at time $t+1$,
  \begin{equation*}
     {\tilde w}({\tilde s},s,t) \equiv P_{\ti{S}{t+1},\ti{S}{t}"|\ts{Y}{0}{T}}
  \left({\tilde s},s"|\ts{y}{0}{T} \right).
  \end{equation*}}
Table~\ref{tab:reestimation} (page \pageref{tab:reestimation})
summarizes the formulas for the updated model parameters after one
pass of the Baum-Welch algorithm.

To derive reestimation formulas for $P_{S(1)}$ and
$P_{\ti{Y}{t}|\ti{S}{t}}$ we will consider a sum over all possible
state sequences $\ts{s}{0}{T}$, \ie,
\begin{equation}
  \label{eq:witFFE}
  w(s,t) = \sum_{\ts{s}{0}{T}:\ti{s}{t}=s}
  P \left(\ts{s}{0}{T}|\ts{y}{0}{T}
  \right).
\end{equation}
Since this is virtually unimplementable, the actual algorithm uses
\begin{equation}
  \label{eq:witalphabeta}
  w(s,t) = \alpha(s,t) \beta(s,t).
\end{equation}
In fact, in Eqn.~\eqref{eq:bw1}, we chose the expression for $\beta$
to make Eqn.~\eqref{eq:witalphabeta} true.

Similarly, we will derive the reestimation formula for
$P_{\ti{S}{t+1}|\ti{S}{t}}$ using
\begin{equation}
  \label{eq:wijtFFE}
  {\tilde w}({\tilde s},s,t) = \sum_{\substack{\ts{s}{0}{T}:\ti{s}{t+1}={\tilde s},\\\ti{s}{t}=s}}
  P \left(\ts{s}{0}{T}|\ts{y}{0}{T} \right),
\end{equation}
but in the algorithm we use
\begin{equation}
  \label{eq:wijtalphabeta}
  {\tilde w}({\tilde s},s,t) = \frac{ \alpha(s,t) \cdot P_{\ti{S}{t+1}|\ti{S}{t}}
    \left({\tilde s}|s \right)  \cdot P_{\ti{Y}{t+1}|\ti{S}{t+1}}
    \left(\ti{y}{t+1}|{\tilde s} \right) \cdot \beta({\tilde s},t+t) }
  {\ti{\gamma}{t+1}}.
\end{equation}
One can verify Eqn.~\eqref{eq:wijtalphabeta} using the model
assumptions, the definitions of $\alpha$, $\beta$, and $\gamma$, and
Bayes rule.

\subsubsection{Reestimation}

With Eqns.~\eqref{eq:witalphabeta} and \eqref{eq:wijtalphabeta} for
$w(i,t)$ and ${\tilde w}({\tilde s},s,t)$ in terms of known quantities
($\alpha$, $\beta$, $\gamma$, $\ts{y}{0}{T}$, and the old model
parameters $\ti{\parameters}{n}$), we are prepared to use the formulas
in Table \ref{tab:reestimation} to calculate new estimates of the
model parameters, $\ti{\parameters}{n+1}$ with higher likelihood.

\begin{table}[htbp]
  \caption[\comment{tab:reestimation }Summary of reestimation formulas.]%
  {Summary of reestimation formulas.\index{reestimation formulas}}
  \centering{\plotsize%
    \begin{minipage}{.7\textwidth}
      Note that formulas for $w(s,t)$ and ${\tilde w}({\tilde s},s,t)$
      appear in Eqns.~\eqref{eq:witalphabeta} and
      \eqref{eq:wijtalphabeta} respectively.
    \end{minipage}\\[1ex]
%    \begin{tabular*}{0.98\textwidth}[H]{|l|r|l|}
    \begin{tabular}[H]{|c|c|c|}
      \hline
      \rule{0pt}{2.5ex}Description & Expression & New Value \\
      \hline
      \rule{0pt}{2.5ex}Initial State Prob.
      & $P_{\ti{S}{1}|\ti{\parameters}{n+1}} \left(s|\ti{\parameters}{n+1} \right)$
      & $ w(s,1) $ \\[1.5ex]
      State Transition Prob.
      & $P_{\ti{S}{t+1}|\ti{S}{t},\ti{\parameters}{n+1}} \left({\tilde
          s}|s, \ti{\parameters}{n+1} \right)$
      & $ \frac {\sum_{t=1}^{T-1} {\tilde w}({\tilde s},s,t)} {\sum_{s'\in\states}
        \sum_{t=1}^{T-1} {\tilde w}(s',s,t)}$ \\[2.5ex]
      Cond. Observation Prob.
      & $P_{\ti{Y}{t}|\ti{S}{t},\ti{\parameters}{n+1}} \left(y|s, \ti{\parameters}{n+1} \right)$
      & $ \frac {\sum_{t:\ti{y}{t}=y} w(s,t)} {\sum_{t} w(s,t)}$ \\[2.0ex]
      \hline
%    \end{tabular*}}
    \end{tabular}}
  \label{tab:reestimation}
\end{table}

To derive the formulas in Table \ref{tab:reestimation}, start with
Step 2 of the EM algorithm (see Eqn.~\eqref{eq:Qdef}) which is to
maximize the auxiliary function
\begin{equation*}
  Q(\parameters',\parameters) \equiv \EV_{P \left(\bS|\y \right),\parameters}
  \left( \log P(\y,\bS|\parameters') \right)
\end{equation*}
with respect to $\parameters'$.  For an HMM, substitute the sequence of
\emph{hidden} states $\ts{s}{0}{T}$ for $\bS$ and the sequence of
observations $\ts{y}{0}{T}$ for $\y$.  Note that the joint probability
of a state sequence $\ts{s}{0}{T}$ and the observation sequence
$\ts{y}{0}{T}$ is
\begin{equation*}
  P\left( \ts{s}{0}{T},\ts{y}{0}{T} \right) = P_{\ti{S}{1}}
  \left(\ti{s}{1} \right) \cdot \prod_{t=2}^T P_{\ti{S}{2}|\ti{S}{1}}
  \left(\ti{s}{t}|\ti{s}{t-1} \right)
  \cdot \prod_{t=1}^T  P_{\ti{Y}{1}|\ti{S}{1}}
  \left(\ti{y}{t}|\ti{s}{t} \right),
\end{equation*}
or equivalently,
\begin{align*}
  \log P\left(\ts{y}{0}{T}, \ts{s}{0}{T} \right) &= \log P_{\ti{S}{1}}
  \left(\ti{s}{1} \right) + \sum_{t=2}^T \log P_{\ti{S}{2}|\ti{S}{1}}
  \left(\ti{s}{t}|\ti{s}{t-1} \right)\\
  &\quad + \sum_{t=1}^T \log  P_{\ti{Y}{1}|\ti{S}{1}}
  \left(\ti{y}{t}|\ti{s}{t} \right).
\end{align*}
We can optimize $Q$ by breaking it into a sum in which each of the
model parameters only appears in one of the terms and then optimizing
each of the terms independently:
\begin{align}
  \label{eq:QHMM}
  Q(\parameters',\parameters) &= \sum_{\ts{s}{0}{T}\in\states^T}
  \left( P
  \left(\ts{s}{0}{T}|\ts{y}{0}{T},\parameters \right) \log P
  \left(\ts{s}{0}{T},\ts{y}{0}{T},\parameters' \right) \right) \\
  \label{eq:QHMMseparate}
  & \equiv Q_{\text{initial}} (\parameters',\parameters) +
  Q_{\text{transition}} (\parameters',\parameters) + Q_{\text{observation}}
  (\parameters',\parameters),
\end{align}
where
\begin{align}
  \label{eq:Qinitial}
  Q_{\text{initial}} (\parameters',\parameters) &\equiv
%
  \sum_{\ts{s}{0}{T}\in\states^T} \left( P
  \left(\ts{s}{0}{T}|\ts{y}{0}{T},\parameters \right) \log P_{\ti{S}{1}|\parameters'}
  \left(\ti{s}{1}|\parameters' \right) \right)\\
%
  \label{eq:Qtransition}
  Q_{\text{transition}} (\parameters',\parameters) &\equiv
%
  \sum_{\ts{s}{0}{T}\in\states^T} \left( P
  \left(\ts{s}{0}{T}|\ts{y}{0}{T},\parameters \right) \sum_{t=2}^T \log
  P_{\ti{S}{2}|\ti{S}{1},\parameters'}
  \left(\ti{s}{t}|\ti{s}{t-1},\parameters' \right) \right) \\
%
  \label{eq:Qoutput}
  Q_{\text{observation}} (\parameters',\parameters) &\equiv
%
  \sum_{\ts{s}{0}{T}\in\states^T} \left( P
  \left(\ts{s}{0}{T}|\ts{y}{0}{T},\parameters \right) \sum_{t=1}^T \log
  P_{\ti{Y}{1}|\ti{S}{1},\parameters'}
  \left(\ti{y}{t}|\ti{s}{t},\parameters' \right) \right)
%
\end{align}

To simplify the appearance of expressions as we optimize $Q$, we
introduce notation for logs of parameters
\begin{align}
  L_{\text{initial}}(i) & \equiv \log P_{\ti{S}{1}|\parameters'}
  \left(s|\parameters' \right) \\
  L_{\text{transition}}(i,j) &\equiv \log P_{\ti{S}{2} |
    \ti{S}{1},\parameters'}({\tilde s}|s,\parameters') \\
  \label{eq:Loutput}
  L_{\text{observation}}(y,i) & \equiv \log
  P_{\ti{Y}{1}|\ti{S}{1},\parameters'} \left(y|s,\parameters' \right).
\end{align}
Now to optimize $Q_{\text{initial}}$ write Eqn.~\eqref{eq:Qinitial} as
\begin{align}
  Q_{\text{initial}} (\parameters',\parameters) &=
  \sum_{\ts{s}{0}{T}\in\states^T} P
  \left(\ts{s}{0}{T}|\ts{y}{0}{T},\parameters \right)
  L_{\text{initial}}(\ti{s}{1})\\
  &= \sum_{s \in \states} L_{\text{initial}}(s)
  \sum_{\ts{s}{0}{T}:\ti{s}{1}=s} P
  \left(\ts{s}{0}{T}|\ts{y}{0}{T},\parameters \right)\\
  &= \sum_s L_{\text{initial}}(s) P_{\ti{S}{1}|\ts{Y}{0}{T},\parameters}
  \left(s|\ts{y}{0}{T},\parameters \right) \text{ see Eqn.~\eqref{eq:witFFE}}\\
  &= \sum_s L_{\text{initial}}(s) w(s,1)
\end{align}
We wish to find the set $\left\{ L_{\text{initial}}(s) \right\}$ that
maximizes $Q_{\text{initial}} (\parameters',\parameters)$ subject to the
constraint
\begin{equation*}
  \sum_s e^{ L_{\text{initial}}(s)} \equiv\sum_s
  P_{\ti{S}{1}|\hat \parameters} \left(s|\hat \parameters \right) = 1.
\end{equation*}
The method of Lagrange multipliers yields
\begin{equation}
  \label{eq:LiSol}
   L_{\text{initial}}(s) = \log w(s,1) ~ \forall s,
\end{equation}
ie, the new estimates of the initial probabilities are
\begin{equation}
  \label{eq:InitialRe}
  P_{\ti{S}{1}|\ti{\parameters}{n+1}} \left(s|\ti{\parameters}{n+1} \right) = w(s,1).
\end{equation}

To derive new estimates of the state transition probabilities, write
\begin{align}
  Q_{\text{transition}} (\parameters',\parameters) &=
  \sum_{\ts{s}{0}{T}\in\states^T} P
  \left(\ts{s}{0}{T}|\ts{y}{0}{T},\parameters \right) \sum_{t=2}^T
  L_{\text{transition}}(\ti{s}{t-1},\ti{s}{t})\\
  &= \sum_{s,{\tilde s}} L_{\text{transition}}(s,{\tilde s})
  \sum_{t=2}^T \sum_{\substack{\ts{s}{0}{T}:\ti{s}{t}={\tilde
        s},\\\ti{s}{t-1}=s}}
  P\left(\ts{s}{0}{T}|\ts{y}{0}{T},\parameters \right)\\
  &= \sum_{s,{\tilde s}} L_{\text{transition}}(s,{\tilde s})
  \sum_{t=1}^{T-1} {\tilde w}({\tilde s},s,t).
\end{align}
Optimization yields
\begin{equation}
  \label{eq:NewTrans}
  P_{\ti{S}{t+1}|\ti{S}{t},\ti{\parameters}{n+1}} \left({\tilde
      s}|s,\ti{\parameters}{n+1} \right) = \frac
  {\sum_{t=1}^{T-1} {\tilde w}({\tilde s},s,t)} {\sum_{s'}
    \sum_{t=1}^{T-1} {\tilde w}(s',s,t)}.
\end{equation}

Similarly, we derive the new estimates of the conditional observation
probabilities from
\begin{align}
  \label{eq:Qout1}
  Q_{\text{observation}} (\parameters',\parameters) &=
  \sum_{\ts{s}{0}{T}\in\states^T} P
  \left(\ts{s}{0}{T}|\ts{y}{0}{T},\parameters \right) \sum_{t=1}^T
  L_{\text{observation}}(\ti{y}{t},\ti{s}{t})\\
  &= \sum_{y\in\outputs,s\in\states} L_{\text{observation}}(y,s)
  \sum_{t:\ti{y}{t}=y}
  \sum_{\ts{s}{0}{T}:\ti{s}{t}=s }P
  \left(\ts{s}{0}{T}|\ts{y}{0}{T},\parameters \right) \\
  &= \sum_{y,s} L_{\text{observation}}(y,s) \sum_{t:\ti{y}{t}=y} w(s,t).
\end{align}
Optimization yields
\begin{equation}
  \label{eq:NewOut}
  P_{\ti{Y}{t}|\ti{S}{t},\ti{\parameters}{n+1}}
  \left(y|s,\ti{\parameters}{n+1} \right) = \frac
  {\sum_{t:\ti{y}{t}=y} w(s,t)} {\sum_{t=1}^T w(s,t)}.
\end{equation}

%%%
%%% fig:train
%%%
\addcontentsline{loa}{section}{Baum-Welch model parameter optimization}
\begin{figure}[htbp]
  \begin{center}
    %\small%
    \def\assign{\leftarrow}%
    \def\oldmodel{\ti{\parameters}{n}}%
    \def\newmodel{\ti{\parameters}{n+1}}%
% Tabbing commands:
% \=    Set a stop
% \>    Skip to the next stop
% \<    Go back a stop
% \\    New line
% \+    Move left margin right one stop
% \-    Move the left margin left one stop
\fbox{
  \begin{minipage}{0.90\textwidth}
    \begin{tabbing}
      XX\=XX\=XX\=XX\=XX\=XX\=XX\=XX\= \kill
      Notation: \> \+ \\ \\
      \begin{minipage}[b]{1.0\textwidth}
        $\oldmodel$ is the model, or equivalently the set of
        parameters, after $n$ iterations of the Baum-Welch algorithm.
      \end{minipage}\\ \\%
      %%
      \begin{minipage}[b]{1.0\textwidth}
        $\bm{\alpha}_n$ is the set of conditional state probabilities
        calculated on the basis of the $n^{\text{th}}$ model and the
        data $\ts{y}{0}{T}$.  See Eqns.~\eqref{eq:alpha} and
        \eqref{eq:forwardD}.
        \begin{equation*}
          \bm{\alpha}_n \equiv \left\{ P_{ S(t)| \ts{Y}{0}{t},\oldmodel} \left( s |
             \ts{y}{0}{t},\oldmodel\right) : \forall s \in \states\, \& \, 1
           \leq t \leq T \right\}
        \end{equation*}
      \end{minipage}\\ \\%
      %%
      \begin{minipage}[b]{1.0\textwidth}
         $\bm{\beta}_n$ is a set of values calculated on the basis of the
         $n^{\text{th}}$ model $\oldmodel$ and the data $\ts{y}{0}{T}$.  See
         Eqns.~\eqref{eq:bw4} and \eqref{eq:backformula}.
         \begin{equation*}
         \bm{\beta}_n \equiv \left\{ \frac{P_{\ts{Y}{t+1}{T}|\ti{S}{t}}
         \left(\ts{y}{t+1}{T} | s \right)} {P
         \left(\ts{y}{t+1}{T}|\ts{y}{0}{t} \right)} : \forall s \in
         \states\, \& \, 1 \leq t < T \right\}
         \end{equation*}
      \end{minipage}\\ \\%
      %%
      \begin{minipage}[b]{1.0\textwidth}
         $\bm{\gamma}_n$ is the set of conditional observation probabilities
         calculated on the basis of the $n^{\text{th}}$ model $\oldmodel$ and
         the data $\ts{y}{0}{T}$.  See Eqns.~\eqref{eq:gamma} and
         \eqref{eq:forwardC}.
         \begin{equation*}
         \bm{\gamma}_n \equiv \left\{P \left(\ti{y}{t} | \ts{y}{0}{t},
         \oldmodel\right) : \, 2 \leq t \leq T \right\}
         \end{equation*}
      \end{minipage}\\ \\%
      %%
      \<\-
      Initialize: \> \+ \\
      Set $n=1$ and choose $\ti{\parameters}{1}$\\ \\ \< \-
      Iterate: \> \+ \\
      $\newmodel \assign \text{reestimate} \left( \ts{y}{0}{T},
      \bm{\alpha}_n, \bm{\beta}_n, \bm{\gamma}_n, \oldmodel\right)$ XX\= \kill
      $\left(\bm{\alpha}_n,\bm{\gamma}_n\right) \assign
      \text{forward}(\ts{y}{0}{T},\oldmodel)$ \> See Section
      \ref{sec:forward} page \pageref{sec:forward}\\
      $\bm{\beta}_n \assign \text{backward}(\bm{\gamma}_n, \ts{y}{0}{T},
      \oldmodel)$ \> See Section \ref{sec:backward}  page \pageref{sec:backward}\\
      $\newmodel \assign \text{reestimate} \left( \ts{y}{0}{T},
      \bm{\alpha}_n, \bm{\beta}_n, \bm{\gamma}_n, \oldmodel\right)$ \> See Table
      \ref{tab:reestimation} page \pageref{tab:reestimation} \\
      $n \assign n+1$ \\
      Test for completion
    \end{tabbing}
  \end{minipage}
}
\caption[\comment{fig:train }Baum-Welch model parameter optimization.]%
{Summary and pseudo-code for optimizing model parameters by iterating
  the Baum-Welch algorithm. \index{Baum-Welch algorithm}}
    \label{fig:train}
  \end{center}
\end{figure}

\section{Remarks}
\label{sec:AlgApp}

\subsection{MAP Sequence of States or Sequence of MAP States?}
\label{sec:sequenceMAP}

\index{sequence of maximum a posteriori state estimates}%

Consider the difference between the sequence of maximum a posteriori states
and the maximum a posteriori sequence of states.  The maximum a posteriori state
at a particular time $t$ is the best guess for where the system was at
that time given all of the observations, \ie
%
$\ti{\hat s}{t} = \argmax_{s'}P(\ti{s'}{t}|\ts{y}{0}{T}) =
\argmax_{s'}\alpha(s',t) \beta(s',t)$
%
(see \eqref{eq:wit} --
\eqref{eq:witalphabeta}).  While it seems reasonable that a sequence
of such guesses would constitute a good guess for the entire
trajectory, it is not an optimal trajectory estimate.  In fact, as the
following example demonstrates, such a trajectory may even be
impossible.

Consider the HMM drawn in Fig.~\ref{fig:sequenceMAP} and the sequence
of observations $\ts{y}{0}{6} = (a,b,b,b,b,c)$.  Any sequence of
states that is consistent with $\ts{y}{0}{6}$ must begin in $e$, end
in $g$, and pass through state $f$ exactly once.  The only unknown
remaining is the time at which the system was in state $f$.  Here is a
tabulation of the four possible state sequences:
\begin{center}
  \begin{tabular}{|cccccc|l|c|}
    \hline
    $\ti{s}{1}$ &  $\ti{s}{2}$ &  $\ti{s}{3}$ & 
    $\ti{s}{4}$ &  $\ti{s}{5}$ &  $\ti{s}{6}$ &
    $P(\ts{y}{0}{6},\ts{s}{0}{6})/z$ & 
    $P(\ts{s}{0}{6}|\ts{y}{0}{6})$ \\
    \hline
    $e$ & $e$ & $e$ & $e$ & $f$ & $g$ & $0.9^3$           & 0.30 \\    $e$ & $e$ & $e$ & $f$ & $g$ & $g$ & $0.9^2\cdot 0.8$  & 0.26 \\
    $e$ & $e$ & $f$ & $e$ & $g$ & $g$ & $0.9 \cdot 0.8^2$ & 0.23 \\
    $e$ & $f$ & $g$ & $g$ & $g$ & $g$ & $0.8^3$           & 0.21 \\\hline
  \end{tabular}
\end{center}
In the table, the term $z$ represents the factors that are common in
$P(\ts{y}{0}{6},\ts{s}{0}{6})$ for all of the possible state
sequences.  Only the factors that are different appear in the seventh
column.  The largest entry in the last column is 0.30 which
corresponds to the MAP estimate: $\ts{\hat s}{1}{6} = (e,e,e,e,f,g,)$.
\begin{figure}[htbp]
  \centering{\plotsize%
    \input{sequenceMAP.pdf_t} 
  }  
  \caption{HMM used to illustrate that the maximum a posteriori sequence of states is
    not the same as the sequence of maximum a posteriori states.}
\label{fig:sequenceMAP}
\end{figure}

The next table displays the values of $P(\ti{s}{t}|\ts{y}{0}{6})$, the
a posteriori probability for the three possible states:
\begin{center}
  \begin{tabular}{cc|cccccc}
    &   & \multicolumn{6}{c}{$t$} \\
    \multicolumn{2}{r|}{$P(\ti{s}{t}|\ts{y}{0}{6})$}
        & 1 & 2 & 3 & 4 & 5 & 6 \\
    \hline
    & $e$ & \textbf{1.0} & \textbf{0.79} & \textbf{0.56} & 0.30 & 0    & 0  \\
$s$ & $f$ & 0   & 0.21 & 0.23 & 0.26 & 0.30 & 0   \\
    & $g$ & 0   & 0    & 0.21 & \textbf{0.44} & \textbf{0.70} & \textbf{1.0}
  \end{tabular}
\end{center}
The table quantifies the intuition that the a posteriori probability starts
in state $e$ at time $t=1$ and sort of diffuses completely over to
state $g$ by time $t=6$.  Notice that although all of the probability
passes through state $f$, at no time is it the most probable state.
Thus the sequence of maximum a posteriori states is $e,e,e,g,g,g$ which is
an \emph{impossible} sequence.  On the other hand, the maximum a posteriori
sequence of states, $e,e,e,e,f,g$, is entirely plausible.

\subsection{Training on Multiple Segments}
\label{sec:MultiSegment}

\index{training on multiple segments}
\index{multiple segments, training on}

Simple modifications to code for the Baum-Welch algorithm enable it to
train on data that consists of a collection of independent segments
$\mathbf{\vec y} \equiv \left\{ \mathbf{y}_1, \mathbf{y}_2, \ldots,
  \mathbf{y}_n \right\}$ where $ \mathbf{y}_k = \ts{(y_k)}{1}{T_k}$.
In particular for each iteration, one should:
\begin{itemize}
\item Run the forward and backward algorithms on each segment
  $\mathbf{y}_k$ to calculate ${\bm \alpha}_k$ and
  ${\bm \beta}_k$
\item Create ${\bm \alpha}$ and ${\bm \beta}$ by concatenating
  ${\bm \alpha}_k~\forall k$ and ${\bm \beta}_k~\forall k$
  respectively.
\item Reestimate all model parameters by applying the formulas in
  Table~\ref{tab:reestimation} to the concatenated  ${\bm \alpha}$,
  ${\bm \beta}$, and $\mathbf{\vec y}$.
\item Modify the reestimated initial state probabilities using
  \begin{equation*}
    P_{\ti{S}{1}|\ti{\parameters}{m+1}} \left(s|\ti{\parameters}{m+1}
    \right) = \frac{1}{n} \sum_{k=1}^n \alpha_k(s,1) \beta_k(s,1)
  \end{equation*}
\end{itemize}

\subsection{Probabilities of the initial state}
\label{sec:Ps0}

Using the procedure of the previous section for a few independent
observation sequences with several iterations of the Baum-Welch
algorithm produces a model in which the estimates of the probabilities
of the initial states, $P_{\ti{S}{1}} \left(s \right)\,\forall s \in
\states$, reflect the characteristics at the beginning of the given
sequences.  Those estimates are appropriate if all observation
sequences come from state sequences that start in a similar fashion.
Such models are not \index*{stationary}.  To accommodate the many
applications in which we wish to model the state dynamics as
\index*{stationary}, we also calculate stationary initial state probability
estimates using
\begin{equation}
  \label{eq:Ps0stationary}
  P_{\ti{S}{1}(\text{stationary})} \left(s \right) =
  \frac{\sum_{t}w(i,t)}{\sum_{j,t}w(j,t)}
\end{equation}



\subsection{Maximizing likelihood over unrealistic classes}
\label{sec:incredible}

We often fit simple hidden Markov models to data that come from
systems that have complicated continuous state spaces.  For example we
fit models with roughly 50 states to electrocardiograms even though we
believe that partial differential equations over vector fields better
describe physiological dynamics that affect the signal.  By fitting
unrealistically simple models we reduce the variance of the parameter
estimates at the expense of having parameters that are harder to
interpret.  It is a version of the classic \index{bias-variance
  tradeoff} bias-variance tradeoff.

\subsection{Multiple Maxima}
\label{sec:MultiMax}

The Baum-Welch algorithm generically converges to a \emph{local}
maximum of the likelihood function.  For example, we obtained the
model used to generate Fig.~\ref{fig:Statesintro} by iterating the
Baum-Welch algorithm on an initial model with random parameters.  By
re-running the experiment with five different seeds for the random
number generator, we obtained the five different results that appear
in Fig.~\ref{fig:TrainChar}.
\begin{figure}
  \centering
  % \resizebox{\textwidth}{!}{\includegraphics{TrainChar.pdf}}
  % The result of resizebox seems to be the same as the
  % includegraphics below.
  \includegraphics[width=1.0\textwidth]{TrainChar.pdf}
  \caption[\comment{fig:TrainChar }Convergence of the Baum-Welch algorithm.]{%
    Convergence of the Baum-Welch algorithm.  Here we have
    plotted the log likelihood per step as a function of the number of
    iterations $n$ of the Baum-Welch algorithm for five different
    initial models $\ti{\parameters}{1}$.  We used the same sequence
    of observations $\ts{y}{0}{T}$ that we used for
    Fig.~\ref{fig:Statesintro}, and we used different seeds for a
    random number generator to make the five initial models.  Note the
    following characteristics: The five different initial models all
    converge to different models with different likelihoods; The curves
    intersect each other as some models improve more with training
    than others; Convergence is difficult to determine because some
    curves seem to have converged for many iterations and later rise
    significantly.  }
  \label{fig:TrainChar}
\end{figure}

\subsection{Disappearing Transitions}
\label{sec:disappear}

For some observation sequences $\ts{y}{0}{T}$ and initial models,
multiple iterations of the Baum-Welch algorithm leads to state
transition probabilities that are too small to be represented in
double precision.  As part of our implementation of the Baum-Welch
algorithm, we prune transitions with conditional probabilities of less
than $1\times 10^{-20}$.  Such pruning:
\begin{itemize}
\item Prevents numerical underflow exceptions
\item Simplifies the models
\item Makes algorithms that process the models run faster
\end{itemize}
Bayesian methods, which are beyond the scope of this book, can address
situations in which one believes that some transitions should be
allowed by a model even though maximizing the likelihood by the
Baum-Welch algorithm drives their probability to zero.

\subsection{Bayesian Estimates Instead of Point Estimates}
\label{sec:EstWholeDist}

A Bayesian parameter estimation scheme begins with an \emph{a priori}
distribution $P_{\parameters}$ that characterizes knowledge about what
values are possible and then uses Bayes rule to combine that knowledge
with observed data $y$ to calculate an \emph{a posteriori}
distribution of parameters $P_{\parameters|y}$.  We don't really
believe that the maximum likelihood estimate (MLE) produced by the
Baum-Welch algorithm is precisely the \emph{one true answer}.  It is
what Bayesians call a \emph{point estimate}.  Parameter values near
the MLE are entirely consistent with the data.  A proper Bayesian
procedure characterizes the plausibility of other parameter values
with an \emph{a posteriori} distribution.  In the next chapter, we
present a variant on the Baum-Welch algorithm that uses a prior
distribution on parameter values and produces an estimate that
maximizes the \emph{a posteriori} probability, \ie, a \emph{MAP}
estimate.  However, like the MLE, the MAP is a point estimate.
Neither does a good job of characterizing the set of plausible
parameters.

In addition to yielding only a point estimate, the Baum-Welch
algorithm is indirect in that each pass optimizes an auxiliary
function rather than optimizing the likelihood, and it converges to
\emph{local} maxima.  A Bayesian \emph{Markov chain Monte Carlo}
approach would address all of these objections at the expense of being
slow.

Although others have obtained Bayesian \emph{a posteriori} parameter
distributions for HMMs using \emph{Markov chain Monte Carlo} and
\emph{variational Bayes} procedures (see for example \cite{Rosales04}
and \cite{Beal03}), we will restrict our attention to point estimates
from the Baum-Welch algorithm and simple variations.


\section{The EM algorithm}
\label{sec:EM}
\index{EM (expectation-maximization) algorithm|textbf}%
\index{expectation-maximization (EM) algorithm|textbf}%

In Section \ref{sec:baum_welch}, we described the Baum-Welch algorithm
for finding parameters of an HMM that maximize the likelihood, and we
noted that the Baum-Welch algorithm is a special case of the EM
algorithm.  Here, we examine the general EM algorithm in more detail.
Readers willing to accept the Baum-Welch algorithm without further
discussion of Eqn.~\ref{eq:EMmax} should skip this section.  Dempster
Laird and Rubin\cite{Dempster77} coined the term \emph{EM algorithm}
in 1977.  More recent treatments include Redner and
Walker\cite{Redner84}, McLachlan and Krishnan\cite{McLachlan96} and
Watanabe and Yamaguchi\cite{Watanabe04}.  \marginpar{Want 2024 recent}
%
Recall that the algorithm operates on models $P_{\Y,\bS,\parameters} $
with parameters $\parameters$ for a mix of data that is observed
$(\Y)$ and data that is unobserved $(\bS)$ and that the steps in the
algorithm are:
\begin{enumerate}
\item Guess a starting value of $\ti{\parameters}{1}$, and set $n=1$.
\item \label{loop} Choose $\ti{\parameters}{n+1}$ to maximize
  an \emph{auxiliary function} $Q$
                                %
  \begin{equation}
    \label{eq:EMmax2}
    \ti{\parameters}{n+1} = \argmax_{\parameters} Q(\parameters,
    \ti{\parameters}{n})
  \end{equation}
  where
  \begin{equation}
    \label{eq:Qdef2}
    Q(\parameters',\parameters) \equiv \EV_{P\left(\bS|\y, \parameters \right)}
    \left( \log P(\y,\bS|\parameters') \right)
  \end{equation}
\item Increment $n$.
\item If not converged, go to \ref{loop}.
\end{enumerate}
Step \ref{loop} does all of the work.  Note that if the unobserved
data $(\bS)$ is discrete, then the auxiliary function $(Q)$ is
$\EV_{P\left(\bS|\y,\parameters \right)} \left( \log
  P(\y,\bS|\parameters') \right) = \sum_\s P(\s|\y,\parameters) \left(
  \log P(\y,\s|_\parameters') \right)$.  Although Dempster Laird and
Rubin \cite{Dempster77} called the characterization of $P
\left(\bS|\y,\parameters \right)$ the \emph{estimation step} and the
optimization of $Q(\parameters, \ti{\parameters}{n})$ over
$\parameters$ the \emph{maximization step}, the steps are now referred
to as \emph{expectation} and \emph{maximization}.

To illustrate the EM algorithm, we generated ten observations from an
\iid Gaussian mixture model \index{Gaussian mixture model}
\begin{equation}
  \label{eq:GaussMix}
  P\left(\ts{y}{0}{T}|\parameters \right) = \prod_{t=1}^T
  \frac{1}{\sqrt{2\pi}} \left( \lambda e^{-\frac{1}{2}
  (\ti{y}{t}-\mu_1)^2} +
  (1-\lambda) e^{-\frac{1}{2} (\ti{y}{t}-\mu_2)^2 } \right)
\end{equation}
with a vector of free parameters $\parameters = (\lambda,\mu_1,\mu_2)$ set
to $(\frac{1}{2},-2,2)$.  (See the plot in Fig.~\ref{fig:GaussMix}.)
Then we tried to recover the parameters from the observations using
the EM algorithm.  The model generates an observation by:
\begin{enumerate}
\item Selecting one of two states or classes $s$ with probability
  $\lambda$ for state $s=0$ and probability $(1-\lambda)$ for state $s=1$
\item Drawing from a Gaussian with mean $\mu_s$ and variance $\sigma^2
  = 1$
\end{enumerate}
Note that for each observation, the class $s$ is unobserved.

As an initial model we set $\ti{\parameters}{1} = \left( \frac{1}{2}, -1, 1
\right)$.  Now for each $\ti{\parameters}{n}$ the EM algorithm consists of
\begin{description}
\item[E-step] For each observation $\ti{y}{t}$, \emph{estimate}
  $\ti{w}{t}$, the probability that the state was $s=0$ given the
  current model parameters and the observation, \ie, $\ti{w}{t} =
  P_{\ti{s}{t}|\ti{y}{t},\ti{\parameters}{n}} \left(0|\ti{y}{t},
    \ti{\parameters}{n}\right)$.
\item[M-step] Adjust the parameters to \emph{maximize} the auxiliary
  function.  Letting $W = \sum_{t=1}^{10} \ti{w}{t}$, some effort
  verifies that the following assignments maximize
  $Q(\parameters,\ti{\parameters}{n})$:
  \begin{itemize}
  \item $\ti{\lambda}{n+1} = \frac{1}{10} W$
  \item $\ti{\mu_1}{n+1} = \frac{1}{W}\sum_{t=1}^{10} \ti{y}{t} \ti{w}{t}$
  \item $\ti{\mu_2}{n+1} = \frac{1}{10-W}\sum_{t=1}^{10} \ti{y}{t} (1-\ti{w}{t})$
  \end{itemize}
\end{description}

Intermediate calculations and density plots for the first two
iterations of the EM algorithm appear in Fig.~\ref{fig:GaussMix}.

\begin{figure}[htbp]
  \centering{\plotsize%
    %%
    %% Data generated by gauss_mix.py
    %%
    \input{gauss_mix_weights} \bigskip \\
    \input{gauss_mix_theta}
    \resizebox{\textwidth}{!}{\includegraphics{GaussMix.pdf}}
  }
  \caption[\comment{fig:GaussMix }Two iterations of the EM
  algorithm.]%
  {Two iterations of the EM algorithm.  We use the algorithm to search
    for parameters of Eqn.~\eqref{eq:GaussMix} that maximize the
    likelihood of the ten simulated observations that appear in the upper
    table in the row labeled $\ti{y}{t}$.  The following rows of the
    upper table report the weighting used for recalculating the
    conditional means for two iterations of the EM algorithm.
    Parameters for the first three models appear in the lower table.
    In the top plot, we illustrate $P(x|\parameters)$ as defined in
    Eqn.~\eqref{eq:GaussMix} for two sets of model parameters: The
    distribution with $\parameters = (0.5, -2, 2)$ that we used to simulate
    the data, and the distribution with
    $\ti{\parameters}{1} = (0.5, -1, 1)$ that we used to initialize
    the EM algorithm.  In the bottom plot the simulated observations
    appear as marks on the abscissa and $P(y|\ti{\parameters}{2})$ and
    $P(y|\ti{\parameters}{3})$ appear as solid lines.}
  \label{fig:GaussMix}
\end{figure}
\afterpage{\clearpage}%% Print this right here please.

The advantages of the EM algorithm are that it is easy to implement
and it monotonically increases the likelihood.  These often outweigh
its slow convergence and the fact that it calculates neither the
second derivative of the likelihood function nor any other indication
of the reliability of the results it reports.  Proving monotonicity is
simple.  If the likelihood is bounded, convergence follows directly
from monotonicity, but convergence of the parameters does not follow.
Also, the likelihood might converge to a local maximum.  Papers by
Baum et al.\cite{Baum70}, Dempster, Laird, and Rubin\cite{Dempster77},
and Wu\cite{Wu83} analyze the issues.  In the next two subsections we
summarize and augment some of that analysis.

\subsection{Monotonicity}

Denoting the log likelihood of the observed data given the model
$\parameters'$ as
\begin{equation*}
  L(\parameters') \equiv \log \left( P \left(\y|\parameters' \right)\right)
\end{equation*}
and the cross entropy of the unobserved data with respect to a model
$\parameters'$ given a model $\parameters$ as
\begin{equation*}
  H(\parameters, \parameters') \equiv - \EV_{P
    \left(\bS|\y,\parameters \right)} \left( \log P(\bS|\y,\parameters')
  \right),
\end{equation*}
we can write the auxiliary function as
\begin{align*}
  Q(\parameters',\parameters) &\equiv \EV_{P \left(\bS|\y ,\parameters\right)}
  \left( \log P(\bS,\y|\parameters') \right)\\
  &= \EV_{P \left(\bS|\y ,\parameters \right)} \left( \log
    P(\bS|\y,\parameters') + \log \left( P(\y|\parameters') \right)
  \right)\\
  &= L(\parameters') - H(\parameters, \parameters')
\end{align*}
or
\begin{equation}
  \label{eq:LQH}
   L(\parameters') = Q(\parameters',\parameters) + H(\parameters,
   \parameters').
\end{equation}
The fact that
\begin{equation}
  \label{eq:GibbsIE}
  H(\parameters, \parameters') \geq  H(\parameters, \parameters)
  \forall \parameters'
\end{equation}
with equality \emph{iff}
\begin{equation*}
  P(\s|\y,\parameters') =  P(\s|\y,\parameters) ~~ \forall \s
\end{equation*}
is called the \emph{\index*{Gibbs Inequality}}\footnote{While many
  statisticians attribute this inequality to Kullback and
  Leibler\cite{Kullback51}, it appears earlier in chapter {\em XI}
  {\em Theorem II} of Gibbs\cite{Gibbs}.}.  It is a consequence of
\index{Jensen's inequality} Jensen's inequality.

Given the model $\ti{\parameters}{n}$ after $n$ iterations of the EM
algorithm, we can write
\begin{equation}
  L(\ti{\parameters}{n}) = Q(\ti{\parameters}{n}, \ti{\parameters}{n}) +
  H(\ti{\parameters}{n}, \ti{\parameters}{n}).
\end{equation}
If for some other model $\parameters'$
\begin{equation}
  \label{eq:GEMcond}
  Q(\parameters',\ti{\parameters}{n}) >
  Q(\ti{\parameters}{n},\ti{\parameters}{n}),
\end{equation}
then the Gibbs inequality implies $H(\ti{\parameters}{n}, \parameters') \geq H(\ti{\parameters}{n},
\ti{\parameters}{n})$ and consequently, $L(\parameters') >
L(\ti{\parameters}{n})$.  Monotonicity of the log function further
implies
\begin{equation*}
  P \left(\y |\parameters'\right) > P \left(\y|\ti{\parameters}{n} \right) .
\end{equation*}
Since the EM algorithm requires the inequality in \eqref{eq:GEMcond},
for $\parameters' = \ti{\parameters}{n+1}$, the algorithm
monotonically increases the likelihood.

\subsection{Convergence}

\ToDo{Do something different with this section.  Perhaps examples of
  degenerate models (flat directions in parameter space)}

\newcommand{\EMmap}{{\cal T}} %
The EM algorithm operates on $\Theta$, a set of allowed parameter
vectors with the function $\EMmap:\Theta \mapsto \Theta$ that
implements an iteration of the algorithm, \ie,
\begin{equation*}
  \ti{\parameters}{n+1} = \EMmap (\ti{\parameters}{n}).
\end{equation*}
Wu\cite{Wu83} has observed that more than one value of $\parameters'$
may maximize $Q(\parameters',\parameters)$.  Consequently he
considered $\EMmap$ to be a point to set map.  However, there are many
model classes --including all of the HMMs the we describe in this
book-- for which one can write algorithms that calculate a unique
$\EMmap(\theta)$.  Thus we consider the case of $\EMmap$ being a
simple function.  If there is a bound\footnote{See
  Section~\ref{sec:regularization} for an example involving
  probability densities with no such bound.  For discrete $\y$,
  however the bound $ P \left(\y|\parameters \right) \leq 1$ holds.}
$\bar P$ on $ P \left(\y|\parameters \right) $ with
\begin{equation*}
   P \left(\y|\parameters \right) \leq \bar P ~ \forall \parameters \in \Theta
\end{equation*}
then the monotonicity of the sequence $\left(
  P(\y|\ti{\parameters}{1}),
  P(\y|\ti{\parameters}{2}), P(\y|\ti{\parameters}{3}), \ldots \right)$ and the bounded
convergence theorem ensures that the limit $ P^* \equiv
\lim_{n\rightarrow \infty} P \left(\y |\ti{\parameters}{n} \right) $
exists.  The bounded convergence theorem does not promise that $P^* =
\sup_{\parameters \in \Theta} P \left(\y | \parameters \right)$ or
that it is even a local maximum, nor does it promise that $\theta^*
\equiv \lim_{n \rightarrow \infty} \ti{\parameters}{n}$ exists.

It would be nice if every trajectory of $\EMmap$ converged to local
maximum of the likelihood.  Since Wu\cite{Wu83} provides an example of
a trajectory that converges to a saddle point of the likelihood, it
certainly is not true.  In the following, we do however give assurance
that, in typical circumstances, trajectories converge to local maxima.

We assume that $\Theta$ is compact and that the functions $L:\Theta
\mapsto \REAL$, $H:\Theta \times \Theta \mapsto \REAL$, and $Q:\Theta
\times \Theta \mapsto \REAL$ are $C^2$, \ie, each function and its
first two derivatives exist and are continuous.  For any starting
point $\ti{\parameters}{1}$, we are interested in the \emph{limit set}
$\Omega(\ti{\parameters}{1})$ defined as the set of points to which
subsequences converge, \ie, for all $\parameters \in
\Omega(\ti{\parameters}{1})$, there is a subsequence of integers
$\left( N(1), N(2), \ldots \right)$ with $N(n+1) > N(n)~ \forall
n$ such that $\lim_{n \rightarrow \infty}
\EMmap^{N(n)}(\ti{\parameters}{1}) = \parameters$.  If $\EMmap$ maps a
compact neighborhood of $\ti{\parameters}{1}$ into itself, then
$\Omega(\ti{\parameters}{1}) \neq \emptyset$.  For discrete HMMs the
assumptions are valid for all points in the interior of $\Theta$.
Continuity and Gibbs inequality ensure that each element of a limit
set has the same log likelihood $L^*$ and that the cross entropy
between pairs of limit points is a constant $H^*$.  Consequently, the
conditional distribution of the unobserved data is constant over all
limit points; call it $P^*(\bS|\y)$.  Also the continuity assumption
ensures that at each point the first derivative of $L$ is zero.  In
summary:
\begin{equation}
  \label{eq:OmegaFlat}
  \forall \parameters \in \Omega(\ti{\parameters}{1})
  \begin{cases}
    L(\parameters) = L^*\\
    H(\EMmap(\parameters), \parameters) = H(\parameters, \parameters) = H^* \\
    P(\s|\y,\parameters) = P^*(\s|\y)~\forall \s \\
    \frac{\partial L(\parameters)}{\partial \parameters} = 0
  \end{cases}
\end{equation}

The limit set must be one of the following types: A single point
$\parameters^*$; A periodic orbit; Or an infinite number of points
none of which is isolated.  Consider each possibility in turn:

\subsubsection{$\Omega = \{\parameters^*\}$ is a single point}
\label{sec:EMsingle}

If $N$ is the dimension of $\Theta$, the Hessian of the log likelihood
is the $N\times N$ matrix $\left. \frac{\partial^2 L(\parameters) }{
    \partial \parameters^2 }\right|_{\parameters^*}$.  Of the $N$
eigenvalues of the Hessian, if $N_+$ are positive, $N_z$ are zero and
$N_-$ are negative, then the triple $(N_+,N_z,N_-)$ is called the
\emph{inertia} of the Hessian.  The following three possibilities are
exhaustive.
\begin{description}
\item[$N_- = N$] This is the desirable case.  The first derivative
  is zero, and the second derivative is negative definite.  These
  are the necessary and sufficient conditions for $\parameters^*$ to
  be a local maximum of $L$.  The analysis of $\EMmap$ below shows
  that such limit points are linearly stable.  Thus nonzero volumes
  of initial conditions in $\Theta$ converge to them.
\item[$N_z >0$] Although this can happen for neighborhoods of values
  for distributions such as the Laplacian ($P(x) = \frac{\lambda}{2}
  e^{-\left|x-\mu\right|}$), it is not generic\footnote{By \emph{not
      generic}, we mean an event that only occurs on a set of
    Lebesgue measure zero.} for any of the distributions we consider
  in this book.
\item[$N_+ >0$] Trajectories that converge to $\parameters^*$ must
  stay on the stable manifold.  Close to $\parameters^*$, if a
  trajectory has \emph{any} component in the unstable subspace, it
  will diverge from $\parameters^*$ exponentially.  The analysis of
  $\EMmap$ below shows that the dimension of the linearly unstable
  subspace is $N_+$.  Thus if $N+ >0$, convergence to
  $\parameters^*$ is not generic.
\end{description}
In summary, convergence to local maxima of the likelihood is generic
and convergence to saddle points of the likelihood is not.

The linear stability of $\EMmap$ at an isolated fixed point
$\parameters^*$ is determined by the eigenvalues
$\left\{\lambda_k:1\leq k \leq N \right\}$ of the $N\times N$ matrix
$\left. \frac{\partial \EMmap(\parameters) }{ \partial \parameters
  }\right|_{\parameters^*}$.  The fixed point is linearly unstable if
any eigenvalue has magnitude greater than one ($\left| \lambda_k
\right| > 1$). We write\footnote{Equation \eqref{eq:DEM}, which
  appears in the original EM paper by Dempster et al.\, can be derived
  by considering the Taylor series expansion of $\frac{\partial
    Q(\parameters,\parameters)}{\partial \parameters}$ about
  $\parameters^*$
  \begin{equation*}
    \left. \frac{\partial Q(\parameters',\parameters)}{\partial \parameters'}
    \right|_{\parameters_a,\parameters_b} = \left. \frac{\partial
        Q(\parameters',\parameters)}{\partial \parameters'}
    \right|_{\parameters^*,\parameters^*} %
    + \left. \frac{\partial^2 Q(\parameters',\parameters)}{\partial
        \parameters' \partial \parameters} \right|_{\parameters^*,\parameters^*}
    (\parameters_b -  \parameters^*) %
    + \left. \frac{\partial^2 Q(\parameters',\parameters)}{\partial
        \parameters'^2} \right|_{\parameters^*,\parameters^*}
    (\parameters_a - \parameters^*) + R,
  \end{equation*}
  where $R$ is a remainder.  Substituting $\ti{\parameters}{n}$ for
  $\parameters_b$ and $\EMmap(\ti{\parameters}{n})$ for
  $\parameters_a$ we find that to first order
  \begin{equation*}
    0 = 0 + \left. \frac{\partial^2 Q(\parameters',\parameters)}{\partial
        \parameters' \partial \parameters} \right|_{\parameters^*,\parameters^*}
    (\ti{\parameters}{n} - \parameters^*) %
    + \left. \frac{\partial^2 Q(\parameters',\parameters)}{\partial
        \parameters'^2} \right|_{\parameters^*,\parameters^*} (\EMmap(\ti{\parameters}{n}) -
    \parameters^*)
  \end{equation*}
  and
  \begin{equation}
    \label{eq:FN1}
    \left. \frac{ \partial \EMmap(\parameters)}{\partial \parameters} \right|_{\parameters^*} = %
    \left[- \left. \frac{\partial^2 Q(\parameters',\parameters)}{\partial
          \parameters'^2} \right|_{\parameters^*,\parameters^*} \right]^{-1} %
    \left. \frac{\partial^2 Q(\parameters',\parameters)}{\partial
        \parameters' \partial \parameters} \right|_{\parameters^*,\parameters^*}.
  \end{equation}
  Direct differentiation of the definition of $H$ yields
  \begin{equation*}
    \left. \frac{\partial^2 H(\parameters' || \parameters)}{\partial
        \parameters'^2} \right|_{\parameters^*,\parameters^*} %
    = \left.- \frac{\partial^2 H(\parameters' || \parameters)}{\partial
        \parameters' \partial \parameters} \right|_{\parameters^*,\parameters^*}.
  \end{equation*}
  Combining this with $Q(\parameters',\parameters) = L(\parameters) -
  H(\parameters' || \parameters)$, we find
  \begin{equation*}
    \left. \frac{\partial^2 Q(\parameters',\parameters)}{\partial \parameters'
        \partial \parameters} \right|_{\parameters^*,\parameters^*} = %
    \left. - \frac{\partial^2 H(\parameters' || \parameters)}{\partial
        \parameters' \partial \parameters} \right|_{\parameters^*,\parameters^*} = %
    \left. \frac{\partial^2 H(\parameters' || \parameters)}{\partial
        \parameters'^2} \right|_{\parameters^*,\parameters^*} =%
    \frac{\partial^2}{\partial \parameters'^2} \left[ L(\parameters') -
      Q(\parameters',\parameters) \right]_{\parameters^*,\parameters^*},
  \end{equation*}
  and substituting back into \eqref{eq:FN1} yields \eqref{eq:DEM}.}
the derivative of $\EMmap$ in terms of derivatives of $L$ and $Q$
\begin{equation}
  \label{eq:DEM}
  \left. \frac{\partial \EMmap(\parameters)}{\partial \parameters}
  \right|_{\parameters^*} = \id -
  \left[ \left. \frac{\partial^2 Q(\parameters',\parameters)}{\partial
        \parameters'^2} \right|_{\parameters^*,\parameters^*} \right]^{-1}
  \left[ \left. \frac{\partial^2 L(\parameters)}{\partial
        \parameters^2} \right|_{\parameters^*} \right].
\end{equation}
Note that $\argmax_\parameters Q(\parameters^*,\parameters) =
\parameters^*$ because $\parameters^*$ is a fixed point of $\EMmap$,
and generically the Hessian $\left[ \left.  \frac{\partial^2
      Q(\parameters,\parameters)}{\partial \parameters^2}
  \right|_{\parameters^*,\parameters^*} \right]$ will be negative
definite.  It follows that the term $A \equiv \left[ -\left.
    \frac{\partial^2 Q(\parameters,\parameters)}{\partial
      \parameters^2} \right|_{\parameters^*,\parameters^*}
\right]^{-1}$ in Eqn.~\eqref{eq:DEM} is positive definite.

If $A$ and $B$ are symmetric and $A$ is positive definite, then one
can \index{Sylvester's inertia theorem} use Sylvester's inertia
theorem\footnote{Gerardo Lafferriere suggested to us the analysis using
  Sylvester's inertia theorem which says that for any symmetric matrix
  $M$ and invertible $X$, both $M$ and $X\transpose M X$ have the same
  inertia.  Because $A^{-1}$ is positive definite, it can be factored
  as $A^{-1} = X X\transpose$.  For this factorization, $A =
  (X\transpose)^{-1}X^{-1}$, and
  \begin{equation}
    \label{eq:inertia}
    X\transpose A B X = X^{-1} B X.
  \end{equation}
  The right hand side of \eqref{eq:inertia} is a similarity
  transformation which preserves the inertia of $B$, while the left
  hand side has the inertia of $AB$.  Thus the inertia of $AB$ is the
  same as the inertia of $B$.} to show that the inertia of $AB$ is
equal to the inertia of $B$. At a saddle point of $L$, $B= \left[
  \left. \frac{\partial^2 L(\parameters)}{\partial
      \parameters^2} \right|_{\parameters^*} \right]$ will have at
least one positive eigenvalue and the product $AB$ will also have at
least one positive eigenvalue $\lambda^+$.  For the eigenvector $v^+$
of $AB$ corresponding to $\lambda^+$,
\begin{equation*}
  (\id +AB) v^+ = (1 + \lambda^+) v^+ .
\end{equation*}
Since $(1+\lambda^+)$ is greater than one and is an eigenvalue of
$\left. \frac{\partial \EMmap(\parameters)}{\partial \parameters}
\right|_{\parameters^*}$, the saddle is an unstable fixed point of
$\EMmap$, and convergence to it is not generic.

Similarly, if $\parameters^*$ is an isolated local maximum of $L$ with
negative definite $B= \left[ \left.  \frac{\partial^2
      L(\parameters)}{\partial \parameters^2} \right|_{\parameters^*}
\right]$, then $AB$ is also negative definite, and each eigenvalue of
$\id + AB$ is less than one.  Negative definiteness of $B$ combined
with the monotonicity of the EM algorithm also implies that every
eigenvalue is greater than $-1$.  Thus $\left| \lambda \right| <1$ for
each eigenvalue $\lambda$ of $\left.  \frac{\partial
    \EMmap(\parameters)}{\partial \parameters}
\right|_{\parameters^*}$, and the maximum is a linearly stable fixed
point of $\EMmap$.  Linear stability implies that for every
$\ti{\parameters}{1}$ in some neighborhood of $\parameters^*$ the
sequence $\left( \ti{\parameters}{1}, \EMmap(\ti{\parameters}{1}),
  \EMmap^2(\ti{\parameters}{1}), \ldots \right)$ converges to
$\parameters^*$.

In summary: Convergence to fixed point of $\EMmap$ that is a local
maximum of the likelihood is generic and convergence to a saddle point
of the likelihood is not generic.

\subsubsection{$\Omega $ is a periodic orbit}
\label{sec:EMperiodic}

At a periodic limit point $\parameters^*$ with period $n$, an analysis
parallel to the derivation of Eqn.~\eqref{eq:DEM} yields
\begin{equation*}
  \label{eq:DEM-periodic}
  \left. \frac{\partial \EMmap^n(\parameters)}{\partial \parameters}
  \right|_{\parameters^*} = \id -
  \left[ \left. \frac{\partial^2 Q(\parameters',\parameters)}{\partial
        \parameters'^2} \right|_{\parameters^*,\parameters^*} \right]^{-1}
  \left[ \left. \frac{\partial^2 L(\parameters)}{\partial
        \parameters^2} \right|_{\parameters^*} \right],
\end{equation*}
and we conclude by the same arguments as for the period one case that
convergence to periodic limit points is generic if they are local
maxima of the likelihood and not generic if they are saddle points of
the likelihood.

\subsubsection{$\Omega $  has an infinite number of points}
\label{sec:EMstrange}

We have not analyzed the possibility of a limit set with an infinite
number of points none of which are isolated.  To detect such behavior,
one could check numerical trajectories of $\EMmap$ for convergence in
both $\parameters$ and $L$.

\subsubsection{A contrived example}
\label{sec:contrived}

\ToDo{Look at analysis of second derivatives in section 2.5 of
https://arxiv.org/pdf/1105.1476.  Also, consider using his notation.}

Figure~\ref{fig:EM} illustrates stable convergence for the case of a
coin being tossed 4 times and observed twice.  By pretending that the
unobserved data is important and handling it via the EM algorithm, we
find that the algorithm implements an obviously stable \emph{linear}
map $\EMmap$.

\begin{figure}[htbp]
  \centering{\resizebox{\textwidth}{!}{\includegraphics{EM.pdf}}
  }
  \caption[\comment{fig:EM }An illustration of the EM algorithm.]%
  {An illustration of the EM algorithm for an experiment in which a
    coin is thrown four times, first a head is observed
    $(\ti{y}{1}=1)$, then a tail is observed $(\ti{y}{2}=0)$, and
    finally two results are unobserved with $s_h$ and $s_t$ being the
    number of unobserved heads and tails respectively.  The goal is to
    find the maximum likelihood value of $\parameters$, the
    probability of heads.  The log likelihood function for the
    complete data is $L_{\parameters} = (s_h + 1)\log(\parameters) +
    (s_t +1)\log(1-\parameters)$.  The auxiliary function
    $Q(\parameters',\parameters) = (1+2\parameters)\log(\parameters')
    + (1 + 2(1-\parameters))\log(1-\parameters')$ appears on the left,
    and the map $\EMmap(\parameters)$ appears on the right.  Note that
    $\parameters^* = \frac{1}{2}$ is the fixed point of $\EMmap$
    (where the plot intersects the slope 1 reference line) and it is stable
    because the slope of $\EMmap$ is less than one.}
  \label{fig:EM}
\end{figure}

%%% Local Variables:
%%% TeX-master: "main"
%%% eval: (load-file "hmmkeys.el")
%%% mode: LaTeX
%%% End:

\input{SGO_values.tex}
\chapter{Variants and Generalizations}
\label{chap:variants}

Hidden Markov models are special cases of discrete time state space
models characterized by a state transition probability function and an
observation probability function, \ie,
\begin{subequations}
  \label{eq:ssm}
  \begin{align}
      &P_{\State_{n+1}|\State_n}\text{, and }\\
      &P_{\Output_n|\State_n}.
  \end{align}
\end{subequations}
In Chapter~\ref{chap:algorithms} we described algorithms for fitting
and using the probability distributions specified in
Eqn.~\eqref{eq:ssm} if both the set of possible states $\states$ and
the set of possible observations $\outputs$ have an unstructured
finite number of discrete values.  However, in many applications the
measurements, and perhaps the states also, are thought of as being
drawn from continuous vector spaces.

Since most experimental observations are measured and recorded
digitally, one could argue that discrete approximations are adequate
and attempt to use the algorithms of Chapter~\ref{chap:algorithms}
anyway.  That approach is disastrous because it precludes exploiting
either the metric or topological properties of the space of
measurements.  Consider the histogram of the first 600 samples of
Tang's laser data in Fig.~\ref{fig:LaserHist}.  Neither 5 nor 93
occurs, but it seems more plausible that 93 will occur in the
remainder of the samples because there are 14 occurrences between 90
and 96 and none between 2 and 8.  To make more effective use of
measured data, one usually approximates the probabilities by functions
with a small number of free parameters.  For many such families of
\emph{parametric models} one can use the algorithms of
Chapter~\ref{chap:algorithms} with minor modifications\footnote{At the
  1988 ICASSP meeting, Poritz\cite{Poritz88} reviewed several HMM
  variants.}.  For a practitioner, the challenge is to find or develop
both a parametric family that closely matches the measured system and
algorithms for fitting and using the models.

\begin{figure}[htbp]
  \centering{\plotsize%
    \includegraphics[width=1.0\textwidth]{LaserHist.pdf}}
  \caption[\comment{fig:LaserHist }Histogram of Tang's laser measurements.]%
  {Histogram of Tang's laser measurements.  Even though neither $y=5$
    nor $y=93$ occurs in $\ts{y}{1}{600}$, it is more plausible that
    $y=93$ would occur in future measurements because of what happens
    in the neighborhood.  Discarding the numerical significance of the
    bin labels would preclude such an observation. }
  \label{fig:LaserHist}
\end{figure}

In this chapter we will describe some model families with Gaussian
observations.  We will use the failure of the maximum likelihood
approach with such models to motivate and develop
\emph{\index*{regularization}}.  Also, we will touch on the
relationships between HMM model families and other kinds of models.

\section{Gaussian Observations}
\label{sec:gaussian}
\index{Gaussian observation}

\subsection{Independent Scalar Observations}
\label{sec:ScalarGaussian}

A simple model for continuously distributed measurements is an HMM
with an independent scalar Gaussian observation model associated with each
state.  In many cases it is adequate, but risky (See
Fig.~\ref{fig:MLEfail}), to simply use the algorithms of
Chapter~\ref{chap:algorithms} with minor modifications for Gaussian
observations.  Such algorithms performed satisfactorily for the exercises
depicted in Fig.~\ref{fig:ScalarGaussian} in which we estimated an
approximate state sequence and model parameters from a sequence of
observations.

%%%
%%% fig:ScalarGaussian
%%%
\begin{figure}[htbp]
  \centering{\plotsize%
    \setlength{\unitlength}{1in}%
    \begin{tabular}[H]{cc}
      \begin{picture}(0,0)
        \put(-0.1,0.8){\makebox{\normalsize\textbf{(a)}}}
      \end{picture}%
      {
      \def\prba{$\ScalarGaussianaPaa$}
      \def\prbb{$\ScalarGaussianaPba$}
      \def\prbc{$\ScalarGaussianaPab$}
      \def\prbd{$\ScalarGaussianaPbb$}
      \def\lbla{\parbox[t]{1.8in}{$\mu=\ScalarGaussianamua$\\$\sigma^2=\ScalarGaussianavara$}}%
      \def\lblb{\parbox[t]{1.8in}{$\mu=\ScalarGaussianamub$\\$\sigma^2=\ScalarGaussianavarb$}}%
      \input{ScalarGaussian.pdf_t}
      }%
      %\smallskip%
      &
      \begin{picture}(0,0)
        \put(0.0,0.8){\makebox{\normalsize\textbf{(b)}}}
      \end{picture}%
      \hspace{2em}
      \resizebox{0.37\textwidth}{!}{\includegraphics{SGO_b.pdf}}
      %
      %\smallskip%
      \\
      \begin{picture}(0,0)
        \put(0.0,0.8){\makebox{\normalsize\textbf{(c)}}}
      \end{picture}%
      \hspace{2em}\resizebox{0.37\textwidth}{!}{\includegraphics{SGO_c.pdf}}%
      \smallskip%
      &
      \begin{picture}(0,0)
        \put(0.0,0.8){\makebox{\normalsize\textbf{(d)}}}
      \end{picture}%
      \hspace{2em}\resizebox{0.37\textwidth}{!}{\includegraphics{SGO_d.pdf}}%
      %\smallskip%
      \\
      \begin{picture}(0,0)
        \put(-0.1,0.8){\makebox{\normalsize\textbf{(e)}}}
      \end{picture}%
      {
      \def\prba{$\ScalarGaussianePaa$}
      \def\prbb{$\ScalarGaussianePba$}
      \def\prbc{$\ScalarGaussianePab$}
      \def\prbd{$\ScalarGaussianePbb$}
      \def\lbla{\parbox[t]{1.8in}{$\mu=\ScalarGaussianemua$\\$\sigma^2=\ScalarGaussianevara$}}%
      \def\lblb{\parbox[t]{1.8in}{$\mu=\ScalarGaussianemub$\\$\sigma^2=\ScalarGaussianevarb$}}%
      \input{ScalarGaussian.pdf_t}
      }
      &
      \begin{picture}(0,0)
        \put(-0.1,0.8){\makebox{\normalsize\textbf{(f)}}}
      \end{picture}%
      {
      \def\prba{$\ScalarGaussianfPaa$}
      \def\prbb{$\ScalarGaussianfPba$}
      \def\prbc{$\ScalarGaussianfPab$}
      \def\prbd{$\ScalarGaussianfPbb$}
        \def\lbla{\parbox[t]{1.8in}{$\mu=\ScalarGaussianfmua$\\$\sigma^2=\ScalarGaussianfvara$}}%
        \def\lblb{\parbox[t]{1.8in}{$\mu=\ScalarGaussianfmub$\\$\sigma^2=\ScalarGaussianfvarb$}}%
        \input{ScalarGaussian.pdf_t}
      }
      \end{tabular}}
  \caption[\comment{fig:ScalarGaussian }An HMM with scalar Gaussian observations.]%
  {An HMM with scalar Gaussian observations.  A state diagram appears in
    \emph{(a)}.  The half-life of the first state is about ten and the half
    life of the second state is about five, \ie, $\ScalarGaussianaPaa^{10} \approx
    \ScalarGaussianaPbb^5 \approx 0.5$.  A simulated state sequence and observation
    sequence appear in \emph{(b)} and \emph{(c)} respectively.  Using the model
    parameters from \emph{(a)} and the observation sequence from \emph{(c)}, the
    Viterbi algorithm estimates the state sequence that appears in \emph{(d)}
    which is satisfyingly similar to the state sequence in \emph{(b)}.
    Finally, starting from the initial model depicted in \emph{(e)} and using
    the observation sequence depicted in \emph{(c)}, 50 iterations of the
    Baum-Welch algorithm produces the model depicted in \emph{(f)} which is
    satisfyingly similar to \emph{(a)}.}
  \label{fig:ScalarGaussian}
\end{figure}

The code that generated the data for Fig.~\ref{fig:ScalarGaussian}
implemented algorithms from Chapter~\ref{chap:algorithms} with the
following modifications:
%%\newpage%%% Look for awkward vertical whitespace under fig:ScalarGaussian
\begin{description} 
\item[$\bm{P_{\ti{Y}{t}|\ti{S}{t}} \left(y|s \right)}$] The Viterbi
  algorithm, the forward algorithm, and the backward algorithm all use
  the observation probability conditioned on the state. In each case one
  simply uses the value of the probability density conditioned on the
  state
  \begin{equation*}
    P_{\ti{Y}{t}|\ti{S}{t}} \left(y|s \right) = \frac{1}{\sqrt{2\pi
    \sigma_s^2}} e^{-\frac{(y-\mu_s)^2}{2\sigma_s^2}}.
  \end{equation*}
\newpage%%% Look for awkward vertical whitespace under fig:ScalarGaussian
\item[Reestimation] Reviewing the derivations in Section
  \ref{sec:reestimation}, we find that the first two formulas in Table
  \ref{tab:reestimation} (those for the initial state probability, %
  $P_{\ti{S}{1}|\ti{\parameters}{n+1}} \left(i|\ti{\parameters}{n+1} \right)$, 
  and the state transition probability,
  $P_{\ti{S}{t+1}|\ti{S}{t},\ti{\parameters}{n+1}} \left({\tilde s}|s \right)$) 
  still work with the new observation model.  To derive reestimation
  formulas for the Gaussian observation model parameters, note that
  Eqn.~\eqref{eq:Loutput} becomes %
  \begin{align}
    L_{\text{observation}}(y,s) & \equiv \log
  P_{\ti{Y}{1}|\ti{S}{1},\parameters'} \left(y|s,\parameters' \right)\\
    \label{eq:LoutputG}
    &= -\frac{1}{2} \log (2\pi) - \log(\sigma_s) -
    \frac{(y-\mu_s)^2}{2\sigma_s^2},
  \end{align}
  and starting from Eqn.~\eqref{eq:Qout1} calculate
\begin{align}
  \label{eq:QoutG1}
  Q_{\text{observation}} (\parameters',\parameters) &=
  \sum_{\ts{q}{1}{T}\in\states^T} P_{\parameters}
  \left(\ts{q}{1}{T}|\ts{y}{1}{T} \right) \sum_{t=1}^T
  L_{\text{observation}}(\ti{y}{t},\ti{q}{t})\\
  \label{eq:QoutG2}
  &= \sum_{s\in\states} \sum_{t=1}^T L_{\text{observation}}(\ti{y}{t},s)
  \sum_{\ts{q}{1}{T}:\ti{q}{t}=s }P_{\parameters}
  \left(\ts{q}{1}{T}|\ts{y}{1}{T} \right) \\
  \label{eq:QoutG3}
  &= - \sum_{s\in\states} \sum_{t=1}^T w(s,t) \left( \frac{1}{2}
    \log(2\pi) + \log(\sigma_s) + \frac{\left( \ti{y}{t}-\mu_s
      \right)^2}{2\sigma_s^2} \right).
\end{align}
Since the formulas
\begin{align}
  \label{eq:NewMuG}
  \mu_s &= \sum_{t=1}^T w(s,t) \ti{y}{t} \\
  \sigma_s^2 &= \sum_{t=1}^T w(s,t)\left( \ti{y}{t}-\mu_s \right)^2
\end{align}
maximize $Q_{\text{observation}} (\parameters',\parameters)$, we use them in place
of the discrete observation reestimation formula of
Chapter~\ref{chap:algorithms} (Table \ref{tab:reestimation} and
Eqn.~\eqref{eq:NewOut}).
%
\end{description}


\subsection{Singularities of the likelihood function and regularization}
\label{sec:regularization}

Running \MLEfailIterations~iterations of the Baum-Welch algorithm on
the observation sequence in Fig.~\ref{fig:ScalarGaussian}~(c) starting
with the model in Fig.~\ref{fig:MLEfail}~(a) produces the model in
Fig.~\ref{fig:MLEfail}~(b) in which the variance of the observations
produced by the second state looks suspiciously small.  In fact with
additional iterations of the Baum-Welch algorithm that variance
continues to shrink, and the code soon stops with a floating point
exception.  The algorithm is pursuing a singularity in the likelihood
function in which the second state fits observation
$\ti{y}{\MLEfailt}$ exactly and the first state fits all of the other
observations.  If $\mu_2 = \ti{y}{\MLEfailt}$ the likelihood
$P_{\ti{Y}{t}|\ti{S}{t}} \left(\ti{y}{\MLEfailt}|2 \right)$ increases
without limit as $\sigma^2_2 \rightarrow 0$, \ie,
\begin{equation*}
  \lim_{\sigma^2_2 \rightarrow 0} P_{\ti{Y}{t}|\ti{S}{t}}
  \left(\ti{y}{\MLEfailt}|2 \right) = \infty.
\end{equation*}

\begin{figure}[htbp]
  \centering{\plotsize%
    \setlength{\unitlength}{1in}%
    \begin{tabular}[H]{cc}
      \begin{picture}(0,0)
        \put(-0.1,0.8){\makebox{\normalsize\textbf{(a)}}}
      \end{picture}%
      {
      \def\prba{$\MLEfailaPaa$}
      \def\prbb{$\MLEfailaPba$}
      \def\prbc{$\MLEfailaPab$}
      \def\prbd{$\MLEfailaPbb$}
      \def\lbla{\parbox[t]{1.8in}{$\mu=-2$\\$\sigma^2=2$}}%
      \def\lblb{\parbox[t]{1.8in}{$\mu=2$\\$\sigma^2=2$}}%
      \input{ScalarGaussian.pdf_t}
      }&
      \begin{picture}(0,0)
        \put(-0.1,0.8){\makebox{\normalsize\textbf{(b)}}}
      \end{picture}%
      {
      \def\prba{$\MLEfailbPaa$}
      \def\prbb{$\MLEfailbPba$}
      \def\prbc{$\MLEfailbPab$}
      \def\prbd{$\MLEfailbPbb$}
      \def\lbla{\parbox[t]{1.8in}{$\mu=\MLEfailbmua$\\$\sigma^2=\MLEfailbvara$}}%
      \def\lblb{\parbox[t]{1.8in}{$\mu=\MLEfailbmub$\\$\sigma^2=\MLEfailbvarb$}}%
      \input{ScalarGaussian.pdf_t}
      }%
    \end{tabular}}%
  \caption[\comment{fig:MLEfail }An illustration of trouble with
  maximum likelihood.]%
  {An illustration of trouble with maximum likelihood.  Here we have
    used the same implementation of the Baum-Welch algorithm that we
    used to produce Fig.~3.2\emph{(f)}, but rather than starting with
    the model in Fig.~\ref{fig:ScalarGaussian}~\emph{(c)}, we started
    the algorithm with the initial model depicted in (a) above.  After
    \MLEfailIterations~iterations of the algorithm we get the
    suspicious model depicted in \emph{(b)} above.}
  \label{fig:MLEfail}
\end{figure}

Such singularities of likelihood are common among parametric
probability density functions.  A particularly simple example is the
\emph{Gaussian mixture model}\index{Gaussian mixture model|textbf}
\begin{equation}
  \label{eq:GaussianMixture}
  f(y) = \lambda \frac{1}{\sqrt{2 \pi \sigma_1^2}}
  e^{-\frac{(y-\mu_1)^2}{2 \sigma_1^2}} + (1-\lambda) \frac{1}{\sqrt{2 \pi \sigma_2^2}}
  e^{-\frac{(y-\mu_2)^2}{2 \sigma_2^2}},
\end{equation}
which has the five parameters $\mu_1,~\sigma_1,~\mu_2,~\sigma_2,$ and $\lambda$.
Assuming that the data are \iid, one might attempt a maximum
likelihood fit to the observations in
Fig.~\ref{fig:ScalarGaussian}~(e) with the likelihood function
\begin{equation}
  \label{eq:GMLike}
  g(\mu_1,\sigma_1,\mu_2,\sigma_2,\lambda) = \prod_{t=1}^T f(\ti{y}{t}).
\end{equation}
While it is possible to find a useful \emph{local} maximum of $g$
near
\begin{equation*}
  \mu_1 = -1,~~\sigma_1 = 1,~~ \mu_2 = 1,~~ \sigma_2 = 1,~~ \lambda =
  \frac{2}{3},
\end{equation*}
the likelihood is higher near the singularities of $g$ specified by
the equations
\begin{align*}
  \mu_s &= \ti{y}{t}\\
  \sigma_s &= 0,
\end{align*}
for each pair $(s,t)\in \left\{1,2\right\} \times \left\{ 1,2,\ldots,T
\right\}$.

If, as is the case here, we want to exclude parameter vectors for
which the likelihood function is larger than its value at the solution
we prefer, then likelihood doesn't really express our goal.
\emph{Regularization} refers to a variation on maximum likelihood
that more accurately reflects what we want.  In the next subsection,
we explain how to use Bayesian \emph{priors} to regularize
\emph{maximum a posteriori} parameter estimates.

%%% \subsection{The EM algorithm for \emph{maximum a posteriori} estimation}
\subsection{The EM algorithm for maximum a posteriori estimation}
%%% variants.tex:297: [Font] Font shape `OT1/cmss/bx/it' undefined using `OT1/cmss/bx/n' instead. (page 51)
\label{sec:EMMAP}

\index*{Bayesian estimation} starts by characterizing the
acceptability of models $P_{\Y|\parameters}$ in terms of a
\emph{prior} probability distribution $P_{\parameters}$.  Initial
observations $\y_e$, called \emph{evidence} or training data, modify
the prior through Bayes rule to yield the \emph{a posteriori}
distribution
\begin{equation}
  \label{eq:posteriori}
  P(\parameters|\y_e) = \frac{P(\y_e,\parameters)}{P(\y_e)} =
  \frac{P(\y_e|\parameters)P(\parameters)}{\int P(\y_e|\parameters)P(\parameters)\,
  d\parameters},
\end{equation}
and the probability of future observations $\y_f$ is
\begin{equation}
  \label{eq:futureBayes}
  P(\y_f|\y_e) = \int  P(\y_f|\parameters) P(\parameters|\y_e)\, d\parameters.
\end{equation}
The parameter vector that \underline{m}aximizes the \emph{\underline{a}
  posteriori} \underline{p}robability, \index{maximum a posteriori
  estimate}
\begin{subequations}
  \begin{align}
    \label{eq:MAPdefa}
    \parameters_{\text{MAP}} &\equiv \argmax_\parameters P(\parameters|\y_e) \\
    \label{eq:MAPdefb}
    &= \argmax_\parameters P(\parameters,\y_e),
  \end{align}
\end{subequations}
is called the MAP estimate.  \index{MAP estimate|see{maximum a
    posteriori estimate}} Using $\parameters_{\text{MAP}}$ one may
approximate\footnote{Although it is every bit as reasonable to use the
  mean to characterize the \emph{a posteriori} distribution as it is to
  use the maximum, we prefer the maximum because changing from MLE to
  MAP requires only minor modifications to the Baum-Welch algorithm.
  A strictly Bayesian approach would retain the entire \emph{a
    posteriori} distribution in parameter space rather than
  characterizing the distribution by a single point estimate.}
Eqn.~\eqref{eq:futureBayes} with $P\left( \y_f|\parameters_{
    \text{MAP} } \right)$.

A slight variation of the algebra in Section~\ref{sec:EM} produces an
EM algorithm for MAP estimation.  Dropping the subscript on $\y_e$ if
we replace the auxiliary function of Eqn.~\eqref{eq:Qdef}, \ie,
\begin{equation*}
  Q_{\text{MLE}}(\parameters',\parameters) \equiv \EV_{P
    \left(\bS|\y,\parameters \right)} \left( \log P(\bS,\y|\parameters') \right),
\end{equation*}
with
\begin{subequations}
  \label{eq:QMAP}
  \begin{align}
  \label{eq:QMAPa}
    Q_{\text{MAP}}(\parameters',\parameters) &\equiv \EV_{P \left(\bS|\y,\parameters \right)}
    \left( \log P(\bS,\y,\parameters') \right) \\
  \label{eq:QMAPb}
    &= Q_{\text{MLE}}(\parameters',\parameters) + \EV_{P \left(\bS|\y,\parameters
      \right)} \left( \log P(\parameters')\right) \\
  \label{eq:QMAPc}
    &= Q_{\text{MLE}}(\parameters',\parameters) + \log P(\parameters'),
  \end{align} 
\end{subequations}
then the derivation of
\begin{equation*}
    Q_{\text{MAP}}(\parameters',\parameters) > Q_{\text{MAP}}(\parameters,\parameters)
    \Rightarrow P(\y,\parameters') > P(\y,\parameters),
\end{equation*}
is completely parallel to the argument on page \pageref{eq:GEMcond}
that concludes $Q_{\text{MLE}}(\parameters',\parameters) >
Q_{\text{MLE}}(\parameters,\parameters)$.  If in addition the
components of $P(\parameters)$ are independent, \ie,
\begin{equation*}
  P(\parameters) = P(\parameters_{\text{initial}}) \cdot
  P(\parameters_{\text{transition}}) \cdot P(\parameters_{\text{observation}}),
\end{equation*}
then like the decomposition of $Q_{\text{MLE}}$ in
Eqn.~\eqref{eq:QHMMseparate} we find
\begin{equation*}
  Q_{\text{MAP}}(\parameters',\parameters) =  Q_{\text{MAP, initial}}
  (\parameters',\parameters) +   Q_{\text{MAP, transition}} (\parameters',\parameters)
  + Q_{\text{MAP, observation}}
  (\parameters',\parameters),
\end{equation*}
with
\begin{subequations}
  \label{eq:QseparateMAP}
  \begin{align}
    Q_{\text{MAP, initial}} &= Q_{\text{MLE, initial}} + \log
    P(\parameters_{\text{initial}}) \\
    Q_{\text{MAP, transition}} &= Q_{\text{MLE, transition}} + \log
    P(\parameters_{\text{transition}}) \\
    \label{eq:QseparateMAPout}
    Q_{\text{MAP, observation}} &= Q_{\text{MLE, observation}} + \log
    P(\parameters_{\text{observation}}).
  \end{align}
\end{subequations}
With a suitable prior, the simple form of Eqn.~\eqref{eq:QseparateMAP}
makes it easy to convert a program that implements the Baum-Welch
algorithm for \index{maximum likelihood estimate} maximum likelihood
estimation into a program that implements maximum a posteriori
estimation.

\subsection{Vector Autoregressive Observations}
\label{sec:ARVGaussian}

\subsubsection{Overview of the model}

Rather than choosing a prior and developing algorithms for the
independent scalar observations of Section~\ref{sec:ScalarGaussian} we will
work on more general models with vector autoregressive Gaussian
observations associated with each hidden state.  If at time $t$ such a
system is in state $s$, then the mean of the conditional distribution
for $\ti{y}{t}$ is a linear function of the $d$ previous observations
$\ts{y}{t-d}{t-1}$ added to a fixed offset.  Specifically, the
conditional distributions for observations are $n$ dimensional vector
Gaussians as follows:
\begin{description}
\item[Covariance] The covariance is a state dependent positive
  definite $n\times n$ matrix $\Sigma_s$.
\item[Mean] Given that the system is in state $s$, the parameters
  $\left\{ c_{s,i,\tau,j}, \bar y_{s,i} \right\}$ and the $d$ previous
  observations determine the mean of the observation distribution
  through an \emph{affine} function, \ie, the sum of a linear function
  and a fixed offset, with components
  \begin{equation}
    \label{eq:ARVGout1}
    \mu_i\left(s,\ts{y}{t-d}{t-1}\right) = \bar y_{s,i} + \sum_{\tau =
    1}^d \sum_{j=1}^n c_{s,i,\tau,j} \ti{y_j}{t-\tau}.
  \end{equation}
  We implement this as a \emph{linear} function of a \emph{context
    vector} $x$ consisting of $d$ previous observations and a constant
  one, \ie,
  \begin{align*}
    \ti{x}{t} & \equiv \left( \overrightarrow{\ti{y}{t-1}},
      \overrightarrow{\ti{y}{t-2}},\ldots
      \overrightarrow{\ti{y}{t-d}}, 1\right)\\
    & \equiv \left(\ti{y_1}{t-1}, \ti{y_2}{t-1}, \ldots,
      \ti{y_n}{t-1}, \ti{y_1}{t-2}, \ldots, \ti{y_n}{t-d}, 1 \right).
  \end{align*}
  Using notation in which the $i\th$ component of the observation
  $\tau$ time steps before time $t$ is $\ti{y_i}{t-\tau}$, the $k\th$
  component of the context at time $t$ is
  \begin{equation}
    \label{eq:context}
    \ti{x_{k}}{t} = \begin{cases} \ti{y_i}{t-\tau} & 1\leq \tau \leq d
    \\ 1 & \tau = d+1,~~ i = 1, \end{cases},
  \end{equation}
  where $k=n\cdot(\tau-1) + i$, and
  \begin{equation}
    \label{eq:ARVGout2}
    \mu_s\left(s,\ts{y}{t-d}{t-1}\right) = A_s \ti{x}{t}
  \end{equation}
  where $A_s$ is an $n\times(nd+1)$ matrix consisting of $\left\{
    c_{s,i,\tau,j} \right\}$ and $\left\{ \bar y_{s,i} \right\}$.
\end{description}
Using this notation the model assumptions are (compare to
Eqns.~\eqref{eq:assume_markov} and \eqref{eq:assume_output} which
describe the assumptions for HMMs with discrete observations.) that
the states follow a Markov process and that the conditional
distribution of an observation given the state $s$ is Gaussian with
mean $A_s \ti{x}{t}$ and covariance $\Sigma_s$, \ie,
\begin{align}
  \label{eq:assume_markovVARG}
  P(\ti{s}{t+1}|\ts{s}{-\infty}{t},\ts{y}{-\infty}{t}) &= 
  P\left(\ti{s}{t+1}|\ti{s}{t}\right)\\
  \label{eq:assume_outputVARG}
  P(\ti{y}{t}|\ts{s}{-\infty}{t},\ts{y}{-\infty}{t-1}) &=
    \frac{1}{\sqrt{ (2 \pi)^n \left|\Sigma_{s(t)}\right|}}
    \exp\left({-\frac{1}{2} z_{s(t)}\transpose(t) \Sigma_{s(t)}^{-1}
    z_{s(t)}(t)} \right)
\end{align}
where
\begin{equation*}
z_s(t) \equiv y(t) - A_s x(t).  
\end{equation*}

The model accounts for relationships between an observation and its
predecessors two ways, first the observation at time $\ti{y}{t}$ is
related to the $d$ previous observations through the matrix $A_s$, and
it is also related to \emph{all} previous observations through the state $s$.

\subsubsection{Reestimation}

A derivation like the one leading to \eqref{eq:QoutG3} yields
\begin{equation}
  \label{eq:QVARGmle1}
  Q_{\text{observation}} (\parameters',\parameters) =  \frac{1}{2}
      \sum_{s\in\states} \sum_{t=1}^T w(s,t) \left[ \log\left( \left|
      \Sigma_s^{-1} \right| \right) - n
      \log(2\pi) - z_s\transpose(t)
      \Sigma_s^{-1} z_s(t)\right].
\end{equation}
Hence each term in the sum over $s$ can be optimized separately.  One
can write code that maximizes $ \sum_{t=1}^T w(s,t) \left[ \log \left|
    \Sigma_s^{-1} \right| - z_s\transpose(t) \Sigma_s^{-1}
  z_s(t)\right]$ using operations on vectors and matrices as
follows:
\begin{enumerate}
\item Create a weighted \emph{context} matrix $X_s$ with columns
  $\ti{x}{t}\sqrt{w(s,t)}$, where $w(s,t) =
  P_{\ti{S}{t}|\ts{y}{1}{T}} \left(s|\ts{y}{1}{T} \right)$ and
  Eqn.~\eqref{eq:context} defines $\ti{x}{t}$.
\item Create a weighted \emph{observation} matrix $Y_s$ with columns
  $\ti{y}{t}\sqrt{w(s,t)}$.
\item \label{step:A} Solve
  \begin{equation}
    \label{eq:VARGnewA}
    A_s = \argmin_M \left|Y_s - MX_s \right|^2
  \end{equation}
  (We use singular value decomposition methods here because they are
  stable and make diagnosing problems easy.)
\item Calculate a matrix of residuals
  \begin{equation*}
    Z_s = Y_s - A_s X_s
  \end{equation*}
\item \label{step:cov} Calculate a new covariance matrix\footnote{One
    may derive this formula by differentiating the right-hand side of
    Eqn.~\ref{eq:QVARGmle1} with respect to the elements of
    $\Sigma_s^{-1}$ and setting the result equal to zero.  The key is
    the observation that for a positive definite matrix $M$,
  \begin{equation*}
    \frac{\partial \log \left| M \right|}{\partial m_{i,j}} =
    \left[M^{-1}\right]_{i,j},
  \end{equation*}
  i.e., the derivative of the log of the determinant with respect to the
  $i,j^{th}$ element of $M$ is the $i,j^{th}$ element of $M^{-1}$.
  
  Since the value of $A_s$ that minimizes $ \sum_{t=1}^T w(s,t) -
  z_s\transpose(t) \Sigma_s^{-1} z_s(t)$ is independent of
$\Sigma_s^{-1}$, it is correct to do step \ref{step:A} before step
\ref{step:cov}. }
  \begin{equation}
    \label{eq:VARGnewSigma}
    \Sigma_s = \frac{\sum_{t=1}^T w(s,t) z_s(t)
      z_s\transpose(t)}{\sum_{t=1}^T w(s,t)},
  \end{equation}
  using $Z_s Z_s\transpose = \sum_{t=1}^T w(s,t) z_s(t)
  z_s\transpose(t)$.
\end{enumerate}

\subsubsection{Regularization}
\index{regularization}

As Eqn.~\eqref{eq:QseparateMAP} indicates, we can influence the a
posteriori distributions of the initial states, the transitions, and
the observations by our selection of the respective priors,
$P(\parameters_{\text{initial}})$, $P(\parameters_{\text{transition}})$, and
$P(\parameters_{\text{observation}})$.  Since singularities in the likelihood
are associated with singular covariance matrices $\Sigma_s$, the prior
for the covariance matrices is most urgent.

Following \cite{Gauvain94, Ormoneit95} we use inverse-\index*{Wishart
distributions} as priors for the inverse covariance matrices.  See
pages 150-155 of Schafer \cite{Schafer97} for a description of the
these distributions.  The inverse-Wishart prior has the following
probability density for an $n\times n$ inverse covariance matrix
\begin{equation*}
  P_{\text{IW}}(\Sigma^{-1}) \equiv C \left| \Sigma^{-1}
                    \right|^{\frac{m+n+1}{2}}
                    e^{-\frac{1}{2} {\bf tr}(\Lambda \Sigma^{-1})},
\end{equation*}
where $C$ is a normalization constant, $m$ is called the \emph{degrees
  of freedom}, and $\Lambda$ is called the \emph{scale}.  The mean and
the maximum of the inverse-Wishart are
\begin{align*}
  \EV \Sigma^{-1} &= \frac{1}{m-n-1} \Lambda^{-1}, \text{ and} \\
  \argmax P_{\text{IW}} \left( \Sigma^{-1} \right) &= \frac{1}{m+n+1}
  \Lambda^{-1}
\end{align*}
respectively.

Assuming neutral priors for the coefficient matrices $A_s$ and the
form
\begin{equation*}
  \Lambda = \beta \id,
\end{equation*}
we find
\begin{equation*}
  P(\parameters_y) \propto \prod_s \left| \Sigma_s^{-1}
                    \right|^{\frac{\alpha}{2}}
                    e^{-\frac{1}{2} {\bf tr}(\beta \Sigma_s^{-1})},
\end{equation*}
(where $\alpha = n+m+1$) and Eqn.~\ref{eq:QseparateMAPout} becomes
\begin{multline*}
  \label{eq:QVARG1}
  Q_{\text{observation}} = C + \sum_s \left( \frac{\alpha}{2}\log\left|
      \Sigma_s^{-1}\right| - \frac{1}{2} {\bf tr}(\beta
    \Sigma_s^{-1}) \right) \\
  + \frac{1}{2} \sum_s \sum_{t=1}^T w(s,t) \left[ \log \left|
      \Sigma_s^{-1} \right| - z_s\transpose(t) \Sigma_s^{-1} z_s(t)\right],
\end{multline*}
where $z_s(t) \equiv y(t) - A_s x(t)$ and $w(s,t)$ is the probability
of being in state $s$ at time $t$ given the data $\ts{y}{1}{T}$.
Reestimation of the regularized model is the same as reestimation for
the unregularized model except that Eqn.~\ref{eq:VARGnewSigma} is
replaced with
\begin{equation}\label{eq:NewSigma}
  \Sigma_s = \frac{\beta \id + \sum_{t=1}^T w(s,t) z_s(t)
    z_s\transpose(t)} {\alpha + \sum_{t=1}^T w(s,t)}.
\end{equation}
Thus in the absence of new information, a covariance matrix is given
the default value $\frac{\beta \id}{\alpha}$, and $\alpha$ is the
weight of that default.

\section{Related Models}
\label{sec:related}

Figure~\ref{fig:VARGstates} illustrates the strength of the HMMs with
vector autoregressive observations described in
Section~\ref{sec:ARVGaussian}, namely that optimization finds regions
over which dynamics are approximately linear and assigns those regions
to single states.  The model class used to make the figure does not
exploit all of the available features of the Lorenz data.  In
particular it does not use the Lorenz equations.  Modeling techniques
exist that can exploit knowledge of nonlinear generating equations,
\eg \emph{extended Kalman filters} and \emph{particle filters} that we
mention below.

\begin{figure}[p]
  \centering{\plotsize%
    \includegraphics[width=1.0\textwidth]{VARGstates}}
  \caption[\comment{fig:VARGstates }Vector autoregressive observation
  models.]%
  {Plots of decoded states using an HMM with vector autoregressive
    observations.  Here the observations are a trajectory of three
    dimensional state vectors from the Lorenz system.  In each state
    the observation $\ti{y}{t}$ is modeled as a Gaussian with a mean
    that is an affine (linear plus fixed offset) function of the
    observation $\ti{y}{t-1}$.}
  \label{fig:VARGstates}
\end{figure}

Here we list a few of the many techniques that are related to the
basic ideas we've described:
\begin{description}
\item[Nonstationary HMMs] The definition of a \emph{stationary} model
  is that all probabilities are independent of shifts in time, \ie, $
  P_{\ts{Y}{1}{t}} = P_{\ts{Y}{1+\tau}{t+\tau}} \forall (t,\tau).$
  Many processes, for example weather, are not even approximately
  stationary.  By making HMMs with state transition probabilities that
  depend on time, many have created nonstationary HMMs for such
  applications.  With Rechtsteiner\cite{Rechtsteiner2000} we have
  implemented weather models with transition probabilities that depend
  on the time of year.  In similar work, Hughes and
  Guttorp\cite{Hughes99} described a local precipitation model with
  state transition probabilities driven by large scale atmospheric
  parameters.
\item[Gaussian mixtures] In a Gaussian mixture, the probability
  density is weighted average of Gaussian densities.  We described a
  simple one dimensional two component example in
  Eqn.~\eqref{eq:GaussianMixture}.  Although a simple Gaussian mixture
  model does not use the notion of \emph{state}, they are frequently
  used as observation models in HMMs.  Note that a Gaussian mixture
  model is equivalent to the degenerate case of an HMM with Gaussian
  observations in which all state transition probabilities are
  identical.
\item[Cluster weighted modeling] For an HMM, the state transition
  probabilities depend only on the state.  Looking at
  Fig.~\ref{fig:VARGstates}, it seems that letting the current
  observation $\ti{y}{t}$ influence the transition probabilities could
  be an improvement, \ie, the probability of the next state would
  depend on both the current state and the current observation.  By
  ignoring the current state entirely, one obtains a \emph{ cluster
    weighted model}\cite{Gershenfeld99}; the observation at time $t+1$
  is a mixture whose component weights are determined by the
  observation at time $t$.
\item[Kalman filter] The Kalman filter is essentially the forward
  algorithm applied to a model with continuous states $\ti{x}{t}$ and
  observations $\ti{y}{t}$ with the form
  \begin{align*}
    \ti{x}{t+1} &= F(\ti{x}{t},t) + \ti{\eta}{t}\\
    \ti{y}{t} &= G(\ti{x}{t}) + \ti{\epsilon}{t}
  \end{align*}
  where the functions $F$ and $G$ are linear in $x$ and the noise
  terms $\ti{\eta}{t}$ and $\ti{\epsilon}{t}$ are Gaussian.  We will
  describe algorithms that operate with such models in the next
  chapter.
\item[Extended Kalman filter] Using linear approximations to nonlinear
  functions $F$ and $G$ in a Kalman filter is called an \emph{extended
  Kalman filter}.  Linear approximations work well as long as the
  covariance of the state distributions are small compared to the
  second derivatives of $F$ and $G$.
\item[Unscented Kalman filter] Rather than use linear approximations
  to $F$ and $G$, an unscented Kalman filter\cite{Julier97} uses exact
  functions and a collection of samples to estimate means and
  variances.
\item[Particle filter] The idea of using the empirical distribution of
  many simulated trajectories (particles) to approximate the
  conditional distribution $P_{\ti{X}{t}|\ts{Y}{1}{t}}$ is called
  particle filtering\cite{Gordon93,Kitagawa96}.  In the procedure,
  particles that seem inconsistent with measurements are eliminated
  and new particles are created. 
\end{description}

%%%
%%% Local Variables:
%%% TeX-master: "main"
%%% eval: (load-file "hmmkeys.el")
%%% End:

\chapter{Continuous States and Observations and Kalman Filtering}
\label{chap:continuous}

We think of \index{state space model}state space systems with continuously distributed
states and observations in terms of equations like
\begin{subequations}
  \label{eq:contnoise}
  \begin{align}
    \ti{x}{t} &= F(\ti{x}{t-1},t) + \ti{\eta}{t}\\
    \ti{y}{t}   &= G(\ti{x}{t},t) + \ti{\epsilon}{t},
  \end{align}
\end{subequations}
where $X\in\REAL^n$ is the state variable, $Y\in\REAL^m$ is the
observation, and $\ti{\eta}{t}$ and $\ti{\epsilon}{t}$ are noise
terms.  Equations~\eqref{eq:contnoise} define the conditional
probability densities
\begin{equation}
  \label{eq:Cstatetran}
  P_{\ti{X}{t+1}|\ti{X}{t}}
\end{equation}
and
\begin{equation}
  \label{eq:Cobs}
  P_{\ti{Y}{t}|\ti{X}{t}}.
\end{equation}
Having each of the noise terms $\ti{\eta}{t}$ and $\ti{\epsilon}{t}$
in Eqns.~\ref{eq:contnoise} be independent of all other noise terms is
sufficient to ensure that the following assumptions hold:
\begin{enumerate}
\item The dynamics of the state variable $X$ are Markov.
\item Given the state at time $t$, the observation $\ti{Y}{t}$ is
  independent of everything else.
\end{enumerate}
These are the same as the assumptions of
Eqns.~\eqref{eq:assume_markov} and \eqref{eq:assume_output} which
characterize HMMs with discrete observations.  In this chapter, we
write forward and backward algorithms by replacing sums of finite
probabilities with integrals over probability densities.  If the
functions $F$ and $G$ are linear in $X$ and the noise terms
$\ti{\eta}{t}$ and $\ti{\epsilon}{t}$ are independent and Gaussian,
then the forward algorithm is called \emph{Kalman Filtering}.

In this chapter we go over the forward an backward algorithms for
continuous states and observations three times.  First in
Section~\ref{sec:integrals} we emphasize the parallel to
Chapter~\ref{chap:algorithms} by simply replacing sums in the
development of that chapter with integrals.  By themselves, the
results are not immediately useful because one must specify parametric
forms for the probability densities and \emph{do} the integrals to
implement the algorithms.  Next, in Section~\ref{sec:LinearGaussian},
we concisely present the algorithms one obtains when the functions $F$
and $G$ in Eqns.~\eqref{eq:contnoise} are linear and the noise terms
$\ti{\eta}{t}$ and $\ti{\epsilon}{t}$ are Gaussian.  We hope this
concise presentation will be useful for readers who want to implement
the algorithms.  Finally, for completeness, in
Section~\ref{sec:KDerive} we demonstrate that in fact the integrals of
Section~\ref{sec:integrals} do yield the formulas of
Section~\ref{sec:LinearGaussian}.

\section{Algorithms with Integrals}
\label{sec:integrals}

Here we write out the forward and backward algorithms for models with
continuously distributed states and observations by simply taking the
descriptions in Chapter \ref{chap:algorithms} and replacing sums with
integrals.

\subsection{Forward Algorithm}
\index{forward algorithm!for continuous states}

As the forward algorithm uses each successive observation, it
calculates two quantities.  First, it calculates the conditional
probability of the of the observation at time $t$ given earlier
observations
\begin{equation*}
  \gamma(t+1) \equiv P\left(\ti{y}{t+1}|\ts{y}{1}{t} \right).
\end{equation*}
Then it calculates the conditional distribution of states given the
observations up to the present time
\begin{equation*}
  \alpha(x,t) \equiv P_{\ti{X}{t}|\ts{Y}{1}{t}} \left(x|\ts{y}{1}{t}
  \right).
\end{equation*}
The equations
\begin{align}
  \label{eq:ContInitial}
  P_{\ti{X}{1},\ti{Y}{1}} \left(x,\ti{y}{1} \right) &=
  P_{\ti{Y}{1}|\ti{X}{1}} \left(\ti{y}{1}|x \right)P_{\ti{X}{1}}
  \left(x \right) \\
                                %
  \gamma(1) = P\left(\ti{y}{1} \right) & = \int
  P_{\ti{X}{1},\ti{Y}{1}}
  \left(x,\ti{y}{1} \right) dx \\
                                %
  \alpha(x,1) \equiv P_{\ti{X}{1}|\ti{Y}{1}} \left(x|\ti{y}{1} \right) &=
  \frac{P_{\ti{X}{1},\ti{Y}{1}} \left(x,\ti{y}{1}
    \right)}{P\left(\ti{y}{1} \right)}
\end{align}
specify the calculation of $\gamma(1)$, the probability of the first
observation, and of $\alpha(x,1)$, the conditional distribution of the
first state given the first observation, that initialize the
algorithm.  Given $\gamma(1)$ and $\alpha(x,1)$, the forward algorithm
executes the following recursion to calculate $\gamma(t)$ and
$\alpha(x,t)$ for each of the remaining observations $\ti{y}{t}$.
\begin{description}
\item[Forecast the state distribution] \index{forecast} Find the
  conditional distribution of the state at the next time given the
  observations up to the present time.
\begin{align}
  \label{eq:t274}
  P\left(\ti{x}{t+1},\ti{x}{t}|\ts{y}{1}{t} \right) &=
  P\left(\ti{x}{t+1}|\ti{x}{t},\ts{y}{1}{t} \right)
  P\left(\ti{x}{t}|\ts{y}{1}{t} \right) \\
  \label{eq:t275}
  &= P\left(\ti{x}{t+1}|\ti{x}{t} \right)
  P\left(\ti{x}{t}|\ts{y}{1}{t} \right) \\
  &= P\left(\ti{x}{t+1}|\ti{x}{t} \right)
  \alpha(\ti{x}{t},t)\\
                                %
  \label{eq:t277}
  P\left(\ti{x}{t+1}|\ts{y}{1}{t} \right) &= \int
  P\left(\ti{x}{t+1},\ti{x}{t}|\ts{y}{1}{t} \right) d\ti{x}{t}\\
%%   &= \int P \left(\ti{x}{t+1}|\ti{x}{t} \right)
%%   P\left(\ti{x}{t}|\ts{y}{1}{t} \right) d\ti{x}{t} \\
  \label{eq:IFore}
  &= \int P \left(\ti{x}{t+1}|\ti{x}{t} \right)
  \alpha(\ti{x}{t},t) d\ti{x}{t}
\end{align}
Here we justify \eqref{eq:t274} by Bayes rule, \eqref{eq:t275} by
the model assumptions, and \eqref{eq:t277} by the definition of a
marginal distribution.
\item[Calculate the joint forecast] Leaving the state at the next time
  as a free parameter but fixing the next observation at $\ti{y}{t}$,
  calculate the joint conditional distribution of the next state and
  observation given the observations up to the present time.
\begin{align}
  \label{eq:t279}
  P\left(\ti{x}{t+1},\ti{y}{t+1}|\ts{y}{1}{t} \right) &=
  P\left(\ti{y}{t+1}|\ti{x}{t+1},\ts{y}{1}{t} \right)
  P\left(\ti{x}{t+1}|\ts{y}{1}{t} \right) \\
  \label{eq:t280}
  &=P\left(\ti{y}{t+1}|\ti{x}{t+1} \right)
  P\left(\ti{x}{t+1}|\ts{y}{1}{t} \right)
\end{align}
Here we justify \eqref{eq:t279} by Bayes rule and \eqref{eq:t280} by
the model assumptions.
\item[Conditional probability of the observation] Integrate out
  $\ti{x}{t+1}$ to get the conditional probability of the next
  observation given the observations up to the present time.
\begin{equation}
  \label{eq:Igamma}
  \ti{\gamma}{t+1} \equiv P\left(\ti{y}{t+1}|\ts{y}{1}{t} \right) = \int
  P\left(\ti{x}{t+1},\ti{y}{t+1}|\ts{y}{1}{t} \right) d\ti{x}{t+1} \\
\end{equation}
\item[Update the conditional state distribution] \index{update} Use
  Bayes rule to combine \eqref{eq:t280} and \eqref{eq:Igamma} and get
  the conditional distribution of the state at time $t+1$ given all of
  the measurements up to time $t+1$.
\begin{align}
  \alpha(x(t+1),t+1) &= P\left(\ti{x}{t+1}|\ts{y}{1}{t+1} \right) \\
  &=  \frac{ P\left(\ti{x}{t+1},\ti{y}{t+1}|\ts{y}{1}{t}
    \right)}{P\left(\ti{y}{t+1}|\ts{y}{1}{t} \right)}\\
  \label{eq:IUpdate}
  &= \frac{ P\left(\ti{x}{t+1},\ti{y}{t+1}|\ts{y}{1}{t}
    \right)}{\ti{\gamma}{t+1}}.
\end{align}
\end{description}

\subsection{Backward Algorithm}
\index{backward algorithm!for continuous states}

Similarly, the backward algorithm finds a function called the
\index*{backwards forecast}
\begin{equation}
  \label{eq:contbeta}
  \beta(x,t) \equiv \frac{P_{\ts{Y}{t+1}{T}|\ti{X}{t}}
  \left(\ts{y}{t+1}{T}|x \right)}{P\left(\ts{y}{t+1}{T}|\ts{y}{1}{t}
  \right)} = \frac{P_{\ti{X}{t}|\ts{Y}{1}{T}} \left(x|\ts{y}{1}{T}
  \right)}{P_{\ti{X}{t}|\ts{Y}{1}{t}} \left(x|\ts{y}{1}{t} \right)}
\end{equation}
for each time $t$ by starting with $\beta(x,T) = 1$ and following with
the recursion
\begin{equation}
  \label{eq:contback}
  \beta(x,t-1) = \int 
  \frac{\beta(x',t) P_{\ti{Y}{t}|\ti{X}{t}}\left(\ti{y}{t}|x' \right)
  P_{\ti{X}{t}|\ti{X}{t-1}}\left(x'|x \right)}
  {P\left(\ti{y}{t}|\ts{y}{1}{t-1} \right)} dx'.
\end{equation}
To justify Eqn.~\eqref{eq:contback} note that
\begin{align}
  \beta(x',t) &= \frac{P_{\ts{Y}{t+1}{T}|\ti{X}{t}}
  \left(\ts{y}{t+1}{T}|x' \right)}{P
  \left(\ts{y}{t+1}{T}|\ts{y}{1}{t} \right)}\\
&= \frac{P_{\ts{Y}{t+1}{T}|\ti{X}{t},\ts{Y}{1}{t}}
  \left(\ts{y}{t+1}{T}|x',\ts{y}{1}{t} \right) P \left(\ti{y}{t}|\ts{y}{1}{t-1}
  \right)}{P \left(\ts{y}{t}{T}|\ts{y}{1}{t-1} \right)} 
\end{align}
and
\begin{alignat}{1}
  P_{\ti{Y}{t}|\ti{X}{t}}&\left(\ti{y}{t}|x' \right) %
  P_{\ti{X}{t}|\ti{X}{t-1}}\left(x'|x \right)\notag\\
  &= P_{\ti{Y}{t}|\ti{X}{t},\ts{Y}{1}{t-1}}%
    \left(\ti{y}{t}|x',\ts{y}{1}{t-1}\right) %
      P_{\ti{X}{t}|\ti{X}{t-1},\ts{Y}{1}{t-1}}%
    \left(x'|x,\ts{y}{1}{t-1}\right)\\
  &= P_{\ti{Y}{t},\ti{X}{t}|\ti{X}{t-1},\ts{Y}{1}{t-1}}%
    \left(\ti{y}{t},x'|x,\ts{y}{1}{t-1}\right)
\end{alignat}
and consider the integrand of \eqref{eq:contback}
\begin{align}
  I &= \frac{\beta(x',t) P_{\ti{Y}{t}|\ti{X}{t}}\left(\ti{y}{t}|x'
    \right) P_{\ti{X}{t}|\ti{X}{t-1}}\left(x'|x \right)}
  {P\left(\ti{y}{t}|\ts{y}{1}{t-1} \right)}\\
   &= \frac{P_{\ts{Y}{t+1}{T}|\ti{X}{t},\ts{Y}{1}{t}}
     \left(\ts{y}{t+1}{T}|x',\ts{y}{1}{t} \right) }{P
    \left(\ts{y}{t}{T}|\ts{y}{1}{t-1} \right)}
   P_{\ti{Y}{t},\ti{X}{t}|\ti{X}{t-1},\ts{Y}{1}{t-1}}
   \left(\ti{y}{t},x'|x,\ts{y}{1}{t-1} \right) \\
  &= \frac{P_{\ts{Y}{t}{T},\ti{X}{t}|\ti{X}{t-1}}
    \left(\ts{y}{t}{T},x'|x \right)}{P
    \left(\ts{y}{t}{T}|\ts{y}{1}{t-1} \right)}.
\end{align}
Integrating out $x'$ leaves $\frac{P_{\ts{Y}{t}{T}|\ti{X}{t-1}}
  \left(\ts{y}{t}{T}|x \right)}{P \left(\ts{y}{t}{T}|\ts{y}{1}{t-1}
  \right)} = \beta(x,t-1)$.

Notice that we have chosen to define $\beta(x,t)$ so that the
conditional distribution of the state at time $t$ given all of the
observations is
\begin{equation}
  \label{eq:contalphabeta}
  \alpha(x,t) \beta(x,t) = P_{\ti{X}{t}|\ts{Y}{1}{T}}
  \left(x|\ts{y}{1}{T} \right).
\end{equation}
Defining and evaluating a \emph{\index*{backwards update}} term
\begin{align}
  \label{eq:bdef}
  b(x,t) &\equiv \frac{\beta(x,t)
    P_{\ti{Y}{t}|\ti{X}{t}}\left(\ti{y}{t}|x
    \right)} {P\left(\ti{y}{t}|\ts{y}{1}{t-1} \right)} \\
  &= \frac{P_{\ti{X}{t}|\ts{Y}{1}{T}} \left(x|\ts{y}{1}{T}
    \right)}{P_{\ti{X}{t}|\ts{Y}{1}{t-1}} \left(x|\ts{y}{1}{t-1}
    \right)}\nonumber\\
  & = \frac{P_{\ts{Y}{t}{T}|\ti{X}{t}} \left(\ts{y}{t}{T}|x
    \right)}{P_{\ts{Y}{t}{T}|\ts{Y}{1}{t-1}}
    \left(\ts{y}{t}{T}|\ts{y}{1}{t-1} \right)} \nonumber
\end{align}
can make evaluating the integral in Eqn.~\eqref{eq:contback} easier.

\section{Linear Gaussian Systems}
\label{sec:LinearGaussian}

If the functions $F$ and $G$ in Eqn.~\eqref{eq:contnoise} are linear
in $x$, the noise terms $\ti{\eta}{t}$ and $\ti{\epsilon}{t}$ are \iid
and Gaussian\footnote{We assume \emph{identical} distributions only to
  simplify the notation.  The procedures generalize easily to time
  dependent noise terms.  The notation $X\sim\Normal(\mu,\Sigma)$
  means, ``The random variable $X$ has a Gaussian distribution with
  mean $\mu$ and covariance $\Sigma$.} with $\ti{\eta}{t} \sim
\Normal(0,\Sigma_\eta)$ and $\ti{\epsilon}{t} \sim
\Normal(0,\Sigma_\epsilon)$, and the distribution of the initial state
$\ti{X}{1}$ is also Gaussian, then we can write
\begin{subequations}
  \label{eq:LinearG}
  \begin{align}
  \ti{X}{1} & \sim \Normal\left(\mu_{\ti{X}{1}},\Sigma_{\ti{X}{1}}
  \right)\\
                                %
  \ti{x}{t} &= \ti{F}{t} \cdot \ti{x}{t-1} + \ti{\eta}{t} &
  \ti{\eta}{t} &\sim \Normal\left( 0, \Sigma_\eta \right) \\
  \ti{y}{t} &= \ti{G}{t} \cdot \ti{x}{t} + \ti{\epsilon}{t} &
  \ti{\epsilon}{t} &\sim \Normal\left( 0, \Sigma_\epsilon \right),
  \end{align}
\end{subequations}
where $\ti{F}{t}$ and $\ti{G}{t}$ are matrices.  We use the notation
$\Normal\left( \mu, \Sigma \right)$ for a Gaussian distribution with
mean $\mu$ and covariance $\Sigma$, and we denote the value of a
Gaussian probability density at $v$ by $\NormalE{\mu}{\Sigma}{v}$ \ie
if $V$ is an $n$-dimensional random variable $V \sim \Normal\left( \mu,
  \Sigma \right)$ then $ P\left(v \right) = \NormalE{\mu}{\Sigma}{v}
\equiv \frac{1}{\sqrt{(2\pi)^n \left|\Sigma\right|}} e^{-\frac{1}{2}
  (v-\mu)\transpose \Sigma^{-1} (v-\mu)}$.

We describe the forward and backward algorithms for linear Gaussian
systems in terms of the quantities listed below.  Our notation for
these quantities emphasizes the similarity to the calculations for
discrete processes in Chapter~\ref{chap:algorithms}.  In particular
the Greek subscripts $\alpha$ and $\beta$ denote the forward updated
distribution and the backwards forecast distribution respectively while
the Roman subscripts $a$ and $b$ denote the forward forecast
distribution and the backwards updated distributions respectively.
The assumptions of Eqns.~\eqref{eq:LinearG} imply that the
distributions described by these parameters are also Gaussian.
\begin{description}
\item[$\bm{\ti{\mu_\alpha}{t}}$ and $\bm{\ti{\Sigma_\alpha}{t}}$]
  (Note \emph{Greek} subscript.) are the parameters of the
  \emph{updated} state distribution\footnote{For the quantities that
    we call $\ti{\mu_\alpha}{t}$ and $\ti{\Sigma_\alpha}{t}$,
    Maybeck\cite{Maybeck82} uses the notation $\hat x(t^+_i)$ and
    $P(t^+_i)$ and Kailath et al.\cite{KSH00} use $\hat x_{i|i}$ and
    $P_{i|i}$ respectively.}, \ie,
  \begin{equation*}
    P\left(\ti{x}{t}|\ts{y}{1}{t} \right) =
    \NormalE{\ti{\mu_\alpha}{t}}{\ti{\Sigma_\alpha}{t}}{\ti{x}{t}}.
  \end{equation*}
\item[$\bm{\ti{\mu_a}{t}}$ and $\bm{\ti{\Sigma_a}{t}}$] (Note
  \emph{Roman} subscript.) are the parameters of the one step
  \emph{forecast} of the state distribution\footnote{For the
    quantities that we call $\ti{\mu_a}{t}$ and $\ti{\Sigma_a}{t}$,
    Maybeck\cite{Maybeck82} uses the notation $\hat x(t^-_i)$ and
    $P(t^-_i)$ and Kailath et al.\cite{KSH00} use $\hat x_{i+1}$ and
    $P_{i+1}$ respectively.}, \ie,
  \begin{equation*}
    P \left(\ti{x}{t}|\ts{y}{1}{t-1} \right) =
    \NormalE{\ti{\mu_a}{t}}{\ti{\Sigma_a}{t}}{\ti{x}{t}}.
  \end{equation*}
\item[$\bm{\ti{\mu_\gamma}{t}}$ and $\bm{\ti{\Sigma_\gamma}{t}}$] are
  the parameters of the conditional probability of the observation at
  time $t$ given all previous observations, \ie
  \begin{equation*}
    P\left(\ti{y}{t}|\ts{y}{1}{t-1} \right) =
    \NormalE{\ti{\mu_\gamma}{t}}{\ti{\Sigma_\gamma}{t}}{\ti{y}{t}}.
  \end{equation*}
  Neither the forward nor the backwards Kalman filter uses the $\gamma$
  terms, but they are useful for calculating the likelihood of model
  parameters.
\item[$\bm{\ti{N_\beta}{t},\ti{\mu_\beta}{t}}$ and
  $\bm{\ti{\Sigma_\beta}{t}}$] are the parameters\footnote{On page
    342, Kailath et al.\cite{KSH00} use $\hat x_i^b$ to denote the
    quantity that we call $\ti{\mu_\beta}{t}$.} of the
  backwards forecast function $\beta(x,t)$, which as a ratio of
  Gaussian functions is itself an unnormalized Gaussian.  We define
  $\ti{N_\beta}{t}$ by
  \begin{equation*}
    \frac{\beta(x,t)}{\ti{N_\beta}{t}} =
    \NormalE{\ti{\mu_\beta}{t}}{\ti{\Sigma_\beta}{t}}{x}.
  \end{equation*}
\item[$\bm{\ti{\mu_b}{t}}$ and $\bm{\ti{\Sigma_b}{t}}$] are the
  mean\footnote{Kailath et al.\cite{KSH00} use $\hat x_{i|i}^b$ to
    denote the quantity that we call $\ti{\mu_b}{t}$.} and covariance
  of the backwards update function $b(x,t)$ (see Eqn.~\eqref{eq:bdef})
  which is an intermediate term in the backward algorithm.  Notice
  that the parameters of the forward forecast have Roman subscripts
  while the parameters of the backward forecast have Greek subscripts.
\item[$\bm{\ti{\mu_{\alpha\beta}}{t}}$ and
  $\bm{\ti{\Sigma_{\alpha\beta}}{t}}$] are the mean and covariance
  of the best estimate of the state at time $t$ given all of the
  observations, \ie $P_{\ti{X}{t}|\ts{Y}{1}{T}} \left(x|\ts{y}{1}{T}
  \right) =
  \NormalE{\ti{\mu_{\alpha\beta}}{t}}{\ti{\Sigma_{\alpha\beta}}{t}}{x}$.
\end{description}


\subsection{Kalman Filter: The Forward Algorithm}
\index{Kalman filter}

The following two step recursion is called Kalman
filtering\footnote{There are many longer presentations of Kalman
  filters, \eg, Maybeck\cite{Maybeck82}, \cite{KSH00}, OLR Jacobs,
  \emph{Introduction to control theory,} Oxford 1993, and RG Brown and
  PYC Hwang \emph{Introduction to Random Signals and Applied Kalman
    Filters} Wiley 1992} and it
implements the forward algorithm:                              %
\newcommand{\G}{ \ti{G}{t} }
\newcommand{\GT}{\left( \G \right)\transpose}
\newcommand{\SX}{ \ti{\Sigma_a}{t} }
\begin{description}
\item[Calculate the forecast of the distribution of the state]
  \begin{subequations}
    \label{eq:KFore}
    \begin{align}
      \label{eq:KForeMu}
      \ti{\mu_a}{t} &= \ti{F}{t} \cdot \ti{\mu_\alpha}{t-1}\\
      \label{eq:KForeSigma}
      \ti{\Sigma_a}{t} &= \ti{F}{t} \cdot \ti{\Sigma_\alpha}{t-1}
      \cdot \left(\ti{F}{t}\right)\transpose + \Sigma_\eta
    \end{align}
  \end{subequations}
\item[Update the distribution of the current state using
  $\bm{\ti{y}{t}}$]
                           %
  \begin{subequations}
    \label{eq:KUpdate}
    \begin{align}
      \label{eq:KUpdateSigma}
      \left( \ti{\Sigma_\alpha}{t} \right)^{-1} &= \left(
        \ti{\Sigma_a}{t} \right)^{-1} + \GT \Sigma_\epsilon^{-1} \G \\
      \ti{\mu_\alpha}{t} &= \ti{\mu_a}{t} +
      \ti{\Sigma_{\alpha}}{t} \left( \ti{G}{t} \right)\transpose
      \Sigma_\epsilon^{-1} \big[\ti{y}{t} - \ti{G}{t}
      \ti{\mu_a}{t}\big]
    \end{align}
  \end{subequations}
\end{description}
Note:
\begin{itemize}
\item Equations~\eqref{eq:KUpdate} are usually presented in the
  equivalent but computationally more efficient form
  \begin{align*}
    \ti{\mu_\alpha}{t} &= \ti{\mu_a}{t} + \ti{K}{t} \left( \ti{y}{t} -
    \ti{G}{t} \ti{\mu_a}{t} \right) \\
    \ti{\Sigma_\alpha}{t} &= \left( \id - \ti{K}{t} \G \right) \SX 
  \end{align*}
  where
  \begin{equation*}
    \ti{K}{t} \equiv \SX \GT \left( \G \SX \GT + \Sigma_\epsilon \right)^{-1}
  \end{equation*}
  is called \emph{the \index*{Kalman gain matrix}} %
  \index{matrix!Kalman gain}%
  (See Eqn.~\eqref{eq:KalmanGain}.).
\item One can calculate the log likelihood of a model by summing the
  increments
  \begin{equation*}
    \log\left( P\left(\ts{y}{1}{T} \right)\right) = \sum_{t=1}^T
    \log\left(P \left(\ti{y}{t}|\ts{y}{1}{t-1} \right) \right)
  \end{equation*}
  and calculating each increment as
  \begin{equation*}
    P \left(\ti{y}{t}|\ts{y}{1}{t-1} \right) =
    \NormalE{\ti{\mu_\gamma}{t}}{\ti{\Sigma_\gamma}{t}}{\ti{y}{t}}
  \end{equation*}
  where
  \begin{align*}
    \ti{\mu_\gamma}{t} &= \ti{G}{t} \ti{\mu_a}{t}\\
    \ti{\Sigma_\gamma}{t} &= \ti{G}{t} \ti{\Sigma_a}{t} \left(
    \ti{G}{t} \right)\transpose + \Sigma_\epsilon\\
    \log\left( P \left(\ti{y}{t}|\ts{y}{1}{t-1} \right) \right) &= -
    \frac{1}{2} \Big( n \log(2\pi) + \log \left( \left|
    \ti{\Sigma_\gamma}{t} \right| \right) \\ & \quad \quad \quad
    + \big( \ti{y}{t} - \ti{\mu_\gamma}{t} \big)\transpose
    \left( \ti{\Sigma_\gamma}{t} \right)^{-1}
    \big( \ti{y}{t} - \ti{\mu_\gamma}{t} \big)
    \Big)
  \end{align*}
\end{itemize}

\subsection{The Backward Algorithm}

We calculate $\ti{\mu_\beta}{t}$ and $\ti{\Sigma_\beta}{t}$ as defined
in Eqn.~\eqref{eq:contbeta} with the following recursion that goes
through the observations in reverse order.

\begin{description}
\item[Update the distribution of the current state using $\ti{y}{t}$]
  \begin{subequations}
    \label{eq:BUpdate}
    \begin{align}
    \left(\ti{\Sigma_b}{t} \right)^{-1} &= \left(\ti{\Sigma_\beta}{t}
    \right)^{-1} + \left( \ti{G}{t} \right)\transpose
    \left(\Sigma_\epsilon
    \right)^{-1} \ti{G}{t} \\
        \ti{\mu_b}{t} &= \ti{\mu_\beta}{t} + \ti{\Sigma_{b}}{t}
        \left( \ti{G}{t} \right)\transpose \Sigma_\epsilon^{-1}
        \big[\ti{y}{t} - \ti{G}{t} \ti{\mu_\beta}{t}\big],
  \end{align}
  \end{subequations}
\item[Forecast of the distribution of the state backward in time]
  \begin{subequations}
    \label{eq:BFore}
    \begin{align}
      \ti{\mu_\beta}{t-1} &= \left( \ti{F}{t} \right)^{-1}
      \ti{\mu_b}{t} \\
      \label{eq:SigmaBeta}
      \ti{\Sigma_\beta}{t-1} &= \left( \ti{F}{t} \right)^{-1} \left(
        \Sigma_\eta + \ti{\Sigma_b}{t} \right) \left( \left( \ti{F}{t}
        \right)^{-1} \right)\transpose
    \end{align}
  \end{subequations}
\end{description}
Note:
\begin{itemize}
\item As for the forward recursion, the update formulas are usually
  presented in a computationally more efficient form using a gain
  matrix (See Eqn.~\eqref{eq:KalmanGainBack}).
\item Ideally one would initialize the backward algorithm with
  \begin{align*}
    \left( \ti{\Sigma_\beta}{T} \right)^{-1} &= 0\\
    \ti{\mu_\beta}{T} &= 0
  \end{align*}
  but that would make $\left( \ti{\Sigma_b}{T} \right)^{-1}$
  uninvertible and preclude using Eqn.~\eqref{eq:SigmaBeta} to
  evaluate $\ti{\Sigma_\beta}{T-1}$.  One can address the problem by
  initializing $\ti{\Sigma_\beta^{-1}}{T}$ with small values, or by using
  the \emph{\index*{inverse covariance form}} of the algorithm (see
  Section~\ref{sec:DetailInverse}.).
\end{itemize}

\subsection{Smoothing}
\index{smoothing}

The conditional distribution of the state at time $t$ given all of the
data is Gaussian and therefore specified by its covariance and mean, \ie,
\begin{equation*}
   P \left(\ti{x}{t}|\ts{y}{1}{T} \right) = \NormalE{\ti{\mu_{\alpha
   \beta}}{t}}{\ti{\Sigma_{\alpha \beta}}{t}}{\ti{x}{t}},
\end{equation*}
where
\begin{subequations}
  \label{eq:smoothing}
  \begin{align}
    \left( \ti{\Sigma_{\alpha \beta}}{t} \right)^{-1} &= \left(
      \ti{\Sigma_{\alpha}}{t} \right)^{-1} + \left(
      \ti{\Sigma_{\beta}}{t} \right)^{-1} ~\text{ and} \\
    \ti{\mu_{\alpha \beta}}{t} &=
    \ti{\Sigma_{\alpha \beta}}{t} \left( \left(
        \ti{\Sigma_{\alpha}}{t} \right)^{-1} \ti{\mu_{\alpha}}{t} +
      \left( \ti{\Sigma_{\beta}}{t} \right)^{-1} \ti{\mu_{\beta}}{t}
    \right)
\end{align}
\end{subequations}
are combinations of the \emph{forward update} parameters and the
\emph{backward prediction} parameters.  Such use of all of the
observations to estimate the state sequence is called
\emph{smoothing}.  Combining forward update parameters and backward
update parameters, \ie, $\alpha$ and $b$, for smoothing is an error.

\section{Algorithm Derivations and Details}
\label{sec:KDerive}

Here we connect the integrals that describe the forward and backward
algorithms (Eqns.~\eqref{eq:ContInitial}-\eqref{eq:bdef}) to the
Kalman formulas (Eqns.~\eqref{eq:KFore}-\eqref{eq:smoothing}).

\subsection{Forward Kalman Filter}
\label{sec:DetailForward}

The basic loop consists of steps \ref{PXforecast} through
\ref{PXupdate} below.  In the first iteration, we use the model
parameters that describe $P\left(\ti{x}{1} \right)$ and proceed
directly to step~\ref{PYforecast}.
\begin{enumerate}
\item \label{PX1} \textbf{Use $\bm{P\left(\ti{x}{1} \right)}$.}  We
  initialize the recursion using the mean $\ti{\mu_a}{1}$ and
  covariance $\ti{\Sigma_a}{1}$ of $P(\ti{x}{1})$ which are parameters
  of the model and entering the loop at step~\ref{PYforecast} to
  calculate $\ti{\mu_\gamma}{1}$ and $\ti{\Sigma_\gamma}{1}$ by
  setting $t=1$.
\item \label{PXforecast} \textbf{Calculate the state forecast,
    $\bm{P\left(\ti{x}{t}|\ts{y}{1}{t-1} \right)}$.}  The distribution
  of the forecast is Gaussian, and therefore determined by its mean
  and covariance.  Thus rather than evaluating the integral in
  Eqn.~\eqref{eq:IFore} to find the distribution
  $P\left(\ti{x}{t}|\ts{y}{1}{t-1} \right)$, we can specify the
  distribution completely by calculating its first two moments.  Since
  \begin{equation*}
    \ti{x}{t} = \ti{F}{t} \ti{x}{t-1} + \ti{\eta}{t}
  \end{equation*}
  the mean is
  \begin{align}
    \ti{\mu_a}{t} &= \EV_{P
      \left(\ti{x}{t-1},\ti{\eta}{t}|\ts{y}{1}{t-1} \right)} \left[
      \ti{F}{t}\ti{X}{t-1} + \ti{\eta}{t} \right] \nonumber \\
    &= \EV_{P\left(\ti{x}{t-1},\ti{\eta}{t}|\ts{y}{1}{t-1} \right)}
      \ti{F}{t}\ti{X}{t-1} +
      \EV_{P\left(\ti{x}{t-1},\ti{\eta}{t}|\ts{y}{1}{t-1} \right)} \ti{\eta}{t} \nonumber \\
    \label{eq:etaIndependent}
    &= \EV_{P \left(\ti{x}{t-1}|\ts{y}{1}{t-1} \right)}
    \ti{F}{t}\ti{X}{t-1}  + \EV_{P (\eta)} \ti{\eta}{t} \\
                                 %
     &= \ti{F}{t}\ti{\mu_\alpha}{t-1}. \nonumber
   \end{align}
   The key step in the above sequence is
   Eqn.~\eqref{eq:etaIndependent} which we justify by observing that
   the distribution of the noise $\ti{\eta}{t}$ is independent of time
   and independent of $\ti{x}{\tau}$, and $\ti{y}{\tau}$ for all
   earlier times $\tau$.  Similarly we calculate the
   variance by
   \begin{align}
     \ti{\Sigma_a}{t} &= \EV_{P
       \left(\ti{x}{t-1},\ti{\eta}{t}|\ts{y}{1}{t-1} \right)} \big[
     \left( \ti{F}{t}\ti{X}{t-1} + \ti{\eta}{t} - \ti{\mu_a}{t-1}
     \right) \nonumber\\
     & \quad \times \left( \ti{F}{t}\ti{X}{t-1} + \ti{\eta}{t} -
       \ti{\mu_a}{t-1} \right)\transpose \big] \\
                                %
    &= \ti{F}{t}\ti{\Sigma_\alpha}{t-1}\ti{F}{t}\transpose +
    \Sigma_\eta .
  \end{align}
  Thus Eqn.~\eqref{eq:KFore} implements the integral of
  Eqn.~\eqref{eq:IFore}.
\item \label{PYforecast} \textbf{Calculate
    $\bm{P\left(\ti{y}{t}|\ts{y}{1}{t-1} \right)}$.}  Using
  \begin{equation*}
    \ti{y}{t} = \ti{G}{t-1}\ti{x}{t} + \ti{\epsilon}{t-1}
  \end{equation*}
  we calculate the mean and covariance of the distribution of the
  forecast observation as follows
  \begin{align}
    \ti{\mu_\gamma}{t} &= \EV_{P
      \left(\ti{x}{t},\ti{\epsilon}{t}|\ts{y}{1}{t-1} \right)}
    \ti{G}{t}\ti{X}{t} + \ti{\epsilon}{t} \nonumber \\
                                %
    \label{eq:mugamma}
    &= \ti{G}{t}\ti{\mu_a}{t}\\
                                %
    \ti{\Sigma_\gamma}{t} &= \EV_{P
      \left(\ti{x}{t},\ti{\epsilon}{t}|\ts{y}{1}{t-1} \right)}
    \left( \ti{G}{t}\ti{X}{t} + \ti{\epsilon}{t} -
      \ti{\mu_\gamma}{t} \right) \nonumber \\
    &\quad\left( \ti{G}{t-1}\ti{X}{t} +
      \ti{\epsilon}{t} - \ti{\mu_\gamma}{t}
    \right)\transpose \nonumber \\
                                %
    \label{eq:Sigmagamma}
    &= \ti{G}{t}\ti{\Sigma_a}{t}\ti{G}{t}\transpose +
    \Sigma_\epsilon
  \end{align}
\item \label{PZforecast} \textbf{Calculate $\bm{P\left(\ti{x}{t},
        \ti{y}{t}|\ts{y}{1}{t-1} \right)}$.}  The conditional
  distribution of the joint variable
  \begin{equation*}
    \ti{z}{t} =
    \begin{bmatrix}
      \ti{x}{t}\\
      \ti{y}{t}
    \end{bmatrix}
  \end{equation*}
  is a Gaussian characterized by
  \begin{equation*}
    \mu_z =
    \begin{bmatrix}
      \ti{\mu_a}{t}\\
      \ti{\mu_\gamma}{t}      
    \end{bmatrix}
  \end{equation*}
  and\footnote{We have introduced the blocks $A$, $B$, and $C$ to
    match the matrix identities in Eqns.~\eqref{eq:MI:ABCDEF} where we
    denote,
    \begin{equation*}
      \begin{bmatrix} A & C \\ C\transpose & B \end{bmatrix}^{-1} \equiv
      \begin{bmatrix} D & F \\ F\transpose & E \end{bmatrix}.
    \end{equation*}}
  \begin{equation}
    \label{eq:cont-temp}
    \Sigma_z \equiv
    \begin{bmatrix} A & C \\ C\transpose & B \end{bmatrix} =
    \begin{bmatrix}
      \ti{\Sigma_a}{t} & \ti{\Sigma_a}{t} (\ti{G}{t})\transpose
      \\ \ti{G}{t} \ti{\Sigma_a}{t}&\ti{\Sigma_\gamma}{t} 
    \end{bmatrix}.
  \end{equation}
  We found the off diagonal terms of $\Sigma_z$ in
  Eqn.~\eqref{eq:cont-temp} as follows
  \begin{align}
    C &= \EV_{P\left(\ti{x}{t},\ti{\epsilon}{t}|\ts{y}{1}{t-1} \right)}
    \big( \ti{x}{t} - \ti{\mu_a}{t} \big) \big( \ti{y}{t}
      - \ti{\mu_\gamma}{t} \big)\transpose \nonumber \\
    &=  \EV_{P\left(\ti{x}{t}|\ts{y}{1}{t-1} \right)}
    \big( \ti{x}{t} - \ti{\mu_a}{t} \big)
    \big( \ti{x}{t} - \ti{\mu_a}{t} \big)\transpose
    \left(\ti{G}{t} \right)\transpose \nonumber \\
    & \quad +
    \EV_{P\left(\ti{x}{t},\ti{\epsilon}{t}|\ts{y}{1}{t-1} \right)}
    \big( \ti{x}{t} - \ti{\mu_a}{t} \big) \big( \ti{\epsilon}{t}
    \big) \nonumber \\
    \label{eq:cont:C}
    &= \ti{\Sigma_{a}}{t} \big(\ti{G}{t} \big)\transpose .
  \end{align}
\item \label{PXupdate} \textbf{Calculate the state update,
    $\bm{P\left(\ti{x}{t}|\ts{y}{1}{t} \right)}$.}  From the
  actual value of $\ti{y}{t}$ and the joint distribution of
  $\ti{x}{t}$ and $\ti{y}{t}$ given $\ts{y}{1}{t-1}$, we use
  Eqn.~\eqref{eq:Gauss-Conditional} to write the parameters of
  $P\left(\ti{x}{t}|\ts{y}{1}{t} \right)$ as
  \begin{subequations}
    \label{eq:UpdateRaw}
    \begin{align}
      \ti{\Sigma_\alpha}{t} &= D^{-1} \\
      \ti{\mu_\alpha}{t} &= \ti{\mu_a}{t} + CB^{-1} \left(
        \ti{y}{t} - \ti{\mu_\gamma}{t}\right).
    \end{align}
  \end{subequations}
  Note the following facts and justifications:
  \begin{align*}
    E^{-1} &= B - C\transpose A^{-1}C && \text{See \eqref{eq:MI:E}}\\
    &= \ti{\Sigma_\gamma}{t} - \ti{G}{t}\ti{\Sigma_a}{t} \left(
    \ti{G}{t}\right)\transpose && \text{See \eqref{eq:cont-temp}}\\
    &= \Sigma_\epsilon && \text{See \eqref{eq:Sigmagamma}}\\
    CB^{-1} &= D^{-1}A^{-1}CE && \text{See \eqref{eq:MI:F}} \\
    &= \ti{\Sigma_\alpha}{t} \left( \ti{G}{t}
    \right)\transpose E && \text{See \eqref{eq:cont:C}}\\
    &= \ti{\Sigma_\alpha}{t} \left( \ti{G}{t}
    \right)\transpose (\Sigma_\epsilon)^{-1}\\
    D &= A^{-1} + A^{-1} CEC\transpose A^{-1} && \text{See \eqref{eq:MI:D}}.
  \end{align*}
  Thus
    \begin{align*}
      \left( \ti{\Sigma_\alpha}{t} \right)^{-1} &= \left(
        \ti{\Sigma_a}{t} \right)^{-1} + \left( \ti{G}{t}
        \right)\transpose \Sigma_\epsilon^{-1} \ti{G}{t} \\
      \ti{\mu_\alpha}{t} &= \ti{\mu_a}{t} +
      \ti{\Sigma_{\alpha}}{t} \left( \ti{G}{t} \right)\transpose
      \Sigma_\epsilon^{-1} \big[\ti{y}{t} - \ti{G}{t}
      \ti{\mu_a}{t}\big]
    \end{align*}
  which are the update equations in \eqref{eq:KUpdate}.
\end{enumerate}

As presented in Eqns.~\eqref{eq:KFore} and \eqref{eq:KUpdate}, the
forward algorithm requires two matrix inversions per iteration.  First
one must invert $\left( \ti{\Sigma_\alpha}{t-1}\right)^{-1}$ for
Eqn.~\eqref{eq:KForeSigma}, then one must invert $\ti{\Sigma_a}{t-1}$
for Eqn.~\eqref{eq:KUpdateSigma}.  Each of these is an $n\times n$
matrix where $n$ is the dimension of the state $x$.  One can avoid
these inversions either by keeping track of state space covariances or
by keeping track of \emph{inverse} state space covariances.  Either
choice improves the numerical speed and accuracy of the algorithm.
Kalman's form keeps track of state space covariances and uses the
\emph{\index*{Kalman gain matrix}} %
\index{matrix!Kalman gain}%
which in the notation of
Eqn.~\eqref{eq:cont-temp} is $K=CB^{-1}$.  To get
Eqn.~\eqref{eq:UpdateRaw} into Kalman's form, note that by
Eqn.~\eqref{eq:MI:D}, $D^{-1} = A -CB^{-1} C\transpose$, and thus
\begin{subequations}
  \label{eq:KalmanGain}
  \begin{align}
    \ti{\Sigma_\alpha}{t} &= \big(\id - \ti{K}{t} \ti{G}{t}
    \big) \ti{\Sigma_a}{t} \\
    \ti{\mu_\alpha}{t} &= \ti{\mu_a}{t} + \ti{K}{t} \left(
      \ti{y}{t} - \ti{\mu_\gamma}{t}\right).
  \end{align}
\end{subequations}
Using Eqns.~\eqref{eq:Sigmagamma}, \eqref{eq:cont-temp} and
\eqref{eq:cont:C}, we find the Kalman gain matrix in terms of known
quantities to be
\begin{align*}
  CB^{-1} &\equiv \ti{K}{t}\\
  &= \ti{\Sigma_{a}}{t} \big(\ti{G}{t} \big)\transpose \big(
  \ti{\Sigma_\gamma}{t}\big)^{-1}\\
  &= \ti{\Sigma_{a}}{t} \big(\ti{G}{t} \big)\transpose \big(\ti{G}{t}\ti{\Sigma_a}{t}\ti{G}{t}\transpose +
    \Sigma_\epsilon \big)^{-1}.
\end{align*}
The formula requires inversion of an $m\times m$ matrix where $m$, the
dimension of the observation $y$, is usually less than the dimension of
the state $x$.

\subsection{Backward Recursion}
\label{sec:DetailBack}
\index{backward algorithm!for continuous states}

Equation \eqref{eq:contback} states the backward recursion as the
integral
\begin{equation*}
  \beta(x,t-1) =  
  \frac{1}{\ti{\gamma}{t}} \int\beta(x',t)
  P_{\ti{Y}{t}|\ti{X}{t}}\left(\ti{y}{t}|x' \right)
  P_{\ti{X}{t}|\ti{X}{t-1}}\left(x'|x \right) dx'.
\end{equation*}
In the context of linear Gaussian systems, $P
\left(\ti{y}{t}|\ti{x}{t} \right)$ is the Gaussian
$\NormalE{\ti{G}{t}\ti{x}{t}}{\Sigma_\epsilon}{\ti{y}{t}}$, $P
\left(\ti{x}{t}|\ti{x}{t-1}\right) $ is the Gaussian $
\NormalE{\ti{F}{t} \ti{x}{t-1}}{\Sigma_\eta)}{\ti{x}{t}}$, and
$\frac{\beta(x,t)}{\ti{N_\beta}{t}} $ is the Gaussian $
\NormalE{\ti{\mu_\beta}{t}}{\ti{\Sigma_\beta}{t}}{x}$.  Thus
\begin{equation*}
  \beta(x,t-1) =  
  \frac{\ti{N_\beta}{t}}{\ti{\gamma}{t}} \int
  \NormalE{\ti{\mu_\beta}{t}}{\ti{\Sigma_\beta}{t}}{x'}
  \NormalE{\ti{G}{t}x'}{\Sigma_\epsilon}{\ti{y}{t}}
  \NormalE{\ti{F}{t}x}{\Sigma_\eta}{x'}   dx'.
\end{equation*}
Since, for smoothing, the only parameters of $\beta$ that we need are
$\mu_\beta$ and $\Sigma_\beta$, we adopt the following strategy to
calculate those parameters without evaluating the entire integral:
\begin{enumerate}
\item \label{BackList1} Drop time from notation in calculations
\item \label{BackList2} Forget normalization and work only with exponents
\item \label{BackList3} Combine $\ti{y}{t}$ with
  $\NormalE{\ti{\mu_\beta}{t}}{\ti{\Sigma_\beta}{t}}{x'}$ and
  $\NormalE{\ti{G}{t}x'}{\Sigma_\epsilon}{\ti{y}{t}}$ to get an
  \emph{updated distribution} for $\ti{x}{t}$ with mean
  $\ti{\mu_b}{t}$ and covariance $\ti{\Sigma_b}{t}$
\item \label{BackList4} Write the exponent of the product
  $\NormalE{\ti{\mu_\beta}{t}}{\ti{\Sigma_\beta}{t}}{x'} \cdot
  \NormalE{\ti{F}{t}x}{\Sigma_\eta}{x'}$ as
  \begin{equation*}
    Q = (x' - \mu)\transpose \Sigma^{-1} (x' - \mu) +
    (x - \ti{\mu_\beta}{t-1})\transpose \left(\ti{\Sigma_\beta}{t-1}
    \right)^{-1}(x - \ti{\mu_\beta}{t-1}) + R
  \end{equation*}
  where $Q$ is quadratic and the remainder $R$ depends on neither $x'$
  nor $x$
\item \label{BackList5} Since the integration over $x'$ eliminates the
  term $(x' - \mu)\transpose \Sigma^{-1} (x' - \mu)$ and $R$ only
  affects normalization, the surviving parameters
  $\ti{\mu_\beta}{t-1}$ and $\ti{\Sigma_\beta}{t-1}$ describe the
  function $\beta(x,t-1)$ up to normalization
\end{enumerate}

We begin with step~\ref{BackList3} in the list above by examining the
quadratic form in the exponent of
$\NormalE{\mu_\beta}{\Sigma_\beta}{x} \cdot
\NormalE{Gx}{\Sigma_\epsilon}{y}$
\begin{align*}
  &(x-\mu_\beta)\transpose \Sigma_\beta^{-1} (x-\mu_\beta) +
  (y-Gx)\transpose \Sigma_\epsilon^{-1} (y-Gx) \\
  &= x\transpose \left(\Sigma_\beta^{-1} + G\transpose
    \Sigma_\epsilon^{-1} G \right) x - 2x\transpose
  \left(\Sigma_\beta^{-1}\mu_\beta + G\transpose \Sigma_\epsilon^{-1}
    y \right) + \mu_\beta \transpose \Sigma_\beta^{-1}\mu_\beta +
  y\transpose \Sigma_\epsilon^{-1} y
\end{align*}
\begin{align*}
  \Sigma_b &= \left( \Sigma_\beta^{-1} + G\transpose
    \Sigma_\epsilon^{-1} G \right)^{-1}\\
  &= \Sigma_\beta - \Sigma_\beta G\transpose \left(G \Sigma_\beta
    G\transpose + \Sigma_\epsilon \right)^{-1} G \Sigma_\beta\\
  \mu_b &= \Sigma_b \left(\Sigma_\beta^{-1}\mu_\beta + G\transpose
    \Sigma_\epsilon^{-1} y \right) \\
  &= \Sigma_b \left( \left( \Sigma_b^{-1} - G\transpose
      \Sigma_\epsilon^{-1} G \right) \mu_\beta + G\transpose
    \Sigma_\epsilon^{-1} y \right)\\
  &= \mu_\beta + \Sigma_b G\transpose \Sigma_\epsilon^{-1}
  (y-G\mu_\beta) \\
  &= \mu_\beta + \Sigma_\beta G\transpose \left(G \Sigma_\beta
    G\transpose + \Sigma_\epsilon \right)^{-1}(y-G\mu_\beta),
\end{align*}
where we justify the last equation by observing that
\begin{align*}
  \Sigma_b G\transpose \Sigma_\epsilon^{-1} &= \Sigma_\beta
  G\transpose \Sigma_\epsilon^{-1} - \Sigma_\beta G\transpose \left(G
    \Sigma_\beta G\transpose + \Sigma_\epsilon \right)^{-1}G
  \Sigma_\beta
  G\transpose \Sigma_\epsilon^{-1} \\
  &= \Sigma_\beta G\transpose \left( \id - \left(G \Sigma_\beta
      G\transpose + \Sigma_\epsilon \right)^{-1}G \Sigma_\beta
    G\transpose \right)\Sigma_\epsilon^{-1} \\
  &= \Sigma_\beta G\transpose \left(G \Sigma_\beta G\transpose +
    \Sigma_\epsilon \right)^{-1} \left(G \Sigma_\beta G\transpose +
    \Sigma_\epsilon - G \Sigma_\beta G\transpose
  \right)\Sigma_\epsilon^{-1}\\
  &= \Sigma_\beta G\transpose \left(G \Sigma_\beta G\transpose +
    \Sigma_\epsilon \right)^{-1}.
\end{align*}
Thus
\begin{subequations}
  \label{eq:KalmanGainBack}
  \begin{align}
    \Sigma_b &= \left(\id  -  K_b G \right) \Sigma_\beta\\
    \mu_b &= \mu_\beta + K_b (y-G\mu_\beta)
  \end{align}
  where
  \begin{equation}
    K_b \equiv \Sigma_\beta G\transpose \left(G \Sigma_\beta
    G\transpose + \Sigma_\epsilon \right)^{-1}
  \end{equation}
  is called the \index*{backwards Kalman gain matrix}.%
  \index{Kalman gain matrix, backwards}%
  \index{matrix,Kalman gain, backwards}%
\end{subequations}

Next, we address step~\ref{BackList4} in the list above, using the
abbreviations
\begin{align*}
  \tilde \Sigma &\equiv (\Sigma_b^{-1} + \Sigma_\eta^{-1})^{-1} \\
  &= \Sigma_\eta - \Sigma_\eta( \Sigma_\eta + \Sigma_b)^{-1}
  \Sigma_\eta \\
  \tilde \mu &\equiv \tilde \Sigma \left( \Sigma_b^{-1} \mu_b +
  \Sigma_\eta^{-1} F x  \right)
\end{align*}
to analyze $Q_{x,x'}$, the exponent of $\NormalE{\mu_b}{\Sigma_b}{x'}
\cdot \NormalE{\ti{F}{t}x}{\Sigma_\eta}{x'}$.
\begin{align*}
  Q_{x,x'} &= (x' - \mu_b)\transpose \Sigma_b^{-1} (x' - \mu_b) + (x' -
  Fx)\transpose \Sigma_\eta^{-1} (x' - Fx)\\
   &= (x')\transpose (\Sigma_b^{-1} + \Sigma_\eta^{-1}) x' -2(x')\transpose
   ( \Sigma_b^{-1} \mu_b - \Sigma_\eta^{-1} F x) + \mu_b\transpose
   \Sigma_b^{-1} \mu_b + x\transpose F\transpose \Sigma_\eta^{-1} F
   x \\
   &= (x' - \tilde \mu)\transpose \tilde \Sigma^{-1}(x' - \tilde \mu) -
   \tilde \mu \transpose \tilde \Sigma^{-1} \tilde \mu +
   \mu_b\transpose \Sigma_b^{-1} \mu_b + x\transpose F\transpose
   \Sigma_\eta^{-1} F x
\end{align*}
Note that $(x' - \tilde \mu)\transpose \tilde \Sigma^{-1}(x' - \tilde
\mu)$ goes away with the integration, and 
\begin{align*}
  \tilde \mu \transpose \tilde \Sigma^{-1} \tilde \mu &= \left(
    \Sigma_b^{-1}\mu_b + \Sigma_\eta^{-1} F x \right)\transpose \left(
    \Sigma_b^{-1} + \Sigma_\eta^{-1} \right)^{-1}
  \left( \Sigma_b^{-1}\mu_b + \Sigma_\eta^{-1} F x \right) \\
  &= x F\transpose \Sigma_\eta^{-1} \tilde \Sigma^{-1}
  \Sigma_\eta^{-1} F x + 2x F\transpose \Sigma_\eta^{-1} \tilde
  \Sigma^{-1} \Sigma_b^{-1}\mu_b + \mu_b\transpose \Sigma_b^{-1}
  \tilde \Sigma^{-1} \Sigma_b^{-1}\mu_b\\
  &= x F\transpose \left( \Sigma_\eta^{-1} - \left(\Sigma_\eta +
      \Sigma_b \right)^{-1} \right) F x + 2x F\transpose
  \left(\Sigma_\eta + \Sigma_b \right)^{-1}\mu_b + \mu_b\transpose
  \Sigma_b^{-1} \tilde \Sigma^{-1} \Sigma_b^{-1}\mu_b .
\end{align*}
After integrating, the exponent of
$\NormalE{\mu_\beta}{\Sigma_\beta}{x}$ is
\begin{equation*}
  Q_x = x\transpose F\transpose (\Sigma_\eta + \Sigma_b)^{-1} Fx - 2x
  F\transpose \left(\Sigma_\eta + \Sigma_b \right)^{-1}\mu_b +
  \text{ scalar terms}. 
\end{equation*}
So
\begin{align*}
  \ti{\Sigma_\beta}{t-1} &= (\ti{F}{t})^{-1} ( \Sigma_\eta +
  \ti{\Sigma_b}{t} ) \left( \left( \ti{F}{t} \right)^{-1}
  \right)\transpose
  \\
  \ti{\mu_\beta}{t-1} &= (\ti{F}{t})^{-1}\ti{\mu_b}{t}.
\end{align*}

\subsection{Smoothing}
\label{sec:DetailSmoothing}
\index{smoothing}

We have defined $\alpha(x,t)$ and $\beta(x,t)$ so that
\begin{equation*}
   P_{\ti{X}{t}|\ts{Y}{1}{T}} \left(x|\ts{y}{1}{T} \right) =
   \alpha(x,t) \beta(x,t).
\end{equation*}
(See Eqn.~\eqref{eq:contalphabeta}.)  In fact
$P_{\ti{X}{t}|\ts{Y}{1}{T}}$ is Gaussian, and we denote its parameters
$\ti{\mu_{\alpha \beta}}{t}$ and $\ti{\Sigma_{\alpha \beta}}{t}$.
Examining the exponential terms in $\alpha(x,t) \beta(x,t)$ we find
(suppressing the time indices)
\begin{align*}
  &\left( x - \mu_\alpha \right)\transpose \Sigma_\alpha^{-1} \left( x
    - \mu_\alpha \right) + \left( x - \mu_\beta \right)\transpose
  \Sigma_\beta^{-1} \left( x - \mu_\beta \right)\\
  & = x\transpose \left( \left( \ti{\Sigma_{\alpha}}{t} \right)^{-1} + \left(
      \ti{\Sigma_{\beta}}{t} \right)^{-1} \right) x - 2x\transpose
  \left( \left( \ti{\Sigma_{\alpha}}{t} \right)^{-1}
  \ti{\mu_{\alpha}}{t} + \left( \ti{\Sigma_{\beta}}{t} \right)^{-1}
  \ti{\mu_{\beta}}{t} \right) \\
& + \mu_\alpha\transpose  \left(
  \ti{\Sigma_{\alpha}}{t} \right)^{-1} \mu_\alpha +
  \mu_\beta\transpose  \left( \ti{\Sigma_{\beta}}{t} \right)^{-1}
  \mu_\beta ,
\end{align*}
which implies
\begin{subequations}
  \begin{align}
    \left( \ti{\Sigma_{\alpha \beta}}{t} \right)^{-1} &= \left(
      \ti{\Sigma_{\alpha}}{t} \right)^{-1} + \left(
      \ti{\Sigma_{\beta}}{t} \right)^{-1} ~\text{ and} \\
    \ti{\mu_{\alpha \beta}}{t} &=
    \ti{\Sigma_{\alpha \beta}}{t} \left( \left(
        \ti{\Sigma_{\alpha}}{t} \right)^{-1} \ti{\mu_{\alpha}}{t} +
      \left( \ti{\Sigma_{\beta}}{t} \right)^{-1} \ti{\mu_{\beta}}{t}
    \right).
\end{align}
\end{subequations}

\subsection{Inverse Covariance Form}
\label{sec:DetailInverse}
 \index{inverse covariance form}

 If the inverse covariance of the state distribution is singular, one
 must propagate inverse covariances rather than covariances (see page
 239 of Maybeck\cite{Maybeck79}).  Kailath et al.\ call this \emph{The
   Information Form} (see page 332 of \cite{KSH00}).
 \index{information form} The procedure is useful when one makes the
 backwards pass through the data for smoothing because the initial
 inverse covariance is zero.

\subsection{Extended Kalman Filter}
\label{sec:EKF}
\index{extended Kalman filter|textbf}

Recall Eqn.~\eqref{eq:contnoise} from the beginning of the chapter
\begin{align*}
  \ti{x}{t+1} &= F(\ti{x}{t},t) + \ti{\eta}{t}\\
  \ti{y}{t}   &= G(\ti{x}{t},t) + \ti{\epsilon}{t},
\end{align*}
and that if the functions $F$ and $G$ are linear and the noise terms
$\ti{\eta}{t}$ and $\ti{\epsilon}{t}$ are independent and Gaussian
that the Kalman filter implements the forward algorithm.  If on the
other hand, the functions $F$ and $G$ are nonlinear but the errors of
first order approximations to them are small compared to the size of
the noise, then one can reasonably apply the same algorithm to the
approximations.  The resulting procedure is called an \emph{extended
  Kalman filter}.  While, as we noted in Section~\ref{sec:related}
there are more robust alternatives, the simplicity of extended Kalman
filters explains their frequent use.  In Section~\ref{sec:laser} we
applied an extended Kalman filter to laser measurements for the
following purposes:
\begin{itemize}
\item Estimate model parameters
\item Estimate state space trajectory
\item Forecast measurements
\end{itemize}
And in the next chapter we will compare the likelihood of a model
implemented as an extended Kalman filter to a performance bound
obtained from Lyapunov exponent estimates.

%%% Local Variables:
%%% TeX-master: "main"
%%% eval: (load-file "hmmkeys.el")
%%% End:

\input{toy_values.tex}
% A typical def in toy_values.tex is \def\toyToyhnt{200}.  The initial
% ``toy'' indicated it is for this chaper, toys.tex.  The following
% ``Toyh'' indicates that the value is defined in the script toy_h.py.
% The final ``nt'' indicates that the value comes from args.n_t in
% that script.

\chapter{Toy Problems and Performance Bounds}
\label{chap:toys}

Having developed algorithms for fitting model parameters, one might
reasonably ask how well the models so produced perform.  In this
chapter we argue that the exercise of fitting models to data from
chaotic dynamical systems is interesting because \emph{Lyapunov
  exponent} calculations give a quantitative benchmark against which
to compare model performance.  The idea is that the stretching or
local instability of dynamics, which Lyapunov exponents characterize,
limits the predictability \index{predictability} of sequences of
observations.  We will start by examining a toy example derived from
the Lorenz system that informally introduces the main ideas.  From
there we will review definitions of entropy and Lyapunov exponents and
results from information theory and ergodic theory that connect the
ideas.  Finally we explain a simple calculation that can determine
that a proposed model is fundamentally suboptimal.

We suppose that a \emph{true} stochastic process that assigns
probabilities to sequences of observations exists.  Many of the terms
that we define are \emph{expected values} with respect to those
probabilities, and we find that sample sequences converge to those
expected values.  We use $P_{*\given \mu}$ to denote these \emph{true}
probabilities\footnote{Following Kolmogorov, modern probability theory
  is cast a subfield of measure theory.  The measure theory literature
  uses the Greek letter $\mu$ for a function or \emph{measure} that
  maps sets to $\REAL^+$.  In earlier chapters, we have used
  $P_{*\given \theta}$ to denote parametric families of distributions.  We
  introduce the mongrel notation $P_{*\given \mu}$ here to make our notation
  for comparisons between a true distribution and a parametric model
  natural.  The meaning of the Greek letter $\mu$ here is not related
  to its use to denote the mean of a distribution.}, and we use them
to define expected values without delving in to theoretical questions
about their existence.

\subsubsection{Lorenz Example}

As an example, we have simulated a version of the %
\index*{Lorenz system} (Eqn.~\eqref{eq:Lorenz}) modified to fit the
form of Eqn.~\eqref{eq:contnoise},
\begin{align*}
  \ti{x}{t+1} &= F(\ti{x}{t},t) + \ti{\eta}{t}\\
  \ti{y}{t}   &= G(\ti{x}{t},t) + \ti{\epsilon}{t}).
\end{align*}
We have used the \index*{extended Kalman filter} described in
chapter~\ref{chap:continuous} to obtain parametric probability
functions $P\left(\ti{y}{t}\given \ts{y}{0}{t} , \theta \right)$ that
approximate $P\left(\ti{y}{t}\given \ts{y}{0}{t}, \mu \right)$, \ie, the
conditional distribution of the measurement at time $t$ given all
previous measurements.  Our code for generating sequences of
measurements has the following characteristics:
  \begin{description}
  \item[State] 
    \begin{equation*}
      x \equiv 
      \begin{bmatrix}
        x_1\\x_2\\x_3
      \end{bmatrix}
    \end{equation*}
  \item[Time step] We obtain the map $F$ by numerically integrating
    Eqn.~\eqref{eq:Lorenz} for time intervals of length $\tau_s$ with an
    absolute error tolerance of $\toyHviewatol$.
  \item[iid state noise] 
    \begin{equation*}
      \ti{\eta}{t} \sim \Normal(0, \begin{bmatrix} 1 & 0 & 0\\0 & 1 &
        0\\0&0&1 \end{bmatrix} \sigma_\eta^2)
    \end{equation*}
  \item[Measurement function] A simple projection
    \begin{equation*}
      G(x) = x_0 = \begin{bmatrix} 1,0,0 \end{bmatrix} \cdot x
    \end{equation*}
  \item[iid measurement noise] 
    \begin{equation*}
      \ti{\epsilon}{t} \sim \Normal(0, \sigma_\epsilon^2)
    \end{equation*}
  \item[Quantization] The observations are quantized with a resolution
    $\Delta=\toyHviewystep$.  We analyze quantized measurements rather than
    continuous measurements because they provide a \emph{finite} rather
    than \emph{infinite} amount of information and they are
    characterized by \emph{coordinate invariant} probability mass
    functions rather than \emph{coordinate dependent} probability
    density functions.
\end{description}
Recall that for the extended Kalman filter the means
$\ti{\mu_\gamma}{t}$ and variances $\ti{\sigma^2_\gamma}{t}$ completely
characterize $ P(\ti{y}{t}\given \ts{y}{0}{t},\theta)$ with
\begin{equation*}
  P(\ti{y}{t}\given \ts{y}{0}{t},\theta) =
  \NormalE{\ti{\mu_\gamma}{t}}{\ti{\sigma^2_\gamma}{t}}{\ti{y}{t}}.
\end{equation*}
(We use a lower case sigma here because the observations are scalars.)
We obtain affine maps for the approximation $F(x+\delta,t) \approx
\left[ DF(x)\right]\delta + F(x)$ %
\nomenclature[rDF]{$\left[ DF(x)\right]\delta$}{The \emph{derivative}
  of the function $F$ at $x$ applied to the vector $\delta$.  This
  notation emphasizes that $DF(x)$ is a linear map.}
%
by numerically integrating both the Lorenz system and the tangent
equations.  We use those approximations with Eqns.~\eqref{eq:KForeMu}
and \eqref{eq:KUpdate} on page \pageref{eq:KUpdate} to implement the
recursive calculation of $\ti{\mu_\gamma}{t}$ and
$\ti{\sigma_\gamma}{t}$ described by Eqns.~\eqref{eq:t274} to
\eqref{eq:IUpdate} on page \pageref{eq:IUpdate}.

Figures~\ref{fig:ToyTS1} and \ref{fig:ToyStretch} depict a simulation
in which dynamical stretching, \ie, the linear instability of
$\left[ DF(x)\right]$, occasionally limits predictability.  We chose
the parameters, specified in the caption of Fig.~\ref{fig:ToyTS1}, so
that dynamical noise and measurement quantization are negligible
compared to the effects of measurement noise and dynamical stretching.
In the middle plot of Fig.~\ref{fig:ToyTS1} notice that while for most
times the forecast deviation of the prediction $\ti{\sigma_\gamma}{t}$
is very close to the size of the state noise
$\sigma_\eta = \toyHviewStateNoise$, occasionally the forecast
deviations are many times larger.  The log likelihood per time step
which appears in the bottom plot of the figure is low when either the
forecast deviations are large or when the difference between the mean
of the forecast and the actual observation is much larger than the
predicted deviation, \ie,
$\ti{\sigma^2_\gamma}{t} <<
\left(\ti{y}{t}-\ti{\mu_\gamma}{t}\right)^2$.

The largest excursion of $\ti{\sigma_\gamma}{t}$ in
Fig.~\ref{fig:ToyTS1} occurs at $t=\toyHviewTMaxVar$.
Figure~\ref{fig:ToyStretch} illustrates the stretching action of the
map $[DF]$ that occurs then.

\begin{figure}[htbp]
  \centering{
    \includegraphics[width=1.0\textwidth]{ToyTS1.pdf}}
  \caption[\comment{fig:ToyTS1 }Extended Kalman filter for one step forecasting.]%
    {Extended Kalman filter for one step forecasting with simulation
    parameters:\\
    \begin{tabular}[c]{ll}
       $\tau_s=\toyHviewdt$ & Sample interval \\
       $\sigma_\eta = \toyHviewStateNoise$ & Standard deviation of state noise \\
       $\sigma_\epsilon = \toyHviewObservationNoise$ & Standard deviation of measurement noise \\
       $\Delta = \toyHviewystep$ & Measurement quantization \\
    \end{tabular}\\
    A time series of observations appears in the upper plot.  The
    middle plot characterizes the one-step forecast distributions
    $P_{\gamma} \left(\ti{y}{t} \right) \equiv P
    \left(\ti{y}{t}\given \ts{y}{0}{t},\theta \right) =
    \NormalE{\ti{\mu_\gamma}{t}}{\ti{\sigma^2_\gamma}{t}}{\ti{y}{t}}$.
    The standard deviations of the forecasts appear in the first
    trace, and the differences between the actual observations
    and the means of the forecasts appear in the second trace.  The
    logs of the likelihoods of the forecasts, $\log(P_{\gamma}
    \left(\ti{y}{t} \right))$, appear in the bottom plot.}
  \label{fig:ToyTS1}
\end{figure}
%%%
\begin{figure}[htbp]
  \centering{\resizebox{\textwidth}{!}{\includegraphics{ToyStretch.pdf}}
  }
  \caption[\comment{fig:ToyStretch }Dynamical stretching increases
  state variance.]%
  {These plots illustrate dynamical stretching increasing the variance
    of the conditional distribution in state space between time steps
    $\toyHviewTMaxVar$ and $\toyHviewTMaxVarPlusOne$ in
    Fig.~\ref{fig:ToyTS1}.  In each plot, the \emph{forecast} state
    distribution ellipse represents
    $P_{a} \left(\ti{x}{t} \right) \equiv P
    \left(\ti{x}{t}\given \ts{y}{0}{t},\theta \right) =
    \NormalE{\mu_a}{\Sigma_a}{\ti{x}{t}}$ and the \emph{update} state
    distribution ellipse represents
    $P_{\alpha} \left(\ti{x}{t} \right) \equiv P
    \left(\ti{x}{t}\given \ts{y}{0}{t+1}, \theta \right) =
    \NormalE{\mu_\alpha}{\Sigma_\alpha}{\ti{x}{t}}$.  For each
    distribution, an ellipse depicts the level set
    $(x-\mu)\transpose \Sigma^{-1} (x-\mu) =1$ in the $x_0\times x_2$
    plane.  To support visual comparisons, the sizes of the ranges for
    $x_0$ and $x_2$ are the same in each of the plots.  In mapping the
    updated distribution at $t=\toyHviewTMaxVar$ (on the left) to the
    forecast distribution at time $t=\toyHviewTMaxVarPlusOne$ (on the
    right), the function $F$ that implements state dynamics stretches
    the ellipse by about a factor of 10 in both directions.}
  \label{fig:ToyStretch}
\end{figure}

Figure~\ref{fig:ToyH} illustrates the behavior of $-\hat h$, the sample
average of the log likelihood of forecasts, for a group of simulations
with parameters that are quite different from those in
Figs.~\ref{fig:ToyTS1} and \ref{fig:ToyStretch}.  Given model parameters
$\theta$ and a sample sequence $\ts{y}{0}{T}$ of length $T$, we define
\begin{align*}
  -\hat h &\equiv \frac{1}{T} \sum_{t=0}^{T-1}
    \log \left( P \left(\ti{y}{t}\given \ts{y}{0}{t}, \theta
      \right)\right)\\
    &=  \frac{1}{T} \log \left( P \left(\ts{y}{0}{T}\given  \theta
      \right)\right).
\end{align*}
The negative of this sample log likelihood is an estimate of the
\emph{cross entropy rate} \index{entropy!cross rate}
\begin{equation*}
  h(\mu||\theta) \equiv  \lim_{T \rightarrow \infty} - \frac{1}{T}
  \EV_{\mu} \log \left( P \left(\ts{Y}{0}{T}\given \theta \right) \right)
\end{equation*}
which in turn is bounded from below by the \emph{entropy rate}.  We
discuss both entropy rate and cross entropy rate in
Section~\ref{sec:pesin}, and in Section~\ref{sec:PesinFormula} we
review the \emph{Pesin Formula}.  That formula says that the largest
\emph{Lyapunov exponent} $\lambda_0$, a characterization of the
stretching action of the dynamics, is equal to the entropy rate, a
characterization of the predictability.  For good models, we expect
the log likelihood of forecasts to fall off with a slope of
$-\lambda_0$ as the sample time $\tau_s$ increases, and for the best
possible model we expect
\begin{equation}
  \label{eq:bound1}
  h(\tau_s) = \lambda_0 \tau_s.
\end{equation}

We have chosen the parameters\footnote{ In making Fig.~\ref{fig:ToyH},
  we wanted simulations close to the bound of Eqn.~\eqref{eq:bound1}.
  We found that at larger values of $\tau_s$ and $\Delta$, extended
  Kalman filters performed better if given models with larger state
  noise than the noise actually used to generate the data, \ie
  $\tilde \sigma_\eta > \sigma_\eta$.  We believe that the effect is
  the result of the larger errors that occur as the affine
  approximation $F(x+\delta) \approx [DF(x)] \delta + F(x)$ fails to
  track the nonlinearities of the Lorenz system over larger intervals
  in state space.  By making $\tilde \sigma_\eta$ larger, the errors
  are accommodated as state noise.  We chose the state noise of the
  generating process to be an order of magnitude larger than the
  absolute integration tolerance of $\toyHviewatol$.  We then chose
  the quantization level and sample times to be as large as possible,
  but still small enough that we could have
  $\tilde \sigma_\eta = \sigma_\eta$ without losing performance.  That
  led to the values
  $\tilde \sigma_\eta = \sigma_\eta = \toyToyhdevstategenerate$,
  $\Delta = \toyToyhystep$, and $0 < \tau_s \leq 0.5$.} specified in
the caption of Fig.~\ref{fig:ToyH} with a measurement quantization
size large enough that the log likelihood is limited primarily by
dynamical stretching and the Gaussian model for
$ P \left(\ti{y}{t}\given \ts{y}{0}{t}, \theta \right)$.  We are pleased
to observe that the overall slope of the plot on the left in
Fig.~\ref{fig:ToyH} is consistent with the estimate
$\hat \lambda_0 = \toyBenettinLambdaOne$ (base $e$) that we obtain
using the Benettin procedure described in Section~\ref{sec:Benettin}.
%%% code/python/lorenzLyap.py

\begin{figure}[htbp]
  \centering{
    \includegraphics[width=1.0\textwidth]{ToyH.pdf}}
  \caption[\comment{fig:ToyH }Average log likelihood of one step forecasts.]%
  {Average log likelihood of one step forecasts as a function of time
    step $\tau_s$ and filter parameter $\tilde \sigma_\epsilon$.  To
    \ToDo{Check code that produces these values}
    simulate measurements for this figure, we used the parameters:\\
    \begin{tabular}[c]{ll}
     $\sigma_\eta = \toyToyhdevstategenerate$ & Standard deviation of state noise \\
     $\sigma_\epsilon = \toyToyhdevmeasurement$ & Standard deviation of measurement noise \\
     $\Delta = \toyToyhystep$ & Measurement quantization \\
     $T=\toyToyhnt$ & Number of samples
    \end{tabular}\\
    For both plots, the vertical axis is the average log likelihood of
    the one-step forecast $-\hat h \equiv \frac{1}{T} \sum_{t=0}^{T-1}
    \log \left( P \left(\ti{y}{t}\given \ts{y}{0}{t}, \theta
      \right)\right)$.  On the left we plot $-\hat h$ as a function of
    both $\tau_s$, the time step, and $\tilde \sigma_\epsilon$, the
    standard deviation of the measurement noise model used by the
    extended Kalman filter.  On the right the top row of dots
    indicates the
    performance of filters that use measurement noise models that depend on the
    sampling time through the formula $\tilde \sigma_\epsilon(\tau_s) =
    10^{m \cdot \tau_s + b}$, with $m =0.4$ and $b=-4.85$ chosen by hand to
    follow the ridge top in the plot
    on the left.  Also on the right, the Bottom row of dots
    indicates the performance of
    filters that use $\tilde \sigma_\epsilon = \toyToyhystep$, \ie the
    measurement quantization level, and the solid line traces
    Eqn.~\eqref{eq:ToyH} in the text.}
   \label{fig:ToyH}
 \end{figure}

In the plot on the right in Fig.~\ref{fig:ToyH} we see that for a
class of filters in which the standard deviation of the model
measurement noise $\tilde \sigma_\epsilon$ is set to the quantization
size $\Delta$, the log likelihood closely follows the approximation
\begin{equation}
  \label{eq:ToyH}
  \hat h(\tau_s) = \log\left(\text{erf}\left(\frac{1}{2\sqrt{2}}\right)\right)
  + \lambda_0 \tau_s,
\end{equation}
where \emph{erf} is the error function\footnote{The error function is
  defined by $\operatorname{erf}(x) = \frac{2}{\sqrt{\pi}}\int_0^x
  e^{-t^2} dt$.}.  We explain the nonzero intercept in
Eqn.~\eqref{eq:ToyH} by observing that in the limit of small sampling
interval ($\tau_s \rightarrow 0$) and zero noise ($\sigma_\eta
\rightarrow 0$ and $\sigma_\epsilon \rightarrow 0$), only one discrete
observation $\ti{y}{t}=\bar y$ is possible given a history
$\ts{y}{0}{t}$.  For data drawn from that limiting case, a Kalman
filter with parameters $\tilde \sigma_\epsilon = \Delta$ and
$\sigma_\eta \rightarrow 0$ would make a forecast with a density
$P(\ti{y}{t}\given \ts{y}{0}{t}) = \NormalE{\bar
  y}{(\Delta)^2}{\ti{y}{t}}$.  Integrating that density over the
quantization interval yields
 \begin{align*}
   P_{\ti{y}{t}\given \theta}(\bar y) &= \int_{\bar y -
     \frac{\Delta}{2}}^{\bar y + \frac{\Delta}{2}} \frac{1}{\sqrt{2\pi
       (\Delta)^2}}e^{-\frac{(y-\bar
       y)^2}{2(\Delta)^2}} dy\\
   &= \int_{-\frac{1}{2}}^{\frac{1}{2}} \frac{1}{\sqrt{2\pi
     }}e^{-\frac{1}{2}s^2} ds\\
   &= \text{erf} \left( \frac{1}{2\sqrt{2}}\right) \\
   &\approx \toyErfSqrtEight \\
   \log \left( P_{\ti{y}{t}\given \theta}(\bar y) \right) &\approx \toyLogErfSqrtEight.
 \end{align*}
Given the simplicity of the analysis, Eqn.~\eqref{eq:ToyH} fits the
simulations in Fig.~\ref{fig:ToyH} remarkably well.


\section{Fidelity Criteria and Entropy}
\label{sec:fidelity}

Stochastic models are fit to an enormous variety of measured phenomena
and the most appropriate measure of the fidelity of a model to
measurements depends on the application.  Such phenomena include long
and short term weather, financial markets, computer data, electric
power demand, and signals and noise in communication or
instrumentation.  In many cases one makes decisions based on a model
and those decisions change the \emph{cost} of future events.  The
expected cost of basing decisions on a model $P_{*\given \theta}$ depends in
a complicated fashion on many factors including how the cost of acting
on a decision depends on lead time and which aspects of the modeled
phenomenon are important.  For the Lorenz example at the beginning of
this chapter we implicitly assumed \emph{stationarity} and
\emph{ergodicity} and characterized model quality in terms of the
average of the log of the likelihood.  For a stationary ergodic
system, the log-likelihood is tied to $D(\mu||\theta)$, the
\emph{relative entropy} of the model $P_{*\given \theta}$ given $P_{*\given \mu}$
(see Eqn.~\eqref{eq:RelativeEntropy}).  Although relative entropy is
not an appropriate performance measure for every application, it is a
common tool for problems in information theory and statistics.  See
Cover and Thomas\cite{Cover91} for many of these including the
application of relative entropy to a theory of gambling.  Relative
entropy is exactly the right performance measure for data compression.
The arithmetic coding algorithm (see the review by Witten, Neal, and
Cleary\cite{Witten87}) for compressing a sequence of symbols uses a
model $P_{*\given \theta}$ to make decisions that affect the cost in a
manner that depends on the symbol values that actually occur.  The
relative entropy $D(\mu||\theta)$ is the expected value of the number
of bits wasted by the algorithm if it uses a model $P_{*\given \theta}$ for
decisions when in fact $P_{*\given \mu}$ is true.  More accurate models lead
to better compression.

\subsection{Definitions}
\label{sec:hDef}

Now, to solidify the discussion, we make some formal definitions.

\subsubsection{Stochastic process}
\index{stochastic process}
We are interested in sequences of states $\ts{X}{0}{T}$ and
measurements  $\ts{Y}{0}{T}$ each of which can be thought of as a
\emph{random function} on the domain $\left\{ 1,2,\ldots,T \right\}$,
\ie, a \emph{stochastic process}.

\subsubsection{Entropy of a discrete random variable}
\index{entropy}
If a discrete random variable $U$ takes on the values
$u_0,u_1,\ldots,u_n$ with probabilities $P(u_0),P(u_1),\ldots,P(u_n)$,
then the \emph{entropy} of $U$ is
\nomenclature[rHU]{$H(U)$}{The entropy of a discrete random variable $U$.}
\begin{equation}
  \label{eq:HDef}
  H(U) \equiv - \EV \log \left( P(U) \right) = -\sum_{k=1}^n P(u_k)
  \log \left( P(u_k) \right).
\end{equation}
Entropy quantifies the the uncertainty in $U$ before its value is
known and the information or degree of surprise in discovering its
value.  If the base of the logarithm in Eqn.~\eqref{eq:HDef} is 2,
then the units of $H(U)$ are called \emph{bits}.  We will use natural
logarithms with Euler constant $e$ as the base.  For natural
logarithms the units of $H(U)$ are called \emph{nats}.

\subsubsection{Differential entropy of a continuous random variable}
\index{entropy!differential of a continuous random variable}
If $U$ is a continuous random variable with a probability density
function $P$ then the \emph{differential entropy} of $U$ is
\newcommand{\hdiff}{\tilde H}
\begin{equation}
  \label{eq:hdiffDef}
  \hdiff(U) \equiv  - \EV \log \left( P(U) \right) = - \int P(u) \log
  \left( P(u) \right) du.
\end{equation}
\nomenclature[rhv]{$\hdiff(U)$}{Differential entropy of the continuous random
  variable $U$.}  Notice that the differential entropy depends on the
coordinates of $U$.

\subsubsection{Conditional entropy}
\index{entropy!conditional}
The \emph{conditional entropy} of $U$ given $V$ is
\nomenclature[rHUV]{$H(U"|V)$}{The \emph{conditional entropy} of $U$ given
  $V$.}
\begin{equation*}
  H(U|V) \equiv - \EV \log \left( P(U|V) \right) = -\sum_{i,j} P(u_i,v_j)
  \log \left( P(u_i|v_j) \right).
\end{equation*}
Factoring the probability of sequences
\begin{equation*}
  P_{\ts{z}{0}{T} } = P _{\ti{z}{0} }
  \prod_{t=0}^{T-1} P _{\ti{z}{t}\given \ts{z}{0}{t} }
\end{equation*}
is equivalent analyzing entropy into the sum
\begin{equation*}
  H(\ts{Z}{0}{T}) = H(\ti{Z}{0}) + \sum_{t=1}^{T-1} H
  \left(\ti{Z}{t}|\ts{Z}{0}{t} \right).
\end{equation*}

\subsubsection{Relative entropy of two probability functions}
\index{entropy!relative} The \emph{relative entropy} between two
probability functions $P_{*\given \mu}$ and $P_{*\given \theta}$ with the same
domain ${\cal Z}$ is %
\nomenclature[rDmu]{$D(\mu"|"|\theta)$}{The \emph{relative entropy}
  between two probability functions $P_{*"\given \mu}$ and $P_{*"\given \theta}$.}
\begin{align}
  \label{eq:RelativeEntropy}
  D(\mu||\theta) &\equiv \EV_{\mu} \log \left(
    \frac{P(Z\given \mu)}{P(Z\given \theta)} \right) \\
  &= \sum_{z \in \cal{Z}} P(z\given \mu) \log\left(
  \frac{P(z\given \mu)}{P(z\given \theta)} \right) \nonumber.
\end{align}
The relative entropy is coordinate independent.  The relative entropy
between two probability functions $P_{*\given \mu}$ and $P_{*\given \theta}$ is never
negative and is zero if and only if the functions are the same on all
sets with finite probability.  We use $D(\mu||\theta)$ to
characterize the fidelity of a model $P_{*\given \theta}$ to a \emph{true}
distribution $P_{*\given \mu}$

\subsubsection{Cross entropy of two probability functions}
\index{entropy!cross}
While some authors use the terms \emph{relative entropy} and
\emph{cross entropy} interchangeably to mean the quantity
$D(\mu||\theta)$ that we defined in Eqn.\eqref{eq:RelativeEntropy}, we
define the cross entropy to be
\begin{align}
  \label{eq:CrossEntropy}
  H(\mu||\theta) &\equiv -\EV_{\mu} \log \left(P(Z\given \theta) \right) \\
  &= -\sum_{z \in \cal{Z}} P(z\given \mu) \log(P(z\given \theta)) \nonumber
\end{align}
and note that
\begin{equation*}
  D(\mu||\theta) = H(\mu||\theta) - H(\mu).
\end{equation*}
The cross entropy is the negative expected log-likelihood of a model.
It is greater than the entropy unless the model $P_{*\given \theta}$ is the same
as $P_{*\given \mu}$ for all sets with finite probability.

\subsubsection{Stationary}
\index{stationary}
A stochastic process is \emph{stationary} if probabilities are
unchanged by constant shifts in time, \ie, for any two integers
$T\geq 1$ and $\tau\geq 0$
\begin{equation*}
  P_{\ts{Z}{0}{T}} =  P_{\ts{Z}{0+\tau}{T+\tau}}.
\end{equation*}

\subsubsection{Ergodic}
Roughly, in an \emph{ergodic} \index{ergodic|textbf} process you can
get anywhere from anywhere else.  Let ${\cal X}$ be the set of states
for a stationary stochastic process with probabilities $P_{*\given \mu}$.
The process is ergodic if for any two subsets of ${\cal X}$, $A$ and
$B$ with $P(A\given \mu) > 0$ and $P(B\given \mu) > 0$ there is a time $T$ such
that the probability of going from set $A$ to set $B$ in time $T$ is
greater than zero.  Birkhoff's ergodic theorem says that for an
ergodic process, time averages converge to expected values with
probability one.

\subsubsection{Entropy rate}
For a discrete stochastic process $X$, the entropy rate \index{entropy
  rate} is \nomenclature[rhx]{$h(X)$}{The \emph{entropy rate} of a
  stochastic process $X$.}
\begin{equation}
  \label{eq:hrate1}
  h(X) \equiv \lim_{T \rightarrow \infty} \frac{1}{T} H(\ts{X}{0}{T}).
\end{equation}
If the process is stationary
\begin{equation}
  \label{eq:hrate2}
  h(X) = \lim_{T \rightarrow \infty} H(\ti{X}{T}|\ts{X}{0}{T}).
\end{equation}
If the process is stationary and Markov
\begin{equation}
  \label{eq:hrate3}
  h(X) =  H(\ti{X}{T+1}|\ti{X}{T})~\forall T.
\end{equation}
And if the process is ergodic
\begin{equation}
  \label{eq:hrate4}
  h(X) = \lim_{T \rightarrow \infty} - \frac{1}{T} \log \left( P \left(\ts{x}{0}{T}\given \mu \right) \right)
\end{equation}
with probability one.

We similarly define the relative entropy rate and the cross entropy
rate.  For an ergodic process $X$ with true probabilities $P_{*\given \mu}$ and
model probabilities $P_{*\given \theta}$
\begin{align}
  \label{eq:hrate5}
  h(\mu||\theta) &\equiv  \lim_{T \rightarrow \infty} - \frac{1}{T}
  \EV_\mu \log \left( P \left(\ts{X}{0}{T}\given \theta \right) \right) \\
  &=  \lim_{T \rightarrow \infty} - \frac{1}{T} \log \left( P
  \left(\ts{x}{0}{T}\given  \theta \right) \right) \nonumber
\end{align}
with probability one.

\subsubsection{Entropy rate of a partition $\mathcal{B}$}
\index{partition} Let ${\cal X}$, the set of states for a
stationary stochastic process with probabilities $P_{*\given \mu}$, be a
continuum, and let $\mathcal{B} = \left\{ \beta_0,\beta_1,\ldots
  \beta_n \right\}$ be a partition of ${\cal X}$ into a finite number
of non-overlapping subsets.  By setting $\ti{b}{t}$ to the index of
the element of $\mathcal{B}$ into which $\ti{x}{t}$ falls, we can map
any sequence $\ts{x}{0}{T}$ into a sequence $\ts{b}{0}{T}$ thus
defining a discrete stationary stochastic process $B$.  Applying the
definition of entropy rate to the process $B$ yields the definition of
the entropy rate as a function of partition $\mathcal{B}$.  Suppose in
particular that on some set ${\cal X}$ the map $F:{\cal X}\mapsto{\cal
  X}$ and the probability $P_{*\given \mu}$ define an ergodic process, that
$\mathcal{B}$ is a partition of ${\cal X}$, and that the model
probability function $P_{*\given \theta}$ assigns probabilities to sequences
of partition indices $\ts{b}{0}{T}$.  We define the entropy rate $
h(\mathcal{B},F,\mu)$ and the \index*{cross entropy rate} %
$h(\mathcal{B},F,\mu||\theta)$ as follows
\begin{align}
  \label{eq:hrate6}
  h(\mathcal{B},F,\mu) &\equiv \lim_{T \rightarrow \infty} \frac{1}{T}
  H(\ts{B}{0}{T}) \\
  &= \lim_{T \rightarrow \infty} -\frac{1}{T} \EV_\mu \log\left(P
  \left(\ts{B}{0}{T}\given  \mu \right) \right) \nonumber\\
  \label{eq:hrate7}
  h(\mathcal{B},F,\mu||\theta) &\equiv \lim_{T \rightarrow \infty}
  -\frac{1}{T} \EV_\mu \log\left(P \left(\ts{B}{0}{T} \given  \theta \right)
  \right). \nonumber  
\end{align}
\subsubsection{Kolmogorov Sinai entropy $h_{KS}$}
\index{Kolmogorov Sinai entropy|textbf}%
\index{entropy!Kolmogorov Sinai|see{Kolmogorov Sinai entropy}}%
\nomenclature[rhw]{$h_{KS}$}{Kolmogorov Sinai entropy.}%
As before, suppose that on some set ${\cal X}$, the map $F:{\cal
  X}\mapsto{\cal X}$ and the probability $P_{*\given \mu}$ define an ergodic
process.  The least upper bound over all partitions $\mathcal{B}$ on
the entropy rate $h(\mathcal{B},F,\mu)$ is the called the
\emph{Kolmogorov Sinai entropy}
\begin{equation}
  \label{eq:hKS}
  h_{KS}(F,\mu) \equiv \sup_{\mathcal{B}} h(\mathcal{B},F,\mu).
\end{equation}

\section{Stretching and Entropy}
\label{sec:pesin}

Here we will outline theory that connects ideas from dynamics to ideas
from probability.  The main results say that average dynamical
stretching (Lyapunov exponents) is proportional to average uncertainty
(entropy) in measurement sequences.  First we will give some examples,
then we will quote definitions and theorems without proof.

\subsection{Maps of the unit circle}
\label{sec:PesinExamples}
\index{unit circle!map of}
\index{map of unit circle}

\subsubsection{Two $x$ mod one}
\label{sec:TwoX}

The map of the unit interval $[0,1)$ into itself defined by
\begin{align}
  \ti{x}{n+1} &= F_2(\ti{x}{n})\\
  \label{eq:twox}
  &\equiv 2x \mod 1
\end{align}
is continuous if we identify the points $0$ and $1$.  Notice that if
we use the partition
\begin{equation}
  \label{eq:partition2}
  \mathcal{B}_2 = \left\{ \beta_0 = [0,\frac{1}{2}),\beta_1 =
  [\frac{1}{2}, 1 )\right\},  
\end{equation}
a symbol sequence $\ts{b}{0}{\infty}$ provides the coefficients of a
base two power series that identifies the starting point, \ie
\begin{equation*}
  \ti{x}{0} = \sum_{t=0}^\infty  \ti{b}{t} \left(\frac{1}{2}\right)^{t+1}.
\end{equation*}
Further, if we assign a uniform probability measure $\mu$ to the
interval, then
\begin{equation*}
  h_{KS}(F_2,\mu) =h(\mathcal{B}_2,F,\mu) = \log(2).
\end{equation*}

\subsubsection{Three $x$ mod one}
\label{sec:ThreeX}

Analogously, by defining the map
\begin{align}
  \ti{x}{n+1} &= F_3(\ti{x}{n})\\
  \label{eq:threex}
  &\equiv 3x \mod 1,
\end{align}
and using the partition
\begin{equation}
  \label{eq:partition3}
  \mathcal{B}_3 = \left\{ \beta_0 = [0,\frac{1}{3}),\beta_1 =
    [\frac{1}{3}, \frac{2}{3} ), \beta_2 = [\frac{2}{3}, 1 )\right\}
\end{equation}
and a uniform probability measure $\mu$, we find
\begin{equation*}
  h_{KS}(F_3,\mu) =h(\mathcal{B}_3,F_3,\mu) = \log(3).
\end{equation*}

This is the result that motivated Kolmogorov and Sinai to define the
entropy $h_{KS}$.  They were addressing the isomorphism problem, \eg, ``Is
there a relabeling of points in the unit interval that makes $F_2$ the
same as $F_3$?''.  Since the characteristic $h_{KS}$ is independent of
the coordinates or labeling used, the fact that $h_{KS}(F_3,\mu) \neq
h_{KS}(F_2,\mu)$ provided a negative answer to the question.

Notice that the Kolmogorov entropy is equal to the average of the log
of the slope of the map.  Specifically, the slope of $F_2$ is 2 and
$h_{KS}(F_2,\mu) = \log(2)$ while the slope of $F_3$ is 3 and
$h_{KS}(F_3,\mu) = \log(3)$.  The rule that entropy is proportional to
the log of the average slope is not true in general.  The next example
provides a counter example and suggests a correction factor.

\subsubsection{Dynamics on a \index*{Cantor set}}
\label{sec:Cantor}

While every point in the entire unit interval can be represented as a
base three power series, \ie
\begin{equation*}
  \forall x \in [0,1),~ \exists d_0^\infty :~x  = \sum_t^\infty
  \ti{d}{t} \left(\frac{1}{3}\right)^{t+1} \text{ with } \ti{d}{t} \in
  \left\{ 0,1,2 \right\} ~ \forall t,
\end{equation*}
the middle third Cantor set consists of the points in the unit
interval that can be represented as base three power series that
exclude the digit ``1''.  The symbol sequences produced by applying
the map $F_3$ and partition $\mathcal{B}_3$ to the middle third Cantor
set are the sequences of coefficients in the base three expansions of
the starting points, \ie, they consist exclusively of $0's$ and $2's$.
Given any finite sequence $d_0^n$, we define the set of infinite
coefficient sequences $\left\{d_0^n,\ldots \right\}$ as those that
begin with the sequence $d_0^n$.  Now we define a probability measure
$\mu_c$ in terms of such sets of infinite sequences,
\begin{equation}
  \label{eq:muC}
  \mu_c\left( \left\{ d_0^n,\ldots \right\} \right) \equiv 
  \begin{cases}
    2^{-(n+1)} & \text{if ``1'' does not appear in } d_0^n\\
    0 & \text{if ``1'' does appear in } d_0^n
  \end{cases}
\end{equation}
With this measure we find
\begin{equation*}
  h_{KS}(F_3,\mu_c) = \log(2).
\end{equation*}

The following isomorphism or relabeling of the unit interval connects
$(F_2,\mu)$ to $(F_3,\mu_c)$:
\begin{enumerate}
\item Find the binary expansion $\ts{b}{0}{\infty}$ of the original
  point $x$
\item Create a new sequence $\ts{d}{0}{\infty}$ by replacing every
  occurrence of ``1'' in  $\ts{b}{0}{\infty}$ with ``2''
\item Map $x$ to $y$ where $y$ is described by the base three
  expansion $\ts{d}{0}{\infty}$
\end{enumerate}

The \index*{Hausdorff dimension}\footnote{Sets and characteristics of
  sets with non-integer dimensions are sometimes called
  \emph{fractal}.} of the middle third Cantor set is
$\delta = \frac{\log(2)}{\log(3)}$, and that is the factor that is
missing in the formula connecting entropy and stretching.
\begin{align}
  h_{KS}(F_3,\mu_c) &= \log(2) \nonumber \\
  &= \frac{\log(2)}{\log(3)} \log(3) \nonumber \\
  \label{eq:CantorCorrect}
  &= \delta \log(\text{stretching factor})
\end{align}
Now we turn to the definitions and theorems that express the above
idea precisely.

\section{Lyapunov Exponents and Pesin's Formula}
\label{sec:PesinFormula}

Vixie\cite{vixie02} has reviewed the work of
Ruelle\cite{ruelle-1978-1}, Pesin\cite{pesin77},
Young\cite{young95-1}, and others who established the relationship
between smooth dynamics and entropy.  Here we reiterate a few of those
results using the following notation:
\begin{description}
\item[$X$] An $n$-dimensional manifold
\item[$F:X\mapsto X$] An invertible $C^2$ (continuous with continuous
  first and second derivatives) map of the manifold into itself
\item[$\mu$] A probability measure on $X$ that is invariant under $F$
\item[$x$] A point on the manifold
\item[$TX(x)$] The tangent space of $X$ at $x$
\item[$v$] An element of $TX(x)$
\end{description}

We define the asymptotic growth rate of the direction $v$ at $x$ as
\begin{equation}
  \label{eq:growthVX}
  \lambda(F,x,v) \equiv \lim_{t\rightarrow \infty} \frac{1}{t} \log
  \left( \left\| [DF^t(x)] v \right\| \right).
\end{equation}
\index*{Oseledec's theorem} ~\cite{young95-1,katok95, mane87} says
that at almost every $x$ the limit exists for every $v$, and that
although the value of the limit depends on $v$, that as $v$ varies, it
only takes on $r \leq n$ discrete values called the \emph{spectrum of
  \index*{Lyapunov exponents}}
\begin{equation}
  \label{eq:LyapunovSpectrum}
  \lambda_0(F,x) > \lambda_1(F,x) > ... > \lambda_r(F,x).
\end{equation}
The tangent space $TX(x)$ is the direct sum of subspaces $E_i \subset
TX(x)$ associated with each exponent, with
\begin{equation*}
  \lambda(F,x,v) = \lambda_i(F,x) ~ \forall v \in E_i
\end{equation*}
and
\begin{equation*}
  TX(x) = \bigoplus_{i=1}^r E_i. %\bigoplus?
\end{equation*}
The dimension of $E_i$ is called the \emph{multiplicity} $m_i$ of the
exponent $\lambda_i$. If $\mu$ is ergodic with respect to $F$, then
the spectrum $\left\{ \lambda_i \right\}$ is the same almost
everywhere.

We want to use \index*{Pesin's formula}\cite{pesin77} which implies
that if $\mu$ is smooth and ergodic, then the entropy is equal to the
sum of the positive Lyapunov exponents, \ie
\begin{equation}
  \label{eq:pesin}
  h_{KS}(F,\mu) = \sum_{i:\lambda_i >0} m_i \lambda_i.
\end{equation}
In light of the correction for fractal dimension that we saw in
Eqn.~\eqref{eq:CantorCorrect} and the ubiquity of fractal measures in
chaotic systems, we should review Ledrappier and Young's explanation
(See \cite{young95-1} for an overview) of the effect of fractal
measures on Pesin's formula.

Ledrappier and Young's formula is given in terms of the dimensions of
the conditional measures on the nested family of \emph{unstable
  foliations} of $F$.  For a point $x\in X$ and $i$ such that
$\lambda_i>0$ we define
\begin{equation}
  \label{eq:Foliation}
  W^i(x) \equiv \left\{ y \in X \text{ such that } \lim_{n \rightarrow
  \infty} \sup \frac{1}{t} \log \left( d\left( F^{-t}(x), F^{-t}(y)
  \right) \right) <
  -\lambda_i \right\}.
\end{equation}
For an intuitive picture, consider a trajectory $x(t)$ that passes
through $x$ at time $t=0$; any trajectory $y(t)$ that has separated
from $x(t)$ at a rate of at least $\lambda_i$, passes through the
manifold $W^i(x)$ at time $t=0$.

Let $\delta_i$ be the Hausdorff dimension of the conditional measure
that $\mu$ defines on $W^i(x)$.  For an ergodic $\mu$, $\delta_i$ will
be constant almost everywhere.  Further, let $\gamma_i$ be the
incremental dimension
\begin{equation*}
  \gamma_i \equiv
  \begin{cases}
    \delta_0 & i=0\\
    \delta_i - \delta_{i-1} & i > 0
  \end{cases}
\end{equation*}
Now \index*{Ledrappier and Young's formula} is
\begin{equation}
  \label{eq:Ledrappier}
  h_{KS}(F,\mu) = \sum_{i : \lambda_i > 0} \lambda_i \gamma_i.
\end{equation}

Note that Pesin's formula holds if the measure $\mu$ is smooth in the
unstable directions.  Such measures are called SRB (Sinai Ruelle
Bowen) measures.  \index{SRB measure} Tucker has found that the Lorenz
system has an SRB measure and says that numerical simulations of
Lorenz's system are ``real''~\cite{tucker99}.

\subsection{A theoretical bound on model likelihood}
\label{sec:TheoreticalBound}

Now we have the terms that we need to discuss theoretical bounds on
the expected log likelihood of models of discrete observations of a
chaotic dynamical system.  Given $X$, $F$, and $\mu$ as described
above, if the multiplicity of each exponent is $m_i = 1$ then we know:
\begin{align}
  \label{eq:Bound1}
  h({\cal{B}},F,\mu) &\equiv - \lim_{t \rightarrow
    \infty} \EV_\mu \log \left( P \left( \ti{b}{t}\given \ts{b}{0}{t}, \mu
    \right) \right) \\
  \label{eq:Bound2}
  h({\cal{B}},F,\mu||\theta) &\equiv - \lim_{t \rightarrow \infty}
  \EV_\mu \log \left( P \left( \ti{b}{t}\given \ts{b}{0}{t}, \theta
    \right) \right) \\
  \label{eq:Bound3}
  h({\cal{B}},F,\mu) &\leq h({\cal{B}},F,\mu||\theta)  &&
  \text{Equality } \iff \theta = \mu \text{ a.e.}\\
  \label{eq:Bound4}
  h({\cal{B}},F,\mu) &\leq h_{KS}(F,\mu)  &&
  \text{Equality } \iff {\cal{B}} \text{ generating} \\
  \label{eq:Bound5}
  h_{KS}(F,\mu) &= \sum_{i : \lambda_i > 0} \lambda_i \gamma_i\\
  \label{eq:Bound6}
  h_{KS}(F,\mu) &\leq \sum_{i : \lambda_i > 0} \lambda_i && \mu \text{
  smooth on } W^i \Rightarrow \text{Equality}
\end{align}
with the following justifications
\begin{description}
\item[\eqref{eq:Bound1} {\mdseries and} \eqref{eq:Bound2}:] Definition
\item[\eqref{eq:Bound3}:] Gibbs inequality, \eqref{eq:GibbsIE}
\item[\eqref{eq:Bound4}:] The definition of $h_{KS}(F,\mu)$ is that it is the
  \emph{supremum} over all partitions ${\cal{B}}$
\item[\eqref{eq:Bound5}:] This is Ledrappier and Young's formula
  \eqref{eq:Ledrappier}
\item[\eqref{eq:Bound6}:] Because in \eqref{eq:Bound5} $0 \leq \gamma_i \leq
  1~ \forall i$
\end{description}
Thus we have the following two theorems:
\begin{theorem}[Lyapunov exponent bound on likelihood]
  If $\mu$ is ergodic and smooth in the unstable directions and
  $\cal{B}$ is a generating partition, then for any model $\theta$ of
  the stochastic process $B$ consisting of $F,\mu$, and $\cal{B}$
  \begin{equation}
    \label{eq:BoundTheorem}
    h({\cal{B}},F,\mu||\theta) \geq \sum_{i : \lambda_i > 0} \lambda_i
  \end{equation}
\end{theorem}
\begin{theorem}[Entropy gap] \label{GapTheorem}
  \index{entropy!gap|textbf}
  If $\mu$ is ergodic (not necessarily smooth in the unstable
  directions).  Then for an optimal model $\theta$ of the stochastic
  process $B$ consisting of $F,\mu$, and $\cal{B}$ ($\cal{B}$ not
  necessarily generating)
  \begin{equation}
    \label{eq:GapTheorem1}
    h({\cal{B}},F,\mu||\theta) = h({\cal{B}},F,\mu) \leq  \chi \equiv \sum_{i :
    \lambda_i > 0} \lambda_i
  \end{equation}
  and if for some other model $\nu$
  \begin{equation}
    \label{eq:GapTheorem2}
    h({\cal{B}},F,\mu||\nu) \geq \chi
    % \equiv \sum_{i : \lambda_i > 0} \lambda_i m_i
  \end{equation}
  then the model $\nu$ is not optimal.
\end{theorem}

In the next section, we will describe a numerical procedure for
estimating Lyapunov exponents, and in the following section we will
argue that one can reasonably use Eqn.~\eqref{eq:GapTheorem2} with
numerical simulations to quantitatively characterize the
non-optimality of a model.

\section{Benettin's Procedure for Calculating Lyapunov Exponents Numerically}
\label{sec:Benettin}

We begin reviewing \index*{Benettin's procedure}\cite{Benettin80} for
estimating Lyapunov exponents by using the Lorenz system as an
example.  The Lorenz system is
\begin{align*}
  \dot x = F(x) = 
  \begin{bmatrix}
    s(x_1-x_0)\\ x_0(r - x_2) -x_1 \\ x_0 x_1 - bx_2.
  \end{bmatrix}
\end{align*}
Note that
\begin{equation*}
  D F(x) = 
  \begin{bmatrix}
    -s & s & 0 \\ r-x_2 & -1 & -x_0 \\ x_1 & x_0 & -b
  \end{bmatrix}
\end{equation*}
where $\left(D F(x)\right)_{i,j} \equiv \frac{\partial F_i(x)}{\partial
  x_j}$.  Let $\Phi$ denote solutions to the differential equation
with
\newcommand{\DM}{{\cal{D}}} % Derivative Matrix, or tangent
\newcommand{\ct}{\tau}      % Continuous time
\begin{equation*}
  \ti{x}{\ct} \equiv \Phi(\ti{x}{0},\ct).
\end{equation*}
Lyapunov exponents are defined (recall Eqn.~\eqref{eq:growthVX}) in
terms of the long time behavior of the derivative matrix
\begin{equation*}
  \ti{\DM}{\ti{x}{0},\ct} \equiv D_{\ti{x}{0}} \Phi(\ti{x}{0},\ct).
\end{equation*}
Interchanging the order of differentiation with respect to $\ti{x}{0}$
and $\ct$ and applying the chain rule yields a \emph{linear}
differential equation for $\DM$:
\begin{align*}
  \dot \DM(\ti{x}{0},\ct) &=  \frac{d}{d\ct} D_{\ti{x}{0}}
  \Phi(\ti{x}{0},\ct)\\
  &= \left. D F(x) \right|_{x=\Phi(\ti{x}{0},\ct)}
  \DM(\ti{x}{0},\ct).
\end{align*}
Thus, given initial conditions $\ti{x}{0}$ and $\ti{\DM}{0} = \id $
%\begin{bmatrix}  1 & 0 & 0 \\ 0 & 1 & 0 \\ 0 & 0 & 1\end{bmatrix}$
one can  use an off-the-shelf routine to find $\begin{bmatrix}
  \ti{x}{\ct}\\ \ti{\DM}{\ct}\end{bmatrix}$ by integrating 
\begin{equation}
  \label{eq:TangentODE}
  \dot {\begin{bmatrix} \ti{x}{\ct}\\\ti{\DM}{\ct}\end{bmatrix}} = 
  \begin{bmatrix}
    F(x) \\  [D F(x)] \DM
  \end{bmatrix}.
\end{equation}

Given a computer with infinite precision, for a range of time
intervals $\tau$, one could:
\begin{enumerate}
\item Integrate Eqn.~\eqref{eq:TangentODE} to obtain $\ti{\DM}{\ct}$
\item \label{item:SVD} Do singular value decompositions (SVD's)
  \begin{equation}
    \label{eq:TangentSVD}
    \ti{U}{\ct} \ti{S}{\ct} \ti{V\transpose}{\ct} = \ti{\DM}{\ct},
  \end{equation}
  where $\ti{U}{\ct}$ and $\ti{V}{\ct}$ are orthogonal and $\ti{S}{\ct}$ is
  diagonal
\item Look for approximate convergence of the finite time Lyapunov
  exponent estimates:
  \begin{equation}
    \label{eq:lambdaSVD}
    \ti{\tilde \lambda_i}{\ct} \equiv \frac{1}{\ct} \log( \ti{S_{i,i}}{\ct} ) .
  \end{equation}
\end{enumerate}
On a real computer, the procedure fails because the ratio of the
largest and smallest singular values
$\frac{\ti{S_0}{\ct}}{\ti{S_d}{\ct}}$ grows exponentially with $\ct$
and becomes larger than the precision of the machine.

Rather than using an SVD decomposition for each $\tau$ in
step~\ref{item:SVD} above, one could use a QR decomposition:
\begin{equation}
  \label{eq:TangentQR}
  \ti{Q}{\ct} \ti{R}{\ct} = \ti{\DM}{\ct}.
\end{equation}
A QR decomposition factors the matrix $\ti{\DM}{\ct}$ into a product
of two matrices the first of which $\ti{Q}{\ct}$ is orthogonal and the
second of which $\ti{R}{\ct}$ is upper triangular.  One could use the
intuitive Gram Schmidt procedure, but there are algorithms that behave
better numerically (see, \eg \cite{GandL3} or \cite{Press92}).
Although the diagonal elements of $\ti{R}{\ct}$ are not equal to the
diagonal elements of $\ti{S}{\ct}$, the finite time estimates
\begin{equation}
  \label{eq:lambdaQR}
  \ti{\hat \lambda_i}{\ct} \equiv \frac{1}{\ct} \log( \left|
  \ti{R_{i,i}}{\ct} \right| )
\end{equation}
and the $\ti{\tilde \lambda_i}{\ct}$ defined in
Eqn.~\eqref{eq:lambdaSVD} converge to the same values\footnote{In the
  SVD of Eqn.~\eqref{eq:TangentSVD}, the first column of $\ti{V}{\ct}$
  specifies the direction of the initial vector in the tangent space
  with the largest stretching.  The exponential stretching rate is the
  Lyapunov exponent $\lambda_0$.  However, with probability one, a
  randomly chosen vector will have the same stretching rate.  The
  estimate $\ti{\hat \lambda_0}{\ct}$ of Eqn~\eqref{eq:lambdaQR} is
  based on the stretching rate of the first standard basis vector, \ie,
  $[1,0,0]$.
  Similar arguments using the growth rates of areas, volumes,
  hyper-volumes, etc. support using the estimates $\ti{\hat
    \lambda_i}{\ct}$ of Eqn~\eqref{eq:lambdaQR} for $i=2,3,\ldots$}.

Using Eqn.~\eqref{eq:lambdaQR} does not address the problem of finite
machine precision for long time intervals $\tau$, but Benettin et al.
recommend calculating $\log( \left| \ti{R_{i,i}}{\ct} \right| )$ by
breaking the interval into $N$ smaller steps of duration $\Delta\tau$
in a way that does address finite precision.  Letting $\ti{A}{n}$
denote the one time step derivative
\begin{equation}
  \label{eq:Adef}
  \ti{A}{n} \equiv D \Phi( \ti{x}{(n-1) \Delta\tau}, \Delta\tau )
\end{equation}
the chain rule implies
\begin{equation*}
  D \Phi( \ti{x}{0}, N \Delta\tau ) = \prod_{n=1}^N \ti{A}{n}.
\end{equation*}
If, for each $n$, one calculates\footnote{To calculate $\ti{Q}{n}$ and
  $\ti{r}{n}$ for each $n$, one can either:
  \begin{enumerate}
  \item Integrate Eqn.~\eqref{eq:TangentODE} for a time interval $\Delta\tau$
    with the initial condition $\begin{bmatrix} \ti{x}{(n-1)\Delta\tau} \\
      \ti{Q}{n-1} \end{bmatrix}$ to obtain $\begin{bmatrix}
      \ti{x}{n\Delta\tau} \\ \ti{A}{n} \ti{Q}{n-1}\end{bmatrix}$ and then
    calculate a QR factorization of $\ti{A}{n} \ti{Q}{n-1}$, the
    second component of the result.
  \item As above, but use the identity matrix instead of $\ti{Q}{n-1}$
    as the second component of the initial condition for the
    integration which yields the result $\begin{bmatrix}
    \ti{x}{n\Delta\tau} \\ \ti{A}{n}\end{bmatrix}$, then calculate a QR
    factorization of the product $\ti{A}{n}\ti{Q}{n-1}$
  \end{enumerate} }
                                %
the pair $\left( \ti{Q}{n}, \ti{r}{n}
\right)$ defined by
\begin{align*}
  \ti{Q}{0} &= \id \\
  \ti{Q}{n} \ti{r}{n} &\equiv \ti{A}{n} \ti{Q}{n-1},
\end{align*}
where $\ti{Q}{n}$ and $\ti{r}{n}$ are obtained by a QR factorization
of the product $\ti{A}{n} \ti{Q}{n-1}$, then induction yields
\begin{equation*}
  \prod_{n=1}^N \ti{A}{n} =\ti{Q}{N} \prod_{n=1}^N \ti{r}{n}.
\end{equation*}
Since $\prod_{n=1}^N \ti{r}{n}$ is upper triangular, we have the QR
factorization
\begin{align}
  \label{eq:QR1}
  D \Phi( \ti{x}{0}, N \Delta\tau ) &= \ti{Q}{N} \ti{R}{N} \\
  \label{eq:QR2}
  \ti{R}{N} &= \prod_{n=1}^N \ti{r}{n}.
\end{align}
Because the diagonal elements are the products
$\ti{R_{i,i}}{N} = \prod_{n=1}^N \ti{r_{i,i}}{n}$, their logs are the
sums
\begin{equation}
  \label{eq:rii}
  \log( \left| \ti{R_{i,i}}{n} \right| ) = \sum_{n=1}^N \log( \left|
    \ti{r_{i,i}}{n} \right| ).
\end{equation}
Substituting this result into Eqn.~\eqref{eq:lambdaQR} constitutes the
Benettin procedure.  The action of a matrix on a unit square is
factored into components $Q$ and $R$ and sketched in
Fig.~\ref{fig:QR}.  Results of applying the procedure to the Lorenz
system appear in Fig.~\ref{fig:benettin}.
\begin{figure}[htbp]
  \centering{\plotsize%
    \def\Mone{$ \begin{bmatrix} \begin{bmatrix} e_0
        \end{bmatrix} & \begin{bmatrix} e_1 \end{bmatrix} \end{bmatrix} $}%
    \def\Mtwo{$ R \begin{bmatrix} \begin{bmatrix} e_0
        \end{bmatrix} & \begin{bmatrix} e_1 \end{bmatrix} \end{bmatrix} $}%
    \def\Mthree{$ QR \begin{bmatrix} \begin{bmatrix} e_0
        \end{bmatrix} & \begin{bmatrix} e_1 \end{bmatrix} \end{bmatrix} $}%
    \def\Mfour{$ \begin{bmatrix} 1 & 0 \\ 0 & 1 \end{bmatrix} $}%
    \def\Mfive{$ \begin{bmatrix} 5 & 0.15 \\ 0 & 0.2 \end{bmatrix} $}%
    \def\Msix{$ \begin{bmatrix}3 & 0.25 \\ -4 & 0 \end{bmatrix} $}%
    \input{QR.pdf_t}}
  \caption[\comment{fig:QR }The action of the $Q$ $R$ factors of a matrix on a unit square.]%
  {The action of the $Q$ $R$ factors of a matrix on a unit square.
    Here $A=
    \begin{bmatrix} 3 & 0.25 \\ -4 & 0 \end{bmatrix}$, $Q=
    \begin{bmatrix} 0.6 & 0.8 \\ -0.8 & 0.6 \end{bmatrix}$, and $R=
    \begin{bmatrix} 5 & 0.15 \\ 0 & 0.2 \end{bmatrix}$.  $R$ stretches the
    $x$ component by a factor of five and shears $y$ components in the
    $x$ direction and shrinks them by a factor of five with a net
    effect of preserving areas (The determinants of
    $A$ and $R$ are both 1.0).  $Q$ simply rotates the stretched figure.}
  \label{fig:QR}
\end{figure}
\begin{figure}[htb]
  \centering{\includegraphics[width=1.0\textwidth]{benettin.pdf}
  }
  \caption[\comment{fig:benettin }Lyapunov exponent calculation for
  the Lorenz system.]%
  {The effect of noise on Lyapunov exponent calculations for the
    Lorenz system.  In the upper plot, the colored lines trace
    $\frac{1}{T} \sum_{t=0}^{T-1} \log\left( \left| \ti{r_{0,0}}{t}
      \right| \right)$ (See Eqn.~\eqref{eq:rii}) for three different
    initial conditions and the black lines trace the 5\% and 95\%
    limits on 1,000 separate runs.  The lower plot is the same except
    that $\left| \ti{r_{0,0}}{t} \right|$ is augmented by a noise term
    with amplitude $\frac{\sigma_\eta}{\Delta} = \toyBenettinratio$
    (See Eqn.~\eqref{eq:LE.aug}).  At $\tau=\toyBenettintrun$ the
    Laypunov exponent estimates from the upper and lower plots are
    $\toyBenettinsdemean \pm \toyBenettinsdestd$ and
    $\toyBenettinaugmentedmean \pm \toyBenettinaugmentedstd$
    respectively.  The difference gives an indication of the
    sensitivity of the estimates to noise.}
  \label{fig:benettin}
\end{figure}

\section{A Practical Performance Bound}
\label{sec:PracticalBound}

Consider the following two cases:
\begin{itemize}
\item State space dynamics perturbed by noise
\item Simulated dynamics perturbed by numerical truncation
\end{itemize}
The definition of \index*{Kolmogorov Sinai entropy} in the two cases yields
extremely different answers.  If the perturbations are random noise,
then the supremum of $h(\mathcal{B},F,\mu)$ over $\mathcal{B}$ does
not exist and $h_{KS}$ is unbounded.  On the other hand, if the
perturbations are numerical truncation and the process is a digital
simulation, then all observation sequences converge to periodic cycles
and $h_{KS} = 0$.  Thus, the \emph{strict} definition of the
Kolmogorov Sinai entropy is useless as a bound on the cross entropy of
models in numerical simulations.  Here we argue however, that
numerical Lyapunov exponent estimates nonetheless provide a
\emph{practical} reference for the performance of models.

If you are working on a new model building procedure that takes
\emph{training} samples $\left\{ \ts{y}{\tau_0}{T_0},
  \ts{y}{\tau_1}{T_1}, \ldots, \ts{y}{\tau_N}{T_N}, \right\}$ and
produces a family of parameterized conditional probability functions
$P \left(\ti{y}{t}\given \ts{y}{0}{t},\theta \right)$, we recommend
numerically estimating the \emph{entropy gap} (see
Theorem~\ref{GapTheorem}) $\delta_{\mu||\theta} =
h({\cal{B}},F,\mu||\theta) - \sum_{i : \lambda_i > 0} \lambda_i$ to
characterize the fidelity of the resulting models $\theta$ to
generating processes.  As a debugging tool, it is reasonable to choose
some parameters $\theta'$ for a model class, use that model to
generate training data, and then verify that as the size of the
training data set increases the proposed model building procedure
recovers the parameters $\theta'$.  However such a test fails to
consider how well the proposed procedure and model class work on the
realistic case of data generated by processes outside the model class.
Even though the test we propose does not provide \emph{correct} model
parameters against which to compare fitted parameters, it does provide
a reference against which to compare model performance.  Specifically,
we advocate the following numerical experiment for evaluating a model
building procedure:
\begin{enumerate}
\item \label{PPB1} Use a numerical simulation of a chaotic dynamical
  system to generate training data and testing data.  For simplicity,
  consider a system with a single positive exponent $\lambda_0$
\item \label{PPB2} Quantize the data with a partition $\mathcal{B}$
\item \label{PPB3} Run the Benettin procedure on the system, to
  estimate Lyapunov exponents
\item \label{PPB4} Substitute the estimated exponents into Ledrappier
  and Young's formula, Eqn.~\eqref{eq:Ledrappier} with $\gamma_0 = 1$
  to get $\hat h(F,\mu)) = \hat \lambda_0$, an estimated entropy rate.
  If $\mathcal{B}$ is fine enough or a generating partition,
  $\hat h(\mathcal{B},F,\mu)) = \hat \lambda_0$ will be a good
  estimate.
\item \label{PPB5} Produce $P_{*\given \theta}$ by applying the new model building
  procedure to the training data
\item \label{PPB6} Estimate the cross entropy by evaluating the
  likelihood of the model on long sequences of testing data
  \begin{equation}
    \label{eq:ToyCE}
    \hat h(\mathcal{B},F,\mu||\theta) = \frac{-1}{T} \sum_{t=1}^T \log \left(
    P \left(\ti{y}{t}\given \ts{y}{0}{t},\theta \right)\right)
  \end{equation}
\item \label{PPB7} Calculate an entropy gap \index{entropy!gap} by
  subtracting the two estimates
  \begin{equation*}
    \hat \delta_{\mu||\theta} = \hat h(\mathcal{B},F,\mu||\theta)
    -\hat h(\mathcal{B},F,\mu)).
  \end{equation*}
  For an optimal model, expect the gap to be zero.  If the gap is much
  larger than zero, conclude that the new procedure is suboptimal.
\end{enumerate}
The test is reasonable only if, subject to some constraints, $\EV \hat
\delta_{\mu||\theta} \geq 0$ is a tight bound and the variance of
$\hat \delta_{\mu||\theta}$ is small.  Below, we argue that a model
that uses knowledge of the generating process and has smooth
probability densities in state space achieves the bound with equality
and thus the bound is tight.

In this class of models, the probability measure for the generating
process is not necessarily ergodic or even stationary; it is derived
from a uniform density over a box that covers possible starting
conditions, and it includes a little bit of noise in the dynamics so
that even in the long time limit it does not become fractal.  Because
the probability is smooth, the models cannot exploit fractal
properties that might exist in the modeled system and consequently
$\gamma$, the Ledrappier and Young correction to the Pesin formula, is
irrelevant.  More specifically we consider a model class with the
following properties: \newcommand{\Lic}{L_{\text{i.c.}}}
\begin{description}
\item[Probability density] The probability density for the initial
  state $P(\ti{x}{1}\given \theta)$ is a uniform distribution on a cube in
  $X$ that has length $\Lic$ on each side.
\item[State noise] The model has noise in the state dynamics,
  \begin{equation}
    \label{eq:PracticalStateMap}
    \ti{x}{t+1} = F(\ti{x}{t}) + \ti{\eta}{t},
  \end{equation}
  where $\ti{\eta}{t}$ are \iid Gaussian with $\ti{\eta}{t} \sim
  \Normal(0,\id \sigma_\eta^2)$.  We suppose that $F$ is the same as
  the \emph{true} function of the \emph{modeled} system, but that
  noise in the modeled system is smaller or zero.
\item[Measurement function] We let the measurement function be the
  same for the model as for the true system, \ie, a discrete partition
  with resolution $\Delta$.  We have in mind a uniform quantization of
  a single component of $X$ such as we used for Fig.~\ref{fig:ToyH}.
\end{description}
The only difference between the true system and the model is that the
state noise in the model may be larger than the state noise in the
true system.

With this framework we can draw samples randomly from a true
distribution $P_{*\given \mu}$ and consider model probabilities
$P_{*\given \theta}$ without having to find a stationary distribution.
In sidestepping the key issue of a stationary distribution, we have
sacrificed ergodicity which is the basis of the definition of a
Lyapunov exponent as a global property.  Empirically, however, the
convergence of the Benettin procedure is similar for any initial
condition (See Fig.~\ref{fig:benettin}).  Relying on this empirical
observation, we suppose for some time interval $\tau$ that the
stretching factor is roughly independent of initial conditions with
\begin{equation}
  \label{eq:Stretch}
  \ti{S}{\tau} \approx \prod_{t=1}^{\tau-1} \ti{r_{0,0}}{t} = e^{\hat \lambda \tau}.
\end{equation}

In the limit of small noise $\sigma_\eta \rightarrow 0$, one can
calculate $P(\ts{y}{0}{T}\given \theta)$ for any sequence of observations as
the probability of the set of initial conditions that are consistent
with  $\ts{y}{0}{T}$, \ie, the pre-image of $\ts{y}{0}{T}$,
\begin{equation*}
  P(\ts{y}{0}{T}\given \theta) = \int_{ \left\{ x:\ts{Y}{0}{T}(x) =
      \ts{y}{0}{T} \right\} } P(x\given \theta) dx.
\end{equation*}
For a $d$ dimensional system, the volume of such pre-images is typically
\begin{equation*}
  \frac{\Delta e^{-\hat \lambda T}}{O(T^d)} < \text{Vol} < \Delta e^{-\hat \lambda T},
\end{equation*}
and since the density of initial conditions is smooth, for large $T$
we find
\begin{subequations}
\begin{equation}
  \label{eq:back.image}
  \frac{1}{T} \log \left( P(\ts{y}{0}{T}\given \theta) \right) \approx -\hat
  \lambda.
\end{equation}

Rather than going backwards in time to analyze the pre-image of
$\ts{y}{0}{T}$, we can think about the forward image of the volume of
initial conditions under the map $\Phi(T)$.  To first order, the
distribution is a uniform probability density over a parallelepiped
that extends a distance $\Lic \ti{s}{T}$ in the direction of the first
column of the orthogonal matrix $\ti{Q}{t}$ in Eqn.~\eqref{eq:QR1}.
The measurement partition divides the image into elements that have a
characteristic size of $\Delta$, yielding
\begin{equation}
  \label{eq:forward.image}
  \frac{1}{T} \log \left( P(\ts{y}{0}{T}\given \theta) \right) \approx -\hat
  \lambda
\end{equation}
\end{subequations}
again.  Given the enormous stretching that occurs, it is clear that
the image of the volume of allowed initial conditions will resemble
steel wool more than a parallelepiped, but the exponential nature of
the stretching is all that matters, and in the small noise limit we
have
\begin{equation}
  \label{eq:zero.noise}
  h({\cal{B}},F,\mu||\theta) \approx \hat \lambda
\end{equation}

One might wonder whether finite state noise $\sigma_\eta$ invalidates
\eqref{eq:zero.noise} or if perhaps a correction of order
$\frac{\sigma_\eta}{\Delta}$ will suffice.  As long as $\sigma_\eta <<
\Lic e^{\hat \lambda T}$, the analysis of the size of the image of the
volume of initial conditions under $\Phi(T)$ that leads to
\eqref{eq:forward.image} is adequate.  The noise term $\sigma_\eta$ in
the model will however transfer probability from observation sequences
permitted by the true system to sequences that it does not allow,
thereby increasing the cross entropy.  At each time step the effect
roughly augments the stretching in each state space direction with a
term of size $\frac{\sigma_\eta}{\Delta}$.  Since the noise at each
time is independent of the noise at all other times, the effects add
in quadrature.  We can estimate an upper bound on the total effect by
replacing $ \left| \ti{r_{i,i}}{n} \right|$ with $ \left|
  \ti{r_{i,i}}{n} \right| + \frac{\sigma_\eta}{\Delta}$ for each $i$
and $n$ in Eqn.~\eqref{eq:rii} of the Benettin procedure, \ie,
\begin{equation}
  \label{eq:LE.aug}
  \hat \lambda_{\text{aug},i} \equiv \frac{1}{N}  \sum_{n=1}^N \log( \left|
    \ti{r_{i,i}}{n} \right|  + \frac{\sigma_\eta}{\Delta}).
\end{equation}
Notice that knowledge of $\hat \lambda_i$ and
$\frac{\sigma_\eta} {\Delta}$, is not sufficient to calculate
$\hat \lambda_{\text{aug},i}$.  If the stretching were uniform with
$\left| \ti{r_{i,i}}{n} \right| = e^{\lambda} ~\forall n$, the
augmented result would be
$\hat \lambda_{\text{aug},i} = \log\left( e^\lambda +
  \frac{\sigma_\eta} {\Delta} \right)$, but the result increases
without bound\footnote{If,
  \begin{equation*}
     \left| \ti{r_{i,i}}{n} \right| = \begin{cases}
       \frac{1}{\delta^{N-1}} e^\lambda & n=1\\
       \delta e^\lambda & \text{otherwise} \end{cases},
   \end{equation*}
   then $\prod_{n=1}^N \left| \ti{r_{i,i}}{n} \right| = e^{N\lambda}$
   and $\hat \lambda_i = \lambda$, but
   \begin{align*}
   \hat \lambda_{\text{aug},i} &=
   \frac{1}{N} \left(
     \log\left(
       \frac{e^\lambda}{\delta^{N-1}} + \frac{\sigma_\eta}{\Delta}
     \right) +
     (N-1) \log\left( \delta e^\lambda + \frac{\sigma_\eta}{\Delta}
     \right) 
                            \right) \\
    \lim_{\delta \to 0} &= \frac{1}{N}
   \left( \lambda + (N-1) \log
     \left( \frac{\sigma_\eta}{\delta\Delta} \right)
   \right) \rightarrow \infty.
   \end{align*}
 } %
as $ \left| \ti{r_{i,i}}{n} \right|$ varies more with $n$.  In
Fig.~\ref{fig:benettin} we compare $\hat \lambda_{\text{aug},1}$ with
$\hat \lambda_{1}$ for the Lorenz system.  For noise with an amplitude
$\frac{\sigma_\eta}{\Delta} = \toyBenettinratio$, the figure indicates
an augmentation of
$\hat \lambda_{\text{aug},1} - \hat \lambda_{1} \approx
\toyBenettinDeltaLambda$, which is roughly an order of magnitude
larger than the augmentation that uniform stretching would produce.
From the figure, we conclude that the Benettin procedure produces a
robust practical upper bound on model performance.

\section{Approaching the Bound}
\label{sec:approach}
\longpage%%% Maybe not required for MathTime... look for half
         %%% paragraph after the fig:LikeLor to see if it is needed.

Although the \emph{slope} of the plot in Fig.~\ref{fig:ToyH} (the log
likelihood per time step attained by extended Kalman filters) matches
the entropy bound, and we are satisfied with our explanation of the
nonzero intercept, an example of a model with a likelihood close to
the bound without any offset would be more satisfying.  To find such
model, we returned to the source of coarsely quantized Lorenz
observations that we used for Fig.~\ref{fig:Statesintro} in the
introduction.  That figure illustrates the association of each of the
twelve discrete hidden states of an HMM with particular regions in
$\REAL^3$, the state space of the Lorenz system.  Although the cross
entropy of that twelve state model is not very close to the
bound\marginpar{ToDo: Entropy per time step or per unit of Lorenz
  time?}  based on our Lyapunov exponent estimate
($\hat h(\mathcal{B},F,\mu)) = \hat \lambda_0 =
\toyLikeLorEntropy~\text{nats} = \toyLikeLorEntropyBits~\text{bits}$),
it seems plausible that by using HMMs with more states, we might get
higher likelihoods.  In fact it is true, but we are surprised at how
many states we need.  As for Fig.~\ref{fig:Statesintro}, we generated
the observations by integrating the Lorenz system with a time step of
$\tau_{\mathtt{sample}}=\toyLikeLortsample$ and quantized the first
component into one of $\toyLikeLornquantized$ levels.

Our first attempt was to train a sequence of models with ever more
hidden states.  We initialized each model randomly and ran many
iterations of the Baum-Welch algorithm on quantized observations.
Even with many iterations, we did not build any promising models.

In our second attempt, we exploited our knowledge of the Lorenz
dynamics in $X=\REAL^3$ as follows:
\begin{enumerate}
\item Generate training data $\ts{x}{0}{T}$ and $\ts{y}{0}{T}$ by
  integrating the Lorenz system.  Here $\ti{x}{t}$ is a point in the
  original state space and $\ti{y}{t}$ is a quantized observation that
  can take one of four values.
\item Generate testing data $\ts{y}{T}{T+N}$ by continuing the integration.
\item Find a set of discrete states $\left\{s_0, s_1, \ldots,
    s_m\right\}$ by partitioning the original space with a uniform
  grid of resolution $\Delta_x$.  We only constructed states for those
  partition elements that were occupied by at least one member of
  $\ts{x}{0}{T}$.
\item Build an HMM using the training sequence.  We set the state
  transition probabilities by counting the frequency with which the
  partitioned $X$ sequence made each possible transition.  Similarly,
  we set the observation model by counting the frequency with which
  each partition element was associated with each possible
  observation.
\item Estimate the cross entropy ($ \hat h(\mathcal{B},F,\mu||\theta)$
  See Eqn.~\eqref{eq:ToyCE}) of the model by calculating its
  log likelihood per time step on the testing data.
\end{enumerate}

As hoped, we found that as we reduced the resolution, the number of
states increased and the cross entropy estimates decreased.  Since
there is no training, \ie, Baum-Welch iterations, in this procedure,
we could calculate the likelihood with a variant of the forward
algorithm that does not store or return $\alpha$ or $\gamma$ values
for the entire sequence.  In fact, given typical observations
$\ts{y}{0}{t+1}$ up to time $t$, only a small fraction of the states
have nonzero probability.  Code that uses sparse matrices to exploit
these features is many orders of magnitude cheaper than vanilla code
for the forward algorithm.

Results of the procedure appear in Fig.~\ref{fig:LikeLor}.  For the
rightmost point on the curve\footnote{While the address space limit of
  32 bit Python constrained the number of states for the first edition
  of this book to about 1,500,000, computation time constrains the
  number of number of states in 2025.  Using a high end 2023 desktop
  computer, it takes about 100 minutes to generate
  Fig.~\ref{fig:LikeLor}.} we found a model with
%
\toyLikeLornstates~hidden states and a cross entropy of
\toyLikeLorcrossentropy~nats or \toyLikeLorcrossentropybits~bits.  By
connecting this model to a simple data compression routine one could
compress the test data (or presumably any other long sequence from the
source) down to \toyLikeLorcrossentropybits~bits per sample, which is
\toyLikeLorGapPercent\% more than the \toyLikeLorEntropyBits~bits per
sample that the Lyapunov exponent estimate suggests is possible.
\begin{figure}[htbp]
  \centering{
    \includegraphics[width=1.0\textwidth]{LikeLor.pdf}}
  \caption[\comment{fig:LikeLor }Entropy gap,
  $\hat \delta_{\mu||\theta}$, vs number of states in HMMs]%
  {Entropy gap, $\hat \delta_{\mu||\theta}$ vs number of states in
    HMMs.  The upper curve indicates estimates of cross entropy
    $\hat h(\mathcal{B},F,\mu||\theta)$ for a sequence of HMMs vs the
    number of discrete states in the models.  We built the models
    using simulated Lorenz state space trajectories as described in
    the text.  The lower line indicates an estimate of the entropy
    rate, $\hat h(F,\mu)) = \hat \lambda_0$, of the true process based
    on Lyapunov exponents estimated by the Benettin procedure.  The
    distance between the curves is the \emph{entropy gap}
    $\hat \delta_{\mu||\theta}$. The gap seems to be going to zero,
    suggesting that an HMM with enough states would perform at least
    as well as any other model based any other technology.  Each model
    was built using the same \toyLikeLorntrain~sample trajectory in
    the original state space, and the cross entropy estimates are
    based on a test sequence of \toyLikeLorntest~observations.}
  \label{fig:LikeLor}
\end{figure}
% \ToDo{remember to cite Yariv Ephraim
% ph\_barbe at hotmail.com recommends Biscarat '94 (a paper on
% stochastic EM)}

%%%
%%% Local Variables:
%%% TeX-master: "main"
%%% eval: (load-file "hmmkeys.el")
%%% mode: LaTeX
%%% End:

\input{apnea_values.tex}
\chapter{Obstructive Sleep Apnea}
\label{chap:apnea}

The challenge for the \emph{Computers in Cardiology} %
\index{Computers in Cardiology 2000 (CINC2000)}%
meeting in 2000 (CINC2000) %
\index{CINC2000 (Computers in Cardiology 2000)}%
was to identify \emph{obstructive sleep apnea} on the basis of
electrocardiograms alone.  Our colleague James \index*{McNames} at
Portland State University, with some support from us, won the prize
for the best minute by minute analysis of data.  We prepared our first
entry using HMMs, and McNames prepared the subsequent (and winning)
entries by hand using spectrograms\footnote{A spectrogram is display
  of power in Fourier spectral bands as a function of time.  The
  x-axis is time, the y-axis is frequency, and the image intensity at
  point $(t,f)$ is the power in that frequency estimated in a window
  centered at that time.} to visualize the data.  \index{spectrogram}

In preparing the first entry, we noticed that apnea produced many
different characteristics in heart rate time series while the
characteristics in non-apnea times were more similar to each other.  We
used a larger number of states in HMMs to model the various
characteristics of apnea and a smaller number of states to model the
more consistent normal times.  However using the maximum a posteriori
probability (MAP) estimate sequence of states from the Viterbi
algorithm, ie,
\begin{equation*}
  \hat {\ts{s}{1}{T}} \equiv
  \argmax_{\ts{s}{1}{T}}P\left(\ts{s}{1}{T} | \ts{y}{1}{T} \right)
\end{equation*}
to estimate the sequence of classifications $\hat {\ts{c}{1}{T}}$ with
\begin{equation*}
  \hat {\ti{c}{t}} \equiv c:\hat {\ti{s}{t}} \in c
\end{equation*}
produced obvious errors.  We found that during many periods of apnea
the probability of being in each of the apnea states was lower than
the probability of being in the normal state with the highest
probability while the sum of the probabilities of apnea states was
larger than the sum over normal states.  Our effort to model the
diversity of apnea characteristics led to decoded sequences that were
normal too often.

We tried to address the issue by estimating the MAP sequence of classes
rather than the MAP sequence of states, ie,
\begin{equation}
  \label{eq:decode_class}
  \hat {\ts{c}{1}{T}} \equiv
  \argmax_{\ts{c}{1}{T}}P\left(\ts{c}{1}{T} | \ts{y}{1}{T}\right).
\end{equation}
While the computational complexity of Viterbi algorithm which finds
the MAP sequence of states is linear in $T$, the length of the
sequence of observations, the computational complexity required to
solve \eqref{eq:decode_class} is exponential\footnote{In the first
  edition of this book, we presented an algorithm that we claimed solved
  \eqref{eq:decode_class} in linear complexity.  In some cases that
  algorithm yields plausible but not necessarily correct results and
  in others it simply crashes.} in $T$.

In this chapter, we will first describe the CINC2000 challenge and how
McNames addressed it.  Then we will address the challenge using HMMs.

\section{The  Challenge and the Data}
\label{sec:challenge}

The PhysioNet
website\footnote{\url{http://www.physionet.org/challenge/2000}}
announced the challenge and described the data as follows:

{\it % italic to signify quote
  {\rm \textbf{Introduction:}}
  %
  Obstructive sleep apnea (intermittent cessation of breathing) is a
  common problem with major health implications, ranging from
  excessive daytime drowsiness to serious cardiac arrhythmias.
  Obstructive sleep apnea is associated with increased risks of high
  blood pressure, myocardial infarction, and stroke, and with
  increased mortality rates. Standard methods for detecting and
  quantifying sleep apnea are based on respiration monitoring, which
  often disturbs or interferes with sleep and is generally expensive.
  A number of studies during the past 15 years have hinted at the
  possibility of detecting sleep apnea using features of the
  electrocardiogram. Such approaches are minimally intrusive,
  inexpensive, and may be particularly well-suited for screening. The
  major obstacle to use of such methods is that careful quantitative
  comparisons of their accuracy against that of conventional
  techniques for apnea detection have not been published.

  We therefore offer a challenge to the biomedical research community:
  demonstrate the efficacy of ECG-based methods for apnea detection
  using a large, well-characterized, and representative set of data.
  The goal of the contest is to stimulate effort and advance the state
  of the art in this clinically significant problem, and to foster
  both friendly competition and wide-ranging collaborations. We will
  award prizes of US\$500 to the most successful entrant in each of two
  events.

  {\rm \textbf{Data for development and evaluation:}}
  %
  Data for this contest have kindly been provided by Dr. Thomas
  Penzel \index{Penzel, Dr.\ Thomas}  of Philipps-University, Marburg,
  Germany [available on the website].

  The data to be used in the contest are divided into a learning set
  and a test set of equal size. Each set consists of 35 recordings,
  containing a single ECG signal digitized at 100 Hz with 12-bit
  resolution, continuously for approximately 8 hours (individual
  recordings vary in length from slightly less than 7 hours to nearly
  10 hours). Each recording includes a set of reference annotations,
  one for each minute of the recording, that indicate the presence or
  absence of apnea during that minute. These reference annotations
  were made by human experts on the basis of simultaneously recorded
  respiration signals. Note that the reference annotations for the
  test set will not be made available until the conclusion of the
  contest. Eight of the recordings in the learning set include three
  respiration signals (oronasal airflow measured using nasal
  thermistors, and chest and abdominal respiratory effort measured
  using inductive plethysmography) each digitized at 20 Hz, and an
  oxygen saturation signal digitized at 1 Hz. These additional signals
  can be used as reference material to understand how the apnea
  annotations were made, and to study the relationships between the
  respiration and ECG signals. [...]

  {\rm \textbf{Data classes:}}
  %
  For the purposes of this challenge, based on these varied criteria,
  we have defined three classes of recordings:
  \begin{description}
  \item[Class A (Apnea):] These meet all criteria. Recordings in class
    A contain at least one hour with an apnea index of 10 or more, and
    at least 100 minutes with apnea during the recording. The learning
    and test sets each contain 20 class A recordings.
  \item[Class B (Borderline):] These meet some but not all of the
    criteria. Recordings in class B contain at least one hour with an
    apnea index of 5 or more, and between 5 and 99 minutes with apnea
    during the recording. The learning and test sets each contain 5
    class B recordings.
  \item[Class C (Control):] These meet none of the criteria, and may
    be considered normal. Recordings in class C contain fewer than 5
    minutes with apnea during the recording. The learning and test
    sets each contain 10 class C recordings.
  \end{description}

  {\rm \textbf{Events and scoring:}}
  %
  Each entrant may compete in one or both of the following events:
  \begin{description}
  \item[1. Apnea screening:] In this event, your task is to design
    software that can classify the 35 test set recordings into class A
    (apnea) and class C (control or normal) groups, using the ECG
    signal to determine if significant sleep apnea is present.  [...]
  \item[2. Quantitative assessment of apnea:] In this event, your
    software must generate a minute-by-minute annotation file for each
    recording, in the same format as those provided with the learning
    set, using the ECG signal to determine when sleep apnea occurs.
    Your annotations will be compared with a set of reference
    annotations to determine your score. Each annotation that matches
    a reference annotation earns one point; thus the highest possible
    score for this event will be approximately 16800 (480 annotations
    in each of 35 records). It is important to understand that scores
    approaching the maximum are very unlikely, since apnea assessment
    can be very difficult even for human experts. Nevertheless, the
    scores can be expected to provide a reasonable ranking of the
    ability of the respective algorithms to mimic the decisions made
    by human experts.
  \end{description}
}%% end \it for quote


\subsection{The Data}
\label{sec:data}

Briefly, one can fetch the following records from PhysioNet:
\begin{description}
\item[a01-a20:] The \emph{a records} from individuals that display
  \emph{apnea}
\item[b01-b05:] The \emph{b records}\footnote{The amplitude of the
    \emph{b05} record varies dramatically over different segments of
    time.  We found it unusable and discarded it entirely.} from
  individuals diagnosed as \emph{borderline}
\item[c01-c10:] The \emph{c records} from \emph{control} or normal
  individuals
\item[a01er-a04er and b01er and c01er:] Identical to \emph{a01-a04}
  and \emph{b01} and \emph{c01} except augmented with respiration and
  $SpO_2$ (percent of arterial hemoglobin saturated with oxygen)
  \nomenclature[rSpO]{$SpO_2$}{Percent of arterial hemoglobin
    saturated with oxygen.} signals
\item[summary\_of\_training:] Expert classifications of each minute in
  the $a$, $b$, and $c$ records
\item[x01-x35:] The test set\footnote{The records \emph{x33} and
    \emph{x34} are so similar that we suspect they are simultaneous
    recordings from different ECG leads.  We did not explicitly exploit
  the similarity in our analysis.}; records without classification
\end{description}

Using data visualization tools\footnote{One can use the script
  hmmds.applications.apnea.explore.py from the software we used to
  create this book to explore aspects of the data.} one can see
striking oscillations in the apnea time series.  The patients stop
breathing for tens of seconds, gasp a few breaths, and stop again.
Each cycle takes about 45 seconds and can go on for most of the night.
We've plotted two periods of such an oscillation from record \emph{a03}
in Fig.~\ref{fig:a03erA}.  The reference or \emph{expert}
classifications provided with the data indicate these are the last
oscillations in the first apnea episode of the night.  Over the entire
night's record of 8 hours and 39 minutes, the patient had apnea for a
total of 4 hours and five minutes in 11 separate episodes.  In that
time, more than once a minute, he was waking up enough to start
breathing.  Ten and a half minutes after the end of the first apnea
episode, the record, as plotted in Fig.~\ref{fig:a03erN}, looks
normal.

\begin{figure}
  \centering{\resizebox{\textwidth}{!}{\includegraphics{a03erA.pdf}}
  }
  \caption[A segment of record a03]%
  {A segment of record a03.  Two cycles of a large apnea
    induced oscillation in $SpO_2$ are drawn in the lower plot.  The
    middle plot is the oronasal airflow signal, and the upper plot is
    the ECG (units of both ONR and ECG are unknown).  The time
    axis is marked in \emph{hours:minutes:seconds}.  Notice the increased
    heart rate just after 0:58:00 and just before 0:59:00.}
  \label{fig:a03erA}
\end{figure}

\begin{figure}
  \centering{\resizebox{\textwidth}{!}{\includegraphics{a03erN.pdf}}
  }
  \caption[A segment of record a03]%
  {A segment of record a03 taken during a period of normal
    respiration.  The signals are the same as in Fig.~6.1.}
  \label{fig:a03erN}
\end{figure}

In plots like Fig.~\ref{fig:a03erA}, the heart rate visibly increases
at the end of the gasping phase and then decreases during the phase of
interrupted respiration.  That heart rate oscillation is the key we
used in our initial attempts to classify periods of apnea.  In
Fig.~\ref{fig:a03erHR}, we've plotted both a heart rate derived from
the ECG \nomenclature[rECG]{ECG}{Electrocardiogram} and the $SpO_2$
signal.  The oscillations in the signals track each other, and the
expert classifies only the region of large heart rate oscillation as
apnea.
\begin{figure}
  \centering{\resizebox{\textwidth}{!}{\includegraphics{a03HR.pdf}}
  }
  \caption[A segment of record 03 at the end of an episode of apnea]%
  {A segment of record 03 at the end of an episode of apnea with
    indications in both the $SpO_2$ signal and the heart rate
    \emph{(HR)} signal.  The expert marked the time before 1:00 as
    apnea and the time afterwards as normal.}
  \label{fig:a03erHR}
\end{figure}

\section{Using Information from Experts to Train}

We first considered the CINC2000 challenge when Andreas Rechtsteiner
told us that he was going to use it as a final project for a class
that James was teaching.  We suggested that Andreas try a two state
HMM with autoregressive observation models, \ie, a model like those
described in section~\ref{sec:ARVGaussian} but with scalar
observations.

To use the expert classification information in training, Andreas
found that one can just modify the observation model.  The technique
is not limited to HMMs with only two states or two classes; it applies
to an arbitrary number of classes and to arbitrary numbers of states
associated with each class.  At each time $t$, let $\ti{c}{t}$ denote
classification information from an expert about which states are
possible.  Specifically $\ti{c}{t}$ is a vector that indicates that
some states are possible and that the others are impossible.  One
simply replaces $P_{\ti{Y}{t}|\ti{S}{t},\ts{Y}{1}{t-1}}$ with
$P_{\ti{Y}{t},\ti{C}{t}|\ti{S}{t},\ts{Y}{1}{t-1}}$ wherever it occurs
in the Baum-Welch algorithm.  The modification forces the system into
states associated with the right class during training.

While the performance of Andreas' two state model was not memorable,
models with more states were promising.

\subsection{The Excellent Eye of Professor McNames}
\label{sec:mcnames}

To diagnose the errors, McNames printed spectrograms of every record.
As he reported in \cite{mcnames2000}, the first step in McNames
analysis was implementing his own QRS detection algorithm.  Then he
derived spectrograms from the resulting QRS analyses.  Although the
spectrograms discarded phase information that we hoped was important,
they clearly indicated the oscillations that we had tried to capture
with complicated HMMs as intense bands of power at frequencies below 2
bpm (beats per minute).%
\nomenclature[rbpm]{bpm}{Beats per minute.}%
In addition to those low frequency oscillations, he noticed bands of
power between 10 and 20 bpm in the spectrograms (See
Fig.~\ref{fig:sgram}).  That higher frequency power was evidence of
respiration.  Using both of those features, he classified the test
data by hand.  First he classified each entire record for \emph{event
  1}.  Since almost none of the minutes in the normal records are
apnea, he classified each minute in those records as normal.  His
third attempt at \emph{event 2} was the best entry at the close of the
contest.

\begin{figure}
  \centering{\resizebox{\textwidth}{!}{\includegraphics{sgram.pdf}}
  }
  \caption[Information about respiration in high
  frequency phase variations]% FixMe: Perhaps use the same parameters
  % for this plot as used for the HMM.  Check after optimizing.
  {Information about respiration in high frequency bands of power
    spectra.  This is derived from the $a11$ record between 20 minutes
    and 2 hours and 30 minutes after the start.  The upper plot is
    heart rate (bandpass filtered 0.09-3.66 bpm), the middle plot is a
    spectrogram of the heart rate, and the lower plot is the expert
    classification.  A single band of spectral power between about 10
    and 20 bpm without much power below the band in the spectrogram
    indicates normal respiration.}
  \label{fig:sgram}
\end{figure}

The question that motivated the contest is ``Is the information in an
ECG alone sufficient to classify apnea?'' McNames work answered the
question affirmatively.  For attempts to do the classification
automatically, his work suggests the following points:
\begin{enumerate}
\item Heart rate oscillations at about 1.3 bpm indicate apnea.
\item A clean band at about 14 bpm in the power spectrum indicates
  normal respiration.
\end{enumerate}

\afterpage{\clearpage}%% Print this right here please.

\section{Using HMMs to Address the Challenge}
\label{sec:apnea_hmms}

In the following sections, as the final example in the book, we use
HMMs to address the CINC2000 challenge.  We begin with an HMM scheme
for estimating heart rate because, like McNames, we find off-the-shelf
QRS detection codes are not adequate for the variety of the CINC2000
data.  From the estimated heart rate signal for each record, we derive
a sequence of two dimensional observations.  The first component is
the low pass filtered heart rate, and the second is roughly the
intensity of the spectrogram in the range of respiration near 15
beats per minute.  Next we fit an HMM with vector autoregressive
observation models as described in Section~\ref{sec:ARVGaussian} to
the two dimensional observations derived from the training data
records.  And finally we apply that HMM to classify each minute of the
test data.

\subsection{Estimating Heart Rate}
\label{sec:ecg_hmms}

\begin{figure}
  \centering{\resizebox{0.8\textwidth}{!}{\includegraphics{elgendi.pdf}}
  }
  \caption[Inadequate off-the-shelf ECG detectors]%
  {In this segment of the ECG for record a03, stars indicate estimates
    of the R wave locations by an off-the-shelf detector\cite{porr},
    and $\times$s indicate estimates from the algorithm described in
    Section~\ref{sec:ecg_hmms}. }
  \label{fig:elgendi}
\end{figure}

\begin{figure}
  \centering{\resizebox{0.8\textwidth}{!}{\includegraphics{a03a10b03c02.pdf}}
  }
  \caption[Segments of ECGs from four records]%
  {Segments of ECGs from four records.  The ECGs for the different
    data records were so different from each other that off-the-shelf
    code was not adequate for estimating heart rate signals.}
  \label{fig:a03a10b03c02}
\end{figure}

\begin{figure}
  \centering{\resizebox{\textwidth}{!}{\includegraphics{constant_a03.pdf}}}
  \caption[Invariant PQRST shape]%
  {At different heart rates the shape and duration of the PQRST
    pattern doesn’t change. Only the delay between the sequences
    changes.  Notice that the lengths of the time intervals in the
    upper and lower plots are identical.}
  \label{fig:constant_a03}
\end{figure}

At first we tried using off-the-shelf code to extract heart rates from
the PhysioNet ECG data.  Figure~\ref{fig:elgendi} illustrates results
of using one of the detectors\footnote{Elgendi, Mohamed \& Jonkman,
  Mirjam \& De Boer, Friso. (2010). "Frequency Bands Effects on QRS
  Detection" The 3rd International Conference on Bio-inspired Systems
  and Signal Processing (BIOSIGNALS2010). 428-431.} implemented by
Porr et al.\footnote{https://github.com/berndporr/py-ecg-detectors},
and Figure~\ref{fig:a03a10b03c02} illustrates the challenging
diversity of waveforms in the PhysioNet data.  To address the
diversity of waveforms, we developed an approach to estimating heart
rates from ECGs based on HMMs that relies on the observation that for
each record the shape and duration of the $PQRST$ pattern doesn't
vary.  For different heart rates the time between the $T$ wave and the
next $P$ wave does change, but the rest of the pattern varies very
little.  Figure~\ref{fig:constant_a03} illustrates the invariance.

\subsubsection{State topology}
\label{sec:state_topology}

\begin{figure}
  \centering{\resizebox{0.5\textwidth}{!}{\input{ecg_hmm.pdf_t}}
  }
  \caption[Topology of HMMs for exploiting invariance of PQRST
  shapes]{Illustration of the topology of HMMs for exploiting
    invariance of PQRST shapes. The chain of 49 fast states models the
    invariant PQRST shape.  The variable duration paths through the
    slow states, $S_1$, $S_2$, and $S_3$, model the flexible time
    between $T$ and $P$ waves.}
  \label{fig:ecg_hmm}
\end{figure}

Figure~\ref{fig:ecg_hmm} illustrates the topology of the HMMs we built
to capture the invariant shapes of the PQRST sequences for each
individual record.  In addition to the states drawn, there is a
special \emph{outlier} state accommodates ECG-lead noise.  Here are
the essential characteristics of the topology:
\begin{itemize}
\item A loop of 52 discrete states.
\item A sequence of 49 \emph{fast} states that don't branch, state $n$ must
  transition to state $n+1$ at each time step.  These fast states
  model the invariant PQRST shape.
\item Three \emph{slow} states that model heart rate variations.
  Each of the three branches to one of the following:
  \begin{itemize}
  \item Itself
  \item Its successor
  \item The first fast state
  \end{itemize}
\end{itemize}
The minimum number of states visited in a loop is 50, or 500~ms since
the ECG data was sampled at 100~Hz.  Consequently the model is not
appropriate for heart rates above 120~bpm.

\subsubsection{Observation models and training}
\label{sec:ecg_training}

We used a Gaussian scalar autoregressive observation model with means
for each state being an affine function of the previous three
observations and variances fit to each state separately.  The models
are one dimensional versions of those described in
Section~\ref{sec:ARVGaussian}.

We used scipy.signal.find\_peaks to supervise training of an initial
model for one of the records from CINC 2000.  we derived models for
the most\footnote{For a few of the records we used other techniques to
  obtain initial models.  However models for all records were trained
  similarly.} other records from that initial model via unsupervised
training.

\subsubsection{Applying trained ECG models}
\label{sec:applying_ECG_models}

As intended, the trained HMMs track PQRST shapes in the ECG data well.
Figure~\ref{fig:a01c02_states} illustrates the performance on two
records, \emph{a01} and \emph{c02}, in which the ECG shapes are quite
different.  To be clear, we repeat that a separate HMM gets trained on
each record.
\begin{figure}
  \centering{\resizebox{0.8\textwidth}{!}{\includegraphics{a01c02_states.pdf}}
  }
  \caption[Decoded state sequences]{State sequences from Viterbi
    decoding of ECG signals for two records appear.  The invariant
    $PQRST$ pattern maps to the lines of constant slope.  The varying
    lengths of the horizontal segments accounts for variable time
    between heart beats.}
  \label{fig:a01c02_states}
\end{figure}

In addition to analyzing measured ECGs, the models can generate
simulated ECGs, eg, Figure~\ref{fig:simulated}.
\begin{figure}
  \centering{\resizebox{\textwidth}{!}{0.8\includegraphics{simulated.pdf}}
  }
  \caption[Simulated ECG]{A plausible ECG appears in the upper plot,
    and the corresponding state sequence appears in the lower plot.
    Driving a model fit to data with a random number generator created
  both plots.}
  \label{fig:simulated}
\end{figure}

\subsubsection{Mapping state sequences to heart rate}
\label{sec:states2hr}

\begin{figure}
  \centering{\resizebox{0.8\textwidth}{!}{\includegraphics{ecg2hr.pdf}}
  }
  \caption[Heart rate derived from ECG.]{An illustration of using
    Viterbi decoding to derive heart rate from an ECG.  An ECG segment
    appears in the upper plot, and the corresponding segment of the
    decoded state sequence appears in the middle plot.  The $\times$
    marks in the middle plot indicate the times when the state is 31.
    The same times are also indicated in the upper plot.  The inverse
    of the difference in time between adjacent $\times$s provides the
    estimated heart rate.  A smoothed version of those estimates
    sampled periodically at 2 Hz appears in the lower plot. }
  \label{fig:ecg2hr}
\end{figure}

We've built HMMs of ECG signals to estimate heart rate signals because
We believe that it is the heart rate that is relevant for detecting
apnea.  Decoded state sequences like those that appear in
Figure~\ref{fig:a01c02_states} are sufficient for estimating heart
rate.  Considering the circular character of the topology depicted in
Figure~\ref{fig:ecg_hmm}, the estimated heart rate is simply the rate
at which decoded sequences go around the circle, and the phase of
state sequence is not important for the estimation.  It is nice
however that one state in the chain of fast states corresponds closely
to the peak of the R wave in the ECGs as appears in
Figure~\ref{fig:ecg2hr}.

\afterpage{\clearpage}%% Print this right here please.

\subsection{An Observation Model and HMMs of Heart Rate}
\label{sec:apnea_observation_model}

While the previous section was about designing HMMs of ECG signals in
order to estimate heart rate signals, here we will address designing
HMMs of heart rate signals in order to detect apnea.  In
Section~\ref{sec:mcnames} we described two characteristics of apnea
that McNames used for classifying apnea, namely:
\begin{itemize}
\item Oscillations of the heart rate with periods of about 45 seconds.
  Such oscillations appear in Fig.~\ref{fig:a03erA}.
\item Modulation of the heart rate by respiration at around 15 beats
  per minute. Such modulation is captured by the spectrogram in
  Fig.~\ref{fig:sgram}.
\end{itemize}
We derive these two characteristics from heart rate signals (as appears
in Fig.~\ref{fig:ecg2hr}) sampled at 2 Hz using spectral techniques
that begin with a fast Fourier transform (FFT) of each entire record
of about 8 hours.  Figure~\ref{fig:explore} illustrates a transition into apnea
in which both characteristics indicate the transition.  The HMM we
built for detecting apnea has six states and uses $8^{\text{th}}$
order vector autoregressive observation models of two dimensional
observations, with components called \textbf{low pass heart rate} and
\textbf{respiration}.

\begin{figure}
  \centering{\resizebox{1.0\textwidth}{!}{\includegraphics{explore.pdf}}
  }
  \caption[Apnea characteristics derived from estimated heart rate
  signal.]{An illustration of apnea characteristic data derived from
    heart rate at a transition into apnea.  The expert marked a
    transition from normal respiration to apnea in record \emph{a03}
    at minute 427.  A low pass Gaussian filter applied to the raw
    heart rate signal produces the trace labeled \emph{Low Pass}, and
    the dots on that trace indicate the data passed as one component
    of the observation data to the HMM.  A Gaussian filter that passes
    frequencies in a band typical for respiration yields the
    \emph{Band Pass} trace.  The \emph{Respiration} trace is a low
    pass filtered version of the envelope of the \emph{Band Pass}
    trace.  Again, the dots indicate data passed to the HMM.  The
    characteristics of apnea are the large slow oscillations in the
    \emph{Low Pass} signal and the drops of the \emph{Respiration}
    signal to low levels.}
  \label{fig:explore}
\end{figure}

A schematic of the HMM topology appears in Figure~\ref{fig:viz}.
There are two states for modeling noise, one during apnea episodes and
one for normal periods.  There are two states each for normal periods
and for apnea episodes.

\begin{figure}
  \centering{\resizebox{1.0\textwidth}{!}{\includegraphics{viz.pdf}} }
  \caption[Topology of HMM for detecting apnea.]{Topology and some
    parameters of a six state HMM for detecting apnea is shown.  The
    figure was generated by the \emph{pygraphviz} software from the
    trained HMM.  A transition between every pair of states is
    possible.  The meaning of the colors of the transitions are:
    \textbf{black} fixed/untrained probability; \textbf{red}
    transition from an apnea state; \textbf{blue} transition from a
    normal state.}
  \label{fig:viz}
\end{figure}

\marginpar{FixMe: Min is at 3 but 4 is used}
\begin{figure}
  \centering{\resizebox{1.0\textwidth}{!}{\includegraphics{errors_vs_fs.pdf}} }
  \caption[Error rate as a function of sample frequency.]{Performance
    vs observations per minute on the training records (a b and c
    records excluding defective records). The models have six states
    and the observation models are vector autoregressive models for
    heart rate and respiration.}
  \label{fig:errors_vs_fs}
\end{figure}

Of the host of parameters, we chose the following for the observation
model by hand using one dimensional studies on the training data like
Figure~\ref{fig:errors_vs_fs}:
% \ArOrder et al come from apnea_values.tex
\begin{description}
\item[Autoregressive Order] \ArOrder
\item[Model Sample Frequency] \ModelSampleFrequency samples per minute
\item[Low Pass Period] \LowPassPeriod seconds
\item[Respiration Center Frequency] \RespirationCenterFrequency bpm
\item[Respiration Filter Width] \RespirationFilterWidth bpm
\item[Respiration Smoothing] \RespirationSmoothing bpm
\end{description}
Figure~\ref{fig:explore} illustrates the derivation of HMM observation
data from the estimated raw heart rate signal using these parameters.

\subsection{Using a Sequence of Class Probabilities}
\label{sec:prettygood}

Equation \eqref{eq:wit}, ie,
\begin{equation*}
  w(s,t) \equiv P_{\ti{S}{t}|\ts{Y}{1}{T}} \left(s|\ts{y}{1}{T}
  \right),
\end{equation*}
provides the probability of each state at each time given the model and
all of the observations.  From such state weights, we construct a
sequence of ratios with
\begin{equation*}
  \ti{r}{t} \equiv \frac{\sum_{s\in A} w(s,t)}{\sum_{s\in N} w(s,t)}
\end{equation*}
where $A$ is the set of apnea states and $N$ is the set of normal
states.  Finally we derive a sequence of classifications,
$\ts{c}{1}{T}$, from the sequence of ratios, $\ts{r}{1}{T}$, by
comparing to a threshold, $R$,
\begin{equation}
  \label{eq:class_sequence}
  \ti{c}{t} =
  \begin{cases}
    N & \ti{r}{t} < R \\
    A & \ti{r}{t} >= R
  \end{cases}.
\end{equation}
A search over parameter values like that illustrated in
Figure~\ref{fig:errors_vs_fs} finds $R=0.7$ minimizes the number of
classification errors on the training data.

\section{Results}
\label{sec:results}

We trained an HMM with the structure and parameters described in
Section~\ref{sec:apnea_observation_model} on the training data (the
usable \emph{a}, \emph{b} and \emph{c} records) and then used it to
classify each minute of the training data as either normal, $N$, or
apnea, $A$.  The results appear in Table~\ref{tab:score}.  Results of
applying the HMM to the test data (the \emph{x} records appear in
Table~\ref{tab:test_score}.  We are disappointed that the score on the
test data does not place among those listed as \emph{CinC Challenge
  2000 Top Scores} at
https://archive.physionet.org/challenge/2000/top-scores.shtml which
have error rates from 0.0738 to 0.1551.  While we were able to get
slightly better performance with more complex ideas, the small
improvements did not seem to justify the complexity.

\begin{table*}
  \centering
  \input{score.tex}
  \caption[Performance]{Performance of the HMM described in
    \ref{sec:apnea_observation_model} on the training data.}
  \label{tab:score}
\end{table*}

\begin{table*}
  \centering
  \input{test_score.tex}
  \caption[Performance]{Performance of the HMM described in
    \ref{sec:apnea_observation_model} on the test data.}
  \label{tab:test_score}
\end{table*}

\section{Classification Versus Estimation}
\label{sec:ClassVsEst}

In the early chapters of this book, we have emphasized choosing model
parameters to maximize the likelihood of the data, the tweaks and
fudges we've used in this chapter are concerned with improving
classification performance rather than improving likelihood.  While it
is true that if one had access to accurate probabilistic
characterizations of observations conditional on class membership the
best classifier would be a likelihood ratio classifier, it is not true
that without such characterizations the best approach to
classification is to estimate them.  This point was made forcefully by
Vapnik\cite{Vapnik98} who said, ``one should solve the
[classification] problem directly and never solve a more general
problem as an intermediate step.''

%%%
%%% Local Variables:
%%% TeX-master: "main"
%%% eval: (load-file "hmmkeys.el")
%%% End:

\appendix

\chapter{Formulas for Matrices and Gaussians}
\label{cha:MatrixFormulas}

Here we review some material necessary for deriving
Eqns.~\eqref{eq:KUpdate}-\eqref{eq:smoothing} on page
\pageref{eq:KUpdate}.  Similar material appears in Appendix A of
Kailath et al.\cite{KSH00}.

\subsubsection{Block Matrix Inverse}
\index{block matrix inverse}%
\index{matrix inverse|see{block matrix inverse}}%

If $G$ is an $n\times n$ invertible matrix, $K$ is an $m\times m$
invertible matrix, and $H$ and $J$ are $n\times m$ and $m\times n$
respectively, then direct matrix multiplication verifies that
\begin{subequations}
  \label{eq:MatrixInverse}
  \begin{align}
    \begin{bmatrix}
      (G-HK^{-1}J)^{-1} & -(G-HK^{-1}J)^{-1}HK^{-1} \\
      -(K-JG^{-1}H)^{-1}JG^{-1} & (K-JG^{-1}H)^{-1}
    \end{bmatrix}
    \begin{bmatrix}
      G & H \\ J & K
    \end{bmatrix}
    &=
    \begin{bmatrix}
      \id & 0 \\ 0 & \id
    \end{bmatrix}\\ \nonumber \\
    \begin{bmatrix}
      G & H \\ J & K
    \end{bmatrix}
    \begin{bmatrix}
      (G-HK^{-1}J)^{-1} & -G^{-1}H(K-JG^{-1}H)^{-1} \\
      -K^{-1}J(G-HK^{-1}J)^{-1} & (K-JG^{-1}H)^{-1}
    \end{bmatrix}
    &=
    \begin{bmatrix}
      \id & 0 \\ 0 & \id
    \end{bmatrix},
  \end{align}
\end{subequations}
assuming that $(G-HK^{-1}J)^{-1}$ and $(K-JG^{-1}H)^{-1}$ exist.
One can derive other expressions for the inverse by using the Sherman
Morrison Woodbury formula (Eqn.~\eqref{eq:SMW}) to expand terms in
Eqn.~\eqref{eq:MatrixInverse}.

By noting
\begin{multline*}
  \begin{bmatrix}
    (G-HK^{-1}J)^{-1} & -(G-HK^{-1}J)^{-1}HK^{-1} \\
    0 & (K-JG^{-1}H)^{-1}
  \end{bmatrix}
  \begin{bmatrix}
    G & H \\ J & K
  \end{bmatrix}\\
  =
  \begin{bmatrix}
    \id & 0 \\
     (K-JG^{-1}H)^{-1}J & (K-JG^{-1}H)^{-1} K
  \end{bmatrix}
\end{multline*}
and taking the determinant of both sides
\renewcommand{\det}[1]{\left| #1 \right|}
\begin{equation*}
  \det{(G-HK^{-1}J)^{-1}} \cdot \det{(K-JG^{-1}H)^{-1}} \cdot 
  \det{\begin{bmatrix}
    G & H \\ J & K
  \end{bmatrix}} = \det{(K-JG^{-1}H)^{-1}} \cdot \det {K}
\end{equation*}
one finds the following formula for determinants
\index{block matrix determinant}
\index{determinant|see{block matrix determinant}}
\begin{equation}
  \label{eq:BlockDet}
  \det{ \begin{bmatrix} G & H \\ J & K \end{bmatrix}} = \det{K}
  \det{(G-HK^{-1}J)}
\end{equation}
\subsubsection{Sherman Morrison Woodbury Formula}
\index{Sherman Morrison Woodbury formula}

If $G$ and $ K $ are invertible matrices, $H$ and $ J $ have
dimensions so that $\left( G + H K J \right)^{-1}$ makes sense and
exists, and $\left( J G^{-1}H + K ^{-1} \right)^{-1}$ exists, then
\begin{equation}
  \label{eq:SMW}
  \left( G + H K  J  \right)^{-1} = G^{-1} - G^{-1}H\left( J G^{-1}H +
   K ^{-1}\right)^{-1} J G^{-1}.
\end{equation}
Multiplying both sides by $\left( G + H K  J  \right)$ verifies the
formula.  Equation~\eqref{eq:SMW} is called the Sherman Morrison
Woodbury formula.

To invert $(A^{-1} + C\transpose B^{-1} C)$, when $A$ is an $n\times
n$ matrix and $B$ is an $m \times m$ matrix, if $n > m$ one can use
\eqref{eq:SMW} to write
\begin{equation}
  \label{eq:MatrixInversion}
  \left(A^{-1} + C\transpose  B ^{-1}C \right)^{-1} = A - AC\transpose
  \left( CAC\transpose +  B  \right)^{-1}CA.
\end{equation}
The right hand side requires inverting an $m \times m$ matrix while
the left hand side requires inverting an $n \times n$ matrix.

\subsubsection{Marginal and Conditional Distributions of a Gaussian}
\index{marginal distribution, Gaussian}
\index{conditional distribution, Gaussian}

Suppose that $W = \begin{bmatrix} U \\ V \end{bmatrix}$ is a Gaussian
random variable with an $n$ dimensional component $U$ and an $m$
dimensional component $V$.  We write its distribution
%
\nomenclature[rNormal]{$\Normal \left( \mu,\Sigma \right)$}{A
  \emph{normal} or Gaussian distribution function; $\mu$ is an $n$
  dimensional vector and $\Sigma$ is an $n\times n$ matrix.  Writing
  $X\sim \Normal \left( \mu,\Sigma \right)$ means $X$ is distributed
  normally with mean $\mu$ and covariance $\Sigma$ and the
  probability density at any particular vector $x$ is
  $\NormalE{\mu}{\Sigma}{x}$.
} % end \nomenclature
%
\nomenclature[rNormalE]{$\NormalE{\mu}{\Sigma}{x}$}{The value of a
  Gaussian probability density function evaluated at the $n$
  dimensional vector $x$, \ie
  \begin{equation*}
    \NormalE{\mu}{\Sigma}{x} \equiv \frac{1}
    {\sqrt{(2\pi)^n\left"|\Sigma\right"|}} e^{ -\frac{1}{2}
      (x-\mu)\transpose \Sigma^{-1} (x-\mu)}
  \end{equation*}
  where $\left"|\Sigma\right"|$ is the determinant of the covariance
  matrix $\Sigma$.
} % end \nomenclature
\begin{equation*}
  W \sim \Normal \left( \mu_W,\Sigma_W \right) \text{ or equivalently
  } P(w) = \NormalE{\mu_W}{\Sigma_W}{w}
\end{equation*}
with
\begin{equation*}
  \mu_W = \begin{bmatrix} \mu_U\\\mu_V \end{bmatrix} \text{ and }
  \Sigma_W =   \begin{bmatrix} \Sigma_{UU} & \Sigma_{UV}\\ \Sigma_{VU}
  & \Sigma_{VV} \end{bmatrix} \equiv \begin{bmatrix} A & C \\
  C\transpose & B \end{bmatrix},
\end{equation*}
where we have introduced $A \equiv \Sigma_{UU}$, $B \equiv
\Sigma_{VV}$, and $C \equiv \Sigma_{UV}$ to shorten the notation.
If we denote
\begin{equation*}
    \Sigma_W^{-1} =   \begin{bmatrix} D & F \\ F\transpose & E
  \end{bmatrix},
\end{equation*}
then from Eqns.~\eqref{eq:MatrixInverse} and ~\eqref{eq:SMW}
\begin{subequations}
  \label{eq:MI:ABCDEF}
  \begin{align}
    \label{eq:MI:D}
    D &= (A - CB^{-1}C\transpose)^{-1} & &= A^{-1} + A^{-1}C E
    C\transpose A^{-1}\\
    \label{eq:MI:E}
    E &= (B - C\transpose A^{-1}C)^{-1} & &= B^{-1} + B^{-1}C\transpose
    D C B^{-1} \\
    \label{eq:MI:F}
    F &= -A^{-1}C E & &= -DCB^{-1}.
  \end{align}
\end{subequations}
In this notation, the marginal distributions are
\begin{subequations}
  \label{eq:Gauss-Marginal}
  \begin{align}
    P\left(u \right) &= \int P\left(u,v \right) dv \\
    &= \NormalE{\mu_U}{A}{u} \\
    P\left(v \right) &= \int P\left(u,v \right) du \\
    &= \NormalE{\mu_V}{B}{v},
  \end{align}
\end{subequations}
and the conditional distributions are
\begin{subequations}
  \label{eq:Gauss-Conditional}
  \begin{align}
    P\left(u \given v \right) &= \frac{ P\left(u,v \right)}{P\left(v \right)} \\
    &= \NormalE{\mu_U + CB^{-1}(v-\mu_v)}{D^{-1}}{u} \\
    P\left(v \given u \right) &= \frac{ P\left(v,u \right)}{P\left(u \right)} \\
    &= \NormalE{\mu_V + C\transpose A^{-1} (u - \mu_U)}{E^{-1}}{v}
  \end{align}
\end{subequations}
Notice that the covariance of the \emph{marginal} distribution of $U$
is given by the $UU$ block of $\Sigma_W$, but that the inverse
covariance of the \emph{conditional} distribution of $U$ is given by
the $UU$ block of $\Sigma_W^{-1}$.

As a check of these formulas, we examine $P\left(u \given v \right) P\left(v
\right)$ and find
\begin{align*}
  P\left(u \given v \right) P\left(v \right) &= \frac{\sqrt{\left| D
      \right|}} {\sqrt{(2\pi)^n}} e^{ -\frac{1}{2} \left(u - \mu_U -
      CB^{-1}(v - \mu_V) \right)\transpose D \left(u - \mu_U -
      CB^{-1}(v - \mu_V) \right)
  } \\
  &\quad \times \frac{1}{\sqrt{(2\pi)^m \left| B \right|}} e^{ -\frac{1}{2}
    (v-\mu_V)\transpose B^{-1} (v-\mu_V)}\\
                                %
    &= \frac{1} {\sqrt{(2\pi)^{n+m} \left| \Sigma_W
      \right| }} \exp \bigg( -\frac{1}{2} \Big[ \\
  &\quad \left( u - \mu_U - CB^{-1}(v - \mu_V) \right)\transpose D
       \left(u - \mu_U - CB^{-1}(v - \mu_V) \right)  \\
     &\quad + (v-\mu_V)\transpose B^{-1} (v-\mu_V) \Big]\bigg)\\
                                  %
     &= \frac{1} {\sqrt{(2\pi)^{n+m} \left| \Sigma_W \right| }}\\
     &\quad\times e^{ -\frac{1}{2} \left((u-\mu_U)\transpose D (u-\mu_U) +
         2 (v-\mu_V)\transpose F\transpose (u-\mu_U) +
         (v-\mu_V)\transpose E (v-\mu_V)\right)}\\
     &= P \left(u,v \right)
\end{align*}
all is right with the world.  In the above, Eqn.~\eqref{eq:BlockDet}
implies that $\frac{\sqrt{\det{D}}}{\sqrt{\det{B}}} =
\frac{1}{\sqrt{\det{\Sigma_W}}}$.


\subsubsection{Completing the Square}
\index{completing the square}

Some of the derivations in section~\ref{sec:KDerive} rely on a
procedure called \emph{completing the square}, which we illustrate
with the following example.  Suppose that the function $f(u)$ is the
product of two $n$ dimensional Gaussians,
$\Normal\left(\mu_1,\Sigma_1\right)$ and
$\Normal\left(\mu_2,\Sigma_2\right)$, \ie
\begin{align}
  \label{eq:Qu.a}
  f(u) &= \frac{1} {\sqrt{(2\pi)^n \left| \Sigma_1 \right| }}
  e^{-\frac{1}{2} ( u - \mu_1)\transpose \Sigma_1^{-1} ( u - \mu_1)}
  \frac{1} {\sqrt{(2\pi)^n \left| \Sigma_2 \right| }}
  e^{-\frac{1}{2} ( u - \mu_2)\transpose \Sigma_2^{-1} ( u - \mu_2)}\\
  \nonumber
       &= \frac{1} {\sqrt{(2\pi)^{2n} \left| \Sigma_1 \right| \left|
        \Sigma_2 \right| }} e^{-\frac{1}{2}\big[ ( u -
    \mu_1)\transpose \Sigma_1^{-1} ( u - \mu_1) + ( u -
    \mu_2)\transpose \Sigma_2^{-1} ( u - \mu_2)\big]}\\
  \nonumber
  &\equiv \frac{1} {\sqrt{(2\pi)^{2n} \left| \Sigma_1 \right| \left|
        \Sigma_2 \right| }} e^{-\frac{1}{2}\big[Q(u)\big]}.
\end{align}
By expanding the function $Q(u)$ in the exponent, we find:
\begin{align}
  \nonumber
  Q(u) &= u\transpose \left( \Sigma_1^{-1} + \Sigma_2^{-1} \right) u -
  2 u\transpose  \left( \Sigma_1^{-1} \mu_1 + \Sigma_2^{-1} \mu_2
  \right) + \mu_1\transpose \Sigma_1^{-1} \mu_1 + \mu_2\transpose
  \Sigma_2^{-1} \mu_2 \\
  \label{eq:Qu.e}
  &= u\transpose q u - 2 u\transpose l + s
\end{align}
where the quadratic, linear, and scalar terms are
\begin{align*}
  q &= \left( \Sigma_1^{-1} + \Sigma_2^{-1} \right) \\
  l &= \left( \Sigma_1^{-1} \mu_1 + \Sigma_2^{-1} \mu_2 \right) \\
  s &= \mu_1\transpose \Sigma_1^{-1} \mu_1 + \mu_2\transpose
  \Sigma_2^{-1} \mu_2
\end{align*}
respectively.

\emph{Completing the square} means finding values $\mu$, $\Sigma$, and
$R$ for which Eqn.~\eqref{eq:Qu.e} takes the form
\begin{equation*}
  Q(u) = (u - \mu)\transpose \Sigma^{-1} (u - \mu) + R,
\end{equation*}
where $R$ is not a function of $u$.  One can verify by substitution
that the solution is
\begin{align*}
  \Sigma^{-1} &= q\\
  \mu &= \Sigma l\\
  R &= s -  \mu\transpose \Sigma^{-1} \mu.
\end{align*}
For the product of Gaussians example \eqref{eq:Qu.a},
\begin{subequations}
  \label{eq:GaussianProduct}
  \begin{align}
  \Sigma^{-1} &= \Sigma_1^{-1} + \Sigma_2^{-1} \\
  \mu &= \Sigma \left( \Sigma_1^{-1} \mu_1 + \Sigma_2^{-1} \mu_2
  \right) \\
  &= \left( \Sigma_1^{-1} + \Sigma_2^{-1} \right)^{-1} \left(
  \Sigma_1^{-1} \mu_1 + \Sigma_2^{-1} \mu_2  \right) \\
  R &= \mu_1\transpose \Sigma_1^{-1} \mu_1 + \mu_2\transpose
  \Sigma_2^{-1} \mu_2 -  \mu\transpose \Sigma^{-1} \mu \\
  &= \mu_1\transpose \Sigma_1^{-1} \mu_1 + \mu_2\transpose
  \Sigma_2^{-1} \mu_2 - \left( \Sigma_1^{-1} \mu_1 + \Sigma_2^{-1}
  \mu_2  \right) \transpose \left( \Sigma_1^{-1} + \Sigma_2^{-1}
  \right)^{-1} \left( \Sigma_1^{-1} \mu_1 + \Sigma_2^{-1} \mu_2
  \right).
\end{align}
\end{subequations}
In words the product of two Gaussian density functions is an
unnormalized Gaussian density function in which the inverse covariance
is the sum of the inverse covariances of the factors and the mean is
the average of the factor means weighted by the inverse covariances.

\chapter{EM Convergence Rate}
\label{chap:em_appendix}

\newcommand{\OldParameters}{\parameters}
\newcommand{\NewParameters}{\parameters'} %
Here we calculate the linear approximation of the behavior of
$\EMmap$, the map implemented by the EM algorithm, in the neighborhood
of a fixed point, $\EMfixedPoint$.  We show that the linear stability
properties of $\EMmap$ are the same as the linear stability properties
of the flow of solutions of the ordinary differential
equation\footnote{In \eqref{eq:LogDynamics} we use the log of the
  likelihood as a potential function while in
  \eqref{eq:GradientDynamics}, we simply used the likelihood.  The two
  equations have the same fixed points, and the linear stability of
  those fixed points are also the same.}
\begin{equation}
  \label{eq:LogDynamics}
  \dot \parameters = \frac{d}{d \parameters} 
  \LogLikelihood(\parameters) ,
\end{equation}
where $\LogLikelihood(\parameters) \equiv \log \left( P \left( y \given
    \parameters \right) \right)$ is the log likelihood. Like trajectories
defined by the differential equation, trajectories of $\EMmap$
generically converge to local maxima of $\LogLikelihood$, and they
avoid local minima and saddles.

The manipulations here require that the probability and likelihood
functions have continuous first and second derivatives and that some
second derivatives have full rank which we implicitly assume.  We let
$\EMmap$ denote the action of one iteration of the algorithm and let
$\EMfixedPoint$ denote a fixed point with
\begin{align*}
  \ti{\parameters}{n+1} &= \EMmap\left( \ti{\parameters}{n} \right) \\
  \EMmap(\EMfixedPoint) &= \EMfixedPoint && \text{with Taylor series} \\
  \EMmap(\parameters) &= \EMfixedPoint + \left[ \frac{\partial
                 \EMmap(\parameters)}{\partial \parameters} \right]_{\EMfixedPoint} (\parameters - \EMfixedPoint) + \text{Remainder}.
\end{align*}
Recall that the EM map depends on the auxiliary function, $Q$, with
\begin{align*}
  Q(\NewParameters, \OldParameters) &\equiv \EV_{S \given y,\OldParameters} \log\left(P(y,S \given
  \NewParameters \right) \\
  \EMmap(\OldParameters) &= \argmax_{\NewParameters} Q(\NewParameters, \OldParameters). 
\end{align*}
For a given value of $\OldParameters$, the derivative of $Q(\NewParameters,
\OldParameters)$ at a maximum is zero, and we write
\begin{align}
  \nonumber
  \Psi(\NewParameters,\OldParameters)
  &\equiv \frac{\partial Q(\NewParameters,
    \OldParameters)}{\partial \NewParameters} \\
  \nonumber
  \Psi(\EMmap(\OldParameters), \OldParameters)
  &= 0 \\
  \nonumber
  \frac{d\Psi(\EMmap(\OldParameters), \OldParameters)}{d \OldParameters}
  &= 0 \\
  \nonumber
  \frac{d\Psi(\EMmap(\OldParameters), \OldParameters)}{d \OldParameters}
  &= \left. \frac{\partial \Psi(\NewParameters, \OldParameters)}{\partial \NewParameters}
    \right|_{\EMmap(\OldParameters), \OldParameters} \left.\frac{\partial
    \EMmap(\parameters)}{\partial \parameters} \right|_{\OldParameters} +
    \left. \frac{\partial \Psi(\NewParameters, \OldParameters)}{\partial \OldParameters}
    \right|_{\EMmap(\OldParameters), \OldParameters} \\
  \nonumber
  &= \left. \frac{\partial^2 Q(\NewParameters, \OldParameters)}{\partial \NewParameters^2}
    \right|_{\EMmap(\OldParameters), \OldParameters} \left.\frac{\partial
    \EMmap(\parameters)}{\partial \parameters} \right|_{\OldParameters} +
    \left. \frac{\partial^2 Q(\NewParameters, \OldParameters)}{\partial \OldParameters
    \partial \NewParameters} \right|_{\EMmap(\OldParameters), \OldParameters} =0 \\
  \label{eq:two_second_derivatives}
  \left.\frac{\partial \EMmap(\parameters)}{\partial \parameters}
  \right|_{\OldParameters}
  &= - \left[ \left. \frac{\partial^2 Q(\NewParameters,  \OldParameters)}{\partial
    \NewParameters^2} \right|_{\EMmap(\OldParameters), \OldParameters} \right]^{-1} \left[
    \left. \frac{\partial^2 Q(\NewParameters, \OldParameters)}{\partial \OldParameters
    \partial \NewParameters} \right|_{\EMmap(\OldParameters), \OldParameters} \right].
\end{align}
Manipulating the first of the two second derivatives in
\eqref{eq:two_second_derivatives} we find
\begin{align}
  \nonumber
  \frac{\partial^2 Q(\NewParameters,  \OldParameters)}{\partial \NewParameters^2}
  &= \frac{\partial^2}{\partial \NewParameters^2} \EV_{S \given y, \OldParameters}
    \log \left( P(y, S \given \NewParameters \right) \\
  \nonumber
  &= \frac{\partial^2}{\partial \NewParameters^2} \EV_{S \given y, \OldParameters}
    \left( \log \left( P(y \given \NewParameters \right) + \log \left( P(S
    \given y, \NewParameters \right) \right)\\
  \nonumber
  &=  \frac{\partial^2}{\partial \NewParameters^2} \log \left( P(y \given
    \NewParameters \right) + \EV_{S \given y, \OldParameters} \left(
    \frac{\partial^2}{\partial \NewParameters^2} \log \left( P(S \given y,
    \NewParameters \right) \right) \\
  \label{eq:second_partial_information}
  &= -J_y - I_{S \given y},
\end{align}
where
\begin{equation*}
  J_y\equiv-\frac{\partial^2}{\partial \parameters^2} \log \left( P(y \given
  \parameters \right)
\end{equation*}
is called the \emph{observed information} that $y$ provides about
$\NewParameters$, and
\begin{equation}
  \label{eq:second_partial_fisher}
  I_{S \given y} = -\EV_{S \given y, \OldParameters} \left(
    \frac{\partial^2}{\partial \NewParameters^2} \log \left( P(S \given y,
    \NewParameters \right) \right)
\end{equation}
is the \emph{Fisher information}\footnote{Fisher Information is
  defined in terms of the squared first derivative, see
  \eqref{eq:score_fisher}.  In many cases a calculation in terms of
  the second derivative like \eqref{eq:second_partial_fisher} is
  equivalent.}  of the
  unobserved data.  Now manipulating the second of the two second
  derivatives in \eqref{eq:two_second_derivatives} we find
\begin{align}
  \nonumber
  \frac{\partial^2 Q(\NewParameters,  \OldParameters)}{\partial \NewParameters \partial
  \OldParameters}
  &= \frac{\partial^2}{\partial \NewParameters \partial \OldParameters} \EV_{S \given y, \OldParameters}
    \log \left( P(y, S \given \NewParameters \right) \\
  \nonumber
  &= \frac{\partial^2}{\partial \NewParameters \partial \OldParameters} \EV_{S \given y, \OldParameters}
    \left( \log \left( P(y \given \NewParameters \right) + \log \left( P(S
    \given y, \NewParameters \right) \right)\\
  \nonumber
  &= \frac{\partial^2}{\partial \NewParameters \partial \OldParameters} \log \left( P(y \given
    \NewParameters \right) + \frac{\partial}{\partial \OldParameters} \EV_{S \given y, \OldParameters} \left(
    \frac{\partial}{\partial \NewParameters} \log \left( P(S \given y,
    \NewParameters \right) \right) \\
  \nonumber
  &=  \frac{\partial}{\partial \OldParameters} \EV_{S \given y, \OldParameters} \left(
    \frac{\partial}{\partial \NewParameters} \log \left( P(S \given y,
    \NewParameters \right) \right) \\
  \nonumber
  &= \frac{\partial}{\partial \OldParameters} \sum_s P(s \given y, \OldParameters) %
    \frac{ %
    \frac{\partial P(s \given y,\NewParameters )}{\partial \NewParameters}
    }{P(s \given y,\NewParameters )}  \\
  \nonumber
  &= \sum_s P(s \given y, \OldParameters) %
    \frac{ %
    \frac{\partial P(s \given y,\OldParameters )}{\partial \OldParameters}
    }{P(s \given y,\OldParameters )}
    \frac{ %
    \frac{\partial P(s \given y,\NewParameters )}{\partial \NewParameters}
    }{P(s \given y,\NewParameters )}
\end{align}
At the fixed point $\EMfixedPoint = \NewParameters = \OldParameters$
and
\begin{align}
  \label{eq:score_fisher}
  \left. \frac{\partial^2 Q(\NewParameters,  \OldParameters)}{\partial \NewParameters
  \partial \OldParameters} \right|_{\EMfixedPoint, \EMfixedPoint}
  &= \EV_{S \given y, \EMfixedPoint} \left( \frac{\partial}{\partial
    \NewParameters} \log \left( P(S \given y, \NewParameters )\right) \right)^2\\
  \label{eq:mixed_partial_information}
  &\equiv I_{S \given y}.
\end{align}
Combining \eqref{eq:mixed_partial_information} and
\eqref{eq:second_partial_information} with
\eqref{eq:two_second_derivatives} we write
\begin{equation}
  \label{eq:information_em_derivative}
  \left.\frac{\partial \EMmap(\parameters)}{\partial \parameters}
  \right|_{\EMfixedPoint} = \left[J_y + I_{S \given y}  \right]^{-1}
  I_{S \given y}.
\end{equation}
Equation~\eqref{eq:information_em_derivative} matches our intuition.
If the observed information, $J_y$, is much larger than the unobserved
information, $I_{S \given y}$, the derivative is small and the
convergence is fast.  Alternatively, if the unobserved information
dominates, then the derivative is close to one and the convergence is
slow\footnote{While Dempster Laird and Rubin\cite{Dempster77} make a
  similar comment about their Theorem 4, Alexis Roche's unpublished
  ``EM algorithm and variants: an informal tutorial'' inspired our
  derivation.}.

\subsection*{Linear Stability of EM}
\label{em_stabiltiy}

Here we use an idea inspired by Sylvester's law of
inertia\footnote{Sylvester's law of inertia is: If $B$ is a symmetric
  matrix, then for any invertible matrix $A$, the number of positive,
  negative and zero eigenvalues (called the inertia of the matrix) of
  $C = A B A\transpose$ is constant.} to show that if at
$\EMfixedPoint$ $J_y$ is positive definite then $\EMmap$ is linearly
stable.

We need the following lemma: If $A$ is positive definite and symmetric
and $B$ is positive definite and symmetric then the eigenvalues of
their product $C = AB$ are positive.  Because $A$ is positive
definite and symmetric, there is an $X$ with
\begin{align*}
  A &= X X\transpose  &&\text{and we can define}\\
  \Gamma &\equiv X^{-1} C X = X^{-1} X X\transpose B X \\
  \Gamma &= X\transpose B X \\
\end{align*}
By assumption $B$ is positive definite so
$y\transpose B y > 0 ~ \forall y \neq 0$.  Now
$\forall z \neq 0, ~ z\transpose X\transpose B X z > 0$ because $Xz$
is a $y$.  So $\Gamma$ is positive definite.  The symmetry of $B$
implies that $\Gamma$ is also symmetric.  Thus $\Gamma$ is positive
definite and symmetric\footnote{The symmetry of $\Gamma$ ensures that
  its eigenvalues are real.}, and all of its eigenvalues are positive.
Because they are related by a similarity transformation, $\Gamma$ and
$C$ have the same eigenvalues, and we know that all of the eigenvalues
of $AB = C$ are positive.

At a fixed point $\EMfixedPoint$, $\EMmap$ is linearly stable if and
only if $\left| \lambda \right| < 1$ for all eigenvalues $\lambda$ of
its derivative $D$.  If $I_{S \given y}$ is full rank, we can rewrite
\eqref{eq:information_em_derivative} as
\begin{align}
  \nonumber
  D &= \left[I_{S \given y}^{-1} J_y + 1 \right]^{-1} \\
  \nonumber
  D^{-1} &= I_{S \given y}^{-1} J_y + 1 \\
  \label{eq:dIJY}
  D^{-1} -1  &= I_{S \given y}^{-1} J_y
\end{align}
Since the right hand side of \eqref{eq:dIJY} satisfies the premises of
the lemma, each of its eigenvalues $\lambda_R$ is positive.  Now for
each eigenvalue of the right hand side there is an eigenvalue of $D$
with
\begin{align}
  \nonumber
  \frac{1}{\lambda_D} - 1 &= \lambda_R  \\
  \nonumber
  \lambda_D &= \frac{1}{1+\lambda_R} &&\text{and since } \lambda_R >0 \\
  \label{eq:D_is_stable}
  0 &< \lambda_D < 1.
\end{align}
Thus if the eigenvalues of $J_y$ are positive (because it is
symmetric, this is equivalent to it being positive definite) then
$\EMmap$ is linearly stable.  A similar argument shows that if
$J_y\equiv-\frac{\partial^2}{\partial \parameters^2} \log \left( P(y \given
  \parameters \right)$ has negative eigenvalues then $\EMmap$ is linearly
unstable.

In summary: Qualitatively $\EMmap$ acts like a gradient flow on the
log likelihood,
$\LogLikelihood(\parameters) \equiv\log \left( P(y \given \parameters
\right)$; convergence to fixed point of $\EMmap$ that is a local
maximum of the likelihood is generic and convergence to a saddle point
of the likelihood is not generic.

\subsubsection{Comments}
\label{sec:ConvergenceRate}

The analysis of this appendix shows that in a neighborhood of a local
maximum of the likelihood the largest eigenvalue,
$\lambda_{\text{max}}$, of $D$ is between 0 and 1.  In the limit of
large $n$ the convergence of the EM algorithm is linear, with an error
$\ti{\epsilon}{n} \equiv \ti{\parameters}{n} - \EMfixedPoint$
dominated by the largest eigenvalue, $\lambda_{\text{max}}$ of $D$,
with
\begin{align*}
  \ti{\epsilon}{n+1} \approx \lambda_{\text{max}} \ti{\epsilon}{n}
  &&\text{for large } n \text{ and }\\
  \lim_{n \rightarrow \infty} \frac{1}{n}\log(\ti{\epsilon}{n}) &= \log(\lambda_{\text{max}}).
\end{align*}

One generally uses the EM algorithm when maximizing the auxiliary
function $Q$ is much easier than estimating derivatives such as
$\delta(\parameters) \equiv \frac{\partial}{\partial \parameters} \log
\left( P(y \given \parameters )\right)$ or
$H(\parameters) \equiv \frac{\partial^2}{\partial \parameters^2} \log
\left( P(y \given \parameters \right)$.  If it is easy to estimate
$\delta$ and $H$, one should use the Newton scheme
\begin{equation*}
  \ti{\parameters}{n+1} = \ti{\parameters}{n} - \left(H(\ti{\parameters}{n})\right)^{-1} \delta(\ti{\parameters}{n})
\end{equation*}
which converges quadratically.  In
Equation~\eqref{eq:information_em_derivative} we have expressed $D$ in
terms of the second derivative $J_y\equiv -H$ and the term $I_{S|y}$
which would be even more difficult to calculate.  We do not suggest
calculating $J_y$, $I_{S|y}$, or $D$, because if you could do that,
you should use a Newton method.  The analysis here both assures that
generic trajectories of the EM algorithm only converge to local maxima
of the likelihood and in Equation~\eqref{eq:information_em_derivative}
gives one a sense of how the convergence rate depends on
characteristics of the likelihood and the unobserved data.

\chapter{Notes on Data and Software}
\label{cha:Software}

We found that writing the text for this book took less time than
writing the supporting software.  All of the software we've used (both
the code we've written and the software that that code depends on) for
this book is free software.  After fetching our code, typing ``make
book'' in the top level directory of the hmmds project will create a
copy of the book in a file called \emph{main.pdf} after some hours of
computation.  \ToDo{For now (2025-02-23) the code specifically for the
  book is at gitlab.com:fraserphysics/hmmds.git, and the more general
  code which that relies on is at gitlab.com:fraserphysics1/hmm.git.
  I develop on NixOS systems.  I intend to repackage the code to make
  it easy for others to use.}

\section*{Data}

We used the following sets of data for examples:
\begin{description}
\item[Tang's laser data] Carl Otto Weiss \index{Weiss, C.\ O.} mailed
  us a CD full of data from various experiments that he and Tang did
  in the 1980s and 1990s.  Although we analyzed many of the files, we
  finally used only a file called \emph{LP5.DAT} in the book (see
  Section~\ref{sec:laser}).  The file \emph{LP5.DAT} is included in
  \emph{hmmdsbook}.
\item[H.~L.~Mencken's \emph{A Book of Prefaces}] We used Mencken's
  book for the parts of speech example in Section~\ref{sec:POSpeech}.
  Although the code fetches the book from www.gutenberg.org as of
  April, 2007, we planned to include the parsed text in
  \emph{hmmdsbook}.
\item[CINC2000 ECG data] We used Dr. Thomas Penzel's ECG measurements
  throughout Chapter~\ref{chap:apnea}.  Although the code fetches the
  the data from\\
  www.physionet.org/physiobank/database/apnea-ecg as of April, 2007,
  we planned to include much smaller files that contain estimates of
  the timing of heart beats in \emph{hmmdsbook}.
\end{description}

\section*{Clarity and Efficiency}

Before the SciPy or NumPy packages existed, we wrote early versions of
the code for this book in C to make it run fast.  Since we wrote those
early versions, SciPy and NumPy have made most of that old C code
obsolete.  Now we have Python code for all of the algorithms described
in the book.

We also provide Cython code for a few of the algorithms.  While the
Cython code is faster, it is harder to read and debug.  The interfaces
to call Cython code match the interfaces to the Python code.  We
recommend developing with the Python code and after that if you need
the speed, try the Cython versions.

Here is the heart of our simple Python implementation of the forward
algorithm described in about 4 pages in Section~\ref{sec:forward}.  It
looks pretty simple here.
\label{code:forward}
\begin{verbatim}
        # last is a conditional distribution of state probabilities.
        # What it is conditioned on changes as the calculations
        # progress.
        last = numpy.copy(self.p_state_initial.reshape(-1))
        for t, likelihood_t in enumerate(self.state_likelihood):
            last *= likelihood_t  # Element-wise multiply
            self.gamma_inv[t] = 1 / last.sum()
            last *= self.gamma_inv[t]
            self.alpha[t, :] = last
            last[:] = numpy.dot(last, self.p_state2state)
\end{verbatim}
And here is the heart of the corresponding implementation of the
backward algorithm.  It is even simpler.
\label{code:backward}
\begin{verbatim}
        # last and beta are analogous to last and alpha in forward(),
        # but the precise interpretations are more complicated.
        last = numpy.ones(self.n_states)
        for t in range(len(self.state_likelihood) - 1, -1, -1):
            self.beta[t, :] = last
            last *= self.state_likelihood[t] * self.gamma_inv[t]
            last[:] = numpy.dot(self.p_state2state, last)
\end{verbatim}

%%% Local Variables:
%%% TeX-master: "main"
%%% eval: (load-file "hmmkeys")
%%% mode: LaTeX
%%% End:

% LocalWords:  Welch Viterbi HMM



\backmatter
% Begin ``Notation'' material
\renewcommand{\nomname}{Notation}%
\renewcommand{\pagedeclaration}[1]{\ (page #1)}%
\setlength{\nomlabelwidth}{2.0cm}%
\renewcommand{\nompreamble}{%
  Throughout the book we write about random variables and stochastic
  processes.  We have tried to select notation that is as simple as
  possible without being ambiguous.  To illustrate the challenge
  suppose that we have been talking about gambling and the weather
  and that we say ``the probability of 5 is 0.16''.  Among the many
  things that we could mean are the following:
  \begin{equation}
    \label{eq:dice}
    \mathbf{Prob}(\text{top face of die}=5) = 0.16
  \end{equation}
  \begin{equation}
    \label{eq:rain}
    \lim_{\epsilon \rightarrow 0} \frac{\mathbf{Prob}(5 < \text{total
        rainfall today in millimeters}<5+\epsilon)}{\epsilon} = 0.16
  \end{equation}
  As a less cumbersome notation we prefer $P_D(5)=0.16$ or
  $P_R(5)=0.16$, where $P$ denotes either a probability mass function
  or a probability density function.  We only use a subscript when it
  is necessary to specify which function we mean.

  In general we use the following conventions:
  \begin{symbdescription}
  \item[$X$] Upper case indicates a random variable.  Although the
    notation does not suggest it, the notion of a random variable
    includes a set of possible values or outcomes and a probability
    distribution for those values.
  \item[$x$] Lower case indicates an \emph{outcome} or value of a
    random variable.
  \item[${\cal X}$] Occasionally we use a calligraphic font to
    indicate the \emph{alphabet} or set of all possible values of a
    random variable $X$.
  \item[$\ts{X}{1}{T}$] A stochastic process indexed by the sequence
    $[1,2,3,\ldots,T]$, \ie,\\
    $\ti{X}{1},\ti{X}{2}, \ldots, \ti{X}{T}$.
  \item[$\ts{x}{1}{T}$] A particular possible outcome of the stochastic
    process $\ts{X}{1}{T}$.
  \item[$\ti{X}{t}$] The random variable that results from picking a
    single component of a stochastic process.
  \item[$P(x)$] The probability (or density) that a random variable
    will have the particular value $x$.  We do not put a subscript on
    $P$ when the context permits us to drop it without ambiguity.
  \item[$P_{\ti{X}{t}} \left(x \right)$] The probability that the
    value of $\ti{X}{t}$ is $x$.
  %% FixMe: This next label is a lot wider than the rest.  I cannot
  %% FixMe: figure out how to move the margin for only one \item so
  %% FixMe: I am leaving it the way it is.  Good Enough?
  \item[$P_{\ti{X}{t+1}|\ti{X}{t}} \left(x_1|x_2 \right)$] The
    conditional probability that the value of $\ti{X}{t+1}$ is $x_1$
    given that the value of $\ti{X}{t}$ is $x_2$.
  \item[$\mu(\beta)$] The probability that an outcome is in the
    \emph{set} $\beta$.  We occasionally use this notation from
    measure theory in Chapter~\ref{chap:toys}.
  \item[$P \left(x|\theta \right)$] Rather than a subscript, we
    occasionally use a conditioning variable to specify one of many
    possible probability distributions.
 \end{symbdescription}

  \section*{Symbols}
  The following symbols usually have the meanings described below:
}
\cleardoublepage%
\markboth{\nomname}{\nomname}%% Required to get correct page heading labels.
\printnomenclature
% End ``Notation'' material

\chapter*{Bibliography}
\addcontentsline{toc}{chapter}{Bibliography}%

\bibliography{hmmds}% \bibliographystyle is set in class definition.

%%% \printindex prints the label ``Index'' already, and also seems to
%%% do a \clearpage first.
%%%
\addcontentsline{toc}{chapter}{Index}%
\printindex[default][Page numbers set in \textbf{bold} type indicate
that the entry refers to a definition.]

\end{document}

ToDo:

fig:LaserLP5: Need 2 traces
fig:LaserForecast: Need 2 traces
Figure 2.4 (MAP sequence of states is not sequence of MAP states) is gone.
fig:LikeLor: Extend n_states to at least 8e6.  Try 10^8.  Draw bound.

%%%---------------
%%% Local Variables:
%%% eval: (load-file "hmmkeys.el")
%%% End:
